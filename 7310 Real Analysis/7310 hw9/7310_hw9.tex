\documentclass[12pt,letterpaper]{article}

%--------Packages--------
\usepackage{amsmath, amsthm, amssymb}
\usepackage{xspace}
\usepackage{graphicx}
\usepackage{hhline}
\usepackage{amssymb}
\usepackage{array}
\usepackage{braket}
\usepackage{multicol}
\usepackage{mathtools}
\usepackage{enumerate}
\usepackage{delarray}
\usepackage{mathtools}
\usepackage{fullpage}
\usepackage{faktor} % For quotients
\usepackage{mathrsfs}

\usepackage[italicdiff]{physics} % For differentials
\usepackage{bbm} % For indicator

% \usepackage{quiver}
\usepackage[linguistics]{forest}




%--------Page Setup--------

\pagestyle{empty}%

\setlength{\hoffset}{-1.54cm}
\setlength{\voffset}{-1.54cm}

\setlength{\topmargin}{0pt}
\setlength{\headsep}{0pt}
\setlength{\headheight}{0pt}

\setlength{\oddsidemargin}{0pt}

\setlength{\textwidth}{195mm}
\setlength{\textheight}{250mm}


%--------Macros--------

\newcommand{\sub}{\subseteq}
\newcommand{\lcm}{\text{lcm}}
\newcommand{\mc}[1]{\mathcal{#1}}
\newcommand{\mf}[1]{\mathfrak{#1}}
\newcommand{\ms}[1]{\mathscr{#1}}
\newcommand{\sO}{\mathcal{O}}
\newcommand{\cyclic}[1]{\langle#1\rangle}
\newcommand{\units}[1]{#1 ^{\times}}
\newcommand{\la}{\langle}
\newcommand{\ra}{\rangle}
\newcommand{\lr}[1]{\left(#1\right)}
\newcommand{\lrvert}[1]{\left\lvert#1\right\rvert}

\DeclarePairedDelimiterX{\inp}[2]{\langle}{\rangle}{#1, #2}

%----Switch phi and varphi
% \let\temp\phi
% \let\phi\varphi
% \let\varphi\temp

\newcommand{\C}{\mathbb{C}}
\newcommand{\F}{\mathbb{F}}
\newcommand{\E}{\mathbb{E}}
\newcommand{\N}{\mathbb{N}\xspace}
\newcommand{\I}{\mathbb{I}\xspace}
\newcommand{\R}{\mathbb{R}\xspace}
\newcommand{\Z}{\mathbb{Z}\xspace}
\newcommand{\Q}{\mathbb{Q}\xspace}
\newcommand{\G}{\mathbb{G}\xspace}

\DeclareMathOperator{\Spec}{Spec}
\DeclareMathOperator{\res}{res}
% \DeclareMathOperator{\Tr}{Tr}
\DeclareMathOperator{\ord}{ord}
\DeclareMathOperator{\Sym}{Sym}
% \DeclareMathOperator{\dv}{div}
\DeclareMathOperator{\alb}{alb}
\DeclareMathOperator{\img}{Im}
\DeclareMathOperator{\et}{et}
\DeclareMathOperator{\ck}{coker}
\DeclareMathOperator{\Reg}{Reg}
\DeclareMathOperator{\Cor}{Cor}
\DeclareMathOperator{\Ac}{at}
\DeclareMathOperator{\supp}{supp}
\DeclareMathOperator{\Hom}{Hom}
\DeclareMathOperator{\Pic}{Pic}
\DeclareMathOperator{\Gal}{Gal}
\DeclareMathOperator{\fc}{frac}
\DeclareMathOperator{\Ann}{Ann}
\DeclareMathOperator{\Mod}{Mod}
\DeclareMathOperator{\Cone}{Cone}
\DeclareMathOperator{\FI}{FI}
\DeclareMathOperator{\End}{End}
\DeclareMathOperator{\Alb}{Alb}
\DeclareMathOperator{\Ext}{Ext}
\DeclareMathOperator{\ab}{ab}
\DeclareMathOperator{\Jac}{Jac}
\DeclareMathOperator{\coker}{coker}
\DeclareMathOperator{\fr}{frac}
\DeclareMathOperator{\Int}{Int}
\let\Span\relax
\DeclareMathOperator{\Span}{Span}
\DeclareMathOperator{\Ran}{Ran}



%----Analysis
\newcommand{\summ}{\sum\limits}
% \newcommand{\norm}[1]{\left\lVert#1\right\rVert}
\newcommand{\thicc}{\bigg}
\newcommand{\eps}{\varepsilon}
\newcommand*\cls[1]{\overline{#1}}
\newcommand{\ind}{\mathbbm{1}}
\DeclareMathOperator{\sgn}{sgn}


%--------Theorem environments--------
\newtheorem{definition}{Definition}[]
\newtheorem{lemma}{Lemma}[]
\newtheorem{corollary}{Corollary}[]
\newtheorem{theorem}{Theorem}[]
\theoremstyle{remark}
\newtheorem*{claim}{Claim}


\newenvironment{solution}
{\begin{proof}[Solution]}
{\end{proof}}


\makeatletter
\newcommand{\thickhline}{%
    \noalign {\ifnum 0=`}\fi \hrule height 1pt
    \futurelet \reserved@a \@xhline
}
\newcolumntype{"}{@{\hskip\tabcolsep\vrule width 1pt\hskip\tabcolsep}}
\makeatother

% --------Problem environment--------
\setlength\parindent{0pt}
\setcounter{secnumdepth}{0}
\newcounter{partCounter}
\newcounter{homeworkProblemCounter}
\setcounter{homeworkProblemCounter}{1}


\newenvironment{homeworkProblem}[1][-1]{
    \ifnum#1>0
        \setcounter{homeworkProblemCounter}{#1}
    \fi
    \section{Problem \arabic{homeworkProblemCounter}}
    \setcounter{partCounter}{1}
    \stepcounter{homeworkProblemCounter}
}


%--------Metadata--------
\title{MATH 7310 Homework 9}
\author{James Harbour}

\begin{document}
\maketitle


\begin{homeworkProblem}
  Let $\nu$ be a complex measure on $(X,\Sigma)$. If $\nu(X) = |\nu|(X)$, prove that $\nu = |\nu|$.

  \begin{proof}
    As $\nu\ll |\nu|$ and $|\nu|$ is a finite measure (so \emph{a fortiori} a $\sigma$-finite measure), $\dd{\nu} = f\dd{|\nu|}$ where $f = \frac{\dd{\nu}}{\dd{|\nu|}}$. Moreover, by Proposition 3.13(b), $|f| = 1$ $|\nu|$-almost everywhere. As $|\Re(f)|\leq 1$ a.e., it follows that $1-\Re(f)\geq$ a.e. Then, we observe by finiteness of $\nu(X)$ that
    \begin{align*}
      0 = |\nu|(X)-\nu(X) = \int 1-f\dd{|\nu|} = \int 1-\Re(f)\dd{|\nu|} - i \int \Im(f)\dd{|\nu|}.
    \end{align*}
    Thus, the real and imaginary parts of the right side of the above equation must both be zero, so the almost everywhere positivity of $1-\Re(f)$ implies that $\Re(f) = 1$ $|\nu|$-almost everywhere. After modifying out by the measure zero sets upon which $|f|$ and $\Re(f)$ are not both equal to one, we obtain that $\Re(f) = |f|$ $|\nu|$-almost everywhere, whence $f = \Re(f)=1$ and $\Im(f) = 0$. Thus, for all $E\in\Sigma$, \[\nu(E) = \int_E f\dd{|\nu|} = \int_E \Re(f)\dd{|\nu|} + i\int_E \Im(F)\dd{|\nu|} = \int_E 1\dd{|\nu|} = |\nu|(E).\]
  \end{proof}
\end{homeworkProblem}


\begin{homeworkProblem}
  Let $\nu$ be a complex measure on $(X,\Sigma)$. If $E\in\Sigma$, define
  \begin{align*}
    \mu_1(E) &= \sup\left\{\sum_1^n |\nu(E_j)|:n\in\N,\,E_1,\ldots,E_n \text{ disjoint},\,E=\bigsqcup_1^n E_j\right\}, \\
    \mu_2(E) &= \sup\left\{\sum_1^\infty |\nu(E_j)|:E_1,E_2,\ldots \text{ disjoint},\,E=\bigsqcup_1^\infty E_j\right\} \\
    \mu_3(E) &= \sup\left\{\left\lvert\int_E f\dd{\nu}\right\rvert: |f|\leq 1\right\}.
  \end{align*}
  Prove that $\mu_1 = \mu_2 = \mu_3$. (\emph{Hint}: First show that $\mu_1\leq\mu_2\leq\mu_3$. To see that $\mu_3 = |\nu|$, let $f = \cls{\dd{\nu}/\dd{|\nu|}}$ and apply Proposition 3.13. To see that $\mu_3\leq \mu_1$, approximate $f$ by simple functions).

  \begin{proof}
    That $\mu_1\leq\mu_2$ is clear as we may take the sequence $E_1,\ldots,E_n,\emptyset,\ldots$ in the set for $\mu_2$ to recover the values in the set for $\mu_1$. \\

    Recall from page 46 of Folland that for any function $f:X\to \C$ we have its polar decomposition $f = \sgn(f)|f|$ where $\sgn(z) = z/|z|$ if $z\neq 0$ and $\sgn(0) = 0$. Moreover, if $f$ is measurable with respect to some positive measure, then so are $\sgn(f)$ and $|f|$. From the polar decomposition of $f$, it follows that $\cls{\sgn(f)}f = |f|$. Using this idea, suppose that$ E_1,E_2,\ldots $ are disjoint with $E = \bigsqcup_{j=1}^\infty E_j$. Then we compute
    \[
      \sum_{j=1}^{\infty}|\nu(E_j)| = \sum_{j=1}^{\infty} \cls{\sgn{\nu(E_j)}}\nu(E_j)
    \]
    Hence, we are led to define $f = \sum_{j=1}^{\infty}\cls{\sgn(\nu(E_j))}\ind_{E_j}$. This function is measurable as it is a pointwise limit of simple functions. Moreover, as the sets $E_j$ are pairwise disjoint and $|\sgn(z)|\leq 1$ for all $z$, it follows that $|f|\leq 1$. Noting that $|\Re(f)|, |\Im(f)|\leq |f|\leq 1\in L^1(|\nu|)$ (as $|\nu(X)|<+\infty$), we may apply the dominated convergence theorem to the positive and negative parts of the partial sums for both $\Re(f)$ and $\Im(f)$ with respect to the positive and negative parts of the signed measures $\Re(\nu)$ and $\Im(\nu)$ to obtain that
    \[
      \left\lvert\int_E f\dd{\nu}\right\rvert = \left\lvert\sum_{j=1}^{\infty}\int \cls{\sgn(\nu(E_j))}\ind_{E_j}\dd{\nu}\right\rvert = \left\lvert\sum_{j=1}^{\infty} \cls{\sgn(\nu(E_j))}\nu(E_j)\right\rvert = \sum_{j=1}^{\infty}|\nu(E_j)|,
    \]
    so $\mu_2\leq\mu_3$.\\

    Now fix $E\in \Sigma$. One one hand, suppose that $|g|\leq 1$. Then by Proposition 3.13(c),
    \[
      \lrvert{\int_E g\dd{\nu}} \leq \int_E |g| \dd{|\nu|} \leq \int_E 1\dd{|\nu|} = |\nu|(E),
    \]
    whence $\mu_3(E)\leq |\nu|(E)$. On the other hand, let $f = \cls{\dd{\nu}/\dd{|\nu|}}$. By Proposition 3.13(b), $|\dd{\nu}/\dd{|\nu|}| = 1$ $|\nu|$-a.e. whence $\dd{\nu}/\dd{|\nu|} = 1/f$ $|\nu|$-a.e. Thus, we compute
    \[
      \lrvert{\int_E f\dd{\nu}} = \lrvert{\int_E f \frac{\dd{\nu}}{\dd{|\nu|}}\dd{|\nu|}} = |\nu|(E),
    \]
    whence $|\nu|(E)\leq \mu_3(E)$. So we have show that $\mu_3 = |\nu|$. \\

    It remains to show that $\mu_3\leq \mu_1$. Fix $E\in\Sigma$ and suppose $f$ is measurable with $|f|\leq 1$. Chose simple functions $(\phi_k)_{k=1}^{\infty}$ such that $0\leq|\phi_1|\leq |\phi_2|\leq\cdots\leq|f|$ and $\phi_k\to f$ pointwise. Write $\phi_k = \sum_{j=1}^{n_k}c_{j}^{(k)}\ind_{E_j^{(k)}}$ where, for all $k\in\N$, $E_1^{(k)},\ldots,E_{n_k}^{(k)}$ are pairwise disjoint such that $X = \bigsqcup_{j=1}^{n_k}E_{j}^{(k)}$ and $c_{j}^{(k)}\in\C$ with $|c_{j}^{(k)}|\leq 1$. \\

    As before, we may apply the dominated convergence theorem to the positive and negative parts for the sequences $\Re(\phi_k\ind_E)$ and $\Im(\phi_k\ind_E)$ with respect to the positive and negative parts of the signed measures $\Re(\nu)$ and $\Im(\nu)$ to obtain that
    \begin{align*}
      \lrvert{\int_E f \dd{\nu}} &= \lrvert{\lim_{k\to\infty}\int \phi_k\ind_E\dd{\nu}} = \lrvert{\lim_{k\to\infty}\int \sum_{j=1}^{n_k}c_{j}^{(k)}\ind_{E_j^{(k)}\cap E}\dd{\nu}} \\
      &= \lim_{k\to\infty}\lrvert{\sum_{j=1}^{n_k}c_{j}^{(k)}\nu(E_j^{(k)}\cap E)} \leq \lim_{k\to\infty}\sum_{j=1}^{n_k}|\nu(E_j^{(k)}\cap E)| \leq \mu_1(E).
    \end{align*}
    As $|f|\leq 1$ was arbitrary, it follows that $\mu_3(E)\leq \mu_1(E)$ as desired.
  \end{proof}
\end{homeworkProblem}


\begin{homeworkProblem}[3]
  \textbf{(a)}: Let $(X,\Sigma)$ be a measurable space. Let $M(\Sigma)$ be the vector space of complex measures on $\Sigma$ with the total variation norm $\norm{\mu} = |\mu|(X)$. Show that $M(\Sigma)$ is a Banach space. \\
  \hspace{10pt}Suggestion: it may be helpful to use that for $\mu\in M(\Sigma)$ we have
  \[
    \sum_{n=1}^{\infty}|\mu(E_n)|\leq\norm{\mu}
  \]
  where $(E_n)_{n=1}^
  \infty$ is a sequence of pairwise disjoint elements of $\Sigma$ (this is a consequence of a prior problem on this homework).

  \begin{proof}
    Let $(\mu_n)_{n=1}^{\infty}$ be a Cauchy sequence in $M(\Sigma)$. Define a new positive measure $\lambda$ on $\Sigma$ by
    \[
      \lambda(E) = \sum_{n=1}^{\infty}2^{-n}\frac{|\mu_n|(E)}{\norm{\mu_n}+1}.
    \]
    Then by nonnegativity, we have that for all $n\in\N$, $\mu_n\ll \lambda$ whence there exists some measurable $f_n$ such that $\dd{\mu_n} = f_n\dd{\lambda}$. Moreover, $f_n\in L^1(\lambda)$ as $\int |f_n|\dd{\lambda} = |\mu_n|(X) <+\infty$. Note that $\lambda$ is necessarily a finite measure, so we utilize part (b) of this exercise. Let $J:L^1(X,\lambda)\to M(\Sigma)$ be as in part (b). Then $\mu_n = J(f_n)$ for all $n\in \N$. As $J$ is an isometry, it follows that $(f_n)_{n=1}^{\infty}$ is Cauchy in $L^1(\lambda)$, whence by completeness there exists some $f\in L^1(\lambda)$ such that $\norm{f_n - f}_{L^1(\lambda)}\xrightarrow{n\to\infty}0$. Let $\mu = J(f)$. Then
    \[
      \norm{\mu_n-\mu} = \norm{f_n-f}_{L^1(\lambda)}\xrightarrow{n\to\infty}0,
    \]
    so $\mu_n\to\mu$ in total variation norm. Thus $M(\Sigma)$ is Banach.
  \end{proof}

  \textbf{(b)}: Fix a positive, $\sigma$-finite measure $\mu$ on $\Sigma$. Show that the map $J : L^1(X,\mu)\to M(\Sigma)$ given by $J(f) = f\dd{\mu}$ is a linear isometry with closed image.

  \begin{proof}
    Let  We wish to show that for $f,g\in L^1(X,\mu)$ and $\alpha\in\C$, $J(\alpha f+g) = \alpha J(f)+J(g)$, after which showing that $\norm{J(f)}_{M(\Sigma)} = \norm{f}_{L^1(\mu)}$ would imply that $J$ is a linear isometry. \\

    % TODO Linearity

    Let $f\in L^1(X,\mu)$. We compute
    \[
      \norm{J(f)}_{M(\Sigma)} = |J(f)|(X) = J(f)(X) = \int_{X}f\dd{\mu} = \norm{f}_{L^1(\mu)}.
    \]

    % TODO image is closed.

    Suppose that $(J(f_n))_{n=1}^{\infty}$ converges to $\nu$ in $M(\Sigma)$ where $(f_n)_{n=1}^{\infty}$ is in $L^1(X,\mu)$. So
    \[
      \norm{J(f_n)-\nu}_{M(\Sigma)}\xrightarrow{n\to \infty}0
    \]
    Suppose that $E\in\Sigma$ is null. Then as $J(f_n)\ll \mu$, $|J(f_n)|(E) = 0$ for all $n\in\N$. So by problem 2,
    \[
      |\nu|(E) \leq |J(f_n)(E-\nu(E))| + |J(f_n)(E)| \leq \norm{J(f_n)-\nu}_{M(\Sigma)}\xrightarrow{n\to \infty}0
    \]
    whence $|\nu|(E) = 0$ i.e. $E$ is null for $\nu$. Thus $\nu\ll \mu$. By the Lebesgue-Radon-Nikodym theorem, there exists some $f\in L^1(X,\mu)$ such that $\nu = f\dd{\mu} = J(f)$, so $\nu$ is in the image of $J$.
  \end{proof}

  \textbf{(c)}: Suppose that $\mu,\nu\in M(\Sigma)$, and let $\dd{\nu} = f\dd{\mu}+\dd{\lambda}$ with $\lambda\perp\mu$ be the Lebesgue-Radon-Nikodym decomposition. Show that
  \[
    \norm{\mu-\nu} = \norm{1-f}_{L^1(\mu)}+\norm{\lambda}.
  \]

  \begin{lemma}[Lemma 1]
    If $\nu_1,\nu_2\in M(\Sigma)$ and $\nu_1\perp\nu_2$, then $\norm{\nu_1+\nu_2} = \norm{\nu_1}+\norm{\nu_2}$.
  \end{lemma}

  \begin{proof}[Proof of Lemma 1]
    Let $\mu = |\nu_1|+|\nu_2|$, so $\mu$ is a positive finite measure such that $\nu_j\ll \mu$. By Radon-Nikodym theorem, there is some $f_j$ ($j=1,2$) such that $\dd{\nu_j} = f_j \dd{\mu}$. Then $\dd{|\nu_j|} = |f_j|\dd{\mu}$ and $\dd{|\nu_1+\nu_2|} = |f_1+f_2|\dd{\mu}$. Let $E,F\sub X$ such that $X = E\cup F$, $E\cap F = \emptyset$, $E$ is null for $\nu_2$, and $F$ is null for $\nu_1$. Then $0 = |\nu_1|(F) = \int_F |f_1|\dd{\mu}$ whence $f_1 = 0$ $\mu$-almost everywhere on $F$ and similarly we obtain that $f_2 = 0$ $\mu$-almost everywhere on $E$. On the other hand, by the finiteness of $|\nu_1|,|\nu_2|$, it follows that $|\nu_1|(X) = |\nu_1|(E)$ and $|\nu_2|(X) = |\nu_2|(F)$. We compute
    \begin{align*}
      \norm{\nu_1+\nu_2} &= \int_X |f_1+f_2|\dd{\mu} = \int_E |f_1+f_2|\dd{\mu} + \int_F |f_1+f_2|\dd{\mu}\\
      &=\int_E |f_1|\dd{\mu} + \int_F |f_2|\dd{\mu} = |\nu_1|(E) + |\nu_2|(F) = \norm{\nu_1}+\norm{\nu_2}.
    \end{align*}
  \end{proof}

  \begin{proof}
    For notational clarity, let $\nu = \delta_{ac}+\delta_{s}$ be the Lebesgue decomposition of $\nu$ with respect to $\mu$, so $\delta_{ac}\ll \mu$ and $\delta_s \perp\mu$. Let $\dd{\delta_{ac}} = f\dd{\mu}$. As $\mu\ll\mu$ and $\delta_{ac}\ll \mu$, we have that $\mu-\delta_{ac}\ll\mu$, whence noting that $-\delta_s\perp \mu$ we conclude that $\mu-\delta_{ac}\perp -\delta_s$. Lastly, observe that for $E\in \Sigma$,
    \[
      \mu(E)-\delta_{ac}(E) = \int_E 1\dd{\mu}-\int_E f\dd{\mu} = \int_E 1-f\dd{\mu},
    \]
    whence $\frac{\dd{(\mu-\delta_{ac})}}{\dd{\mu}} = 1-f$. Hence, in the notation of part (b), $\mu-\delta_{ac} = J(1-f)$. So, noting that $J$ is an isometry and applying Lemma 1 to $\mu-\delta_{ac}\perp -\delta_s$, we find that
    \[
      \norm{\mu-\nu} = \norm{\mu-\delta-{ac}-\delta_s} = \norm{\mu-\delta_{ac}} + \norm{\delta_s} = \norm{1-f}_{L^1(\mu)}+\norm{\delta_s}.
    \]
  \end{proof}
\end{homeworkProblem}

\begin{homeworkProblem}
  If $E$ is a Borel set in $\R^n$, the density $D_E(x)$ of $E$ at $x$ is defined as
  \[
    D_E(x) = \lim_{r\to0}\frac{m(E\cap B_r(x))}{m(B_r(x))},
  \]
  whenever the limit exists.\\

  \textbf{(a)}: Show that $D_E(x) = 1$ for a.e. $x\in E$ and $D_E(x) = 0$ for a.e. $x\in E^c$.

  \begin{proof}
    Define a new Borel measure $\nu$ by $\nu(F) = \mu(E\cap F)$ for Borel $F$. Then $\nu\ll m$ and observe that, for Borel sets $F$,
    \[
      \nu(F) = \int \ind_{E\cap F}\dd{m} = \int \ind_E \ind_F\dd{m} = \int_F \ind_E\dd{m},
    \]
    so $\frac{\dd{\nu}}{\dd{m}} = \ind_E$ $m$-a.e. by uniqueness in Lebesgue-Radon-Nikodym theorem. Moreover, as $\nu(F)\leq m(F)$ for all Borel $F$, it follows that $\nu$ is finite on compacts and thus regular, so by the Lebesgue differentiation theorem, the following limit exists for $m$-a.e. $x$ and is equal to
    \[
      D_E(x) = \lim_{r\to 0}\frac{\nu(B_r(x))}{m(B_r(x))} = \ind_E(x),
    \]
    whence the claim follows.
  \end{proof}

  \textbf{(b)}: Find examples of $E$ and $x$ such that $D_E(x)$ is a given number $\alpha\in (0,1)$, or such that $D_E(x)$ does not exist.

  \begin{proof}[Solution]
    For an example where $E$ and $x$ are such that $D_E(x)$ is any given number $\alpha\in (0,1)$, consider $X= \R^2$, $x = (0,0)$, and $E$ a sector of the unit disk centered at $(0,0)$ making an $\alpha\cdot 2\pi$ radians angle with $(0,0)$ and the $x$-axis. By definition of angles, $D_E((0,0))$ is clearly $\alpha$.\\
  \end{proof}
\end{homeworkProblem}


\begin{homeworkProblem}[5]
  Let $\psi:\R\to\R$ be given as $\psi = \ind_{[0,1/2)}-\ind_{[1/2,1]}$. For $n,k\in\Z$ define $h_{n,k}(t) = 2^{n/2}\psi(2^n t-k)$. Show that $\mc{E} = \{ 1\}\cup\{h_{n,k}:n\in \N\cup\{ 0\},\, 0\leq k<2^n \}$ is an orthonormal basis for $L^2([0,1])$.

  \begin{proof}
    Let $A_{nk}=[k2^{-n},(k+1/2)2^{-n})$ and $B_{nk}=[(k+1/2)2^{-n},(k+1)2^{-n}]$.
    Note that
    \[
      h_{n,k} = 2^{n/2}\lr{\ind_{[k2^{-n},(k+1/2)2^{-n})}-\ind_{[(k+1/2)2^{-n},(k+1)2^{-n}]}} = 2^{n/2}\lr{\ind_{A_{nk}}-\ind_{B_{nk}}}.
    \]
    Then we compute
    \begin{align*}
      \inp{h_{n,k}}{h_{m,l}} &= 2^{(n+m)/2}\int\lr{\ind_{A_{nk}}-\ind_{B_{nk}}}\lr{\ind_{A_{ml}}-\ind_{B_{ml}}}\dd{t}\\
      &=2^{(n+m)/2}\lr{m(A_{nk}\cap A_{ml}) + m(B_{nk}\cap B_{ml}) - m(A_{nk}\cap B_{ml}) - m(A_{ml}\cap B_{nk})}.
    \end{align*}
  \end{proof}
\end{homeworkProblem}

\begin{homeworkProblem}
  Fix $n\in\N$, and $1\leq p< +\infty$. For $y\in\R^n$, define $\tau_y:L^p(\R^n)\to L^p(\R^n)$ by $\tau_y(f)(x) = f(x-y)$. Show that if $f\in L^p(\R^n)$, then
  \begin{equation}
    \norm{\tau_y f}_p = \norm{f}_p.
  \end{equation}
  \begin{equation}
    \lim_{y\to0}\norm{\tau_y f - f}_p = 0.
  \end{equation}
  Hint: use (1) to show that the set of $f'$s for which (2) is true is a closed, linear subspace of $L^p(\R^n)$. Then check (2) on a dense set of $f'$s where (2) is easier to see.

  \begin{proof}
    Part (1) follows trivially from the change of variables $x-y\rightsquigarrow x$. Let $S\sub L^p(\R^n)$ be the set of $f$'s in $L^p(\R^n)$ for which (2) is true. Suppose $(f_n)_{n=1}^{\infty}$ is a sequence in $S$ and $f\in L^p(\R^n)$ such that $\norm{f_n-f}_p\xrightarrow{n\to\infty}0$. Observe that
    \begin{align*}
      \norm{\tau_y f - f}_p \leq \norm{\tau_y f - f_n}_p + \norm{f_n - f}_p &\leq \norm{\tau_y (f - f_n)}_p + \norm{\tau_y f_n - f_n}_p + \norm{f_n - f}_p \\&= 2\norm{f_n-f}_p + \norm{\tau_y f_n - f_n},
    \end{align*}
    which implies that $f\in S$. \\

    Property (2) is clearly invariant under scaling by a real number, so to show $S$ is a linear subspace of $L^p(\R^n)$ it suffices to show that it is closed under sums. Suppose that $f,g\in S$. Then
    \[
      \norm{\tau_y(f+g)-(f+g)}_p \leq \norm{\tau_y f -f}_p + \norm{\tau_y g -g}_p
    \]
    whence $f+g\in S$.\\

    We show that (2) holds for $C_c(\R^n)$. Fix $f\in C_c(\R^n)$ and let $K$ be the closure of the support of $f$. By continuity of $f$, $\tau_y f\to f$ pointwise. Without loss of generality, we restrict our limit to over $y<1$. Then, $|\tau_y f| \leq \sup_{x\in\R^n}|f(x)| \ind_{K+B_1(0)}\in L^p(\R^n)$, whence by the dominated convergence theorem
    \[
      \lim_{y\to 0}\int|f(x-y)-f(x)|^p\dd{x} = \int\lim_{y\to0}|f(x-y)-f(x)|^p\dd{x} = 0.
    \]
    Thus, (2) holds for a dense subspace of $L^p(\R^n)$, so it holds for all of $L^p(\R^n)$.
  \end{proof}
\end{homeworkProblem}

\end{document}

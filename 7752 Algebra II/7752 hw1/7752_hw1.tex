\documentclass[12pt,letterpaper]{article}

%--------Packages--------
\usepackage{amsmath, amsthm, amssymb}
\usepackage{xspace}
\usepackage{graphicx}
\usepackage{hhline}
\usepackage{amssymb}
\usepackage{array}
\usepackage{braket}
\usepackage{multicol}
\usepackage{mathtools}
\usepackage{enumerate}
\usepackage{delarray}
\usepackage{mathtools}
\usepackage{fullpage}
\usepackage{faktor} % For quotients


% \usepackage{quiver}
\usepackage[linguistics]{forest}




%--------Page Setup--------

\pagestyle{empty}%

\setlength{\hoffset}{-1.54cm}
\setlength{\voffset}{-1.54cm}

\setlength{\topmargin}{0pt}
\setlength{\headsep}{0pt}
\setlength{\headheight}{0pt}

\setlength{\oddsidemargin}{0pt}

\setlength{\textwidth}{195mm}
\setlength{\textheight}{250mm}


%--------Macros--------

\newcommand{\sub}{\subseteq}
\newcommand{\lcm}{\text{lcm}}
\newcommand{\mc}[1]{\mathcal{#1}}
\newcommand{\mf}[1]{\mathfrak{#1}}
\newcommand{\sO}{\mathcal{O}}
\newcommand{\cyclic}[1]{\langle#1\rangle}
\newcommand{\units}[1]{#1 ^{\times}}
\newcommand{\la}{\langle}
\newcommand{\ra}{\rangle}
%----Switch phi and varphi
\let\temp\phi
\let\phi\varphi
\let\varphi\temp

\newcommand{\C}{\mathbb{C}}
\newcommand{\F}{\mathbb{F}}
\newcommand{\N}{\mathbb{N}\xspace}
\newcommand{\I}{\mathbb{I}\xspace}
\newcommand{\R}{\mathbb{R}\xspace}
\newcommand{\Z}{\mathbb{Z}\xspace}
\newcommand{\Q}{\mathbb{Q}\xspace}
\newcommand{\G}{\mathbb{G}\xspace}
\DeclareMathOperator{\Spec}{Spec}
\DeclareMathOperator{\res}{res}
\DeclareMathOperator{\Tr}{Tr}
\DeclareMathOperator{\ord}{ord}
\DeclareMathOperator{\Sym}{Sym}
\DeclareMathOperator{\dv}{div}
\DeclareMathOperator{\alb}{alb}
\DeclareMathOperator{\img}{Im}
\DeclareMathOperator{\et}{et}
\DeclareMathOperator{\ck}{coker}
\DeclareMathOperator{\Reg}{Reg}
\DeclareMathOperator{\Cor}{Cor}
\DeclareMathOperator{\Ac}{at}
\DeclareMathOperator{\supp}{supp}
\DeclareMathOperator{\Hom}{Hom}
\DeclareMathOperator{\Pic}{Pic}
\DeclareMathOperator{\Gal}{Gal}
\DeclareMathOperator{\fc}{frac}
\DeclareMathOperator{\Ann}{Ann}
\DeclareMathOperator{\Mod}{Mod}
\DeclareMathOperator{\Cone}{Cone}
\DeclareMathOperator{\FI}{FI}
\DeclareMathOperator{\End}{End}
\DeclareMathOperator{\Alb}{Alb}
\DeclareMathOperator{\Ext}{Ext}
\DeclareMathOperator{\ab}{ab}
\DeclareMathOperator{\Jac}{Jac}
\DeclareMathOperator{\coker}{coker}
\DeclareMathOperator{\fr}{frac}


%----Analysis
\newcommand{\dd}[2][]{\frac{\partial^{#1}}{\partial {#2}^{#1}}}
\newcommand{\summ}{\sum\limits}
\newcommand{\norm}[1]{\left \vert \left \vert #1 \right \vert \right \vert}
\newcommand{\thicc}{\bigg}
\newcommand{\eps}{\varepsilon}


%--------Theorem environments--------
\newtheorem{definition}{Definition}[]
\newtheorem{lemma}{Lemma}[]
\newtheorem{corollary}{Corollary}[]
\newtheorem{theorem}{Theorem}[]
\theoremstyle{remark}
\newtheorem*{claim}{Claim}


\newenvironment{solution}
{\begin{proof}[Solution]}
{\end{proof}}


\makeatletter
\newcommand{\thickhline}{%
    \noalign {\ifnum 0=`}\fi \hrule height 1pt
    \futurelet \reserved@a \@xhline
}
\newcolumntype{"}{@{\hskip\tabcolsep\vrule width 1pt\hskip\tabcolsep}}
\makeatother

% --------Problem environment--------
\setlength\parindent{0pt}
\setcounter{secnumdepth}{0}
\newcounter{partCounter}
\newcounter{homeworkProblemCounter}
\setcounter{homeworkProblemCounter}{1}


\newenvironment{homeworkProblem}[1][-1]{
    \ifnum#1>0
        \setcounter{homeworkProblemCounter}{#1}
    \fi
    \section{Problem \arabic{homeworkProblemCounter}}
    \setcounter{partCounter}{1}
    \stepcounter{homeworkProblemCounter}
}


%--------Metadata--------
\title{MATH 7752 Homework 1}
\author{James Harbour}


\begin{document}
\maketitle

\begin{homeworkProblem}
  Let $R$ be a ring and $M$ an $R$-module. \\

  \textbf{(a)} Prove that for every $m\in M$, the map $r\mapsto rm$ from $R$ to $M$ is a homomorphism of $R$-modules. \\

  \begin{proof}
    Fix $m\in M$ and let $\phi$ denote the map $r\mapsto rm$. Fix $x,y\in R$ and $r\in R$. Observe that
    \[
      \phi(x+y) = (x+y)m = xm + ym = \phi(x) + \phi(y)
    \]
    and
    \[
      \phi(rx) = (rx)m = r(xm) = r\phi(x),
    \]
    so $\phi$ is an $R$-module homomorphism.
  \end{proof}

  \textbf{(b)} Assume that $R$ is commutative and $M$ an $R$-module. Prove that there is an isomorphism  $\Hom_R(R,M)\simeq M$ as $R$-modules.

  \begin{proof}
    For $m\in M$, let $\phi_m$ denote the $R$-module homomorphism in part (a). \\
    Consider the map $\psi: M \to \Hom_R(R,M)$ given by $\psi(m) = \phi_m$. For $m,n \in M$ and $r,x\in R$,
    \[
      \psi(m+n)(x) = \phi_{m+n}(x) = x(m+n) = xm+xn = \phi_m(x) +\phi_n(x) = (\psi(m)+\psi(n))(x)
    \]
    so $\psi(m+n) = \psi(m)+\psi(n)$, and
    \[
      \psi(rm)(x) = \phi_{rm}(x) =x(rm) = r(xm) = r\phi_{m}(x)=(r\psi(m))(x)
    \]
    so $\psi(rm) = r\psi(m)$. \\

    Suppose $\psi(m) = \psi(n)$. Then $m = \phi_m(1) = \psi(m)(1) = \psi(n)(1) = \phi_n(1) = n$, so $\psi$ is injective. \\

    Suppose $\phi \in \Hom_R(R,M)$. For $r \in R$,
    \[
      \psi_{\phi(1)}(r) = r\phi(1) = \phi(r),
    \]
    so $\psi_{\phi(1)} = \phi$, i.e. $\psi$ is surjective.
  \end{proof}
\end{homeworkProblem}

\begin{homeworkProblem}
  Give an explicit example of a map $f:A\to B$ with the following properties:
  \begin{itemize}
    \item $A,B$ are $R$-modules.
    \item $f$ is a group homomorphism.
    \item $f$ is not an $R$-module homomorphism.
  \end{itemize}

  \begin{solution}
    Consider $A=B=\C$ viewed as $\C$-modules over themselves. Let $f:A\to B$ be complex conjugation. For $z,w\in A$, $f(z+w) = \overline{z+w} = \overline{z}+\overline{w} = f(z)+f(w)$, so $f$ is a group homomorphism. However, for $z\in A\setminus \{ 0\}$, $f(i z)= -i \overline{z} \neq i \overline{z} = if(z)$, so $f$ is not an $R$-module homomorphism.
  \end{solution}

\end{homeworkProblem}

\begin{homeworkProblem}
  Let $R$ be a ring and $M$ an $R$-module. \\

  \textbf{(a)} Let $N$ be a subset of $M$. The \textit{annihilator} of $N$ is defined to be the set \[\Ann_R(N):=\{r\in R: rn=0, \text{ for all }n\in N\}.\] Prove that $\Ann_R(N)$ is a left ideal of $R$.

  \begin{proof}
    Let $x,y\in I$ and $r\in R$. Fix $n\in N$. Noting that $xn = 0 = yn$, it follows that
    \[
      (x+ry)n = xn + (ry)n = xn + r(yn) = 0.
    \]
    Thus $x+ry\in \Ann_R(N)$. Since all elements chosen were arbitrary, $\Ann_R(N)$ is a left ideal of $R$.
  \end{proof}

  \textbf{(b)} Show that if $N$ is an $R$-submodule of $M$, then $\Ann_R(N)$ is an ideal of $R$ (i.e. it is two-sided ideal). \\

  \begin{proof}
    By part (a), it suffices to show that $\Ann_R(N)$ is a right ideal of $R$. Moreover, part (a) shows \textit{a fortiori} that $\Ann_R(N)$ is already an abelian group, so we need only address its multiplicative structure. Let $y\in \Ann_R(N)$ and $r\in R$. Fix $n\in N$. As $N$ is an $R$-submodule of $M$, $yn\in N$, whence $(yr)n = y(rn) = 0$ by definition. Hence $\Ann_R(N)$ is a two-sided ideal of $R$.
  \end{proof}

  \textbf{(c)} For a subset $I$ of $R$ the \textit{annihilator} of $I$ in $M$ is defined to be the set,
  \[
    \Ann_M(I):=\{m\in M:xm=0, \text{ for all }x\in I\}.
  \]
  Find a natural condition on $I$ that guarantees that $\Ann_M(I)$ is a submodule of $M$. \\

  \begin{claim}
     $\Ann_M(I)$ is an $R$-submodule of $M$ if $I$ is a right ideal of $R$.
  \end{claim}

  \begin{proof}
    Suppose $I$ is a right ideal of $R$. As $x\cdot 0 = 0$ for all $x\in I$, $\Ann_M(I) \neq \emptyset$. Suppose $m,n \in \Ann_M(I)$ and $r\in R$. Fix $x\in I$. By definition $x\cdot m = 0$. As $I$ is a right ideal, $xr \in I$, so $x\cdot (m + r\cdot n) = x\cdot m + (xr)\cdot n = 0$. Thus $\Ann_M(I)$ is an $R$-submodule of $M$.
  \end{proof}


  \textbf{(d)} Let $R$ be an integral domain. Prove that every finitely generated torsion $R$-module has a nonzero annihilator.

  \begin{proof}
    Let $M$ be a finitely generated torsion $R$-module. Taking a generating set $m_1,\ldots, m_n\in M$ of $M$, for each $k\in \{ 1,\ldots, n\}$ there exists an $x_k\in \units{R} = R\setminus \{ 0\}$ such that $x_k m_k = 0$. As $\units{R}$ is closed under multiplication, $r := x_1 \cdots x_n \in \units{R}$ whence $r\neq 0$.

    Now suppose that $m\in M$. Then there exist $r_1, \ldots, r_n \in R$ such that $m = r_1 m_1 + \cdots + r_n m_n$. Observe that, by the commutativity of $R$,
    \[
      rm = (x_1 \cdots x_n)(r_1 m_1 + \cdots + r_n m_n) = \sum_{k=1}^{n} \left(\prod_{i\neq k}x_i\right) (x_k m_k) = 0.
    \]
    Thus $0\neq r\in \Ann_R(M)$, so $M$ has nonzero annihilator.
  \end{proof}
\end{homeworkProblem}

\begin{homeworkProblem}
  In class we obtained a simple characterization of $R$-modules when $R=\Z$, and $R=F[x]$, with $F$ a field. Imitate the method to find similar characterizations for $R$-modules in the following cases: \\

  \textbf{(a)} $R=\Z/n\Z$, for some $n\geq 2$.
  \begin{claim}
    $\{\Z/n\Z \text{-modules}\}\longleftrightarrow\{n\text{-torsion abelian groups}\}$
  \end{claim}

  \begin{proof}\ \\
    \underline{$\implies$}: Let $A$ be a $\Z/n\Z$-module. We write $\Z/n\Z = \Set{[0], [1], \ldots, [n-1]}$. Define a $\Z$-module structure on $A$ by letting $m\cdot a := [m]\cdot a$ for $m\in \Z$. This gives $A$ the structure of an abelian group. To see that $A$ is $n$-torsion, observe that, for any $a\in A$,
    \[
      n\cdot a = [n]\cdot a = [0]\cdot a = 0.
    \]

    \underline{$\impliedby$}: Let $A$ be an $n$-torsion abelian group. Then $A$ has a natural $\Z$-module structure given by repeated addition. Define a $\Z/n\Z$-module structure on $A$ by letting $[m]\cdot a = m\cdot a$. To see that this definition is well-defined, suppose that $[m] = [m']$. Then $n$ divides $m-m'$, whence there exists a $k\in \Z$ such that $m-m' = kn$. Now, $(m-m')\cdot a = (kn)\cdot a = k\cdot(n\cdot a) = 0$ by $n$-torsion, so $m\cdot a = m'\cdot a$. That this action gives a $\Z/n\Z$-structure follows from the fact that it descends from a $\Z$-module structure.
  \end{proof}

  \textbf{(b)} $R=\Z[x]$.
  \begin{claim}
    \[
      \Set{ \Z[x] \text{-modules}}\longleftrightarrow\Set{(A, T)\ | \begin{array}{c}
        A \text{ an abelian group} \\
        T:A\to A \text{ an abelian group endomorphism}
        \end{array}
      }
    \]
  \end{claim}

  \begin{proof}\ \\
    \underline{$\implies$}: Let $A$ be a $\Z[x]$-module. Via restriction of scalars under the natural inclusion $\Z \hookrightarrow \Z[x]$, $A$ has an abelian group structure. Define a map $T:A\to A$ be $T(a) = x\cdot a$. By distributivity, $T$ is an abelian group endomorphism. Moreover, given any $p(x)\in \Z[x]$, we have by linearity that $p(x)\cdot a = p(T)a$. \\

    \underline{$\impliedby$}: Let $A$ be an abelian group and $T:A\to A$ an abelian group endomorphism. Define a $\Z[x]$-module structure on $A$ by declaring $p(x)\cdot a = p(T)a$ for all $p(x)\in \Z[x]$. That this action gives a $\Z[x]$-module structure is clear.
  \end{proof}

  \textbf{(c)} $R=F[x,y]$.
  \begin{claim}
    \[
      \Set{\F[x,y] \text{-modules}} \longleftrightarrow \Set{(V, T, S)\ |
        \begin{array}{c}
          V \text{ an $F$-vector space} \\
          T,S:V\to V \text{ are $F$-linear maps such that } TS = ST
        \end{array}
      }
    \]
  \end{claim}

  \begin{proof}\ \\
    \underline{$\implies$}: Let $V$ be an $F[x,y]$-module. Via restriction of scalars under the natural inclusion $F\hookrightarrow F[x,y]$, $V$ has the structure of an $F$-vector space. Define maps $T,S:V\to V$ by $T(v) := x\cdot v$ and $S(v) := y\cdot v$ for all $v\in V$. For $\lambda \in F$ and $v,w\in A$, observe that \[T(\lambda v+w) = x\cdot (\lambda v+w) = \lambda\cdot( x\cdot v) + x\cdot w = \lambda \cdot T(v) + T(w)\] and same for $S$ (mutatis mutandis), so $T$ and $S$ are $F$-linear endomorphisms of $V$. Moreover, for $v\in V$,
    \[
      (ST) (v) = S(x\cdot v) = y\cdot (x\cdot v) = (yx)\cdot v = (xy) \cdot v = x\cdot(y\cdot v) = (TS)(v),
    \]
    so $ST = TS$. \\

    \underline{$\impliedby$}: Let $V$ be an $F$-vector space and $T,S:V\to V$ be commuting $F$-linear endomorpisms of $V$. Define an $F[x,y]$-module structure on $V$ by setting $\lambda \cdot v :=\lambda v$ for $\lambda\in F$, $x \cdot v := T(v)$, $y\cdot v := S(v)$, and extending to $F(x,y)$ by linearity and distributivity, i.e. $p(x,y)\cdot v := p(T,S)v$. That the action of $p(T,S)$ on $V$ is well-defined follows from the assumption that $S$ and $T$ commute. Moreover, this ensures that this action is respects multiplication within $F[x,y]$, whence we receive an $F[x,y]$-module.
  \end{proof}
\end{homeworkProblem}

\begin{homeworkProblem}
  An $R$-module $M$ is called \textit{simple} (or \textit{irreducible}) if its only submodules are $\{0\}$ and $M$. An $R$-module $M$ is called \textit{indecomposable} if $M$ is not isomorphic to $N\oplus Q$ for some non-zero submodules $N,Q$. Show that every simple $R$-module is indecomposable, but the converse is not true.

  \begin{proof}
    Let $M$ be a simple $R$-module. Then $M\neq 0$ as otherwise there would only be one submodule. Suppose, for the sake of contradiction, that $M$ is not indecomposable. Then there exist some nonzero submodules $N,Q\sub M$ such that $M \cong N\oplus Q$. By simplicity of $M$, it follows that $N,Q = M$. But then $M\cong M\oplus M$. Moreover, $0\oplus M$ is then a nonzero proper submodule of $M\oplus M$, whence via the isomorphism $M\oplus M \cong M$ we obtain a nonzero proper submodule of $M$, contradicting the simplicity of $M$. \\

    To see that the converse does not hold, consider $R=\Z$ and $M=\Z$ considered as a $\Z$-module over itself. Note that $2\Z\sub \Z$ is a nonzero proper submodule fo $\Z$, so $M$ is not simple. to see that $M$ is indecomposable, note that all nonzero submodules of $M$ are of the form $a\Z$ for some $a\in \Z\setminus \{ 0\}$, and for any $a,b\in \Z\setminus\{ 0\}$, $ab\in a\Z \cap b\Z$. Hence, no sum of the required form would be direct.
  \end{proof}
\end{homeworkProblem}
,.
\begin{homeworkProblem}
  Let $R$ be a ring. An $R$-module $M$ is called \textit{cyclic} if it is generated as an $R$-module by a single element. \\

  \textbf{(a)} Prove that every cyclic $R$-module is of the form $R/I$ for some left ideal $I$ of $R$.

  \begin{proof}
    Let $M$ be a cyclic $R$-module. Then there exists an $m\in M$ such that $M = Rm$. Consider the map $\phi:R\to M$ given by $\phi(r)= rm$ for $r\in R$. By problem 1 part (a), $\phi$ is an $R$-module homomorphism; moreover, $\phi$ is surjective since $m$ generates $M$. Let $I = \ker(\phi)$, a left ideal of $R$ (actually two-sided, but we are identifying $R$ with its left regular module over itself so a priori $I$ is just a left R-submodule). Then, by the first isomorphism theorem, $M = \phi(R)\cong R/\ker(\phi) = R/I$.
  \end{proof}

  \textbf{(b)} Show that the simple $R$-modules are precisely the ones which are isomorphic to $R/\mathfrak{m}$ for some maximal left ideal $\mathfrak{m}$.

  \begin{proof}
    On one hand, $\mf{m}$ be a maximal left ideal of $R$. By the correspondence theorem applied to the natural projection, the only $R$-submodules of $R/\mf{m}$ are $\{ 0\}$ and $R/\mf{m}$, so $R/\mf{m}$ is simple (and so is every $R$-module isomorphic to it). \\

    On the other hand, suppose $M$ is a nonzero simple $R$-module. Take $m\in M\setminus\{ 0\}$. Then by the simplicity of $M$, $Rm = M$ i.e. $M$ is a cyclic module generated by $m$. Part (a) implies that there is some left ideal $\mf{m}$ of $R$ such that $M\cong R/\mf{m}$. Suppose that $I$ is a proper left ideal of $R$ such that $\mf{m}\sub I\subsetneq R$. Applying the natural projection, we see that $0 \sub I/\mf{m} \subsetneq R/\mf{m}$, whence simplicity of $R/\mf{m}$ implies that $I/\mf{m}$ is trivial i.e. $I = \mf{m}$. Thus by definition $\mf{m}$ is a maximal left ideal.
  \end{proof}

  \textbf{(c)} Show that any non-zero homomorphism of simple $R$-modules is an isomorphism. Deduce that if $M$ is simple, its endomorphism ring $\End_R(M):=\Hom_R(M,M)$ is a division ring. This result is known as \textit{Schur's Lemma}. \\

  \begin{proof}
    Suppose that $M,N$ are simple $R$-modules and let $f:M\to N$ be a nonzero $R$-module homomorphism.  As $f(M)\neq 0$ is a submodule of $N$, by simplicity $f(M) = N$ i.e. $f$ is surjective. As $f$ is nonzero, $\ker(f)$ is a nonzero submodule of $M$ whence $\ker(f) = 0$ i.e. $f$ is injective. Hence $f$ is an isomorphism. \\

    Suppose $M$ is simple and $f\in \End_R(M)\setminus\{ 0\}$. Then $f$ is an isomorphism, so the set-theoretic inverse $f^{-1}$ is in fact an R-module isomorphism and $f^{-1}\in \End_R(M)$. Hence $\End_R(M)$ is a divison ring.
  \end{proof}


\end{homeworkProblem}

\begin{homeworkProblem}
  Show that $\Q$ is not a free $\Z$-module, that is $\Q$ is not isomorphic to a direct sum of the form $\displaystyle\bigoplus_I \Z$, for any index set $I$. More generally, let $R$ be a PID which is not a field and $K=\fr(R)$ be its fraction field. Show that $K$ is not a free $R$-module.

  \begin{proof}
    Suppose, for the sake of contradiction, that $\Q$ is a free $\Z$-module. Let $X\sub \Q\setminus\{ 0\}$ be a $\Z$-basis for $\Q$. Fix $\frac{a}{b}, \frac{c}{d}\in X$. Noting that $a,b,c,d \neq 0$, observe that then
    \[
      (-bc)\cdot\frac{a}{b} + (da)\frac{c}{d} = \frac{-ac}{1} + \frac{ac}{1} = 0.
    \]
    Thus, for $X$ to be a basis, $|X| = 1$. Then by assumption, there exists a $\Z$-module isomorphism $f:\Q\to\Z$. Now, by surjectivity there exists a $\frac{p}{q}\in \Q$ such that $f(\frac{p}{q}) = 1$. Then,
    \[
      1 = f\left(\frac{p}{q}\right) = f\left(2\cdot\frac{p}{2q}\right) = 2\cdot f\left(\frac{p}{2q}\right),
    \]
    which is absurd. \\

    Now let $R$ be a PID which is not a field and $K=\fr(R)$ be its fraction field. Suppose, for the sake of contradiction, that $K$ is a free $R$-module. As before, take $X\sub K\setminus\{ 0\}$ to be an $R$-basis for $K$. Again, fix $\frac{a}{b}, \frac{c}{d}\in X$. Since $R$ is an integral domain and $a,b,c,d \neq 0$, $-bc\neq 0$ and $da\neq 0$. Observe that then
    \[
      (-bc)\cdot\frac{a}{b} + (da)\frac{c}{d} = \frac{-ac}{1} + \frac{ac}{1} = 0,
    \]
    so for $X$ to be a basis, $|X| = 1$. Let $f:R\to K$ be an $R$-module isomorphism and $\iota:R\to K$ be the natural localization map. As $R\setminus \{ 0\}$ has no zero divisors, $\iota$ is injective. From the fact that $f$ is an $R$-module isomorphism, we see that $K=f(R)=R\cdot f(1)$. \\

    Write $f(1) = \frac{a}{s} = \iota(a)\frac{1}{s}$. Then there exists an $r\in R$ such that
    \[
      \frac{1}{s^2} = r\cdot\frac{a}{s} = \iota(r)\frac{a}{s} \implies \frac{1}{s} = \iota(ra) \in \iota(R)
    \]
    but then, $f(1) = \iota(a)\iota(ra) = \iota(ara) \in \iota(R)$, whence $K = R\cdot f(1) = \iota(R)f(1) = \iota(R)$. As $\iota$ is injective, it follows that $K$ is ring-isomorphic to $R$, contradicting that $R$ is not a field.
  \end{proof}
\end{homeworkProblem}

\begin{homeworkProblem}
  Let $R$ be a commutative ring. Recall that an ideal $I$ of $R$ is called \textit{nilpotent} if there exists some $n\in\N$ such that $I^n=0$. \\

  \textbf{(a)} Let $i\in I$. Show that the element $r=1-i$ is invertible in $R$.

  \begin{proof}
    As $I$ is a nilpotent ideal, there exists an $n\in \N$ such that $I^n = 0$. Then $i^n = 0$, so
    \[
      1 = 1 - i^n = (1-i)(1+i+\cdots+i^{n-1}),
    \]
    whence $1-i \in \units{R}$.
  \end{proof}

  \textbf{(b)} Let $M,N$ be $R$-modules and let $\phi:M\rightarrow N$ be an $R$-module homomorphism. Show that $\phi$ induces an $R$-module homomorphism, $\overline{\phi}:M/IM\rightarrow N/IN$.

  \begin{proof}
    Let $\pi_M:M\to M/IM$ and $\pi_N:N\to N/IN$ be the natural projections. Define a map $\overline{\phi}:M/IM\to N/IN$ by $\overline{\phi}(m+IM) := \phi(m)+IN = (\pi_N \circ\phi)(m)$. To see that this map is well defined, suppose that $m + IM = m' + IM$. Then there exist $i_1, \ldots, i_s \in I$ and $m_1, \ldots, m_s \in M$ such that $m-m' = i_1 m_1 + \cdots + i_s m_s$. So
    \[
      \phi(m-m')= \phi(i_1 m_1 + \cdots + i_s m_s) = i_1 \phi(m_1) + \cdots + i_s \phi(m_s) \in IN,
    \]
    whence $\pi_N (\phi(m))-\pi_N (\phi(m'))=\pi_N (\phi(m-m')) = 0$, so $\pi_N (\phi(m))=\pi_N (\phi(m'))$.

    % TODO Show this is an R-module homomorphism.
    That $\overline{\phi}$ is an abelian group homomorphism is clear from the fact the $\phi$ is one. Suppose $r\in R$ and $m\in M$. Then
    \[
      \overline{\phi}(r\cdot (m+IM)) = \overline{\phi}(rm+IM) = \phi(rm)+IN = r\phi(m)+IN = r\cdot(\phi(m)+IN) = r\cdot\overline{\phi}(m+IM),
    \]
    so $\overline{\phi}$ is an $R$-module homomorphism.
  \end{proof}

  \textbf{(c)} Prove that if $\overline{\phi}$ is sujective, then $\phi$ is itself surjective.

  \begin{proof}
    Suppose that $\overline{\phi}$ is surjective. Then $N/IN=\overline{\phi}(M/IM) = (\overline{\phi}\circ\pi_M)(M) = (\pi_N \circ \phi)(M) = \phi(M)/IN$. Take $n\in N$. Then there exists an $m\in M$ such that $n+IN = \overline{\phi}(m+IM) = \phi(m) + IN$, whence $n-\phi(m)\in IN$. It follows that $n = \phi(m) + (n-\phi(m)) \in \phi(M) + IN$, whence $N=\phi(M) + IN$. As $I$ is a nilpotent ideal, there exists a $k\in \N$ such that $I^k = 0$. Observe that
    \[
      N = \phi(M) + IN = \phi(M) + I(\phi(M)+IN) = \phi(M) + I^2 N = \cdots = \phi(M) + I^k N =\phi(M),
    \]
    so $\phi$ is surjective. (Note: I couldn't see a way to use part (a) for this proof).
  \end{proof}
\end{homeworkProblem}

% \begin{homeworkProblem}
%   Let $G$ be a finite group and $k$ a field. Consider the group ring $k[G]$. \\
%
%   \textbf{(a)} Let $M$ be a $k$-vector space with a $G$-action. Show that $M$ becomes a $k[G]$-module. Conversely, if $M$ is a $k[G]$-module, show that $M$ is a $G$-set. \\
%
%   \begin{proof}
%     \underline{$\implies$}: Let $M$ be a $k$-vector space with a $G$-action. Define a $k[G]$-action on $M$ in the natural way, i.e. for $m\in M$ let
%     \[
%       \left(\sum_{g\in G} a_g g\right)\cdot m := \sum_{g\in G}a_g (g\cdot m).
%     \]
%     By group action axioms, $e\cdot m = m$ for all $m\in M$, so $1_{k[G]} = e$ acts trivially on $M$ as desired. For $$
%   \end{proof}
%
%   \textbf{(b)} Let $M,N$ be two $k[G]$-modules. Show that $\Hom_{k}(M,N)$ becomes a $k[G]$-module with the following $G$-action: For $g\in G$ and $\phi:M\to N$ a $k[G]$-homomorphism define
%   \[
%     (g\cdot\phi)(m):=g\phi(g^{-1}m),\text{ for }m\in M.
%   \]
% \end{homeworkProblem}


\end{document}

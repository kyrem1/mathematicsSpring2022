\documentclass[12pt,letterpaper]{scrartcl}

%--------Packages--------
\usepackage{amsmath, amsthm, amssymb}
\usepackage{xspace}
\usepackage{graphicx}
\usepackage{hhline}
\usepackage{amssymb}
\usepackage{array}
\usepackage{braket}
\usepackage{multicol}
\usepackage{mathtools}
\usepackage{enumerate}
\usepackage{delarray}
\usepackage{mathtools}
\usepackage{fullpage}
\usepackage{faktor} % For quotients
\usepackage{mathrsfs}



\usepackage[italicdiff]{physics} % For differentials
\usepackage{bbm} % For indicator

% \usepackage{quiver}
\usepackage[linguistics]{forest}




%--------Page Setup--------

\pagestyle{empty}%

\setlength{\hoffset}{-1.54cm}
\setlength{\voffset}{-1.54cm}

\setlength{\topmargin}{0pt}
\setlength{\headsep}{0pt}
\setlength{\headheight}{0pt}

\setlength{\oddsidemargin}{0pt}

\setlength{\textwidth}{195mm}
\setlength{\textheight}{250mm}

\setlength{\parindent}{0pt}

%--------Macros--------

\newcommand{\sub}{\subseteq}
\newcommand{\lcm}{\text{lcm}}
\newcommand{\mc}[1]{\mathcal{#1}}
\newcommand{\mf}[1]{\mathfrak{#1}}
\newcommand{\ms}[1]{\mathscr{#1}}
\newcommand{\sO}{\mathcal{O}}
\newcommand{\cyclic}[1]{\langle#1\rangle}
\newcommand{\units}[1]{#1 ^{\times}}
\newcommand{\la}{\langle}
\newcommand{\ra}{\rangle}
\newcommand{\lr}[1]{\left(#1\right)}
\newcommand{\lrvert}[1]{\left\lvert#1\right\rvert}

\DeclarePairedDelimiterX{\inp}[2]{\langle}{\rangle}{#1, #2}

%----Switch phi and varphi
% \let\temp\phi
% \let\phi\varphi
% \let\varphi\temp

\newcommand{\C}{\mathbb{C}}
\newcommand{\F}{\mathbb{F}}
\newcommand{\E}{\mathbb{E}}
\newcommand{\D}{\mathbb{D}}
\newcommand{\N}{\mathbb{N}\xspace}
\newcommand{\I}{\mathbb{I}\xspace}
\newcommand{\R}{\mathbb{R}\xspace}
\newcommand{\Z}{\mathbb{Z}\xspace}
\newcommand{\Q}{\mathbb{Q}\xspace}
\newcommand{\G}{\mathbb{G}\xspace}

\DeclareMathOperator{\Spec}{Spec}
\DeclareMathOperator{\res}{res}
% \DeclareMathOperator{\Tr}{Tr}
\DeclareMathOperator{\ord}{ord}
\DeclareMathOperator{\Sym}{Sym}
% \DeclareMathOperator{\dv}{div}
\DeclareMathOperator{\alb}{alb}
\DeclareMathOperator{\img}{Im}
\DeclareMathOperator{\et}{et}
\DeclareMathOperator{\ck}{coker}
\DeclareMathOperator{\Reg}{Reg}
\DeclareMathOperator{\Cor}{Cor}
\DeclareMathOperator{\Ac}{at}
\DeclareMathOperator{\supp}{supp}
\DeclareMathOperator{\Hom}{Hom}
\DeclareMathOperator{\Pic}{Pic}
\DeclareMathOperator{\Gal}{Gal}
\DeclareMathOperator{\fc}{frac}
\DeclareMathOperator{\Ann}{Ann}
\DeclareMathOperator{\Mod}{Mod}
\DeclareMathOperator{\Cone}{Cone}
\DeclareMathOperator{\FI}{FI}
\DeclareMathOperator{\End}{End}
\DeclareMathOperator{\Alb}{Alb}
\DeclareMathOperator{\Ext}{Ext}
\DeclareMathOperator{\ab}{ab}
\DeclareMathOperator{\Jac}{Jac}
\DeclareMathOperator{\coker}{coker}
\DeclareMathOperator{\fr}{frac}
\DeclareMathOperator{\Int}{Int}
\let\Span\relax
\DeclareMathOperator{\Span}{Span}
\DeclareMathOperator{\Ran}{Ran}



%----Analysis
\newcommand{\summ}{\sum\limits}
% \newcommand{\norm}[1]{\left\lVert#1\right\rVert}
\newcommand{\thicc}{\bigg}
\newcommand{\eps}{\varepsilon}
\newcommand*\cls[1]{\overline{#1}}
\newcommand{\ind}{\mathbbm{1}}
\DeclareMathOperator{\sgn}{sgn}


%--------Theorem environments--------
\newtheoremstyle{mystyle}%                % Name
  {}%                                     % Space above
  {}%                                     % Space below
  {}%                                     % Body font
  {}%                                     % Indent amount
  {\bfseries}%                            % Theorem head font
  {.}%                                    % Punctuation after theorem head
  { }%                                    % Space after theorem head, ' ', or \newline
  {\thmname{#1}\thmnote{ #3}}%            % Theorem head spec (can be left empty, meaning `normal')

\theoremstyle{mystyle}
\newtheorem*{problem}{Problem}

\theoremstyle{plain}
\newtheorem{definition}{Definition}[]
\newtheorem{lemma}{Lemma}[]
\newtheorem{corollary}{Corollary}[]
\newtheorem{theorem}{Theorem}[]
\theoremstyle{remark}
\newtheorem*{claim}{Claim}


\newenvironment{solution}
{\begin{proof}[Solution]}
{\end{proof}}


%--------Metadata--------
\title{Operator Algebras Reading Course}
\subtitle{Meeting 1 Exercises}
\author{James Harbour}

\begin{document}
\maketitle

\begin{problem}
    Let $X$ be a topological vector space over $\F\in\{\R,\C\}$ and $\phi:X\to \F$ a linear functional. Then $\phi$ is continuous if and only if $\ker(\phi)$ is closed.
\end{problem}

\vspace{3pt}
\begin{problem}
  Let $V$ be a vector space over $\F\in\{\R,\C\}$ and $f,f_1,\ldots,f_n$ linear functionals on $V$. Then there exist $c_1,\ldots,c_n\in\F$ such that $f = \sum_{j}c_j f_j$ if and only if $\ker(f)\supseteq \bigcap_{j=1}^{n}\ker(f_j)$.
\end{problem}

\begin{proof}
  The forward direction is clear by definition. Now suppose that $\ker(f)\supseteq \bigcap_{j=1}^{n}\ker(f_j)$. If $f=0$ then we are done so suppose that $f\neq 0$. Then $f$ is surjective. Consider $T:V\to\F^n$ given by $T(x) = (f_1(x),\ldots,f_n(x))$. This map is linear and $\ker(T) = \bigcap_{j=1}^{n}\ker(f_j)\sub \ker(f)$, so $f$ factors through $V/\ker(T)$. Moreover, $V/\ker(T)$ is isomorphic to $T(V)$ which is a subspace of $\F^n$, so there exists a map $g: \F^n\to \F$ such that $f = g\circ T$. Now, let $c_1,\ldots, c_n\in\F$ be such that $g(x_1,\cdots,x_n) = \sum_{j=1}^{n}c_j x_j$. Then $f(x) = h(f_1(x),\ldots,f_n(x)) = \sum_{j=1}^{n}c_j f_j(x)$ as desired.
\end{proof}

\vspace{3pt}
\begin{problem}[IV-1.23]
  Let $X$ and $Y$ be locally convex spaces and $T:X\to Y$ a linear transformation. Show that $T$ is continuous if and only if for every continuous seminorm $p$ on $Y$, $p\circ T$ is a continuous seminorm on $X$.
\end{problem}

\begin{proof}
  Note that if $p$ is a seminorm on $Y$, then by linearity of $T$ it is clear that $p\circ T$ is a seminorm on $X$. Hence, the forward direction follows immediately from transitivity of continuity. \\

  Now suppose that for every continuous seminorm $p$ on $Y$, $p\circ T$ is a continuous seminorm on $X$.
  % TODO finish proof after completing proposition 1.15

  By linearity, it suffices to prove that $T$ is continuous at $0$. Moreover, by proposition 1.15, we have a basis for the neighborhood system at $0$ given by the collection of all open, convex, balanced subsets of $Y$. As such, let $C\sub Y$ by open, convex, and balanced. It is enough to show that $T^{-1}(C)$ is open, as $0\in T^{-1}(C)$.\\

  By proposition 1.14, the Minkowski function $p$ of $C$ is the unique seminorm on $Y$ such that
  \[
    C = \{y\in Y : p(y) < 1\}.
  \]
  As $\{y\in Y : p(y)<1\}$ is open by assumption, proposition 1.3(b) implies that $p$ is continuous whence $p\circ T$ is a continuous seminorm on $X$. Thus, proposition 1.3 implies that the set $V:=\{x\in X: (p\circ T)(x) < 1\}$ is open. Noting that
  \[
    x\in T^{-1}(C) \iff Tx \in C \iff (p\circ T)(x)<1,
  \]
  it follows that $T^{-1}(C)= V$ is open.
\end{proof}

\vspace{3pt}
\begin{problem}[IV-2.4]
  A subset $B$ of a TVS $X$ is \emph{bounded} if for every open set $U$ containing $0$, there exists an $\eps>0$ such that $\eps B\sub U$.
  Let $X$ be a topological vector space. Prove the following.
\end{problem}

  \textbf{(a)}: If $B$ is a bounded subset of $X$, then so is $\cls{B}$.

  \begin{lemma}
    If $U$ is a neighborhood of $0$, then there is a neighborhood $V$ of $0$ such that $\cls{V}\sub U$.
  \end{lemma}

  \begin{proof}[Proof of Lemma 1]
    By an exercise we had in algebraic topology on topological groups, there exists a symmetric neighborhood $V$ of $0$ such that $V+V\sub U$. Take $x\in \cls{V}$. Then $x + V$ is an open neighborhood of $x$, so there is some $y\in V\cap (x + V)$. Writing $y = x+v$ for some $v\in V$, it follows that $x = y-v\in V-V = V+V\sub U$.
  \end{proof}

  \begin{proof}


  \end{proof}

  \textbf{(b)}: The finite union of bounded sets is bounded.\\

  \textbf{(c)}: Every compact set is bounded.

  \begin{lemma}
    If $U$ is a neighborhood of $0$, then there is a balanced neighborhood $W$ of $0$ such that $W\sub U$.
  \end{lemma}

  \begin{proof}[Proof of Lemma 2]
    The multiplication map $m:\F\times X\to X$ is continuous and $U$ is a neighborhood of $0$, so there exist $\delta>0$ and a neighborhood $V$ of 0 such that $m(B(0, \delta) \times V) \sub U$, i.e. $\alpha V\sub U$ for all $|\alpha|<\delta$. Let $W = \bigcup_{|\alpha|<\delta}\alpha V$. \\

    Suppose $w\in W$ and $|\beta|<1$. Writing $w = \alpha v$ for some $|\alpha|<\delta$ and $v\in V$, as $|\alpha\beta|<\delta$ it follows that $\beta w = \alpha\beta v \in W$. Thus $W$ is a balanced open neighborhood of $0$ constructed such that $W\sub U$.
  \end{proof}

  \begin{proof}
    Let $U$ be a neighborhood of $0$. By the above lemma, there is a balanced neighborhood $V$ of $0$ such that $V\sub U$.

    As $V$ is absorbing, $X = \bigcup_{n=1}^{\infty}nV$. Suppose that $K$ is compact. Then $K\sub \bigcup_{n=1}^{\infty}nV$ whence there are some $n_1<\cdots<n_k$ such that
    \[
      K\sub n_1 V\cup n_2 V \cdots n_k V.
    \]
    As $V$ is balanced, $n_1 V\cup n_2 V \cdots n_k V = n_k V$
  \end{proof}

  \textbf{(d)}: If $B\sub X$, then $B$ is bounded if and only if for every sequence $(x_n)$ contained in $B$ and for every $(\alpha_n)$ in $c_0$, $\alpha_n x_n\to 0$ in $X$.

  \begin{proof}\ \\
    \underline{$\implies$}: Suppose $(x_n)$ is a sequence in $B$ and $(\alpha_n)$ in $c_0$. Let $U$ be a neighborhood of $0$. By the lemma, there is a balanced neighborhood $V$ of $0$ such that $V\sub U$. The boundedness of $B$ implies that there is some $\eps>0$ such that $\eps B \sub V$. Let $N\in \N$ be such that $|\alpha_n| < \eps$ for all $n\geq N$. Then, as $V$ is balanced, for $n\geq N$
    \[
      \alpha_n x_n = \frac{\alpha}{\eps}\eps x_n \in \frac{\alpha}{\eps}\eps B \sub  \frac{\alpha}{\eps}V \sub V \sub U.
    \]

    \underline{$\impliedby$}: Suppose, for the sake of contradiction, that $B$ is not bounded. Then there is some neighborhood $U$ of $0$ such that $\eps B \not\sub U$ for all $\eps>0$. Thus, for $n\in \N$ there exists $x_n\in B\setminus nV$. Then $\frac{1}{n}x_n\not\in V$ for all $n\in \N$, contradicting the assumption that $\frac{1}{n}x_n\to 0$ in $X$.
  \end{proof}

  \textbf{(e)}: If $Y$ is a TVS, $T:X\to Y$ continuous linear, and $B$ is a bounded subset of $X$, then $T(B)$ is a bounded subset of $Y$.

  \begin{proof}
    Suppose that $V\sub Y$ is a neighborhood of $0$. Then $T^{-1}(V)$ is a neighborhood of $0$ in $X$, so there is some $\eps>0$ such that $\eps B\sub T^{-1}(V)$. Applying $T$, it follows that $\eps T(B) = T(\eps B)\sub T(T^{-1}(V))\sub V$.
  \end{proof}

  \textbf{(f)}: If $X$ is a LCS and $B\sub X$, then $B$ is bounded if and only if for every continuous seminorm $p$, $\sup\{p(b):b\in B\}<\infty$.

  \begin{proof}\ \\
    \underline{$\implies$}: Suppose that $p$ is a continuous seminorm. Then $V : =\{x: p(x)<1\}$ is an open neighborhood of $0$, so there is some $\eps>0$ such that $\frac{1}{\eps}B \sub V$, whence
    \[
      B\sub \eps V  = \{\eps x : p(x)<1\} = \{x: p(x) < \eps\}
    \]
    so $\sup\{p(b):b\in B\}\leq \eps < + \infty$.\\

    \underline{$\impliedby$}: Suppose that U is an open neighborhood of $0$. As $X$ is locally convex, there is an open balanced convex set $C$ such that $C\sub U$. Let $p$ be the Minkowski function corresponding to $C$, so $C = \{x : p(x)<1\}$. Letting $\delta > \sup\{p(b):b\in B\}$, it follows that
    \[
      \delta C = \{\delta x: p(x)< 1\} = \{x: p(x)<\delta\} \implies B\sub \delta C\sub \delta U.
    \]

  \end{proof}

  \textbf{(g)}: If $X$ is a normed space and $B\sub X$, then $B$ is bounded if and only if $\sup\{\norm{b}: b\in B\}<\infty$.

  \begin{proof}\ \\
    \underline{$\implies$}: The unit ball $B(0,1)$ is an open neighborhood of $0$ so there is some $r>0$ such that $B \sub rB(0,1) \sub B(0,r)$, so $\sup\{\norm{b}: b\in B\}\leq r < \infty$.\\

    \underline{$\impliedby$}: Suppose that $U$ is an open neighborhood of $0$. Then there exists some $\eps > 0$ such that $B(0,\eps)\sub U$. Letting $\delta =\sup\{\norm{b}: b\in B\}$, it follows that
    \[
      B\sub B(0,\delta) = \frac{\delta}{\eps}B(0,\eps)\sub \frac{\delta}{\eps}U.
    \]

  \end{proof}

  \textbf{(i)}: The translate of a bounded set is bounded.

  \begin{proof}
    Let $a\in X$. As $\{a\}$ is compact, the previous part implies that $\{a\}$ is bounded. Let $U$ be a neighborhood of $0$ and $V$ a neighborhood of $0$ such that $V+V\sub U$. Let $\eps_1,\eps_2>0$ such that $\eps_1 B\sub V$ and $\eps_2 a\in V$. Letting $\delta = \max\{\frac{1}{\eps_1},\frac{1}{\eps_2}\}$, it follows that
    \[
      B+a \sub \frac{1}{\eps_1}V + \frac{1}{\eps_2}V \sub \delta V  + \delta V\sub \delta U.
    \]
  \end{proof}


\begin{problem}
  Do IV-3 problems 1, 2, 6, 8-13.
\end{problem}

% Challenge problems are those on the strict topology for $C_b(X)$.

\end{document}

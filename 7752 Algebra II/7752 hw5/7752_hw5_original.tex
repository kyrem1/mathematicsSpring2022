\documentclass[12pt,
psamsfonts]{amsart}

%-------Packages---------
\usepackage{amssymb,amsfonts,amsmath}
\usepackage[all,arc]{xy}
\usepackage{enumerate}
\usepackage{mathrsfs}
\usepackage{fullpage}
\usepackage{xspace}
\usepackage[margin=1.0in]{geometry}
\usepackage{tcolorbox}
\usepackage{tikz-cd}
\usepackage{color}
\usepackage{aliascnt}
\usepackage[foot]{amsaddr}
\usepackage{hyperref}


%--------Theorem Environments--------
%theoremstyle{plain} --- default
\newtheorem{thm}{Theorem}[section]

%----Theorem
\newaliascnt{theo}{thm}
\newtheorem{theo}[theo]{Theorem}
\aliascntresetthe{theo}
\newcommand{\theoautorefname}{Theorem}
%----Corollary
\newaliascnt{cor}{thm}
\newtheorem{cor}[cor]{Corollary}
\aliascntresetthe{cor}
\newcommand{\corautorefname}{Corollary}
%----Proposition
\newaliascnt{prop}{thm}
\newtheorem{prop}[prop]{Proposition}
\aliascntresetthe{prop}
\newcommand{\propautorefname}{Proposition}
%----Lemma
\newaliascnt{lem}{thm}
\newtheorem{lem}[lem]{Lemma}
\aliascntresetthe{lem}
\newcommand{\lemautorefname}{Lemma}
%----Conjecture
\newaliascnt{conj}{thm}
\newtheorem{conj}[conj]{Conjecture}
\aliascntresetthe{conj}
\newcommand{\conjautorefname}{Conjecture}
%----Question
\newaliascnt{que}{thm}
\newtheorem{que}[que]{Question}
\aliascntresetthe{que}
\newcommand{\queautorefname}{Question}
%----Assumption
\newaliascnt{ass}{thm}
\newtheorem{ass}[ass]{Assumption}
\aliascntresetthe{ass}
\newcommand{\assautorefname}{Assumption}
%----Definition
\newaliascnt{defn}{thm}
\newtheorem{defn}[defn]{Definition}
\aliascntresetthe{defn}
\newcommand{\defnautorefname}{Definition}




%Style
\theoremstyle{remark}
%----Remark
\newaliascnt{rem}{thm}
\newtheorem{rem}[rem]{Remark}
\aliascntresetthe{rem}
\newcommand{\remautorefname}{Remark}

\newtheorem*{ack}{Acknowledgements}




\newtheorem{Proof}{Proof}

\theoremstyle{definition}
%\newtheorem{defn}[thm]{Definition}
\newtheorem{defns}[thm]{Definitions}
\newtheorem{con}[thm]{Construction}
\newtheorem{exmp}[thm]{Example}
\newtheorem{exmps}[thm]{Examples}
\newtheorem{notn}[thm]{Notation}
\newtheorem{notns}[thm]{Notations}
\newtheorem{addm}[thm]{Addendum}
\newtheorem{exer}[thm]{Exercise}
\newtheorem{conv}[thm]{Convention}

\newtheorem{case}[thm]{Case}


\newtheorem{rems}[thm]{Remarks}
\newtheorem{warn}[thm]{Warning}
%\newtheorem{sch}[thm]{Scholium}
\newtheorem{notation}[thm]{Notation}
\newtheorem{ex}[thm]{Examples}
\newtheorem{note}[thm]{Note}



\newcommand{\N}{\mathbb{N}\xspace}
\newcommand{\I}{\mathbb{I}\xspace}
\newcommand{\R}{\mathbb{R}\xspace}
\newcommand{\Z}{\mathbb{Z}\xspace}
\newcommand{\Q}{\mathbb{Q}\xspace}
\newcommand{\C}{\mathbb{C}\xspace}
\newcommand{\G}{\mathbb{G}\xspace}
\DeclareMathOperator{\Spec}{Spec}
\DeclareMathOperator{\res}{res}
\DeclareMathOperator{\Tr}{Tr}
\DeclareMathOperator{\ord}{ord}
\DeclareMathOperator{\Sym}{Sym}
\DeclareMathOperator{\dv}{div}
\DeclareMathOperator{\alb}{alb}
\DeclareMathOperator{\img}{Im}
\DeclareMathOperator{\et}{et}
\DeclareMathOperator{\ck}{coker}
\DeclareMathOperator{\Reg}{Reg}
\DeclareMathOperator{\Cor}{Cor}
\DeclareMathOperator{\Ac}{at}
\DeclareMathOperator{\supp}{supp}
\DeclareMathOperator{\Hom}{Hom}
\DeclareMathOperator{\Pic}{Pic}
\DeclareMathOperator{\Gal}{Gal}
\DeclareMathOperator{\fc}{frac}
\DeclareMathOperator{\Ann}{Ann}
\DeclareMathOperator{\Mod}{Mod}
\DeclareMathOperator{\Cone}{Cone}
\DeclareMathOperator{\FI}{FI}
\DeclareMathOperator{\End}{End}
\DeclareMathOperator{\rk}{rk}
\DeclareMathOperator{\Ext}{Ext}
\DeclareMathOperator{\ab}{ab}
\DeclareMathOperator{\Jac}{Jac}
\DeclareMathOperator{\coker}{coker}
\DeclareMathOperator{\fr}{frac}
\makeatletter
\let\c@equation\c@theo
\makeatother
\numberwithin{equation}{section}

\bibliographystyle{plain}
%\newcommand{\textlatin }




%--------Meta Data: Fill in your info------
\title{Math 7752 - Homework 5\\
Due Friday 02/25/22}

\begin{document}

\maketitle

\begin{enumerate}
\item Let $F=\Z^3$ be the free $\Z$-module of rank $3$. Let $N$ be the submodule of $F$ generated by $v_1=(1,2,3), (5,4,6)$, and $(7,8,9)$. 
\begin{enumerate}
\item Find compatible bases for $F$ and $N$, that is, bases satisfying the submodule theorem 1. 
\item Describe the quotient $F/N$ in the IF form.
\item Describe in IF form the abelian group given by the presentation 
\[\langle a,b,c|\;a+2b+3c=0,5a+4b+6c=0, 7a+8b+9c=0\rangle.\]
\end{enumerate} 
\medskip 
\item Let $R$ be a PID. For an $R$-module $M$ define $\rk(M)$ to be the minimal size of a generating set of $M$.  \begin{enumerate}
[(a)]\item Let $M$ be a finitely generated $R$-module and $R/a_1R\oplus\cdots\oplus R/a_mR\oplus R^s$ be its invariant factor decomposition. That is, $s\geq 0$ and the elements $a_1,\ldots, a_m$ are non-zero, non-units such that $a_1|a_2|\cdots|a_m$. Prove that $\rk(M)=m+s$. \textbf{Warning:} It is not true in general that $\rk(P\oplus Q)=\rk(P)\oplus\rk(Q)$. 
\item Let $F$ be a free $R$-module of rank $n$ with basis $e_1,\ldots e_n$. Let $N$ be the submodule of $F$ generated by some elements $v_1,\ldots,v_n\in F$. Let $A\in Mat_n(F)$ be the matrix such that 
\[\left(\begin{array}{ccc}
v_1\\
\vdots\\
v_n
\end{array}\right)=A\left(\begin{array}{ccc}
e_1\\
\vdots\\
e_n
\end{array}\right).\] Find a simple condition on the entries of $A$ which holds if and only if $\rk(F/N)=n$. 
\end{enumerate} 
\medskip
%\item Solve problem 12 from Dummit/Foote, p. 469.  \\

\item  In this problem $R$ will be a commutative domain. An $R$-module $P$ is called \textit{projective} if it is a direct summand of a free $R$-module. That is, if there exist a free $R$-module $F$ and a submodule $Q$ of $F$ such that $F=P\oplus Q$. 
\begin{enumerate}
\item Let $P,M,N$ be $R$-modules and suppose $f:M\rightarrow N$ is a surjective $R$-module homomorphism. The map $f$ induces a homomorphism of $R$-modules,
\begin{eqnarray*}
f_\star:&&\Hom_R(P,M)\rightarrow\Hom_R(P,N)\\
&&[\phi:P\rightarrow M]\mapsto [f\circ\phi:P\rightarrow N].
\end{eqnarray*} Prove that if $P$ is finitely generated and projective, then $f_\star$ is surjective.\\ \textbf{Hint:} The universal property of free $R$-modules will be useful. 
\item Show that if $R$ is a PID and $P$ is finitely generated, then $P$ is projective if and only if $P$ is free. 
\\
\end{enumerate}
\item Determine the number of possible RCF's of $8\times 8$ matrices $A$ over $\Q$ with $\chi_A(x)=x^8-x^4$. Explain your argument in detail. 
\\
\item \textbf{(Optional: extra practice for the midterm)} Solve Problem 9 from p. 489 of Dummit-Foote. 




\end{enumerate}





\end{document}
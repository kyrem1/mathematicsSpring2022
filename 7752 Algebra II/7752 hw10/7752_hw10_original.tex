\documentclass[12pt,
psamsfonts]{amsart}

%-------Packages---------
\usepackage{amssymb,amsfonts,amsmath}
\usepackage[all,arc]{xy}
\usepackage{enumerate}
\usepackage{mathrsfs}
\usepackage{fullpage}
\usepackage{xspace}
\usepackage[margin=1.0in]{geometry}
\usepackage{tcolorbox}
\usepackage{tikz-cd}
\usepackage{color}
\usepackage{aliascnt}
\usepackage[foot]{amsaddr}
\usepackage{hyperref}


%--------Theorem Environments--------
%theoremstyle{plain} --- default
\newtheorem{thm}{Theorem}[section]

%----Theorem
\newaliascnt{theo}{thm}
\newtheorem{theo}[theo]{Theorem}
\aliascntresetthe{theo}
\newcommand{\theoautorefname}{Theorem}
%----Corollary
\newaliascnt{cor}{thm}
\newtheorem{cor}[cor]{Corollary}
\aliascntresetthe{cor}
\newcommand{\corautorefname}{Corollary}
%----Proposition
\newaliascnt{prop}{thm}
\newtheorem{prop}[prop]{Proposition}
\aliascntresetthe{prop}
\newcommand{\propautorefname}{Proposition}
%----Lemma
\newaliascnt{lem}{thm}
\newtheorem{lem}[lem]{Lemma}
\aliascntresetthe{lem}
\newcommand{\lemautorefname}{Lemma}
%----Conjecture
\newaliascnt{conj}{thm}
\newtheorem{conj}[conj]{Conjecture}
\aliascntresetthe{conj}
\newcommand{\conjautorefname}{Conjecture}
%----Question
\newaliascnt{que}{thm}
\newtheorem{que}[que]{Question}
\aliascntresetthe{que}
\newcommand{\queautorefname}{Question}
%----Assumption
\newaliascnt{ass}{thm}
\newtheorem{ass}[ass]{Assumption}
\aliascntresetthe{ass}
\newcommand{\assautorefname}{Assumption}
%----Definition
\newaliascnt{defn}{thm}
\newtheorem{defn}[defn]{Definition}
\aliascntresetthe{defn}
\newcommand{\defnautorefname}{Definition}




%Style
\theoremstyle{remark}
%----Remark
\newaliascnt{rem}{thm}
\newtheorem{rem}[rem]{Remark}
\aliascntresetthe{rem}
\newcommand{\remautorefname}{Remark}

\newtheorem*{ack}{Acknowledgements}




\newtheorem{Proof}{Proof}

\theoremstyle{definition}
%\newtheorem{defn}[thm]{Definition}
\newtheorem{defns}[thm]{Definitions}
\newtheorem{con}[thm]{Construction}
\newtheorem{exmp}[thm]{Example}
\newtheorem{exmps}[thm]{Examples}
\newtheorem{notn}[thm]{Notation}
\newtheorem{notns}[thm]{Notations}
\newtheorem{addm}[thm]{Addendum}
\newtheorem{exer}[thm]{Exercise}
\newtheorem{conv}[thm]{Convention}

\newtheorem{case}[thm]{Case}


\newtheorem{rems}[thm]{Remarks}
\newtheorem{warn}[thm]{Warning}
%\newtheorem{sch}[thm]{Scholium}
\newtheorem{notation}[thm]{Notation}
\newtheorem{ex}[thm]{Examples}
\newtheorem{note}[thm]{Note}



\newcommand{\N}{\mathbb{N}\xspace}
\newcommand{\I}{\mathbb{I}\xspace}
\newcommand{\R}{\mathbb{R}\xspace}
\newcommand{\Z}{\mathbb{Z}\xspace}
\newcommand{\Q}{\mathbb{Q}\xspace}
\newcommand{\C}{\mathbb{C}\xspace}
\newcommand{\G}{\mathbb{G}\xspace}
\newcommand{\F}{\mathbb{F}\xspace}
\DeclareMathOperator{\Spec}{Spec}
\DeclareMathOperator{\res}{res}
\DeclareMathOperator{\Tr}{Tr}
\DeclareMathOperator{\ord}{ord}
\DeclareMathOperator{\Sym}{Sym}
\DeclareMathOperator{\dv}{div}
\DeclareMathOperator{\alb}{alb}
\DeclareMathOperator{\img}{Im}
\DeclareMathOperator{\et}{et}
\DeclareMathOperator{\ck}{coker}
\DeclareMathOperator{\Reg}{Reg}
\DeclareMathOperator{\Cor}{Cor}
\DeclareMathOperator{\ch}{char}
\DeclareMathOperator{\supp}{supp}
\DeclareMathOperator{\Hom}{Hom}
\DeclareMathOperator{\Aut}{Aut}
\DeclareMathOperator{\Gal}{Gal}
\DeclareMathOperator{\fc}{frac}
\DeclareMathOperator{\Ann}{Ann}
\DeclareMathOperator{\Mod}{Mod}
\DeclareMathOperator{\Cone}{Cone}
\DeclareMathOperator{\FI}{FI}
\DeclareMathOperator{\End}{End}
\DeclareMathOperator{\rk}{rk}
\DeclareMathOperator{\Ext}{Ext}
\DeclareMathOperator{\ab}{ab}

\DeclareMathOperator{\coker}{coker}
\DeclareMathOperator{\fr}{frac}
\makeatletter
\let\c@equation\c@theo
\makeatother
\numberwithin{equation}{section}

\bibliographystyle{plain}
%\newcommand{\textlatin }




%--------Meta Data: Fill in your info------
\title{Math 7752 - Homework 10\\
Due Wednesday 04/15/22 at 1 p.m.}

\begin{document}

\maketitle

%\textbf{Reminders:} If $K/F$ and $L/K$ are finite extensions, then 
%$L/F$ is finite and \[[L:F]=[L:K][K:F].\] In particular, both $[K:F]$ and $[L:K]$ divide $[L:F]$. 
%\\

%\textbf{Reminder:} A field $F$ of characteristc $p>0$ is perfect if the Frobenius map $\varphi:x\mapsto x^p$ is surjective. \\

\begin{enumerate}
%\item Let $K/L/F$ be a tower of algebraic extensions. Show that $K/F$ is separable if and only if $K/L$ and $L/F$ are separable. \\
%\item Let $K/F$ be a finite separable extension. Show that there is a finite number of fields $L$ such that $F\subseteq L\subseteq K$. \\
%\item Let $F$ be a field of $\ch(F)=p>0.$ Show that $F$ admits a finite inseparable extension $K/F$ if and only if $F$ is not perfect. \\
\item Let $F$ be a field, $f(x)\in F[x]$ be an irreducible separable polynomial over $F$ of degree $n$ and let $K$ be a splitting field of $f(x)$. 
\begin{enumerate}
\item Prove that $|\Gal(K/F)|$ is a multiple of $n$ and divides $n!$. 
\item Let $n=3$. Prove that $\Gal(K/F)$ is isomorphic to either $\Z/3\Z$ or $S_3$. 
\item Let $n=4$ and assume that $|\Gal(K/F)|=8$. Determine the isomorphism class of $\Gal(K/F)$.  
\end{enumerate} 
\medskip 
\medskip 
\item Let $f(x)\in\Q[x]$ be an irreducible polynomial of degree $n$, and let $K$ be a splitting field of $f(x)$ contained in $\C$. Label the roots of $f(x)$ by $\alpha_1,\ldots, \alpha_n$ (in some order), and let $\rho:\Gal(K/\Q)\hookrightarrow S_n$ be the associated embedding. 
 \begin{enumerate}
[(a)]\item Assume that $f(x)$ has at least one non-real root. Prove that the complex conjugation gives an element $\tau$ of $\Gal(K/\Q)$ of order $2$. What can you say about $\tau$ if $f(x)$ has precisely two non-real roots?
\item Suppose that the degree $n$ of $f(x)$ is a  prime number, and that $f(x)$ has precisely two non-real roots. Prove that $\Gal(K/\Q)$ is isomorphic to $S_n$. \textbf{Hint:} You might need to recall some facts from Algebra I about generators of $S_n$. 
%\item Suppose that the integers $n_1,\ldots, n_r$ are square-free and pairwise relatively prime. Prove that $[\Q_r:\Q]=2^r$. Conclude that the extension $L=\Q(T)$, where $T=\{\sqrt{n}:n\in\N, n \text{ square free}\}$ is an infinite algebraic extension of $\Q$. 
\end{enumerate} 
\medskip
\medskip 
\item Let $K$ be the splitting field of $f(x)=x^4-2\in\Q[x]$. 
\begin{enumerate}
\item  Choose an order on the set of roots of $f(x)$ and describe the associated embedding $\Gal(K/\Q)\hookrightarrow S_4$. (You can use the information you obtained in Homework 8).  
\item Describe all subgroups of $\Gal(K/\Q)$ and the corresponding subfields of $K$. 
\end{enumerate}
\medskip
%\item Let $n\geq 1$. Consider the finite field $\F_{p^n}=\{\alpha\in\overline{\F}_p: \alpha^{p^n}=\alpha\}$. Consider the Frobenius endomorphism $\varphi:\F_{p^n}\to\F_{p^n}$ given by $x\mapsto x^{p}$.  
%\begin{enumerate}
%\item Prove that $\varphi$ is an isomorphism of $\F_{p^n}$ to itself (and hence an automorphism). 
%\item  Determine the rational canonical form of the $\F_p$-linear transformation $\varphi$. 
%\end{enumerate}
%\medskip
%\item Let $F$ be a field and $p(x)\in F[x]$ be a non-constant polynomial. Show that $p(x)$ is separable if and only if $\gcd(p(x),p'(x))=1$.\\
%\item  The purpose of this problem is to prove that the group $\Aut_\Q(\R)$ is trivial by following the suggested steps. Let $\sigma:\R\to\R$ be an automorphism of $\R$.
%\begin{enumerate}
%\item  Show that $\sigma$ is strictly increasing. 
%\item Use the density of $\Q$ in $\R$ to show that $\sigma$ is continuous at $x=0$. 
%\item Deduce that $\sigma$ is continuous on $\R$, and hence $\sigma(x)=x$.  
%\end{enumerate}
%
%\medskip
\item Let $K/F$ and $L/F$ be field extensions.  \begin{enumerate}
\item Assume that $L/F$ is finite Galois. Show that $KL/K$ is also  Galois. 
\item Suppose that both $K/F$ and $L/F$ are Galois extensions. 
\begin{enumerate}
\item Prove that the extension $KL/F$ is also Galois and there is a natural embedding $\iota:\Gal(KL/F)\rightarrow\Gal(K/F)\times \Gal(L/F)$.
\item Assume now that $K/F$ and $L/F$ are both finite. Prove that the map $\iota$ in part (i) is an isomorphism if and only if $K\cap L=F$. 
\end{enumerate} 
\end{enumerate} 
%\medskip
%\medskip
%\item Dummit and Foote, Problems 17 \& 18 on pages 582-583. 




\end{enumerate}





\end{document}
\documentclass[12pt,
psamsfonts]{amsart}

%-------Packages---------
\usepackage{amssymb,amsfonts,amsmath}
\usepackage[all,arc]{xy}
\usepackage{enumerate}
\usepackage{mathrsfs}
\usepackage{fullpage}
\usepackage{xspace}
\usepackage[margin=1.0in]{geometry}
\usepackage{tcolorbox}
\usepackage{tikz-cd}
\usepackage{color}
\usepackage{aliascnt}
\usepackage[foot]{amsaddr}
\usepackage{hyperref}


%--------Theorem Environments--------
%theoremstyle{plain} --- default
\newtheorem{thm}{Theorem}[section]

%----Theorem
\newaliascnt{theo}{thm}
\newtheorem{theo}[theo]{Theorem}
\aliascntresetthe{theo}
\newcommand{\theoautorefname}{Theorem}
%----Corollary
\newaliascnt{cor}{thm}
\newtheorem{cor}[cor]{Corollary}
\aliascntresetthe{cor}
\newcommand{\corautorefname}{Corollary}
%----Proposition
\newaliascnt{prop}{thm}
\newtheorem{prop}[prop]{Proposition}
\aliascntresetthe{prop}
\newcommand{\propautorefname}{Proposition}
%----Lemma
\newaliascnt{lem}{thm}
\newtheorem{lem}[lem]{Lemma}
\aliascntresetthe{lem}
\newcommand{\lemautorefname}{Lemma}
%----Conjecture
\newaliascnt{conj}{thm}
\newtheorem{conj}[conj]{Conjecture}
\aliascntresetthe{conj}
\newcommand{\conjautorefname}{Conjecture}
%----Question
\newaliascnt{que}{thm}
\newtheorem{que}[que]{Question}
\aliascntresetthe{que}
\newcommand{\queautorefname}{Question}
%----Assumption
\newaliascnt{ass}{thm}
\newtheorem{ass}[ass]{Assumption}
\aliascntresetthe{ass}
\newcommand{\assautorefname}{Assumption}
%----Definition
\newaliascnt{defn}{thm}
\newtheorem{defn}[defn]{Definition}
\aliascntresetthe{defn}
\newcommand{\defnautorefname}{Definition}




%Style
\theoremstyle{remark}
%----Remark
\newaliascnt{rem}{thm}
\newtheorem{rem}[rem]{Remark}
\aliascntresetthe{rem}
\newcommand{\remautorefname}{Remark}

\newtheorem*{ack}{Acknowledgements}




\newtheorem{Proof}{Proof}

\theoremstyle{definition}
%\newtheorem{defn}[thm]{Definition}
\newtheorem{defns}[thm]{Definitions}
\newtheorem{con}[thm]{Construction}
\newtheorem{exmp}[thm]{Example}
\newtheorem{exmps}[thm]{Examples}
\newtheorem{notn}[thm]{Notation}
\newtheorem{notns}[thm]{Notations}
\newtheorem{addm}[thm]{Addendum}
\newtheorem{exer}[thm]{Exercise}
\newtheorem{conv}[thm]{Convention}

\newtheorem{case}[thm]{Case}


\newtheorem{rems}[thm]{Remarks}
\newtheorem{warn}[thm]{Warning}
%\newtheorem{sch}[thm]{Scholium}
\newtheorem{notation}[thm]{Notation}
\newtheorem{ex}[thm]{Examples}
\newtheorem{note}[thm]{Note}



\newcommand{\N}{\mathbb{N}\xspace}
\newcommand{\I}{\mathbb{I}\xspace}
\newcommand{\R}{\mathbb{R}\xspace}
\newcommand{\Z}{\mathbb{Z}\xspace}
\newcommand{\Q}{\mathbb{Q}\xspace}
\newcommand{\G}{\mathbb{G}\xspace}
\DeclareMathOperator{\Spec}{Spec}
\DeclareMathOperator{\res}{res}
\DeclareMathOperator{\Tr}{Tr}
\DeclareMathOperator{\ord}{ord}
\DeclareMathOperator{\Sym}{Sym}
\DeclareMathOperator{\dv}{div}
\DeclareMathOperator{\alb}{alb}
\DeclareMathOperator{\img}{Im}
\DeclareMathOperator{\et}{et}
\DeclareMathOperator{\ck}{coker}
\DeclareMathOperator{\Reg}{Reg}
\DeclareMathOperator{\Cor}{Cor}
\DeclareMathOperator{\Ac}{at}
\DeclareMathOperator{\supp}{supp}
\DeclareMathOperator{\Hom}{Hom}
\DeclareMathOperator{\Pic}{Pic}
\DeclareMathOperator{\Gal}{Gal}
\DeclareMathOperator{\fc}{frac}
\DeclareMathOperator{\Ann}{Ann}
\DeclareMathOperator{\Mod}{Mod}
\DeclareMathOperator{\Cone}{Cone}
\DeclareMathOperator{\FI}{FI}
\DeclareMathOperator{\End}{End}
\DeclareMathOperator{\Alb}{Alb}
\DeclareMathOperator{\Ext}{Ext}
\DeclareMathOperator{\ab}{ab}
\DeclareMathOperator{\Jac}{Jac}
\DeclareMathOperator{\coker}{coker}
\DeclareMathOperator{\fr}{frac}
\makeatletter
\let\c@equation\c@theo
\makeatother
\numberwithin{equation}{section}

\bibliographystyle{plain}
%\newcommand{\textlatin }




%--------Meta Data: Fill in your info------
\title{Math 7752 - Homework 1\\
Due Friday 01/28/22 at 1 p.m.}

\begin{document}

\maketitle
\textbf{Convention:} All rings considered below will have $1$ (but not necessarily commutative, unless stated). Additionally, by an $R$-module we will always mean a left $R$-module. 
\begin{enumerate}
\item Let $R$ be a ring and $M$ an $R$-module. 
\begin{enumerate}
[(a)]\item Prove that for every $m\in M$, the map $r\mapsto rm$ from $R$ to $M$ is a homomorphism of $R$-modules. 
\item Assume that $R$ is commutative and $M$ an $R$-module. Prove that there is an isomorphism  $\Hom_R(R,M)\simeq M$ as $R$-modules. 
\end{enumerate}
\item Give an explicit example of a map $f:A\to B$ with the following properties: \begin{itemize}
\item $A,B$ are $R$-modules.
\item $f$ is a group homomorphism.
\item $f$ is not an $R$-module homomorphism. 
\end{itemize}
\item Let $R$ be a ring and $M$ an $R$-module. 
\begin{enumerate}
[(a)]\item Let $N$ be a subset of $M$. The \textit{annihilator} of $N$ is defined to be the set \[\Ann_R(N):=\{r\in R: rn=0, \text{ for all }n\in N\}.\] Prove that $\Ann_R(N)$ is a left ideal of $R$. 
\item Show that if $N$ is an $R$-submodule of $M$, then $\Ann_R(N)$ is an ideal of $R$ (i.e. it is two-sided ideal). 
\item For a subset $I$ of $R$ the \textit{annihilator} of $I$ in $M$ is defined to be the set, 
\[\Ann_M(I):=\{m\in M:xm=0, \text{ for all }x\in I\}.\] Find a natural condition on $I$ that guarantees that $\Ann_M(I)$ is a submodule of $M$. 
\item Let $R$ be an integral domain. Prove that every finitely generated torsion $R$-module has a nonzero annihilator. 
\end{enumerate}
\item In class we obtained a simple characterization of $R$-modules when $R=\Z$, and $R=F[x]$, with $F$ a field. Imitate the method to find similar characterizations for $R$-modules in the following cases: (a) $R=\Z/n\Z$, for some $n\geq 2$; (b) $R=\Z[x]$; (c) $R=F[x,y]$.  
\item An $R$-module $M$ is called \textit{simple} (or \textit{irreducible}) if its only submodules are $\{0\}$ and $M$. An $R$-module $M$ is called \textit{indecomposable} if $M$ is not isomorphic to $N\oplus Q$ for some non-zero submodules $N,Q$. Show that every simple $R$-module is indecomposable, but the converse is not true.  
\item Let $R$ be a ring. An $R$-module $M$ is called \textit{cyclic} if it is generated as an $R$-module by a single element. 
\begin{enumerate}
[(a)]\item Prove that every cyclic $R$-module is of the form $R/I$ for some left ideal $I$ of $R$. 
\item Show that the simple $R$-modules are precisely the ones which are isomorphic to $R/\mathfrak{m}$ for some maximal left ideal $\mathfrak{m}$. 
\item Show that any non-zero homomorphism of simple $R$-modules is an isomorphism. Deduce that if $M$ is simple, its endomorphism ring $\End_R(M):=\Hom_R(M,M)$ is a division ring. This result is known as \textit{Schur's Lemma}. 
\end{enumerate}
\item Show that $\Q$ is not a free $\Z$-module, that is $\Q$ is not isomorphic to a direct sum of the form $\displaystyle\bigoplus_I \Z$, for any index set $I$. More generally, let $R$ be a PID which is not a field and $K=\fr(R)$ be its fraction field. Show that $K$ is not a free $R$-module. 
\item Let $R$ be a commutative ring. Recall that an ideal $I$ of $R$ is called \textit{nilpotent} if there exists some $n\in\N$ such that $I^n=0$. 
\begin{enumerate}
[(a)]\item Let $i\in I$. Show that the element $r=1-i$ is invertible in $R$. 
\item Let $M,N$ be $R$-modules and let $\phi:M\rightarrow N$ be an $R$-module homomorphism. Show that $\phi$ induces an $R$-module homomorphism, $\overline{\phi}:M/IM\rightarrow N/IN$. 
\item Prove that if $\overline{\phi}$ is sujective, then $\phi$ is itself surjective. 
\end{enumerate}




\end{enumerate}
\bigskip
\begin{center}
\textbf{Extra Problem (optional)}
\begin{enumerate}
\item Let $G$ be a finite group and $k$ a field. Consider the group ring $k[G]$. 
\begin{enumerate}
\item Let $M$ be a $k$-vector space with a $G$-action. Show that $M$ becomes a $k[G]$-module. Conversely, if $M$ is a $k[G]$-module, show that $M$ is a $G$-set. 
\item Let $M,N$ be two $k[G]$-modules. Show that $\Hom_{k}(M,N)$ becomes a $k[G]$-module with the following $G$-action: For $g\in G$ and $\phi:M\to N$ a $k[G]$-homomorphism define
\[(g\cdot\phi)(m):=g\phi(g^{-1}m),\text{ for }m\in M.\]
\end{enumerate}
\end{enumerate}
\end{center}



\end{document}
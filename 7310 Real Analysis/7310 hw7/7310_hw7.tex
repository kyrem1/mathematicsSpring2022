\documentclass[12pt,letterpaper]{article}

%--------Packages--------
\usepackage{amsmath, amsthm, amssymb}
\usepackage{xspace}
\usepackage{graphicx}
\usepackage{hhline}
\usepackage{amssymb}
\usepackage{array}
\usepackage{braket}
\usepackage{multicol}
\usepackage{mathtools}
\usepackage{enumerate}
\usepackage{delarray}
\usepackage{mathtools}
\usepackage{fullpage}
\usepackage{faktor} % For quotients
\usepackage{mathrsfs}

\usepackage[italicdiff]{physics} % For differentials
\usepackage{bbm} % For indicator

% \usepackage{quiver}
\usepackage[linguistics]{forest}




%--------Page Setup--------

\pagestyle{empty}%

\setlength{\hoffset}{-1.54cm}
\setlength{\voffset}{-1.54cm}

\setlength{\topmargin}{0pt}
\setlength{\headsep}{0pt}
\setlength{\headheight}{0pt}

\setlength{\oddsidemargin}{0pt}

\setlength{\textwidth}{195mm}
\setlength{\textheight}{250mm}


%--------Macros--------

\newcommand{\sub}{\subseteq}
\newcommand{\lcm}{\text{lcm}}
\newcommand{\mc}[1]{\mathcal{#1}}
\newcommand{\mf}[1]{\mathfrak{#1}}
\newcommand{\ms}[1]{\mathscr{#1}}
\newcommand{\sO}{\mathcal{O}}
\newcommand{\cyclic}[1]{\langle#1\rangle}
\newcommand{\units}[1]{#1 ^{\times}}
\newcommand{\la}{\langle}
\newcommand{\ra}{\rangle}
\newcommand{\lr}[1]{\left(#1\right)}
%----Switch phi and varphi
% \let\temp\phi
% \let\phi\varphi
% \let\varphi\temp

\newcommand{\C}{\mathbb{C}}
\newcommand{\F}{\mathbb{F}}
\newcommand{\N}{\mathbb{N}\xspace}
\newcommand{\I}{\mathbb{I}\xspace}
\newcommand{\R}{\mathbb{R}\xspace}
\newcommand{\Z}{\mathbb{Z}\xspace}
\newcommand{\Q}{\mathbb{Q}\xspace}
\newcommand{\G}{\mathbb{G}\xspace}
\DeclareMathOperator{\Spec}{Spec}
\DeclareMathOperator{\res}{res}
% \DeclareMathOperator{\Tr}{Tr}
\DeclareMathOperator{\ord}{ord}
\DeclareMathOperator{\Sym}{Sym}
% \DeclareMathOperator{\dv}{div}
\DeclareMathOperator{\alb}{alb}
\DeclareMathOperator{\img}{Im}
\DeclareMathOperator{\et}{et}
\DeclareMathOperator{\ck}{coker}
\DeclareMathOperator{\Reg}{Reg}
\DeclareMathOperator{\Cor}{Cor}
\DeclareMathOperator{\Ac}{at}
\DeclareMathOperator{\supp}{supp}
\DeclareMathOperator{\Hom}{Hom}
\DeclareMathOperator{\Pic}{Pic}
\DeclareMathOperator{\Gal}{Gal}
\DeclareMathOperator{\fc}{frac}
\DeclareMathOperator{\Ann}{Ann}
\DeclareMathOperator{\Mod}{Mod}
\DeclareMathOperator{\Cone}{Cone}
\DeclareMathOperator{\FI}{FI}
\DeclareMathOperator{\End}{End}
\DeclareMathOperator{\Alb}{Alb}
\DeclareMathOperator{\Ext}{Ext}
\DeclareMathOperator{\ab}{ab}
\DeclareMathOperator{\Jac}{Jac}
\DeclareMathOperator{\coker}{coker}
\DeclareMathOperator{\fr}{frac}
\DeclareMathOperator{\Int}{Int}



%----Analysis
\newcommand{\summ}{\sum\limits}
% \newcommand{\norm}[1]{\left\lVert#1\right\rVert}
\newcommand{\thicc}{\bigg}
\newcommand{\eps}{\varepsilon}
\newcommand*\cls[1]{\overline{#1}}
\newcommand{\ind}{\mathbbm{1}}
\DeclareMathOperator{\Span}{Span}

%--------Theorem environments--------
\newtheorem{definition}{Definition}[]
\newtheorem{lemma}{Lemma}[]
\newtheorem{corollary}{Corollary}[]
\newtheorem{theorem}{Theorem}[]
\theoremstyle{remark}
\newtheorem*{claim}{Claim}


\newenvironment{solution}
{\begin{proof}[Solution]}
{\end{proof}}


\makeatletter
\newcommand{\thickhline}{%
    \noalign {\ifnum 0=`}\fi \hrule height 1pt
    \futurelet \reserved@a \@xhline
}
\newcolumntype{"}{@{\hskip\tabcolsep\vrule width 1pt\hskip\tabcolsep}}
\makeatother

% --------Problem environment--------
\setlength\parindent{0pt}
\setcounter{secnumdepth}{0}
\newcounter{partCounter}
\newcounter{homeworkProblemCounter}
\setcounter{homeworkProblemCounter}{1}


\newenvironment{homeworkProblem}[1][-1]{
    \ifnum#1>0
        \setcounter{homeworkProblemCounter}{#1}
    \fi
    \section{Problem \arabic{homeworkProblemCounter}}
    \setcounter{partCounter}{1}
    \stepcounter{homeworkProblemCounter}
}


%--------Metadata--------
\title{MATH 7310 Homework 7}
\author{James Harbour}

\begin{document}
\maketitle

\begin{homeworkProblem}[2]
  If $X,Y$ are sets, and $f:X\to\C$, $g:Y\to\C$, we define $f\otimes g:X\times Y\to\C$ by $(f\otimes g)(x,y) = f(x)g(y)$. Fix $1\leq p<+\infty$.\\

  \textbf{(a)}: Let $(X,\Sigma,\mu),(Y,\mc{F},\nu)$ be $\sigma$-finite measure spaces. Show that if $f\in L^p(X,\mu),g\in L^p(Y,\nu)$, then $\norm{f\otimes g}_p = \norm{f}_p \norm{g}_p$.\\

  \textbf{(b)}: Let $(Z,\mc{O},\zeta)$ be a finite measure space. Suppose that $\mc{A}\sub\mc{O}$ is an algebra which generates the $\sigma$-algebra of $\mc{O}$. Use the monotone class lemma to show that $\{\ind_A:A\in\mc{A}\}$ is dense in $\{\ind_E:E\in\mc{O}\}$ in the $L^p$-norm for all $1\leq p <+\infty$.\\

  \textbf{(c)}: Let $(X,\Sigma,\mu),(Y,\mc{F},\nu)$ be finite measure spaces. Use the previous part to show that $\{\ind_E : E\in \Sigma\otimes\mc{F}\}\sub\cls{\Span}^{\norm{\cdot}_p}\{\ind_E\otimes\ind_F:E\in\Sigma,F\in\mc{F}\}$. Use this to show that $\cls{\Span}^{\norm{\cdot}_p}\{\ind_E\otimes\ind_F:E\in\Sigma,F\in\mc{F}\} = L^p(X\times Y,\mu\otimes\nu)$.\\

  \textbf{(d)}: Let $(X,\Sigma,\mu),(Y,\mc{F},\nu)$ be $\sigma$-finite measure spaces. Suppose that $D_X\sub L^p(X,\mu)$, $D_Y\sub L^p(Y,\nu)$ and that
  \[
    \cls{\Span}^{\norm{\cdot}_p}(D_X) = L^1(X,\mu),\hspace{10pt}\cls{\Span}^{\norm{\cdot}_p}(D_Y) = L^1(Y,\nu).
  \]
  Show that $\cls{\Span}^{\norm{\cdot}_p}(\{f\otimes g: f\in D_X,g\in D_Y\}) = L^p(X\times Y,\mu\otimes\nu)$.

\end{homeworkProblem}


\begin{homeworkProblem}
  Suppose that $f\in L^p\, \cap\, L^\infty$ for some $p<+\infty$ so that $f\in L^q$ for all $q>p$. Prove that then $\norm{f}_\infty = \lim_{q\to\infty}\norm{f}_q$.
\end{homeworkProblem}


\begin{homeworkProblem}
  If $f$ is a measurable function on $X$, define the \emph{essential range} $R_f$ of $f$ to be the set of all $z\in\C$ such that $\{x: |f(x)-z|<\eps\}$ has positive measure for all $\eps > 0$.\\

  \textbf{(a)}: Prove that $R_f$ is closed.

  \begin{proof}
    Let $z\in \cls{R_f}$. Then there exists a sequence $(z_n)_{n=1}^{\infty}$ in $R_f$ such that $z_n\to z$. Fix $\eps>0$. There is some $N\in\N$ such that $n\geq N\implies B_{\eps/2}(z_n)\sub B_\eps(z)$. Then $f^{-1}(B_{\eps/2}(z_n))\sub f^{-1}(B_\eps(z))$, whence $0<\mu(f^{-1}(B_{\eps/2}(z_n)))\leq \mu(f^{-1}(B_\eps(z)))$. Hence $z\in R_f$, so $R_f$ is closed.
  \end{proof}

  \textbf{(b)}: Prove that if $f\in L^\infty$, then $R_f$ is compact and $\norm{f}_\infty = \max\{|z|: z\in R_f\}$.


\end{homeworkProblem}


\begin{homeworkProblem}
  Suppose that $1\leq p < +\infty$ and $(f_n)_{n=1}^{\infty}$ in $L^p$. Prove that $(f_n)_{n=1}^{\infty}$ is Cauchy in the $L^p$-norm if an only if the following three conditions hold:
  \begin{enumerate}
    \item $(f_n)$ is Cauchy in measure;
    \item the sequence $(|f_n|^p)_{n=1}^{\infty}$ is uniformly integrable
    \item for every $\eps>0$ there exists $E\sub X$ such that $\mu(E) < +\infty$ and $\int_{E^c}|f_n|^p\dd{\mu}<\eps$ for all $n\in\N$.
  \end{enumerate}
\end{homeworkProblem}


\begin{homeworkProblem}
    Prove that if $E$ is a subset of a Hilbert space $\mc{H}$, then $(E^\perp)^\perp$ is the smallest closed subspace of $\mc{H}$ containing $E$.

    \begin{claim}
      If $M$ is a closed linear subspace of $\mc{H}$, then $(M^\perp)^\perp = M$.
    \end{claim}

    \begin{proof}[Proof of Claim]
      Note that we have $\mc{H} = M\oplus M^\perp$. Let $y\in (M^\perp)^\perp$. Then there exist unique $x\in M$, $x^\perp\in    M^\perp$ such that $y = x + x^\perp$. Noting that $M\sub (M^\perp)^\perp$, we have that $x^\perp = y-x \in M^\perp\cap (M^\perp)^\perp = \{0\}$, whence $x^\perp = 0$ and $y=x\in M$. Thus $M = (M^\perp)^\perp$.
    \end{proof}

    \begin{proof}
      On one hand, note that $E\sub \cls{\Span(E)}\implies (E^\perp)^\perp\sub (\cls{\Span(E)}^\perp)^\perp \overset{\text{claim}}{=}\cls{\Span(E)}$. On the other hand, as $(E^\perp)^\perp$ is a closed linear subspace of $\mc{H}$ and $E\sub(E^\perp)^\perp$, it follows that $\cls{\Span(E)}\sub (E^\perp)^\perp$.
    \end{proof}

    % TODO Show E^\perp is a closed linear subspace of $\mc{H}$.
\end{homeworkProblem}


\end{document}

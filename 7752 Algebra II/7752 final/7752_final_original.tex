\documentclass[12pt,
psamsfonts]{amsart}

%-------Packages---------
\usepackage{amssymb,amsfonts,amsmath}
\usepackage[all,arc]{xy}
\usepackage{enumerate}
\usepackage{mathrsfs}
\usepackage{fullpage}
\usepackage{xspace}
\usepackage[margin=1.0in]{geometry}
\usepackage{tcolorbox}
\usepackage{tikz-cd}
\usepackage{color}
\usepackage{aliascnt}
\usepackage[foot]{amsaddr}
\usepackage{hyperref}


%--------Theorem Environments--------
%theoremstyle{plain} --- default
\newtheorem{thm}{Theorem}[section]

%----Theorem
\newaliascnt{theo}{thm}
\newtheorem{theo}[theo]{Theorem}
\aliascntresetthe{theo}
\newcommand{\theoautorefname}{Theorem}
%----Corollary
\newaliascnt{cor}{thm}
\newtheorem{cor}[cor]{Corollary}
\aliascntresetthe{cor}
\newcommand{\corautorefname}{Corollary}
%----Proposition
\newaliascnt{prop}{thm}
\newtheorem{prop}[prop]{Proposition}
\aliascntresetthe{prop}
\newcommand{\propautorefname}{Proposition}
%----Lemma
\newaliascnt{lem}{thm}
\newtheorem{lem}[lem]{Lemma}
\aliascntresetthe{lem}
\newcommand{\lemautorefname}{Lemma}
%----Conjecture
\newaliascnt{conj}{thm}
\newtheorem{conj}[conj]{Conjecture}
\aliascntresetthe{conj}
\newcommand{\conjautorefname}{Conjecture}
%----Question
\newaliascnt{que}{thm}
\newtheorem{que}[que]{Question}
\aliascntresetthe{que}
\newcommand{\queautorefname}{Question}
%----Assumption
\newaliascnt{ass}{thm}
\newtheorem{ass}[ass]{Assumption}
\aliascntresetthe{ass}
\newcommand{\assautorefname}{Assumption}
%----Definition
\newaliascnt{defn}{thm}
\newtheorem{defn}[defn]{Definition}
\aliascntresetthe{defn}
\newcommand{\defnautorefname}{Definition}




%Style
\theoremstyle{remark}
%----Remark
\newaliascnt{rem}{thm}
\newtheorem{rem}[rem]{Remark}
\aliascntresetthe{rem}
\newcommand{\remautorefname}{Remark}

\newtheorem*{ack}{Acknowledgements}




\newtheorem{Proof}{Proof}

\theoremstyle{definition}
%\newtheorem{defn}[thm]{Definition}
\newtheorem{defns}[thm]{Definitions}
\newtheorem{con}[thm]{Construction}
\newtheorem{exmp}[thm]{Example}
\newtheorem{exmps}[thm]{Examples}
\newtheorem{notn}[thm]{Notation}
\newtheorem{notns}[thm]{Notations}
\newtheorem{addm}[thm]{Addendum}
\newtheorem{exer}[thm]{Exercise}
\newtheorem{conv}[thm]{Convention}

\newtheorem{case}[thm]{Case}


\newtheorem{rems}[thm]{Remarks}
\newtheorem{warn}[thm]{Warning}
%\newtheorem{sch}[thm]{Scholium}
\newtheorem{notation}[thm]{Notation}
\newtheorem{ex}[thm]{Examples}
\newtheorem{note}[thm]{Note}



\newcommand{\N}{\mathbb{N}\xspace}
\newcommand{\I}{\mathbb{I}\xspace}
\newcommand{\R}{\mathbb{R}\xspace}
\newcommand{\Z}{\mathbb{Z}\xspace}
\newcommand{\Q}{\mathbb{Q}\xspace}
\newcommand{\C}{\mathbb{C}\xspace}
\newcommand{\G}{\mathbb{G}\xspace}
\newcommand{\F}{\mathbb{F}\xspace}
\DeclareMathOperator{\Spec}{Spec}
\DeclareMathOperator{\res}{res}
\DeclareMathOperator{\Tr}{Tr}
\DeclareMathOperator{\ord}{ord}
\DeclareMathOperator{\Sym}{Sym}
\DeclareMathOperator{\dv}{div}
\DeclareMathOperator{\alb}{alb}
\DeclareMathOperator{\img}{Im}
\DeclareMathOperator{\et}{et}
\DeclareMathOperator{\ck}{coker}
\DeclareMathOperator{\Reg}{Reg}
\DeclareMathOperator{\Cor}{Cor}
\DeclareMathOperator{\ch}{char}
\DeclareMathOperator{\supp}{supp}
\DeclareMathOperator{\Hom}{Hom}
\DeclareMathOperator{\Aut}{Aut}
\DeclareMathOperator{\Gal}{Gal}
\DeclareMathOperator{\fc}{frac}
\DeclareMathOperator{\Ann}{Ann}
\DeclareMathOperator{\Mod}{Mod}
\DeclareMathOperator{\Cone}{Cone}
\DeclareMathOperator{\FI}{FI}
\DeclareMathOperator{\End}{End}
\DeclareMathOperator{\rk}{rk}
\DeclareMathOperator{\Ext}{Ext}
\DeclareMathOperator{\ab}{ab}

\DeclareMathOperator{\coker}{coker}
\DeclareMathOperator{\fr}{frac}
\makeatletter
\let\c@equation\c@theo
\makeatother
\numberwithin{equation}{section}

\bibliographystyle{plain}
%\newcommand{\textlatin }




%--------Meta Data: Fill in your info------
\title{Algebra II - FINAL EXAM\\
Opens: Monday 05/02/22 at 9 a.m. \\
Closes: Friday 05/06/22 at 11.59 p.m.}

\begin{document}

\maketitle

\textbf{Directions:} 
\begin{itemize}
\item There are six problems in this exam. The total number of points is 110, but you only need 100 to get $100\%$.\\
%\item There are a few bonus points in this exam. You should try those, even if you choose not to do that problem. (Some of those are pretty easy). \\
\item Please show all your work and justify any statements you make. \\
\item Vague statements and hand-waving arguments won't be appreciated. State clearly any result you are referring to. \\ 
\item You may assume the statement of an earlier part of a problem to do a later part. \\
\item After completion, please submit an electronic copy of your exam to Gradescope. Typed solutions are preferred, but hand-written solutions are also allowed, \textbf{as long as the writing is legible, and the scanned copy is clear}. 
\end{itemize}

\vspace{100pt}
\textbf{Rules:} \begin{itemize}
\item You are only allowed to use the lecture notes and statements from homework assignments. Answers of the form: ``It follows from Theorem blah in Dummit and Foote" will only be given partial credit. \\ 
\item No outside resources are allowed. Moreover, \textbf{NO} collaboration is allowed. \\

\item In the beginning of your exam, please sing the pledge:

``\textit{In my honor as a student, I pledge that I have neither given nor received help on this assignment''.}
\end{itemize}

\newpage

%\textbf{Reminders:} If $K/F$ and $L/K$ are finite extensions, then 
%$L/F$ is finite and \[[L:F]=[L:K][K:F].\] In particular, both $[K:F]$ and $[L:K]$ divide $[L:F]$. 
%\\

%\textbf{Reminder:} A field $F$ of characteristc $p>0$ is perfect if the Frobenius map $\varphi:x\mapsto x^p$ is surjective. \\

\begin{enumerate}
\item[\textbf{1.}] Let $F$ be a field, $n\in\N$ and $A\in Mat_n(F)$.
\begin{enumerate}
\item ($10$ points) Assume that $\ch(F)=0$. Prove that the matrix $A$ is nilpotent if and only if $\Tr(A^k)=0$, for all $k\in\N$. 
\item ($4$ points) Show that the assertion of (a) is false if $\ch(F)\neq 0$. \\
\end{enumerate}

\item[\textbf{2.}] Let $A\in GL_n(\F_p)$, where $p$ is a prime number and $n\in\N$. 
\begin{enumerate}
\item ($6$ points) Suppose that the matrix $A$ is \textbf{diagonalizable over the algebraic closure} $\overline{\F}_p$.  Show that the order of $A$ in the group $GL_n(\F_p)$ is equal to the lcm of the orders of the eigenvalues of $A$ in $\overline{\F}_p^\times$. 
\item ($10$ points) Prove that $GL_n(\F_p)$ has an element $B$ of exact order $p^n-1$. 
\item ($4$ points) Construct explicitly an element $B\in GL_2(\F_3)$ of order $8$. \\
\end{enumerate}

%\item Let $f(x)=x^4-x^2+1\in\Q[x]$. 
%\begin{enumerate}
%\item Describe a splitting field $K$ for $f$ over $\Q$, and find its degree. 
%\item Describe explicitly the Galois group $G=\Gal(K/\Q)$ and determine its isomorphism class. 
%\item Find all subfields of $K/\Q$. 
%\end{enumerate}

\item[\textbf{3.}] Let $f(x)=(x^2-2)(x^3-3)\in\Q[x]$, and let $K$ be a splitting field of $f(x)$ over $\Q$. 
\begin{enumerate}
\item ($10$ points) Prove that $\Gal(K/\Q)\simeq S_3\times\Z/2\Z$.
\item ($8$ points) Find a primitive element of $K$ over $\Q$ (and prove your answer). \\
\end{enumerate} 

\item[\textbf{4.}] \begin{enumerate}
\item ($10$ points) Let $F\subseteq K\subseteq L$ be a tower of algebraic extensions. Let $\alpha\in L$, and let $\mu_{\alpha,F}(x)$ be its minimal polynomial over $F$. Prove an isomorphism of $K$-algebras \[K\otimes_F F(\alpha)\simeq K[x]/(\mu_{\alpha,F}(x)).\]
\item ($13$ points) Let $K_1, K_2$ be two algebraic extensions (not necessarily finite) of a field $F$. Suppose that $F$ has characteristic $0$ and that we have $F$-embeddings $K_1, K_2\hookrightarrow\overline{F}$, to a fixed algebraic closure of $F$. Prove that the $F$-algebra $K_1\otimes_F K_2$ has no nonzero nilpotent elements. \textbf{Hint:} Reduce to the case when one of the extensions is finite over $F$. \\
\end{enumerate}

\item[\textbf{5.}]
\begin{enumerate}
\item ($6$ points) Let $K/F$ be a finite Galois extension and let $G=\Gal(K/F)$. Assume that $G$ is a simple group (recall: this means that $G$ has no nontrivial proper normal subgroups). Let $\alpha\in K\setminus F$ and $\mu_{\alpha,F}(x)\in F[x]$ its minimal polynomial over $F$. Prove that $K$ is a splitting field for $\mu_{\alpha,F}(x)$. 
\item ($8$ points) Let $F\subseteq L\subseteq K$ be a tower of finite extensions. Suppose that both $L/F$ and $K/F$ are Galois and the Galois group $H=\Gal(K/L)$ of $K/L$ is cyclic. Let $M$ be any subfield of $K/L$. Prove that the extension $M/F$ is Galois. \\
\end{enumerate}

\item[\textbf{6.}] Let $p$ be an odd prime. Consider the $p^{th}$ cyclotomic field, $K=\Q(\zeta)$, where $\zeta\in\C$ is a primitive $p^{th}$ root of unity. 
\begin{enumerate}
\item ($6$ points) Prove that $K$ contains a unique subfield of the form $\Q(\sqrt{m})$, where $m\in\Z$ is a square-free integer.
\item ($10$ points) Now suppose $p=5$. Find the integer $m$ explicitly. \textbf{Hint:} Finding an explicit generator of the Galois group $G=\Gal(K/\Q)$  might be useful. 
\item ($5$ points) Do the same for $p=7$. How is your answer different from (b)? What would you expect the general answer to be? 
\end{enumerate}
\medskip 
\medskip
%\item[\textbf{7.}] 
%\begin{enumerate}
%\item ($5$ points) Let $p$ be a prime and $n\in\N$.
% Consider the polynomial $f(x)=x^{p^n}-x\in\F_p[x]$. Prove that $f(x)$ is equal to the product of \textbf{ALL monic irreducible polynomials} $g(x)\in\F_p[x]$ whose degree divides $n$ (with each of the polynomials appearing in the factorization only once). 
%\item ($5$ points)  Consider the polynomial $p(x)=x^5-1\in\F_7[x]$. Let $K$ be a splitting field for $p(x)$ over $\F_7$. Compute the Galois group $G=\Gal(K/\F_7)$. 
%%We already know that the finite field $\F_{p^n}$ is a splitting field for $f(x)=x^{p^n}-x$. Let $m\in\N$ be the minimal positive integer  such that $\F_{p^n}$ is a splitting field over $\F_p$ for some polynomial of degree $m$ (not necessarily irreducible). Prove that $m=q_1^{l_1}+\cdots+q_r^{l_r}$, where $n=q_1^{l_1}\cdots q_r^{l_r}$ is the prime factorization of $n$. 
%\end{enumerate}
\medskip 
\medskip



\end{enumerate}





\end{document}
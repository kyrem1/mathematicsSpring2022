\documentclass[12pt,letterpaper]{article}

%--------Packages--------
\usepackage{amsmath, amsthm, amssymb}
\usepackage{xspace}
\usepackage{graphicx}
\usepackage{hhline}
\usepackage{amssymb}
\usepackage{array}
\usepackage{braket}
\usepackage{multicol}
\usepackage{mathtools}
\usepackage{enumerate}
\usepackage{delarray}
\usepackage{mathtools}
\usepackage{fullpage}
\usepackage{faktor} % For quotients
\usepackage{mathrsfs}

\usepackage[italicdiff]{physics} % For differentials
\usepackage{bbm} % For indicator

% \usepackage{quiver}
\usepackage[linguistics]{forest}




%--------Page Setup--------

\pagestyle{empty}%

\setlength{\hoffset}{-1.54cm}
\setlength{\voffset}{-1.54cm}

\setlength{\topmargin}{0pt}
\setlength{\headsep}{0pt}
\setlength{\headheight}{0pt}

\setlength{\oddsidemargin}{0pt}

\setlength{\textwidth}{195mm}
\setlength{\textheight}{250mm}


%--------Macros--------

\newcommand{\sub}{\subseteq}
\newcommand{\lcm}{\text{lcm}}
\newcommand{\mc}[1]{\mathcal{#1}}
\newcommand{\mf}[1]{\mathfrak{#1}}
\newcommand{\ms}[1]{\mathscr{#1}}
\newcommand{\sO}{\mathcal{O}}
\newcommand{\cyclic}[1]{\langle#1\rangle}
\newcommand{\units}[1]{#1 ^{\times}}
\newcommand{\la}{\langle}
\newcommand{\ra}{\rangle}
\newcommand{\lr}[1]{\left(#1\right)}
%----Switch phi and varphi
% \let\temp\phi
% \let\phi\varphi
% \let\varphi\temp

\newcommand{\C}{\mathbb{C}}
\newcommand{\F}{\mathbb{F}}
\newcommand{\N}{\mathbb{N}\xspace}
\newcommand{\I}{\mathbb{I}\xspace}
\newcommand{\R}{\mathbb{R}\xspace}
\newcommand{\Z}{\mathbb{Z}\xspace}
\newcommand{\Q}{\mathbb{Q}\xspace}
\newcommand{\G}{\mathbb{G}\xspace}
\DeclareMathOperator{\Spec}{Spec}
\DeclareMathOperator{\res}{res}
% \DeclareMathOperator{\Tr}{Tr}
\DeclareMathOperator{\ord}{ord}
\DeclareMathOperator{\Sym}{Sym}
% \DeclareMathOperator{\dv}{div}
\DeclareMathOperator{\alb}{alb}
\DeclareMathOperator{\img}{Im}
\DeclareMathOperator{\et}{et}
\DeclareMathOperator{\ck}{coker}
\DeclareMathOperator{\Reg}{Reg}
\DeclareMathOperator{\Cor}{Cor}
\DeclareMathOperator{\Ac}{at}
\DeclareMathOperator{\supp}{supp}
\DeclareMathOperator{\Hom}{Hom}
\DeclareMathOperator{\Pic}{Pic}
\DeclareMathOperator{\Gal}{Gal}
\DeclareMathOperator{\fc}{frac}
\DeclareMathOperator{\Ann}{Ann}
\DeclareMathOperator{\Mod}{Mod}
\DeclareMathOperator{\Cone}{Cone}
\DeclareMathOperator{\FI}{FI}
\DeclareMathOperator{\End}{End}
\DeclareMathOperator{\Alb}{Alb}
\DeclareMathOperator{\Ext}{Ext}
\DeclareMathOperator{\ab}{ab}
\DeclareMathOperator{\Jac}{Jac}
\DeclareMathOperator{\coker}{coker}
\DeclareMathOperator{\fr}{frac}
\DeclareMathOperator{\Int}{Int}



%----Analysis
\newcommand{\summ}{\sum\limits}
% \newcommand{\norm}[1]{\left\lVert#1\right\rVert}
\newcommand{\thicc}{\bigg}
\newcommand{\eps}{\varepsilon}
\newcommand*\cls[1]{\overline{#1}}
\newcommand{\ind}{\mathbbm{1}}
\DeclareMathOperator{\Span}{Span}

%--------Theorem environments--------
\newtheorem{definition}{Definition}[]
\newtheorem{lemma}{Lemma}[]
\newtheorem{corollary}{Corollary}[]
\newtheorem{theorem}{Theorem}[]
\theoremstyle{remark}
\newtheorem*{claim}{Claim}


\newenvironment{solution}
{\begin{proof}[Solution]}
{\end{proof}}


\makeatletter
\newcommand{\thickhline}{%
    \noalign {\ifnum 0=`}\fi \hrule height 1pt
    \futurelet \reserved@a \@xhline
}
\newcolumntype{"}{@{\hskip\tabcolsep\vrule width 1pt\hskip\tabcolsep}}
\makeatother

% --------Problem environment--------
\setlength\parindent{0pt}
\setcounter{secnumdepth}{0}
\newcounter{partCounter}
\newcounter{homeworkProblemCounter}
\setcounter{homeworkProblemCounter}{1}


\newenvironment{homeworkProblem}[1][-1]{
    \ifnum#1>0
        \setcounter{homeworkProblemCounter}{#1}
    \fi
    \section{Problem \arabic{homeworkProblemCounter}}
    \setcounter{partCounter}{1}
    \stepcounter{homeworkProblemCounter}
}


%--------Metadata--------
\title{MATH 7310 Homework 7}
\author{James Harbour}

\begin{document}

\maketitle

\begin{homeworkProblem}
  Let $(X,\Sigma,\mu)$ be a measure space.

  \textbf{(i)}: Prove that if $\mu(E_n)<+\infty$ for $n\in\N$ and $\ind_{E_n}\to f$ in $L^1$, then $f$ is (a.e. equal to) the characteristic function of a measurable set.\\

  \textbf{(ii)}: Let $\Sigma_f = \{ E\in\Sigma: \mu(E)<+\infty\}$. Define an equivalence relation on $\Sigma_f$ by $E\sim F$ if $\mu(E\Delta F)=0$. Let $\Omega = \Sigma_f/\sim$, and define a metric $\rho$ on $\Omega$ be $\rho([E],[F]) = \mu(E\Delta F)$. Show that the map $\iota:\Omega\to L^1(X,\mu)$ given by $\iota([E]) = \ind_E$ is an isometry with closed image.\\

  \textbf{(iii)}: Show that $(\Omega,\rho)$ is a complete metric space.
\end{homeworkProblem}

\begin{homeworkProblem}[2]
  If $X,Y$ are sets, and $f:X\to\C$, $g:Y\to\C$, we define $f\otimes g:X\times Y\to\C$ by $(f\otimes g)(x,y) = f(x)g(y)$. Fix $1\leq p<+\infty$.\\

  \textbf{(a)}: Let $(X,\Sigma,\mu),(Y,\mc{F},\nu)$ be $\sigma$-finite measure spaces. Show that if $f\in L^p(X,\mu),g\in L^p(Y,\nu)$, then $\norm{f\otimes g}_p = \norm{f}_p \norm{g}_p$.\\

  \textbf{(b)}: Let $(Z,\mc{O},\zeta)$ be a finite measure space. Suppose that $\mc{A}\sub\mc{O}$ is an algebra which generates the $\sigma$-algebra of $\mc{O}$. Use the monotone class lemma to show that $\{\ind_A:A\in\mc{A}\}$ is dense in $\{\ind_E:E\in\mc{O}\}$ in the $L^p$-norm for all $1\leq p <+\infty$.\\

  \textbf{(c)}: Let $(X,\Sigma,\mu),(Y,\mc{F},\nu)$ be finite measure spaces. Use the previous part to show that $\{\ind_E : E\in \Sigma\otimes\mc{F}\}\sub\cls{\Span}^{\norm{\cdot}_p}\{\ind_E\otimes\ind_F:E\in\Sigma,F\in\mc{F}\}$. Use this to show that $\cls{\Span}^{\norm{\cdot}_p}\{\ind_E\otimes\ind_F:E\in\Sigma,F\in\mc{F}\} = L^p(X\times Y,\mu\otimes\nu)$.\\

  \textbf{(d)}: Let $(X,\Sigma,\mu),(Y,\mc{F},\nu)$ be $\sigma$-finite measure spaces. Suppose that $D_X\sub L^p(X,\mu)$, $D_Y\sub L^p(Y,\nu)$ and that
  \[
    \cls{\Span}^{\norm{\cdot}_p}(D_X) = L^1(X,\mu),\hspace{10pt}\cls{\Span}^{\norm{\cdot}_p}(D_Y) = L^1(Y,\nu).
  \]
  Show that $\cls{\Span}^{\norm{\cdot}_p}(\{f\otimes g: f\in D_X,g\in D_Y\}) = L^p(X\times Y,\mu\otimes\nu)$.

\end{homeworkProblem}


\begin{homeworkProblem}
  Suppose that $f\in L^p\, \cap\, L^\infty$ for some $p<+\infty$ so that $f\in L^q$ for all $q>p$. Prove that then $\norm{f}_\infty = \lim_{q\to\infty}\norm{f}_q$.
\end{homeworkProblem}


\begin{homeworkProblem}
  If $f$ is a measurable function on $X$, define the \emph{essential range} $R_f$ of $f$ to be the set of all $z\in\C$ such that $\{x: |f(x)-z|<\eps\}$ has positive measure for all $\eps > 0$.\\

  \textbf{(a)}: Prove that $R_f$ is closed.

  \begin{proof}
    Let $z\in \cls{R_f}$. Then there exists a sequence $(z_n)_{n=1}^{\infty}$ in $R_f$ such that $z_n\to z$. Fix $\eps>0$. There is some $N\in\N$ such that $n\geq N\implies B_{\eps/2}(z_n)\sub B_\eps(z)$. Then $f^{-1}(B_{\eps/2}(z_n))\sub f^{-1}(B_\eps(z))$, whence $0<\mu(f^{-1}(B_{\eps/2}(z_n)))\leq \mu(f^{-1}(B_\eps(z)))$. Hence $z\in R_f$, so $R_f$ is closed.
  \end{proof}

  \textbf{(b)}: Prove that if $f\in L^\infty$, then $R_f$ is compact and $\norm{f}_\infty = \max\{|z|: z\in R_f\}$.


\end{homeworkProblem}


\begin{homeworkProblem}
  Suppose that $1\leq p < +\infty$ and $(f_n)_{n=1}^{\infty}$ in $L^p$. Prove that $(f_n)_{n=1}^{\infty}$ is Cauchy in the $L^p$-norm if and only if the following three conditions hold:
  \begin{enumerate}
    \item $(f_n)$ is Cauchy in measure;
    \item the sequence $(|f_n|^p)_{n=1}^{\infty}$ is uniformly integrable
    \item for every $\eps>0$ there exists $E\sub X$ such that $\mu(E) < +\infty$ and $\int_{E^c}|f_n|^p\dd{\mu}<\eps$ for all $n\in\N$.
  \end{enumerate}

  \begin{lemma}
    Any finite subset $\{f_k\}_{k=1}^{n}\sub L^1(\mu)$ is uniformly integrable.
  \end{lemma}
  \begin{proof}[Proof of Lemma 1]
    We show first that $f\in L^1(\mu)$ is uniformly integrable. Note that, if $f\in L^1(\mu)$, then $|f|\ind_{\{|f|> m\}}\searrow 0$ pointwise a.e. as $\{|f| = +\infty\} = \bigcap_{M\in\N}\{|f|> M\}$ implies that $\lim_{M\to\infty}\mu({|f|>M})=\mu(\{|f| = +\infty\}) = 0$. Moreover, for all $M\in\N$, $|f\ind_{|f|>M}|\leq |f|\in L^1(\mu)$, so by the dominated convergence theorem
    \begin{equation}
      \lim_{M\to\infty}\int_{\{|f|>M\}} |f|\dd{\mu} = 0.
    \end{equation}
    For any $E\sub X$ measurable and $M\in\N$, we have that
    \begin{equation}
      \int_E |f|\dd{\mu} = \int_{E\,\cap\{ |f|\leq M\}} |f|\dd{\mu} + \int_{E\,\cap\{ |f|> M\}} |f|\dd{\mu} \leq M\cdot\mu(E) + \int_{\{ |f|> M\}} |f|\dd{\mu}.
    \end{equation}
    Fix $\eps > 0$. By (1), there exists some $N\in \N$ such that $\int_{\{|f|>N\}} |f|\dd{\mu}<\frac{\eps}{2}$. Choose $\delta = \frac{\eps}{2N}$. Then, for any $E\sub X$ measurable such that $\mu(E)<\delta$, we have by (2) that
    \[
      \left\lvert\int_E f\dd{\mu}\right\rvert\leq \int_E |f|\dd{\mu} < N\cdot\delta + \frac{\eps}{2} = \eps.
    \]

    Now suppose that $\{f_k\}_{k=1}^{n}\sub L^1(\mu)$ is a finite subset of $L^1(\mu)$. Fix $\eps > 0$. By uniform integrability of each of the singletons, for each $k\in \{1,\ldots,n\}$ there exists a $\delta_k>0$ such that $\mu(E)<\delta_k\implies |\int_{E} f_k|<\eps$. Choosing $\delta = \min\{\delta_1,\ldots,\delta_n\}>0$, the claim follows.
  \end{proof}

  \begin{lemma}
    Suppose $(f_n)_{n=1}^{\infty}$ is a sequence in $L^1(\mu)$ and $f\in L^1(\mu)$ such that $\norm{f_n-f}_1 \xrightarrow{n\to\infty} 0$. Then $\{f_n\}_{n=1}^{\infty}$ is uniformly integrable.
  \end{lemma}
  \begin{proof}[Proof of Lemma 2]
    Observe that, for any measurable $E\sub X$ and $n\in \N$,
    \[
      \int_E |f_n|\dd{\mu} \leq \int_E |f|\dd{\mu} + \int_E |f_n - f|\dd{\mu} \leq \int_E |f|\dd{\mu} + \norm{f_n-f}_1.
    \]
    Fix $\eps > 0$ and choose $N\in\N$ such that for $n\geq N$ we have $\norm{f_n-f}_1 < \frac{\eps}{2}$. By Lemma 1, $\{ f\}$ is uniformly integrable, so there is some $\delta'>0$ such that $\mu(E)<\delta'$ implies that $\int_E |f|\dd{\mu} < \frac{\eps}{2}$. \\

    Again by Lemma 1, $\{ f_k\}_{k=1}^{N-1}$ is uniformly integrable, so there is some $\delta''>0$ such that $\mu(E)<\delta''$ implies $\int_E |f_k|\dd{\mu} < \eps$ for all $k\in\{1,\ldots,N-1\}$. Setting $\delta = \min\{\delta',\delta''\}$, the claim follows.
  \end{proof}

  \begin{proof}[Proof of Theorem]\ \\
    \underline{$\implies$}: Suppose that $(f_n)_{n=1}^{\infty}$ is Cuachy in the $L^p$-norm. Then by completeness, there is some $f\in L^p(\mu)$ such $\norm{f-f_n}_p \xrightarrow{n\to\infty} 0$. For $\eps>0$, noting that $\{ |f_n-f|\geq\eps\} = \{|f_n-f|^p / \eps^p\geq 1\}$, we have that
    \[
      \mu(\{|f_n-f|\geq\eps\}) = \int_{\{|f_n-f|\geq\eps\}}\frac{|f_n-f|^p}{\eps^p}\dd{\mu} \leq\frac{1}{\eps^p}\norm{f_n-f}_p^p \xrightarrow{n\to\infty} 0.
    \]

    Thus $f_n\to f$ in measure, whence $(f_n)_{n=1}^{\infty}$ is Cauchy in measure. \\

    By the reverse triangle inequality,
    \[
      \left\lvert\norm{f_n}_p - \norm{f}_p\right\rvert \leq \norm{f_n-f}_p\xrightarrow{n\to\infty}0,
    \]
    so $\norm{|f_n|^p}_1\xrightarrow{n\to\infty}\norm{|f|^p}_1\in L^1(\mu)$. Now by Lemma 2, $(|f_n|^p)_{n=1}^{\infty}$ is uniformly integrable.


  \end{proof}
\end{homeworkProblem}


\begin{homeworkProblem}
    Prove that if $E$ is a subset of a Hilbert space $\mc{H}$, then $(E^\perp)^\perp$ is the smallest closed subspace of $\mc{H}$ containing $E$.

    \begin{claim}
      If $M$ is a closed linear subspace of $\mc{H}$, then $(M^\perp)^\perp = M$.
    \end{claim}

    \begin{proof}[Proof of Claim]
      Note that we have $\mc{H} = M\oplus M^\perp$. Let $y\in (M^\perp)^\perp$. Then there exist unique $x\in M$, $x^\perp\in    M^\perp$ such that $y = x + x^\perp$. Noting that $M\sub (M^\perp)^\perp$, we have that $x^\perp = y-x \in M^\perp\cap (M^\perp)^\perp = \{0\}$, whence $x^\perp = 0$ and $y=x\in M$. Thus $M = (M^\perp)^\perp$.
    \end{proof}

    \begin{proof}
      On one hand, note that $E\sub \cls{\Span(E)}\implies (E^\perp)^\perp\sub (\cls{\Span(E)}^\perp)^\perp \overset{\text{claim}}{=}\cls{\Span(E)}$. On the other hand, as $(E^\perp)^\perp$ is a closed linear subspace of $\mc{H}$ and $E\sub(E^\perp)^\perp$, it follows that $\cls{\Span(E)}\sub (E^\perp)^\perp$.
    \end{proof}

    % TODO Show E^\perp is a closed linear subspace of $\mc{H}$.
\end{homeworkProblem}


\end{document}

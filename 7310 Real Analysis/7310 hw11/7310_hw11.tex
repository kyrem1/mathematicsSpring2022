\documentclass[12pt,letterpaper]{article}

%--------Packages--------
\usepackage{amsmath, amsthm, amssymb}
\usepackage{xspace}
\usepackage{graphicx}
\usepackage{hhline}
\usepackage{amssymb}
\usepackage{array}
\usepackage{braket}
\usepackage{multicol}
\usepackage{mathtools}
\usepackage{enumerate}
\usepackage{delarray}
\usepackage{mathtools}
\usepackage{fullpage}
\usepackage{faktor} % For quotients
\usepackage{mathrsfs}

\usepackage[italicdiff]{physics} % For differentials
\usepackage{bbm} % For indicator

% \usepackage{quiver}
\usepackage[linguistics]{forest}




%--------Page Setup--------

\pagestyle{empty}%

\setlength{\hoffset}{-1.54cm}
\setlength{\voffset}{-1.54cm}

\setlength{\topmargin}{0pt}
\setlength{\headsep}{0pt}
\setlength{\headheight}{0pt}

\setlength{\oddsidemargin}{0pt}

\setlength{\textwidth}{195mm}
\setlength{\textheight}{250mm}


%--------Macros--------

\newcommand{\sub}{\subseteq}
\newcommand{\lcm}{\text{lcm}}
\newcommand{\mc}[1]{\mathcal{#1}}
\newcommand{\mf}[1]{\mathfrak{#1}}
\newcommand{\ms}[1]{\mathscr{#1}}
\newcommand{\sO}{\mathcal{O}}
\newcommand{\cyclic}[1]{\langle#1\rangle}
\newcommand{\units}[1]{#1 ^{\times}}
\newcommand{\la}{\langle}
\newcommand{\ra}{\rangle}
\newcommand{\lr}[1]{\left(#1\right)}
\newcommand{\lrvert}[1]{\left\lvert#1\right\rvert}

\DeclarePairedDelimiterX{\inp}[2]{\langle}{\rangle}{#1, #2}

%----Switch phi and varphi
% \let\temp\phi
% \let\phi\varphi
% \let\varphi\temp

\newcommand{\C}{\mathbb{C}}
\newcommand{\F}{\mathbb{F}}
\newcommand{\E}{\mathbb{E}}
\newcommand{\N}{\mathbb{N}\xspace}
\newcommand{\I}{\mathbb{I}\xspace}
\newcommand{\R}{\mathbb{R}\xspace}
\newcommand{\Z}{\mathbb{Z}\xspace}
\newcommand{\Q}{\mathbb{Q}\xspace}
\newcommand{\G}{\mathbb{G}\xspace}

\DeclareMathOperator{\Spec}{Spec}
\DeclareMathOperator{\res}{res}
% \DeclareMathOperator{\Tr}{Tr}
\DeclareMathOperator{\ord}{ord}
\DeclareMathOperator{\Sym}{Sym}
% \DeclareMathOperator{\dv}{div}
\DeclareMathOperator{\alb}{alb}
\DeclareMathOperator{\img}{Im}
\DeclareMathOperator{\et}{et}
\DeclareMathOperator{\ck}{coker}
\DeclareMathOperator{\Reg}{Reg}
\DeclareMathOperator{\Cor}{Cor}
\DeclareMathOperator{\Ac}{at}
\DeclareMathOperator{\supp}{supp}
\DeclareMathOperator{\Hom}{Hom}
\DeclareMathOperator{\Pic}{Pic}
\DeclareMathOperator{\Gal}{Gal}
\DeclareMathOperator{\fc}{frac}
\DeclareMathOperator{\Ann}{Ann}
\DeclareMathOperator{\Mod}{Mod}
\DeclareMathOperator{\Cone}{Cone}
\DeclareMathOperator{\FI}{FI}
\DeclareMathOperator{\End}{End}
\DeclareMathOperator{\Alb}{Alb}
\DeclareMathOperator{\Ext}{Ext}
\DeclareMathOperator{\ab}{ab}
\DeclareMathOperator{\Jac}{Jac}
\DeclareMathOperator{\coker}{coker}
\DeclareMathOperator{\fr}{frac}
\DeclareMathOperator{\Int}{Int}
\let\Span\relax
\DeclareMathOperator{\Span}{Span}
\DeclareMathOperator{\Ran}{Ran}



%----Analysis
\newcommand{\summ}{\sum\limits}
% \newcommand{\norm}[1]{\left\lVert#1\right\rVert}
\newcommand{\thicc}{\bigg}
\newcommand{\eps}{\varepsilon}
\newcommand*\cls[1]{\overline{#1}}
\newcommand{\ind}{\mathbbm{1}}
\DeclareMathOperator{\sgn}{sgn}


%--------Theorem environments--------
\newtheorem{definition}{Definition}[]
\newtheorem{lemma}{Lemma}[]
\newtheorem{corollary}{Corollary}[]
\newtheorem{theorem}{Theorem}[]
\theoremstyle{remark}
\newtheorem*{claim}{Claim}


\newenvironment{solution}
{\begin{proof}[Solution]}
{\end{proof}}


\makeatletter
\newcommand{\thickhline}{%
    \noalign {\ifnum 0=`}\fi \hrule height 1pt
    \futurelet \reserved@a \@xhline
}
\newcolumntype{"}{@{\hskip\tabcolsep\vrule width 1pt\hskip\tabcolsep}}
\makeatother

% --------Problem environment--------
\setlength\parindent{0pt}
\setcounter{secnumdepth}{0}
\newcounter{partCounter}
\newcounter{homeworkProblemCounter}
\setcounter{homeworkProblemCounter}{1}


\newenvironment{homeworkProblem}[1][-1]{
    \ifnum#1>0
        \setcounter{homeworkProblemCounter}{#1}
    \fi
    \section{Problem \arabic{homeworkProblemCounter}}
    \setcounter{partCounter}{1}
    \stepcounter{homeworkProblemCounter}
}


%--------Metadata--------
\title{MATH 7310 Homework 11}
\author{James Harbour}

\begin{document}
\maketitle

\begin{homeworkProblem}[1]

\end{homeworkProblem}

\begin{homeworkProblem}
  Let $(X,\Sigma,\mu)$ be a probability space. Fix $p\in[1,+\infty]$ and $f\in L^p(X,\Sigma,\mu)$, let $p'\in[1,+\infty]$ be the conjugate exponent. Let $\mc{F}\sub \Sigma$ be a sub-$\sigma$-algebra.\\

  \textbf{(a)}: Show that $f\in L^1(X,\Sigma,\mu)$.

  \begin{proof}
    By proposition 6.12, $\norm{f}_1\leq \mu(X)^{1-\frac{1}{p}}\norm{f}_p < +\infty$, whence $f\in L^1(X,\Sigma,\mu)$.
  \end{proof}

  \textbf{(b)}: Let $\E_\mc{F}(f)$ be the conditional expectation onto $\mc{F}$. Show that
  \[
    \norm{\E_\mc{F}(f)g}_1 \leq \norm{f}_p\norm{g}_{p'}
  \]
  for all $\mc{F}$-measurable simple functions $g$. Use this to show that $\E_\mc{F}(f)\in L^p(X,\mc{F},\mu)$ and that
  \[
    \norm{\E_\mc{F}(f)}_p\leq \norm{f}_p.
  \]

  \begin{proof}
    Let $\alpha:X\to\C$ be the $\mc{F}$-measurable function such that $\alpha\E_\mc{F}(f) = |\E_\mc{F}(f)|$ and $|\alpha|=1$. Note that then by finiteness $\alpha\in L^\infty(X,\mc{F},\mu\vert_\mc{F})$. Then, for $\mc{F}$-measurable simple functions $g$, by Homework 8 problems 5(a) and 4(b),
    \begin{align*}
      \norm{\E_\mc{F}(f)g}_1 &= \int \E_\mc{F}(f)\alpha |g|\dd{\mu\vert_{\mc{F}}} = \int \E_\mc{F}(f\alpha) |g|\dd{\mu\vert_{\mc{F}}} \\
      &= \int f\alpha |g|\dd{\mu} = \lrvert{\int f\alpha |g|\dd{\mu}} \leq \int |f| |g|\dd{\mu} \leq \norm{f}_p\norm{g}_{p'}.
    \end{align*}
    Now by $L^p$-$L^{p'}$ duality,
    \begin{align*}
      \norm{\E_\mc{F}(f)}_p &= \sup\left\{\lrvert{\int\E_\mc{F}(f)g\dd{\mu\vert_\mc{F}}}:g\in L^{p'}(X,\mu\vert_{\mc{F}})\text{ simple with }\norm{g}_{p'}=1\right\} \\
      &\leq\sup\left\{\norm{\E_\mc{F}(f)g}_1:g\in L^{p'}(X,\mu\vert_{\mc{F}})\text{ simple with }\norm{g}_{p'}=1\right\} \\
      &\leq\sup\left\{\norm{f}_p\norm{g}_{p'}:g\in L^{p'}(X,\mu\vert_{\mc{F}})\text{ simple with }\norm{g}_{p'}=1\right\}  = \norm{f}_p\\
    \end{align*}
  \end{proof}
\end{homeworkProblem}


\begin{homeworkProblem}
  Suppose that $(X,\Sigma,\mu)$ and $(Y,\mc{F},\nu)$ are $\sigma$-finite measure spaces and $K\in L^2(X\times Y,\mu\otimes\nu)$. If $f\in L^2(Y,\nu)$, the integral $Tf(x) = \int_Y K(x,y)f(y)\dd{\nu(y)}$ converges for a.e. $x\in X$, $Tf\in L^2(X,\mu)$, and $\norm{Tf}_2\leq \norm{K}_2 \norm{f}_2$.

  \begin{proof}
    By Holder's inequality with $p=2$,
    \[
      \int |K(x,y)||f(y)|\dd{\nu(y)} = \norm{K(x,\cdot)}_{L^2(\nu)}\norm{f}_{L^2(\nu)} < +\infty,
    \]
    so the integral converges absolutely for a.e. $x$. Now, we compute

    \[
      \norm{Tf}_{L^2(\mu)} \leq \norm{\int|K(x,y)||f(y)|\dd{\nu(y)}}_{L^2(\mu)} \leq \norm{\norm{K(\cdot,\cdot)}_{L^2(\mu)}}_{L^2(\nu)}\norm{f}_{L^2(\nu)} = \norm{K}_{L^2(\mu\otimes\nu)} \norm{f}_{L^2(\nu)} < +\infty
    \]
    so $Tf\in L^2(\mu)$.
  \end{proof}
\end{homeworkProblem}


\begin{homeworkProblem}
  Let $\eta(t) = e^{-1/t}$ for $t>0$ and $\eta(t) = 0$ for $t \leq 0$.\\

  \textbf{(a)}: For $k\in \N$, $t>0$, prove that $\eta^{(k)}(t) = P_k(1/t)e^{-1/t}$ where $P_k$ is a polynomial of degree $2k$.

  \begin{proof}
    We induct on $k\in \N$. We compute that $\eta'(t) = \frac{1}{t^2}e^{-1/t}$, so $P_1(x) = x^2$ is degree $2$ and thus satisfies the hypothesis. Now suppose that $\eta^{(k)}(t) = P_k(\frac{1}{t})e^{-1/t}$ where $P_k$ is a polynomial of degree $2k$. Then
    \begin{align*}
      \eta^{(k+1)}(t) = (P_k(\frac{1}{t})e^{-1/t})' = \frac{1}{t^2}P_k'(\frac{1}{t})e^{-1/t} + \frac{1}{t^2}P_k(\frac{1}{t})e^{-1/t} =(\frac{1}{t^2}P_k'(\frac{1}{t}) + \frac{1}{t^2}P_k(\frac{1}{t}))e^{-1/t}
    \end{align*}
    so $P_{k+1}(x) = x^2 P_k'(x)+x^2 P_k(x)$ is a degree $2(k+1)$ polynomial we have satisfied the hypothesis.
  \end{proof}

  \textbf{(b)}: Prove that $\eta^{(k)}(0)$ exists and is zero for all $k\in\N$.

  \begin{proof}
    We compute that $\lim_{t\to0^+}\frac{\eta(t)}{t} = 0$, so $\eta'(0)$ exists and equals zero. Now suppose $\eta^{(k)}(0)$ exists and equals zero. Then
    \[
      \lim_{t\to0+}\frac{\eta^{(k)}(t)-\eta^{(k)}(0)}{t} =\lim_{t\to0+}\frac{P_k(\frac{1}{t})e^{\frac{-1}{t}}}{t} = \lim_{t\to0+}\frac{1}{t P_k(\frac{1}{t})e^{\frac{1}{t}}} = 0,
    \]
    so $\eta^{(k+1)}(0)$ exists and equals zero. By induction, we are done.
  \end{proof}
\end{homeworkProblem}

\begin{homeworkProblem}
  Let $E$ be a measurable subset of $\R^n$ of positive measure. Show that $E-E$ contains an open set $U$ with $0\in U$.

  \begin{proof}
    Suppose first that $E\sub \R^n$ has $m(E)<+\infty$. Let $U = \{x: \ind_E*\ind_{-E}(x) >0\}$. As $\ind_E*\ind_{-E}$ is continuous, $U = (\ind_E*\ind_{-E})^{-1}((0,+\infty))$ is open. Moreover,
    \[
      \ind_E*\ind_{-E}(0) = \int \ind_E(y)\ind_{-E}(-y)\dd{y} = \int\ind_E\dd{y} = m(E) > 0,
    \]
    so $0\in U$. Lastly, suppose $x\in U$. Then $0 < \ind_E*\ind_{-E}(x) = \int \ind_E(y)\ind_{-E}(x-y)\dd{y}$, whence by positivity of the integrand there exists some $y\in\R^n$ such that $\ind_E(y)\ind_{-E}(x-y) \neq 0$. But then $y\in E$ and $x-y\in -E$, so $x = y + (x-y)\in E-E$. Thus $U\sub E-E$.\\

    Now suppose $m(E) = +\infty$. By $\sigma$-finiteness, there exists some $F\sub E$ measurable such that $0<m(F)<+\infty$. By the previous case, we may find some open $U\sub F-F$ with $0\in U$, whence $U\sub F-F\sub E-E$, as desired. 
  \end{proof}
\end{homeworkProblem}

\end{document}

\documentclass[12pt,
psamsfonts]{amsart}

%-------Packages---------
\usepackage{amssymb,amsfonts,amsmath}
\usepackage[all,arc]{xy}
\usepackage{enumerate}
\usepackage{mathrsfs}
\usepackage{fullpage}
\usepackage{xspace}
\usepackage[margin=1.0in]{geometry}
\usepackage{tcolorbox}
\usepackage{tikz-cd}
\usepackage{color}
\usepackage{aliascnt}
\usepackage[foot]{amsaddr}
\usepackage{hyperref}


%--------Theorem Environments--------
%theoremstyle{plain} --- default
\newtheorem{thm}{Theorem}[section]

%----Theorem
\newaliascnt{theo}{thm}
\newtheorem{theo}[theo]{Theorem}
\aliascntresetthe{theo}
\newcommand{\theoautorefname}{Theorem}
%----Corollary
\newaliascnt{cor}{thm}
\newtheorem{cor}[cor]{Corollary}
\aliascntresetthe{cor}
\newcommand{\corautorefname}{Corollary}
%----Proposition
\newaliascnt{prop}{thm}
\newtheorem{prop}[prop]{Proposition}
\aliascntresetthe{prop}
\newcommand{\propautorefname}{Proposition}
%----Lemma
\newaliascnt{lem}{thm}
\newtheorem{lem}[lem]{Lemma}
\aliascntresetthe{lem}
\newcommand{\lemautorefname}{Lemma}
%----Conjecture
\newaliascnt{conj}{thm}
\newtheorem{conj}[conj]{Conjecture}
\aliascntresetthe{conj}
\newcommand{\conjautorefname}{Conjecture}
%----Question
\newaliascnt{que}{thm}
\newtheorem{que}[que]{Question}
\aliascntresetthe{que}
\newcommand{\queautorefname}{Question}
%----Assumption
\newaliascnt{ass}{thm}
\newtheorem{ass}[ass]{Assumption}
\aliascntresetthe{ass}
\newcommand{\assautorefname}{Assumption}
%----Definition
\newaliascnt{defn}{thm}
\newtheorem{defn}[defn]{Definition}
\aliascntresetthe{defn}
\newcommand{\defnautorefname}{Definition}




%Style
\theoremstyle{remark}
%----Remark
\newaliascnt{rem}{thm}
\newtheorem{rem}[rem]{Remark}
\aliascntresetthe{rem}
\newcommand{\remautorefname}{Remark}

\newtheorem*{ack}{Acknowledgements}




\newtheorem{Proof}{Proof}

\theoremstyle{definition}
%\newtheorem{defn}[thm]{Definition}
\newtheorem{defns}[thm]{Definitions}
\newtheorem{con}[thm]{Construction}
\newtheorem{exmp}[thm]{Example}
\newtheorem{exmps}[thm]{Examples}
\newtheorem{notn}[thm]{Notation}
\newtheorem{notns}[thm]{Notations}
\newtheorem{addm}[thm]{Addendum}
\newtheorem{exer}[thm]{Exercise}
\newtheorem{conv}[thm]{Convention}

\newtheorem{case}[thm]{Case}


\newtheorem{rems}[thm]{Remarks}
\newtheorem{warn}[thm]{Warning}
%\newtheorem{sch}[thm]{Scholium}
\newtheorem{notation}[thm]{Notation}
\newtheorem{ex}[thm]{Examples}
\newtheorem{note}[thm]{Note}



\newcommand{\N}{\mathbb{N}\xspace}
\newcommand{\I}{\mathbb{I}\xspace}
\newcommand{\R}{\mathbb{R}\xspace}
\newcommand{\Z}{\mathbb{Z}\xspace}
\newcommand{\Q}{\mathbb{Q}\xspace}
\newcommand{\C}{\mathbb{C}\xspace}
\newcommand{\G}{\mathbb{G}\xspace}
\DeclareMathOperator{\Spec}{Spec}
\DeclareMathOperator{\res}{res}
\DeclareMathOperator{\Tr}{Tr}
\DeclareMathOperator{\ord}{ord}
\DeclareMathOperator{\Sym}{Sym}
\DeclareMathOperator{\dv}{div}
\DeclareMathOperator{\alb}{alb}
\DeclareMathOperator{\img}{Im}
\DeclareMathOperator{\et}{et}
\DeclareMathOperator{\ck}{coker}
\DeclareMathOperator{\Reg}{Reg}
\DeclareMathOperator{\Cor}{Cor}
\DeclareMathOperator{\Ac}{at}
\DeclareMathOperator{\supp}{supp}
\DeclareMathOperator{\Hom}{Hom}
\DeclareMathOperator{\Pic}{Pic}
\DeclareMathOperator{\Gal}{Gal}
\DeclareMathOperator{\fc}{frac}
\DeclareMathOperator{\Ann}{Ann}
\DeclareMathOperator{\Mod}{Mod}
\DeclareMathOperator{\Cone}{Cone}
\DeclareMathOperator{\FI}{FI}
\DeclareMathOperator{\End}{End}
\DeclareMathOperator{\rk}{rk}
\DeclareMathOperator{\Ext}{Ext}
\DeclareMathOperator{\ab}{ab}
\DeclareMathOperator{\Jac}{Jac}
\DeclareMathOperator{\coker}{coker}
\DeclareMathOperator{\fr}{frac}
\makeatletter
\let\c@equation\c@theo
\makeatother
\numberwithin{equation}{section}

\bibliographystyle{plain}
%\newcommand{\textlatin }




%--------Meta Data: Fill in your info------
\title{Math 7752 - Homework 4\\
Due Friday 02/18/22}

\begin{document}

\maketitle

\begin{enumerate}
 
% Give an example of a commutative domain $R$, and a finitely generated free $R$-module $M$ which has a submodule $N$ which is not free. 
%\end{enumerate}
%\medskip
\item Let $V$ and $W$ be finite dimensional vector spaces over a field $F$. Let $\{v_1,\ldots,v_n\}$, $\{w_1,\ldots,w_m\}$ be bases of $V,W$ respectively. Consider the $F$-linear transformation $\varphi:V\otimes_F W\rightarrow M_{n\times m}(F)$ defined by $\varphi(v_i\otimes w_j)=e_{ij}$, where $e_{ij}$ is the matrix with $1$ at the $(i,j)$-entry and $0$ elsewhere. 
\begin{enumerate}
\item Verify that such a linear transformation exists and is in fact an isomorphism of $F$-vector spaces.
\item Prove that for every $A\in M_{n\times m}(F)$ the following statements are equivalent:
\begin{enumerate}
[(i)]\item There exists some $v\in V, w\in W$ such that $A=\varphi(v\otimes w)$ ($v,w$ need not be basis elements).
\item $\rk(A)\leq 1$. 
\end{enumerate}
\end{enumerate} 
\medskip 
\item  Let $R=\oplus_{n=0}^\infty R_n$ be a graded ring. Recall that an element $r\in R$ is called \textit{homogeneous} if $r\in R_n$, for some $n\geq 0$. Notice that every $r\in R$ can be written uniquely as $\displaystyle r=\sum_{n=0}^\infty r_n$, where $r_n\in R_n$ and all but finitely many $r_n$'s are equal to zero. The elements $\{r_n\}$ are called \textit{the homogeneous components} of $r$. 
\begin{enumerate}
\item Let $I$ be an ideal of $R$. Prove that the following statements are equivalent:
\begin{enumerate}
[(i)]\item $I$ is a graded ideal, i.e. $\displaystyle I=\oplus_{n=0}^\infty(I\cap R_n)$.
\item For each $r\in I$, all homogeneous components of $r$ lie also in $I$.
\end{enumerate}
\item Let $I$ be an ideal of $R$ generated by homogeneous elements. Prove that $I$ is graded. 
\end{enumerate}
\medskip 
\item
\begin{enumerate}
\item  Let $R$ be a PID. Prove that $R$ is Noetherian.
\item Let $R$ be a commutative ring and $M$ be an $R$-module.
Recall that $M$ is called \textit{Noetherian} if every ascending chain $M_1\subset M_2\subset \cdots M_n\subset\cdots$ of submodules of $M$ eventually stabilizes. 
\begin{enumerate}[(i)]\item Let $N$ be a submodule of $M$. Prove that the following are equivalent: 
 \begin{enumerate}
 [(i)]\item $M$ is Noetherian.
 \item $N$ and $M/N$ are both Noetherian. 
 \end{enumerate}
 \item Let $R$ be a commutative Noetherian ring. Use (a) to prove that $R^n$ is Noetherian, for every $n\geq 1$. 
 \item Prove that if $R$ is Noetherian, then every submodule of a finitely generated $R$-module is finitely generated. 
\end{enumerate}
\end{enumerate} 
\medskip 
\item Let $A$ be a ring (with $1$) and $B$ be a subring of $A$. The ring $B$ is called \textit{a retract} of $A$ if there exists a surjective ring homomorphism, $\varphi:A\rightarrow B$ such that $\varphi|_{B}=1_B$. 

Let $M$ and $N$ be $R$-modules. Prove that the tensor algebra $T(M)$ is (naturally isomorphic to) a subalgebra of $T(M\oplus N)$ and this subalgebra is a retract. Prove that the same is true for symmetric algebras. \\

\item Finish the proof we started in class. Namely: Let $V$ be a  $F$-vector space of dimension $n$, where $F$ is a field. Let $\varphi:V\rightarrow V$ be a $F$-linear transformation. Consider the linear transformation $\Phi_{ext,n}:\wedge^n(V)\rightarrow\wedge^n(V)$ induced by $\varphi$. Prove that $\Phi_{ext,n}$ is given by scalar multiplication by $\det(\varphi)$. 

%\item Let $M=\Z^3$ and $N$ be the $\Z$-submodule of $M$ generated by $(1,2,3)$, $(5,4,6)$ and $(7,8,9)$. 
%\begin{enumerate}
%\item 
%\end{enumerate}

\end{enumerate}





\end{document}
\documentclass[12pt,letterpaper]{article}

%--------Packages--------
\usepackage{amsmath, amsthm, amssymb}
\usepackage{xspace}
\usepackage{graphicx}
\usepackage{hhline}
\usepackage{amssymb}
\usepackage{array}
\usepackage{braket}
\usepackage{multicol}
\usepackage{mathtools}
\usepackage{enumerate}
\usepackage{delarray}
\usepackage{mathtools}
\usepackage{fullpage}
\usepackage{faktor} % For quotients
\usepackage{mathrsfs}

% \usepackage{quiver}
\usepackage[linguistics]{forest}




%--------Page Setup--------

\pagestyle{empty}%

\setlength{\hoffset}{-1.54cm}
\setlength{\voffset}{-1.54cm}

\setlength{\topmargin}{0pt}
\setlength{\headsep}{0pt}
\setlength{\headheight}{0pt}

\setlength{\oddsidemargin}{0pt}

\setlength{\textwidth}{195mm}
\setlength{\textheight}{250mm}


%--------Macros--------

\newcommand{\sub}{\subseteq}
\newcommand{\lcm}{\text{lcm}}
\newcommand{\mc}[1]{\mathcal{#1}}
\newcommand{\mf}[1]{\mathfrak{#1}}
\newcommand{\ms}[1]{\mathscr{#1}}
\newcommand{\sO}{\mathcal{O}}
\newcommand{\cyclic}[1]{\langle#1\rangle}
\newcommand{\units}[1]{#1 ^{\times}}
\newcommand{\la}{\langle}
\newcommand{\ra}{\rangle}
%----Switch phi and varphi
\let\temp\phi
\let\phi\varphi
\let\varphi\temp

\newcommand{\C}{\mathbb{C}}
\newcommand{\F}{\mathbb{F}}
\newcommand{\N}{\mathbb{N}\xspace}
\newcommand{\I}{\mathbb{I}\xspace}
\newcommand{\R}{\mathbb{R}\xspace}
\newcommand{\Z}{\mathbb{Z}\xspace}
\newcommand{\Q}{\mathbb{Q}\xspace}
\newcommand{\G}{\mathbb{G}\xspace}
\DeclareMathOperator{\Spec}{Spec}
\DeclareMathOperator{\res}{res}
\DeclareMathOperator{\Tr}{Tr}
\DeclareMathOperator{\ord}{ord}
\DeclareMathOperator{\Sym}{Sym}
\DeclareMathOperator{\dv}{div}
\DeclareMathOperator{\alb}{alb}
\DeclareMathOperator{\img}{Im}
\DeclareMathOperator{\et}{et}
\DeclareMathOperator{\ck}{coker}
\DeclareMathOperator{\Reg}{Reg}
\DeclareMathOperator{\Cor}{Cor}
\DeclareMathOperator{\Ac}{at}
\DeclareMathOperator{\supp}{supp}
\DeclareMathOperator{\Hom}{Hom}
\DeclareMathOperator{\Pic}{Pic}
\DeclareMathOperator{\Gal}{Gal}
\DeclareMathOperator{\fc}{frac}
\DeclareMathOperator{\Ann}{Ann}
\DeclareMathOperator{\Mod}{Mod}
\DeclareMathOperator{\Cone}{Cone}
\DeclareMathOperator{\FI}{FI}
\DeclareMathOperator{\End}{End}
\DeclareMathOperator{\Alb}{Alb}
\DeclareMathOperator{\Ext}{Ext}
\DeclareMathOperator{\ab}{ab}
\DeclareMathOperator{\Jac}{Jac}
\DeclareMathOperator{\coker}{coker}
\DeclareMathOperator{\fr}{frac}
\DeclareMathOperator{\Int}{Int}


%----Analysis
\newcommand{\dd}[2][]{\frac{\partial^{#1}}{\partial {#2}^{#1}}}
\newcommand{\summ}{\sum\limits}
\newcommand{\norm}[1]{\left \vert \left \vert #1 \right \vert \right \vert}
\newcommand{\thicc}{\bigg}
\newcommand{\eps}{\varepsilon}
\newcommand*\cls[1]{\overline{#1}}


%--------Theorem environments--------
\newtheorem{definition}{Definition}[]
\newtheorem{lemma}{Lemma}[]
\newtheorem{corollary}{Corollary}[]
\newtheorem{theorem}{Theorem}[]
\theoremstyle{remark}
\newtheorem*{claim}{Claim}


\newenvironment{solution}
{\begin{proof}[Solution]}
{\end{proof}}


\makeatletter
\newcommand{\thickhline}{%
    \noalign {\ifnum 0=`}\fi \hrule height 1pt
    \futurelet \reserved@a \@xhline
}
\newcolumntype{"}{@{\hskip\tabcolsep\vrule width 1pt\hskip\tabcolsep}}
\makeatother

% --------Problem environment--------
\setlength\parindent{0pt}
\setcounter{secnumdepth}{0}
\newcounter{partCounter}
\newcounter{homeworkProblemCounter}
\setcounter{homeworkProblemCounter}{1}


\newenvironment{homeworkProblem}[1][-1]{
    \ifnum#1>0
        \setcounter{homeworkProblemCounter}{#1}
    \fi
    \section{Problem \arabic{homeworkProblemCounter}}
    \setcounter{partCounter}{1}
    \stepcounter{homeworkProblemCounter}
}


%--------Metadata--------
\title{MATH 7310 Homework 2}
\author{James Harbour}

\begin{document}
\maketitle

\begin{homeworkProblem}[1]
  Let $\mu$ be a finitely additive measure. \\

  \textbf{(a)} Prove that $\mu$ is a measure if and only if it is continuous from below as in Theorem 1.8c.

  \begin{proof}
    Theorem 1.8c shows the forward direction so it suffices to show the reverse direction. Suppose that $\mu$ is continuous from below. Let $(E_j)_{j=1}^{\infty}$ be a sequence of disjoint elements in the sigma algebra $\mc{M}$ corresponding to $\mu$. Define a new sequence $(F_n)_{n=1}^{\infty}$ in $\mc{M}$ by $F_n = \bigsqcup_{j=1}^{n}E_j$. Then $\bigsqcup_{n=1}^{\infty}E_n = \bigcup_{n=1}^{\infty} F_n$. As $(F_n)_{n=1}^{\infty}$ is an increasing sequence in $\mc{M}$, we have that
    \[
      \mu\left(\bigsqcup_{n=1}^{\infty}E_n\right) = \mu\left(\bigcup_{n=1}^{\infty}F_n\right) = \lim_{n\to\infty}\mu(F_n) \lim_{n\to\infty}\sum_{j=1}^{n}\mu(E_j) = \sum_{j=1}^{\infty} \mu(E_j),
    \]
    so $\mu$ is a measure.
  \end{proof}

  \textbf{(b)} If $\mu(X)<\infty$, prove that $\mu$ is a measure if and only if it is continuous from above as in Theorem 1.8d.

  \begin{proof}
    Theorem 1.8d shows the forward direction so it suffices to show the reverse direction. Suppose that $\mu$ is continuous from above. Let $(E_j)_{j=1}^{\infty}$ be a sequence of disjoint elements in $\mc{M}$. Define a new sequence $(F_n)_{n=1}^{\infty}$ in $\mc{M}$ by $F_n = \bigsqcup_{j=1}^{n}E_j$. Observe that $F_1^c\supset F_2^c \subset F_3^c \supset \cdots$ is a decreasing sequence in $\mc{M}$ with $\mu(F_1^c) = \mu(X)-\mu(F_1) < +\infty$. Hence, by continuity from above,
    \begin{align*}
      \mu\left(\bigsqcup_{j=1}^{\infty}E_j\right)&=\mu\left(\bigcup_{n=1}^{\infty}F_n\right) = \mu\left(X\setminus\bigcap_{n=1}^{\infty}F_n^c\right) = \mu(X)-\mu\left(\bigcap_{n=1}^{\infty}F_n^c\right) = \mu(X)-\lim_{n\to\infty}(F_n^c) \\
      &= \mu(X)-\lim_{n\to\infty}\mu\left(X\setminus \bigsqcup_{j=1}^{n}E_j\right) = \mu(X)-\lim_{n\to\infty}\mu(X)-\mu\left(X\setminus \bigsqcup_{j=1}^{n}E_j\right) = \lim_{n\to\infty}\sum_{j=1}^{n}\mu(E_j) = \sum_{j=1}^{\infty}\mu(E_j),
    \end{align*}
    so $\mu$ is a measure.
  \end{proof}
\end{homeworkProblem}

\begin{homeworkProblem}[2]
  Let $(X, \mc{M}, \mu)$ be a finite measure space. \\

  \textbf{(a)} If $E,F\in \mc{M}$ and $\mu(E\Delta F) = 0$, then $\mu(E)=\mu(F)$.

  \begin{proof}
    Observe that
    \[
      0 = \mu(E\Delta F) = \mu((E\setminus F)\sqcup(F\setminus E)) = \mu(E\setminus F) + \mu(F\setminus E).
    \]
    As $\mu(E\setminus F),\mu(F\setminus E)\geq 0$, it follows that $\mu(E\setminus F),\mu(F\setminus E) = 0$. Then as $E = (E\setminus F)\sqcup (E\cap F)$ and $F = (F\setminus E) \sqcup (F\cap E)$, $\mu(E) = \mu(F)$.
  \end{proof}

  \textbf{(b)} Say that $E\sim F$ if $\mu(E\Delta F) = 0$; show that $\sim$ is an equivalence relation on $\mc{M}$.

  \begin{proof}\ \\
    \emph{(Reflexivity):} Note that $E\Delta E = E\setminus E = \emptyset \implies \mu(E\Delta E) = 0$, so $E\sim E$.

    \emph{(Symmetry):} Note that $E\Delta F = (E\setminus F) \sqcup (F\setminus E) = F\Delta E$, so $E\sim F\implies F\sim E$.

    \emph{(Transitivity):} Suppose that $E\sim F$ and $F\sim G$. Observe that
    \begin{align*}
      E\setminus G &= ((E\setminus F)\sqcup (E\cap F))\setminus G = ((E\setminus F)\setminus G) \cup ((E\cap F)\setminus G) \sub (E\setminus F) \cup (F\setminus G) \\
      G\setminus E &= ((G\setminus F)\sqcup (G\cap F))\setminus E = ((G\setminus F)\setminus E) \cup ((G\cap F)\setminus E) \sub (G\setminus F) \cup (F\setminus E) \\
    \end{align*}
    so by monotonicity and subadditivity,
    \[
      \mu(E\Delta G) \leq \mu((E\setminus F) \cup (F\setminus G)) + \mu((G\setminus F) \cup (F\setminus E)) \leq \mu(E\setminus F)+\mu(F\setminus E) + \mu(F\setminus G) + \mu(G\setminus F) = \mu(E\Delta F) + \mu(F\Delta G) = 0,
    \]
    hence $E\sim G$.
  \end{proof}

  \textbf{(c)} For $E,F\in\mc{M}$, define $\rho(E,F)=\mu(E\Delta F)$. Then $\rho(E,G)\leq \rho(E,F)+\rho(F,G)$, and hence $\rho$ defines a metric on the space $\mc{M}/\sim$.

  \begin{proof}
    Note that the inequality used in the proof of transitivity above held regardless of the assumptions that the symmetric differences were zero, whence
    \[
      \rho(E,G) = \mu(E\Delta G) \leq \mu(E\Delta F) + \mu(F\Delta G) = \rho(E,F) + \rho(F,G).
    \]
  \end{proof}
\end{homeworkProblem}


\begin{homeworkProblem}
  Let $\mc{A}$ be the collection of finite unions of sets of the form $(a,b]\cap \Q)$ where $-\infty\leq a\leq b\leq +\infty$. \\

  \textbf{(i)} Show that $\mc{A}$ is an algebra on $\Q$. (Use Proposition 1.7.)

  \begin{proof}
    Let $\mc{E}$ be the collection of sets of the form $(a,b]\cap \Q$ with $-\infty\leq a<b\leq +\infty$. By Proposition 1.7, it suffices to show that $\mc{E}$ is an elementary family.\\

    Note that for an $a\in \R$, $(a,a]\cap \Q = \emptyset$, so $\emptyset\in \mc{E}$. \\
    Suppose $E,F\in \mc{E}$.


  \end{proof}

  \textbf{(ii)} Show that the $\sigma$-algebra generated by $\mc{A}$ is $\mc{P}(\Q)$.

  \begin{proof}
    As $\mc{A}\sub\mc{P}(\Q)$, by minimality $\Sigma(\mc{A})\sub \mc{P}(\Q)$. Now take $q\in Q$. Observe that $(q-\frac{1}{n}, q]\cap\Q\in\mc{A}$ for all $n\in \N$, whence $\{ q\} =  \bigcap_{n=1}^{\infty}(q-\frac{1}{n}, q]\cap\Q \in \Sigma(\mc{A})$. Hence, $\Sigma(\mc{A})$ contains all finite and countable subsets of $\Q$, so countability of $\Q$ implies that $\mc{P}(\Q)\sub\Sigma(\mc{A})$.
  \end{proof}

  \textbf{(ii)} Define $\mu_0$ on $\mc{A}$ by $\mu_0(\emptyset)=0$ and $\mu_0(A)=\infty$ for $A\neq\emptyset$. Prove that $\mu_0$ is a premeasure on $\mc{A}$, and that there is more than one measure on $\mc{P}(\Q)$ whose restriction to $\mc{A}$ is $\mu_0$.

  \begin{proof}

  \end{proof}
\end{homeworkProblem}

\begin{homeworkProblem}
  Let $\mc{A}$ be an alegbra, and let $\mu:\mc{A}\to[0,+\infty]$ be a finitely additive measure.\\

  \textbf{(i)} Suppose $(A_j)_{j=1}^{\infty}$ are pairwise disjoint elements of $\mc{A}$, and that $A=\bigcup_{j=1}^{\infty}A_j\in\mc{A}$. Show that
  \[
    \mu(A)\geq\sum_{j=1}^{\infty}\mu(A_j).
  \]

  \begin{proof}
    Since $\mu$ is finitely additive, it is also finitely subadditive. Then by monotonicity, for any $n\in \N$,
    \[
      \mu(A)\geq\mu\left(\bigsqcup_{j=1}^{n}A_j\right) = \sum_{j=1}^{n}\mu(A_j).
    \]
    Hence, it follows that $\mu(A)\geq \sum_{j=1}^{\infty}\mu(A_j)$.
  \end{proof}

  \textbf{(ii)} Show that the following are equivalent:
  \begin{enumerate}
    \item $\mu$ is a premeasure,
    \item $\mu\left(\bigcup_{j=1}^{\infty}A_j\right)\leq \sum_{j=1}^{\infty}\mu(A_j)$ for any sequence $(A_j)_{j=1}^{\infty}$ with $\bigcup_{j=1}^{\infty}A_j\in\mc{A}$,
    \item for any increasing sequence $(E_j)_{j=1}^{\infty}$ in $\mc{A}$ with $\bigcup_{j=1}^{\infty}E_j\in\mc{A}$, we have
    \[
      \mu\left(\bigcup_{j}E_j\right) = \lim_{n\to\infty}\mu(E_n).
    \]
  \end{enumerate}

  \begin{proof}\ \\
    \emph{(1$\implies$2)}: Let $(A_j)_{j=1}^{\infty}$ be a sequence in $\mc{A}$ such that $\bigcup_{j=1}^{\infty}A_j\in\mc{A}$. Define a new sequence $(A_n')_{n=1}^{\infty}$ in $\mc{A}$ by $A_1'=A_1$ and $A_n' = A_n \setminus \bigcup_{j=1}^{n-1}A_j$ for $n\geq 2$. Then by disjoint countable additivity and monotonicity,
    \[
      \mu\left(\bigcup_{j=1}^{\infty}A_j\right) = \mu\left(\bigsqcup_{j=1}^{\infty}A_j'\right) = \sum_{j=1}^{\infty}\mu(A_j')\leq \sum_{j=1}^{\infty}\mu(A_j).
    \]

    \emph{($1\implies 3$)}:



    \emph{($3\implies 1$)}: Follows identically to the proof in problem 1 part (b) by assuming the sequence is such that the countable union stays in $\mc{A}$.

    \emph{($3\implies 2$)}: Suppose that  $(A_j)_{j=1}^{\infty}$ is a sequence in $\mc{A}$ with $\bigcup_{j=1}^{\infty}A_j\in\mc{A}$. Then, by finite subadditivity (which follows from finite additivity),
    \[
      \mu\left(\bigcup_{j=1}^{\infty}A_j\right) = \mu\left(\bigcup_{n=1}^{\infty}\bigcup_{j=1}^{n}A_j\right) = \lim_{n\to\infty}\mu\left(\bigcup_{j=1}^{n}A_j\right) \leq \lim_{n\to\infty}\sum_{j=1}^{n}\mu(A_j) = \sum_{j=1}^{\infty}\mu(A_j).
    \]
  \end{proof}

  \textbf{(iii)} If $\mu(X)<+\infty$, show that $\mu$ is a premeasure if and only if for every decreasing sequence $(E_n)_{n=1}^{\infty}$ of sets in $\mc{A}$ with $\bigcap_{n=1}^{\infty}E_n = \emptyset$, we have
  \[
    \lim_{n\to\infty}\mu(E_n) = 0.
  \]

  \begin{proof}\ \\
    \underline{$\implies$}: Let $(E_n)_{n=1}^{\infty}$ be a decreasing sequence of sets in $\mc{A}$ with $\bigcap_{n=1}^{\infty}E_n = \emptyset$

  \end{proof}
\end{homeworkProblem}


\begin{homeworkProblem}[5]
  A \emph{metric measure space} is a triple $(X,d,\mu)$ where $(X,d)$ is a metric space and $\mu:\mc{B}_{(X,d)} \to [0,+\infty]$ is a measure. We say that $E\sub X$ is a \emph{continuity set} if $\mu(\cls{E}\setminus \Int(E)) = 0$. For this problem, fix a metric measure space $(X,d,\mu)$. \\

  \textbf{(i)} Show that the collection of continuity sets forms an algebra of sets.

  \begin{proof}
    Suppose that $E_1,\ldots, E_n\sub X$ are continuity sets. Then $\mu(\cls{E_j}\setminus \Int(E_j)) = 0$ for $1\leq j\leq n$. As there are finitely many sets, the union of closures is equal to the closure of the union. Hence
    \[
      \cls{\bigcup_{j=1}^{n} E_j} \setminus \Int\left(\bigcup_{j=1}^{n}E_j\right) = \bigcup_{j=1}^{n} \cls{E_j} \setminus \Int\left(\bigcup_{j=1}^{n}E_j\right) \sub \bigcup_{j=1}^{n} \cls{E_j} \setminus \bigcup_{j=1}^{n}\Int(E_j) = \bigcup_{j=1}^{n} \cls{E_j} \setminus \Int(E_j),
    \]
    so by subadditivity,
    \[
      \mu\left(\cls{\bigcup_{j=1}^{n} E_j} \setminus \Int\left(\bigcup_{j=1}^{n}E_j\right)\right) = \mu\left(\bigcup_{j=1}^{n} \cls{E_j} \setminus \Int(E_j)\right) \leq \sum_{j=1}^{n}\mu(\cls{E_j}\setminus \Int(E_j)) = 0
    \]
    whence $E_1\cup\cdots\cup E_n$ is a continuity set. Now suppose that $E\sub X$ is a continuity set.
    \[
      (\cls{X\setminus E})\setminus \Int(X\setminus E) = (X\setminus \Int(E)) \setminus \Int (X\setminus E) = (X\setminus \Int(E)) \setminus (X\setminus \cls{E}) = \cls{E}\setminus \Int(E)
    \]
    so $\mu((\cls{X\setminus E})\setminus \Int(X\setminus E)) = \mu(\cls{E}\setminus \Int(E)) = 0$, whence $X\setminus E$ is also a continuity set.
  \end{proof}

  \textbf{(ii)} Show that if $x\in X$, $r>0$ and $\mu(B_r(x,d))<+\infty$, then there is an $s\in (0,r)$ so that $B_s(x,d)$ is a continuity set.

  \begin{proof}
    Suppose, for the sake of contradiction, that $\mu(\cls{B_s(x)}\setminus\Int(B_s(x)))\neq 0$ for all $s\in(0,r)$. For $n\in \N$, define a set
    \[
      A_n=\{ s\in(0,r): \frac{1}{n}\leq \mu(\cls{B_s(x)}\setminus\Int(B_s(x))) < \frac{1}{n-1}\}
    \]
    where $1/0 := \infty$ by convention. Then $(0,r) = \bigcup_{n=1}^{\infty}A_n$, so there exists an $n\in \N$ such that $A_n$ is infinite. Take a countably infinite subset $\{ s_1, s_2, \ldots\}\sub A_n$. Note that, for any fixed $t\in(0,+\infty)$, $\cls{B_t(x)}\setminus\Int(B_t(x))\sub\{y\in X:d(x,y) = t\}$, whence
    \[
      \mu\left(\bigsqcup_{j=1}^{\infty}\cls{B_{s_j}(x)}\setminus\Int(B_{s_j}(x)) \right) = \sum_{j=1}^{\infty}\mu(\cls{B_{s_j}(x)}\setminus\Int(B_{s_j}(x))) = \infty
    \]
    contradicting that $\mu(B_r(x))<\infty$.
  \end{proof}

  \textbf{(iii)} Suppose that $(X,d)$ is separable and that for every $x\in X$, there is an $r>0$ so that $\mu(B_r(x,d))<+\infty$. Show that there is a countable basis consisting of open continuity sets. (Hint: given a countable dense $D\sub X$ and $x\in D$, use the preceding part to choose a countable set $J_x\sub (0,+\infty)$ with the property that $\inf_{t\in J_x}t = 0$ and so that $B_t(x,d)$ is a continuity set for all $t\in J_x$).

  \begin{proof}
    Let $D\sub X$ be a countable dense subset of $X$. Fix $x\in D$. For $n\in \N$, appeal to part (i) to find a $t_n\in(0,r)$ such that $B_{t_n}(x,d)$ is a continuity set. Letting $J_x = \{ t_n:n\in\N\}$, it follows that $J_x$ has the property that $\inf_{t\in J_x}t = 0$ and so that $B_t(x,d)$ is a continuity set for all $t\in J_x$. Let
    \[
      \ms{J} = \{(x,t):x\in D, t\in J_x\}
    \]
    Note that $\ms{J}$ is countable. Let $\ms{B} = \{ B_t(x,d): (x,t)\in J_x\}$. We claim that $\ms{B}$ is a basis for the metric topology on $(X,d)$. As $\ms{B}$ covers $D$ and $D$ is dense in $X$, it is clear that $\ms{B}$ covers $X$.

    Suppose that $x\in X$ and $ B_t(y,d),B_{t'}(z,d)\in \ms{B}$ such that $x\in B_t(y,d) \cap B_{t'}(z,d)$. As $\inf_{t\in J_x}t = 0$, there exists a $t''\in J_x$ such that $t\leq t,t'$, whence $x\in B_{t''}(x,d)\sub B_{t}(x,d)\cap B_{t}(x,d)$.
  \end{proof}
\end{homeworkProblem}

\begin{homeworkProblem}
  Let $(X,d)$ be a metric space and $\mu,\nu$ be finite Borel measures on $X$ with $\mu(X) = \nu(X)$. Let $\mc{A}=\{ E\in\mc{B}_{(X,d)} : \mu(E)=\nu(E)\}$. \\

  \textbf{(i)} Show that if $F\sub E$ and $F,E\in\mc{A}$, then $E\setminus F\in\mc{A}$. Also show that if $(E_n)_{n=1}^{\infty}$ is an increasing sequence of elements of $\mc{A}$, then $\bigcup_{n=1}^{\infty}E_n\in\mc{A}$.

  \begin{proof}
    As $E,F\in\mc{A}$, $\mu(E)=\nu(E)$ and $\mu(F)=\nu(F)$. Then
    \[
      \mu(E\setminus F) = \mu(E)-\mu(F) = \nu(E)-\nu(F) = \nu(E\setminus F)
    \]
    so $E\setminus F\in \mc{A}$. Now suppose that $(E_n)_{n=1}^{\infty}$ is an increasing sequence of elements of $\mc{A}$. By continuity from above,
    \[
      \mu\left(\bigcup_{n=1}^{\infty}E_n\right) = \lim_{n\to\infty}\mu(E_n) = \lim_{n\to\infty}\nu(E_n) = \nu\left(\bigcup_{n=1}^{\infty}E_n\right),
    \]
    so $\bigcup_{n=1}^{\infty}E_n\in\mc{A}$.
  \end{proof}

  \textbf{(ii)} Given a nonempty $F\sub X$ closed and $x\in X$, define $d(x,F)=\inf_{y\in F}d(x,y)$. Show that $x\mapsto d(x,F)$ is continuous and $F=\{ x\in X : d(x,F) = 0\}$.

  \begin{proof}
    Suppose $x,y\in X$. For $z\in F$,
    \[
      d(x,F) \leq d(x,z) \leq d(x,y)+ d(y,z) \implies d(x,F)-d(x,y)\leq d(y,z).
    \]
    As this holds for arbitrary $z\in F$, it follows that $d(x,F) - d(x,y) \leq d(y,F)$, so $d(x,F)- d(y,F) \leq d(x,y)$. By symmetry, $d(y,F)-d(x,F) \leq d(x,y)$, so $|d(x,F)-d(y,F)|\leq d(x,y)$. Thus, the function $x\mapsto d(x,F)$ is $1-Lipschitz$ whence it is continuous. \\

    Clearly $F\sub \{ x\in X : d(x,F) = 0\}$, so it suffices to show the reverse containment. Suppose that $x\in X$ such that $d(x,F) = 0$. For all $n\in \N$, there exists an $f_n\in F$ such that $0\leq d(x,f_n)<\frac{1}{n}$. It follows that $d(x,f_n)\xrightarrow{n\to\infty}0$, so $f_n\xrightarrow{n\to\infty}x$. Thus $x$ is a limit point of $F$, so $F$ being closed implies that $x\in F$.
  \end{proof}

  \textbf{(iii)} Show that $\{ U\sub X: U\text{ is open}\}\sub\mc{A}$ if an only if $\{ F\sub X : F\text{ is closed}\}\sub\mc{A}$.

  \begin{proof}\ \\
    \underline{$\implies$}: Suppose that $\{ U\sub X: U\text{ is open}\}\sub\mc{A}$. Take $F\sub X$ such that $F$ is closed.

    \underline{$\impliedby$}:

  \end{proof}
\end{homeworkProblem}
\end{document}

%
\documentclass{article}
\usepackage{fullpage,graphicx,amsfonts,amssymb,amsmath,amsthm,titling}
\usepackage[all]{xy}

\theoremstyle{plain} %--default
\newtheorem{theorem}    {Theorem}
\newtheorem{theoremletter}{Theorem}
\renewcommand*{\thetheoremletter}{\Alph{theoremletter}}
\newtheorem{lemma}      [theorem]{Lemma}
\newtheorem{corollary}  [theorem]{Corollary}
\newtheorem{proposition}[theorem]{Proposition}
\newtheorem{algorithm}  [theorem]{Algorithm}
\newtheorem{criterion}  [theorem]{Criterion}
\newtheorem{conjecture} [theorem]{Conjecture}
\newtheorem{question}   [theorem]{Question}

\theoremstyle{definition}
\newtheorem{definition} [theorem]{Definition}
\newtheorem{condition}  [theorem]{Condition}
\newtheorem{example}    [theorem]{Example}
\newtheorem{problem}    [theorem]{Problem}
\newtheorem{exercise}   [theorem]{Exercise}
\newtheorem{solution}   [theorem]{Solution}

\theoremstyle{remark}
\newtheorem{remark}              {Remark}
\newtheorem{note}                {Note}
\newtheorem{claim}               {Claim}
\newtheorem{summary}             {Summary}
\newtheorem{case}                {Case}
\newtheorem{acknowledgment}      {Acknowledgments}
\newtheorem{conclusion}          {Conclusion}
\newtheorem{notation}            {Notation}

\numberwithin{equation}{section}

\newenvironment{spmatrix}{\left(\begin{smallmatrix}}{\end{smallmatrix}\right)}

\newcommand{\sect}{%
  \removelastskip%
  \vskip.5\baselineskip%
  \refstepcounter{subsection}%
  \noindent%
  {\bf (\thesubsection)}%
  \hspace{3pt}}

%% --OPERDEFNS--
\renewcommand{\Re}{\operatorname{Re}}
\renewcommand{\Im}{\operatorname{Im}}
% ----------------------------------------------------

\title{Undergraduate Workshop Notre Dame 2022\\Exercise sets}
\author{Andrei Jorza\and Evan O'Dorney\and Claudiu Raicu}
\date{}

\renewcommand{\k}{{\mathbf k}}
\newcommand{\defi}[1]{{\bf\upshape\sffamily #1}}
\newcommand{\mf}[1]{\mathfrak{#1}}
\newcommand{\bb}[1]{\mathbb{#1}}


\begin{document}

\maketitle

\section{Exercises for Monday}
These problems are designed to give you hands-on familiarity with the material in the lectures. Some of them are easier with a computer. Don't worry if you don't have the background to solve a few of the problems; just move on.
\subsection{Algebraic Curves}
\begin{enumerate}
 \item\label{ex-varieties:1} \begin{itemize}
 \item Show that the set
 \[ C = \{ (t^2,t^3) : t\in \k \}\]
 is the same as the affine algebraic set
 \[ V(y^2-x^3).\]
 \item Show then that
 \[ I(C) = \langle y^2-x^3 \rangle \subset \k[x,y],\]
 and conclude that $C$ is an affine algebraic variety.
 \item Show that $\dim(C)=1$ ($C$ is called the \defi{cuspidal cubic curve}).
 \end{itemize}
 \item\label{ex-varieties:1} If $X,Y$ are affine algebraic sets in $\k^n$ and $I,J$ are ideals in $A$, show that
 \begin{itemize}
  \item If $X\subseteq Y$ then $I(X) \supseteq I(Y)$.
  \item If $I \subseteq J$ then $V(I) \supseteq V(J)$.
  \item If we let $\sqrt{I} = \{ a\in A : a^r\in I\mbox{ for some }r>0\}$ denote the \defi{radical} of the ideal $I$ then
  \[V(I) = V(\sqrt{I}).\]
  \item Using the fact that \[\sqrt{I} = \bigcap_{\substack{\mf{P}\supseteq I \\ \mf{P}\mbox{ prime ideal}}} \mf{P}\] and Theorem 1.1 (see notes) show that for every ideal $I\subset A$ we have
  \[ I(V(I)) = \sqrt{I}.\]
 \end{itemize}
 \item\label{ex-varieties:2} Show that for any two ideals $I,J\subseteq A$ we have
 \[ V(I\cdot J) = V(I \cap J) = V(I) \cup V(J).\]
 Prove that if $X$ is an algebraic set then we have an equivalence
 \[X \mbox{ is irreducible }\Longleftrightarrow I(X)\mbox{ is a prime ideal}.\]

\item\label{ex-varieties:3} State and prove the analogous statements in Exercises~\ref{ex-varieties:1} and~\ref{ex-varieties:2} for homogeneous ideals and projective algebraic sets/varieties.

\item Verify that if $X= \{ [a_0:\cdots:a_n] \}$ consists of a single point in $\bb{P}^n$ then the homogeneous ideal of $X$ is
\[ I(X) = \langle a_i x_j - a_j x_i : 0\leq i<j \leq n \rangle \subset S = \k[x_0,\cdots,x_n],\]
and that the homogeneous coordinate ring of $X$ is isomorphic to a polynomial ring in one variable.

By looking at specific points in $\bb{P}^n$, $n=1,2,3,\cdots$, convince yourself that in general $I(X)$ can be generated by only $n$ linear homogeneous polynomials (so the $n+1\choose 2$ polynomials that I wrote down give in general a reduntant set of generators for $I(X)$).

\item\label{ex-varieties:6} Verify that you can decompose
\[ \bb{P}^n = \bb{A}^n \sqcup \bb{P}^{n-1},\]
where $\bb{A}^n \cong \{ [1:a_1:\cdots:a_n]\}$ and $\bb{P}^{n-1}\cong \{ [0:a_1:\cdots:a_n]\}$. We call $\bb{P}^{n-1}$ the \defi{hyperplane at infinity}, parametrizing the directions of the lines through the origin in $\bb{A}^n$.

In fact, if we let
\[ U_i = \{ [a_0:\cdots:a_{i-1}:1:a_{i+1}:\cdots:a_n] \} \subset \bb{P}^n,\quad i=0,\cdots,n,\]
then we can naturally identify $U_i$ with $\bb{A}^n$, and its complement $H_i = \bb{P}^n \setminus U_i$ with $\bb{P}^{n-1}$.

\item Consider the subset 
\[ X = \{ [s^3: st^2 : t^3] : s,t\in\k,\text{ not both $s,t$ are }0\} \subset \bb{P}^2 = \{ [w:x:y]\}.\]
Show that 
\[ I(X) = \langle y^2 w - x^3 \rangle \subset \k[w,x,y],\]
and conclude that $X$ is a projective variety. Using the notation in Exercise~\ref{ex-varieties:6}, show that $X \cap U_0$ is naturally identified with the cuspidal curve $C$ in $\bb{A}^2=U_0$, and that
\[ X = C \cup \{[0:0:1]\},\]
where you can think of $[0:0:1]$ as the \defi{point at infinity} on the cuspidal curve. Note also that $y^2w-x^3$ is the \defi{homogenization} of the equation $y^2-x^3$ of $C$, with respect to the variable $w$.

\end{enumerate}

\subsection{Elliptic Curves}
\begin{enumerate}
\item How many constants are needed in the general equation of a plane curve of degree $n$?
(Check that your formula gives the right answer, $10$, for the case $n = 3$.)
\item Let $f (x) = x^3 + Ax + B$ where $A$ and $B$ are real numbers. Let $\Delta = -(4A^3 + 27B^2)$. Prove
that
\begin{enumerate}
  \item $f (x)$ has a multiple root if and only if $\Delta = 0$.
  \item $f (x)$ has three distinct real roots if and only if $\Delta > 0$.
  \item $f (x)$ has one real root and two non-real roots if and only if $\Delta < 0$.
\end{enumerate}
(Hint: $f (x)$ factors completely into linear factors over the complex numbers. Since
there is no $x^2$ term in $f (x)$, the sum of the zeros of $f (x)$ is 0, and the factorization has
the form
\[
  f (x) = (x - r)(x - s)(x + r + s)
\]
for some complex numbers $r$ and $s$. Calculate $\Delta$ in terms of $r$ and $s$ and factor it.)

The number $\Delta$ is called the \emph{discriminant}; it plays a role analogous to that of $b^2 - 4ac$
for quadratic polynomials.
\item It turns out that the real points on the elliptic curve $y^2 = x^3 + Ax + B$ form two
connected components if $\Delta > 0$ and only one connected component if $\Delta < 0$. (Loosely
speaking, a connected component is a piece you can draw without lifting your pencil
from the paper.) Can you explain this, using the previous
problem?
\item \begin{enumerate}
  \item Parametrize the rational points on the hyperbola $x^2 - 2y^2 = 1$.
  \item Find some \emph{integer} points on this hyperbola. (In general, an equation of the form $x^2 - ky^2 = \pm 1$, where we seek integer solutions $(x,y)$, is called a \emph{Pell equation.})
  \item Prove that there are infinitely many integer points on this hyperbola. (Hint: Show that if $(x_1, y_1)$ and $(x_2, y_2)$ are solutions, so is $(x_1 x_2 + 2 y_1 y_2, x_1 y_2 + y_1 x_2)$. Where does this formula come from?)
\end{enumerate}

\end{enumerate}
\subsection{Modular Forms}
The {\bf Bernoulli numbers} $B_n$ are defined by
the Taylor expansion $\displaystyle \dfrac{x}{e^x-1}=\sum B_n\dfrac{x^n}{n!}=1-\frac{1}{2}x+\frac{1}{12}x^2-\frac{1}{720}x^4+\cdots$.
\begin{enumerate}
\item Show that $B_n=0$ for all {\bf odd} $n>1$, and that
$B_n\in \mathbb{Q}$ for all $n$.
\item Show that \[z \cot(z) = 1+\sum_{n=1}^\infty (-4)^n
B_{2n}\frac{z^{2n}}{(2n)!}\] [Hint: Plug in $x=2i z$.]
\item A beautiful result from complex analysis implies that
\begin{equation}\label{cot}\pi\cot(\pi z)=\frac{1}{z}+\sum_{n\geq
  1}\left(\frac{1}{z+n}+\frac{1}{z-n}\right)=\sum_{n\in
  \mathbb{Z}}\frac{1}{z+n}.
\end{equation}
(Same zeros and same poles!)
Show that
\[z\cot(z) = 1+2\sum_{n=1}^\infty\frac{z^2}{z^2-n^2\pi^2}\]
and expand into geometric series each $\dfrac{z^2}{z^2-n^2\pi^2}$ to show that
\[\zeta(2n) =
\frac{(-1)^{n+1}B_{2n}(2\pi)^{2n}}{2(2n)!}.\]
\item Let $q=e^{2\pi i z}$.
\begin{enumerate}
\item Show directly from the definition of $\cot z$ that
\[\pi\cot(\pi z) = \pi i \frac{q-1}{q+1}=\pi i -2\pi
i\sum_{n=0}^\infty q^n\]
\item Differentiate equation \eqref{cot} $k-1$ times to show that
\[\sum_{n\in \mathbb{Z}}\frac{1}{(z+n)^k} = \frac{(-2\pi i)^k}{(k-1)!}\sum_{n=1}^\infty n^{k-1}q^n.\]
\end{enumerate}

% \item Let $k$ be an integer. Show that if $f(z+1)=f(z)$ and
% $f(-1/z) = z^k f(z)$ then $f(g\cdot z) = (cz+d)^k f(z)$ for all
% $g\in \SL(2,\mathbb{Z})$. Here $g=\begin{pmatrix} a&b\\c&d\end{pmatrix}$. [Hint: The group $\SL(2,\mathbb{Z})$ is generated by $S=\begin{pmatrix} 0&-1\\1&0\end{pmatrix}$ and $T=\begin{pmatrix} 1&1\\0&1\end{pmatrix}$.]

% \item Let $S$ be the set of pairs of integers $(m,n)$ not both zero. Suppose $\begin{pmatrix} a&b\\c&d\end{pmatrix}\in
% \SL(2,\mathbb{Z})$. Show that 
% \[\{(ma+nc, mb+nd)\mid (m,n)\in S\}=S\]

\end{enumerate}
\newpage
\maketitle
\section{Exercises for Tuesday}
\subsection{Algebraic Curves}
\subsection{Elliptic Curves}
\subsection{Modular Forms}
\newpage
\maketitle

\section{Exercises for Wednesday}
\subsection{Algebraic Curves}
\subsection{Elliptic Curves}
\subsection{Modular Forms}
\newpage
\maketitle

\section{Exercises for Thursday}
\subsection{Algebraic Curves}
\subsection{Elliptic Curves}
\subsection{Modular Forms}
\newpage
\maketitle

\section{Exercises for Friday}
\subsection{Algebraic Curves}
\subsection{Elliptic Curves}
\subsection{Modular Forms}

\end{document}


%%% Local Variables:
%%% mode: latex
%%% TeX-master: t
%%% End:

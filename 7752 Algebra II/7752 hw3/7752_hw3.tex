\documentclass[12pt,letterpaper]{article}

%--------Packages--------
\usepackage{amsmath, amsthm, amssymb}
\usepackage{xspace}
\usepackage{graphicx}
\usepackage{amssymb}
\usepackage{array}
\usepackage{braket}
\usepackage{multicol}
\usepackage{mathtools}
\usepackage{enumerate}
\usepackage{delarray}
\usepackage{mathtools}
\usepackage{fullpage}
\usepackage{faktor} % For quotients
\usepackage{mathrsfs}

% \usepackage{quiver}
\usepackage[linguistics]{forest}




%--------Page Setup--------

\pagestyle{empty}%

\setlength{\hoffset}{-1.54cm}
\setlength{\voffset}{-1.54cm}

\setlength{\topmargin}{0pt}
\setlength{\headsep}{0pt}
\setlength{\headheight}{0pt}

\setlength{\oddsidemargin}{0pt}

\setlength{\textwidth}{195mm}
\setlength{\textheight}{250mm}


%--------Macros--------

\newcommand{\sub}{\subseteq}
\newcommand{\lcm}{\text{lcm}}
\newcommand{\ms}[1]{\mathscr{#1}}
\newcommand{\mc}[1]{\mathcal{#1}}
\newcommand{\mf}[1]{\mathfrak{#1}}
\newcommand{\sO}{\mathcal{O}}
\newcommand{\cyclic}[1]{\langle#1\rangle}
\newcommand{\units}[1]{#1 ^{\times}}
\newcommand{\la}{\langle}
\newcommand{\ra}{\rangle}
%----Switch phi and varphi
\let\temp\phi
\let\phi\varphi
\let\varphi\temp

\newcommand{\C}{\mathbb{C}}
\newcommand{\F}{\mathbb{F}}
\newcommand{\N}{\mathbb{N}\xspace}
\newcommand{\I}{\mathbb{I}\xspace}
\newcommand{\R}{\mathbb{R}\xspace}
\newcommand{\Z}{\mathbb{Z}\xspace}
\newcommand{\Q}{\mathbb{Q}\xspace}
\newcommand{\G}{\mathbb{G}\xspace}
\DeclareMathOperator{\Spec}{Spec}
\DeclareMathOperator{\res}{res}
\DeclareMathOperator{\Tr}{Tr}
\DeclareMathOperator{\ord}{ord}
\DeclareMathOperator{\Sym}{Sym}
\DeclareMathOperator{\dv}{div}
\DeclareMathOperator{\alb}{alb}
\DeclareMathOperator{\img}{Im}
\DeclareMathOperator{\et}{et}
\DeclareMathOperator{\ck}{coker}
\DeclareMathOperator{\Reg}{Reg}
\DeclareMathOperator{\Cor}{Cor}
\DeclareMathOperator{\Ac}{at}
\DeclareMathOperator{\supp}{supp}
\DeclareMathOperator{\Hom}{Hom}
\DeclareMathOperator{\Pic}{Pic}
\DeclareMathOperator{\Gal}{Gal}
\DeclareMathOperator{\fc}{frac}
\DeclareMathOperator{\Ann}{Ann}
\DeclareMathOperator{\Mod}{Mod}
\DeclareMathOperator{\Cone}{Cone}
\DeclareMathOperator{\FI}{FI}
\DeclareMathOperator{\End}{End}
\DeclareMathOperator{\Alb}{Alb}
\DeclareMathOperator{\Ext}{Ext}
\DeclareMathOperator{\ab}{ab}
\DeclareMathOperator{\Jac}{Jac}
\DeclareMathOperator{\coker}{coker}
\DeclareMathOperator{\fr}{frac}
\DeclareMathOperator{\spn}{span}


%----Analysis
\newcommand{\dd}[2][]{\frac{\partial^{#1}}{\partial {#2}^{#1}}}
\newcommand{\summ}{\sum\limits}
\newcommand{\norm}[1]{\left \vert \left \vert #1 \right \vert \right \vert}
\newcommand{\thicc}{\bigg}
\newcommand{\eps}{\varepsilon}


%--------Theorem environments--------
\newtheorem{definition}{Definition}[]
\newtheorem{lemma}{Lemma}[]
\newtheorem{corollary}{Corollary}[]
\newtheorem{theorem}{Theorem}[]
\theoremstyle{remark}
\newtheorem*{claim}{Claim}


\newenvironment{solution}
{\begin{proof}[Solution]}
{\end{proof}}


\makeatletter
\newcolumntype{"}{@{\hskip\tabcolsep\vrule width 1pt\hskip\tabcolsep}}
\makeatother

% --------Problem environment--------
\setlength\parindent{0pt}
\setcounter{secnumdepth}{0}
\newcounter{partCounter}
\newcounter{homeworkProblemCounter}
\setcounter{homeworkProblemCounter}{1}


\newenvironment{homeworkProblem}[1][-1]{
    \ifnum#1>0
        \setcounter{homeworkProblemCounter}{#1}
    \fi
    \section{Problem \arabic{homeworkProblemCounter}}
    \setcounter{partCounter}{1}
    \stepcounter{homeworkProblemCounter}
}


%--------Metadata--------
\title{MATH 7752 Homework 3}
\author{James Harbour}


\begin{document}
\maketitle

\begin{homeworkProblem}
  Let $R$ be a commutative domain, and let $M$ be a free $R$-module with basis $X=\{e_1,\ldots,e_k\}$, with $k\geq 2$. Prove that the element $e_1\otimes e_2+e_2\otimes e_1$ cannot be written as simple tensor $m\otimes n$, for some $m,n\in M$.

  \begin{proof}
    Suppose, for the sake of contradiction, that there exist $m,n\in M$ such that $m\otimes n = e_1\otimes e_2+e_2\otimes e_1$. Write $m = \sum_{i=1}^{n}r_i e_i$ and $n=\sum_{j=1}^{n}s_j e_j$ for some $r_i,s_j\in R$. Then
    \[
      e_1\otimes e_2+e_2\otimes e_1 = \left(\sum_{i=1}^{n}r_i e_i\right) \otimes \left(\sum_{j=1}^{n}s_j e_j\right) = \sum_{i,j}r_i s_j e_i\otimes e_j
    \]
    Under this isomorphism $M\cong R^n$ induced by the basis $X$, we have that \[M\otimes M\cong R^n\otimes R^n \cong (R^n\otimes R)^n \cong (R\otimes R)^{n^2} \cong R^{n^2}\] as $R$-modules. By the previous homework, as $R$ is commutative, it follows that $M\otimes M$ has well defined rank given by $rank(M) = n^2$.
  \end{proof}
\end{homeworkProblem}

\begin{homeworkProblem}
   Let $R$ be a commutative ring (with $1$) and $n,m\in\N$. Prove that there is an isomorphism of \textbf{$R$-algebras} $R^n\otimes R^m\simeq R^{nm}$. (Here by $R^n$ we mean the direct sum $\underbrace{R\oplus\cdots\oplus R}_n$.)
\end{homeworkProblem}

\begin{homeworkProblem}
  \textbf{(a)} Let $V$ be a finite-dimensional $\C$-vector space. Then $V$ can be considered as a vector over $\R$ (by restriction of scalars), and it holds $\dim_{\R}V=2\dim_{\C}V$. Prove that $V\otimes_{\C}V$ is not isomorphic to $V\otimes_{\R}V$ as $\R$-vector spaces, and compute their dimensions over $\R$. \\

  \textbf{(b)} Let $R$ be an integral domain (commutative), and let $K$ be its fraction field. Prove that there is an isomorphism of $F$-modules, $F\otimes_R F\simeq F\otimes_F F\simeq F$, where the $F$-module structure on $F\otimes_R F$ is given by \textbf{extension of scalars} (i.e. tensor product of Type I).


\end{homeworkProblem}

\begin{homeworkProblem}
  The purpose of this problem is to classify all $2$-dimensional $\R$-algebras (where $\R$ are the real numbers). That means, to classify (up to algebra isomorphism) those $\R$-algebras that are $2$-dimensional $\R$ vector spaces.

  Let $A$ be a $2$-dimensional $\R$-algebra (with $1$). \\

  \textbf{(a)} Let $u\in A$ be any element that is $\R$-linearly independent from $1$. Prove that
  \begin{enumerate}
  [(i)] \item $u$ generates $A$ as an $\R$-algebra. That is, the minimal $\R$-subalgebra of $A$ containing $u$ and $1$ is $A$ itself.
  \item The element $u$ satisfies a quadratic equation $au^2+bu+c=0$, for some $a,b,c\in\R$ with $a\neq 0$. Conclude that $A$ is necessarily commutative.
  \end{enumerate}

  \begin{proof}
    Noting that the subalgebra generated by $u$ contains $\spn_\R(\{ 1,u\})$ which has dimension 2 as an $\R$-vector space, it follows that the subalgebra generated by $u$ is in fact $A$. \\

    Since the subalgebra generated by $u$ is $A$, it follows that there exist $a,b\in \R$ such that $u^2 = au + b1$, whence $u^2 -au - b = 0$. This implies the algebra $A$ is commutative as multiplication is hence defined by the relations $u\cdot 1 = u = 1\cdot u$ and $1=1\cdot 1$, which are all commutative.
  \end{proof}

  \textbf{(b)} Show that there exists some $v\in A$ which is $\R$-linearly independent from $1$ and is such that $v^2=-1$, or $v^2=1$, or $v^2=0$.

  \begin{proof}

  \end{proof}


  \textbf{(c)} Deduce from part (b) that $A$ is isomorphic as an $\R$-algebra to one of the following: $\R[x]/(x^2+1)$, or $\R[x]/(x^2-1)$, or $\R[x]/(x^2)$. \\

  \textbf{(d)} Prove that the algebras $\R[x]/(x^2+1)$, $\R[x]/(x^2-1)$, and $\R[x]/(x^2)$ are pairwise non-isomorphic. \textbf{Hint:} This can be shown with almost no computation.
\end{homeworkProblem}

\begin{homeworkProblem}
  The purpose of this problem is to prove the following theorem: Let $D$ be a finite dimensional division algebra over $\R$. Then $D$ is isomorphic to $\R,\C$ or $\mathbb{H}$ (the quaternions). One way to proceed is to use the following steps: \\

  \textbf{(a)} Let $\alpha\in D$ be an element $\R$-linearly independent from $1$. Show that $\alpha$ satisfies a quadratic irreducible polynomial $p_\alpha(x)=x^2+ax+b\in\R[x]$.

  \begin{proof}
    Since $D$ is finite-dimensional over $\R$, there exists an $n\in\N$ such that the set $\{ 1, \alpha, \ldots, \alpha^n\}$ is $\R$-linearly dependent. Hence $\alpha$ is algebraic over $\R$, so the set $I_\alpha = \{ f(x)\in\R[x]: f(\alpha) = 0\}$.

    % TODO, prove $I_\alpha$ is an ideal.
    As $I_\alpha$ is an ideal and $\R[x]$ is a PID, there exists a (without loss of generality) monic polynomial $p_\alpha(x)\in\R[x]$ such that $I_\alpha = (p_\alpha)$. As $\alpha$ is algebraic, $p_\alpha=\neq 0$. Moreover, $p_\alpha$ is nonconstant by $p_\alpha(\alpha) = 0$. Hence, $p_\alpha$ is not a unit in $\R[x]$. If $f\in I_\alpha = (p_\alpha)$ is irreducible, then in writing $f = gp_\alpha$ for some $g\in \R[x]$, irreducibility implies that $(f) = (p_\alpha) = I_\alpha$. Moreover, this implies that $\deg(f) = \deg(p_\alpha)$, so $p_\alpha$ being monic implies that $p_\alpha$ is the unique irreducible monic element of $I_\alpha$.

    As $\alpha \not\in \R\cdot 1$, $\deg(p_\alpha)\geq 2$. By the Fundamental Theorem of Algebra, it follows then that $p_\alpha$ must be quadratic, so there exist $a,b\in \R[x]$ such that $p_\alpha(x) = x^2 + ax +b$.

  \end{proof}

  \textbf{(b)} Let $V=\{\alpha\in D:\alpha^2\in\R_{\leq 0}\}$. Show that $V$ is an $\R$-linear subspace of $D$. \textbf{Hint:} Show there is an $\R$-linear map $f:D\to\R$ with kernel $V$.\\

  \begin{proof}
    For $\alpha\in D$, define an $\R$-endomorphism $T_\alpha$ of $D$ via left multiplication by $\alpha$. This furnished a linear map $D\to \End_\R(D)$. We claim that $V$ is the kernel of the composition of the $\R$-linear maps
    \[
      D\rightarrow \End_\R(D) \xrightarrow{\Tr} \R.
    \]
    Fix $\alpha\in D$ such that $\alpha\not\in\R\cdot 1$. Then by part (a) there exist $a,b\in\R$ such that $\alpha$ satisfies a quadratic irreducible polynomial $p_\alpha(x) = x^2+ax +b$. Observe that, for $v\in D$,
    \[
      p_\alpha(T_\alpha)(v) = T_\alpha^2(v) + aT_\alpha(v) + b(v) = \alpha^2 v +a\alpha v + bv = (\alpha^2+a\alpha + b\alpha)(v) = 0
    \]
    so $p_\alpha(T_\alpha) = 0\in\End_\R(D)$. Irreducibility of $p_\alpha$ then implies that $p_\alpha$ is the minimal polynomial for the operator $T\alpha$. Let $\chi_{\alpha}(x)$ be the characteristic polynomial for $T_\alpha$. Then $p_{\alpha}(x)\vert \chi_\alpha(x)$  and there exists a $k\in \N$ such that $\chi_\alpha(x)\vert (p_\alpha(x))^k$. As $\chi_\alpha$ is monic and $p_\alpha$ is irreducible, there exists an $l\in\N$ such that $\chi_\alpha(x) = (p_\alpha(x))^l$. By multinomial expansion,

    \begin{align*}
      \chi_\alpha(x) = (p_\alpha(x))^l = \sum_{\substack{n_1+n_2+n_3 = l \\ n_1,n_2,n_3\geq 0}}\binom{l}{n_1,\ n_2,\ n_3} x^{2n_1+n_2}a^{n_2}b^{n_3}
    \end{align*}

    This polynomial has $x^{2l-1}$ coefficient
    \[
      \binom{l}{l-1,\ 1,\ 0}a = l\cdot a
    \]
    However, the $x^{2l-1}$ coefficient of $\chi_\alpha$ is also $\pm\Tr(T_\alpha)$, so $\pm\Tr(T_\alpha) = l\cdot a$. Moreover, as $p_\alpha(x)$ is irreducible, $a^2-4b < 0\implies b>\frac{a^2}{4}\geq 0$. Hence, if $\alpha$ is such that $\Tr(\alpha) = 0$, then $a = 0$ whence $0 = p_\alpha(\alpha) = \alpha^2 + b \implies \alpha^2 = -b \leq 0$, i.e. $\alpha\in V$. Conversely, suppose that $\alpha\in D\setminus\{0\}$ is such that $\alpha^2 < 0$. Then $\alpha$ is linearly independent from $1$, so there exist $a,b\in \R$. such that $\alpha^2 + a\alpha + b = 0$. Note that, as $\alpha^2\in \R$, linear independence of $\alpha$ from $1$ implies that $a=0$ and $\alpha^2+b = 0$. Then, $\Tr(T_\alpha) = 0$, as desired. 
  \end{proof}

  \textbf{(c)} Define $B:V\times V\to\R$, $\displaystyle B(\alpha,\beta):=-\frac{\alpha\beta+\beta\alpha}{2}$. Show that $B$ defines an inner product on $V$ (i.e. $B$ is a symmetric, positive definite bilinear form on $V$).\\

  \textbf{(d)} Let $W$ be a linear subspace of $V$ that generates $D$ as an $\R$-algebra. Let $n=\dim_{\R}W$. Choose an orthonormal basis of $W$, i.e. a basis $\{e_i\}$ of $W$ such that $B(e_i,e_i)=1$ for all $i$ and $B(e_i,e_j)=0$ for all $i\neq j$ (such a basis always exists). Using this orthonormal basis show that if $n\geq 2$, then $D$ has a subalgebra isomorphic to $\mathbb{H}$.\\

  \textbf{(e)} \textbf{Bonus:} Suppose $n\geq 2$. Prove that $A=H$. \textbf{Hint:} One way to proceed is to show that if $n>2$, then the multiplication  in $D$ cannot be associative.
\end{homeworkProblem}


\end{document}

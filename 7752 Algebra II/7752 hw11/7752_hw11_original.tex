\documentclass[12pt,
psamsfonts]{amsart}

%-------Packages---------
\usepackage{amssymb,amsfonts,amsmath}
\usepackage[all,arc]{xy}
\usepackage{enumerate}
\usepackage{mathrsfs}
\usepackage{fullpage}
\usepackage{xspace}
\usepackage[margin=1.0in]{geometry}
\usepackage{tcolorbox}
\usepackage{tikz-cd}
\usepackage{color}
\usepackage{aliascnt}
\usepackage[foot]{amsaddr}
\usepackage{hyperref}


%--------Theorem Environments--------
%theoremstyle{plain} --- default
\newtheorem{thm}{Theorem}[section]

%----Theorem
\newaliascnt{theo}{thm}
\newtheorem{theo}[theo]{Theorem}
\aliascntresetthe{theo}
\newcommand{\theoautorefname}{Theorem}
%----Corollary
\newaliascnt{cor}{thm}
\newtheorem{cor}[cor]{Corollary}
\aliascntresetthe{cor}
\newcommand{\corautorefname}{Corollary}
%----Proposition
\newaliascnt{prop}{thm}
\newtheorem{prop}[prop]{Proposition}
\aliascntresetthe{prop}
\newcommand{\propautorefname}{Proposition}
%----Lemma
\newaliascnt{lem}{thm}
\newtheorem{lem}[lem]{Lemma}
\aliascntresetthe{lem}
\newcommand{\lemautorefname}{Lemma}
%----Conjecture
\newaliascnt{conj}{thm}
\newtheorem{conj}[conj]{Conjecture}
\aliascntresetthe{conj}
\newcommand{\conjautorefname}{Conjecture}
%----Question
\newaliascnt{que}{thm}
\newtheorem{que}[que]{Question}
\aliascntresetthe{que}
\newcommand{\queautorefname}{Question}
%----Assumption
\newaliascnt{ass}{thm}
\newtheorem{ass}[ass]{Assumption}
\aliascntresetthe{ass}
\newcommand{\assautorefname}{Assumption}
%----Definition
\newaliascnt{defn}{thm}
\newtheorem{defn}[defn]{Definition}
\aliascntresetthe{defn}
\newcommand{\defnautorefname}{Definition}




%Style
\theoremstyle{remark}
%----Remark
\newaliascnt{rem}{thm}
\newtheorem{rem}[rem]{Remark}
\aliascntresetthe{rem}
\newcommand{\remautorefname}{Remark}

\newtheorem*{ack}{Acknowledgements}




\newtheorem{Proof}{Proof}

\theoremstyle{definition}
%\newtheorem{defn}[thm]{Definition}
\newtheorem{defns}[thm]{Definitions}
\newtheorem{con}[thm]{Construction}
\newtheorem{exmp}[thm]{Example}
\newtheorem{exmps}[thm]{Examples}
\newtheorem{notn}[thm]{Notation}
\newtheorem{notns}[thm]{Notations}
\newtheorem{addm}[thm]{Addendum}
\newtheorem{exer}[thm]{Exercise}
\newtheorem{conv}[thm]{Convention}

\newtheorem{case}[thm]{Case}


\newtheorem{rems}[thm]{Remarks}
\newtheorem{warn}[thm]{Warning}
%\newtheorem{sch}[thm]{Scholium}
\newtheorem{notation}[thm]{Notation}
\newtheorem{ex}[thm]{Examples}
\newtheorem{note}[thm]{Note}



\newcommand{\N}{\mathbb{N}\xspace}
\newcommand{\I}{\mathbb{I}\xspace}
\newcommand{\R}{\mathbb{R}\xspace}
\newcommand{\Z}{\mathbb{Z}\xspace}
\newcommand{\Q}{\mathbb{Q}\xspace}
\newcommand{\C}{\mathbb{C}\xspace}
\newcommand{\G}{\mathbb{G}\xspace}
\newcommand{\F}{\mathbb{F}\xspace}
\DeclareMathOperator{\Spec}{Spec}
\DeclareMathOperator{\res}{res}
\DeclareMathOperator{\Tr}{Tr}
\DeclareMathOperator{\ord}{ord}
\DeclareMathOperator{\Sym}{Sym}
\DeclareMathOperator{\dv}{div}
\DeclareMathOperator{\alb}{alb}
\DeclareMathOperator{\img}{Im}
\DeclareMathOperator{\et}{et}
\DeclareMathOperator{\ck}{coker}
\DeclareMathOperator{\Reg}{Reg}
\DeclareMathOperator{\Cor}{Cor}
\DeclareMathOperator{\ch}{char}
\DeclareMathOperator{\supp}{supp}
\DeclareMathOperator{\Hom}{Hom}
\DeclareMathOperator{\Aut}{Aut}
\DeclareMathOperator{\Gal}{Gal}
\DeclareMathOperator{\fc}{frac}
\DeclareMathOperator{\Ann}{Ann}
\DeclareMathOperator{\Mod}{Mod}
\DeclareMathOperator{\Cone}{Cone}
\DeclareMathOperator{\FI}{FI}
\DeclareMathOperator{\End}{End}
\DeclareMathOperator{\rk}{rk}
\DeclareMathOperator{\Ext}{Ext}
\DeclareMathOperator{\ab}{ab}

\DeclareMathOperator{\coker}{coker}
\DeclareMathOperator{\fr}{frac}
\makeatletter
\let\c@equation\c@theo
\makeatother
\numberwithin{equation}{section}

\bibliographystyle{plain}
%\newcommand{\textlatin }




%--------Meta Data: Fill in your info------
\title{Math 7752 - Homework 11\\
Due Friday 04/29/22}

\begin{document}

\maketitle

%\textbf{Reminders:} If $K/F$ and $L/K$ are finite extensions, then 
%$L/F$ is finite and \[[L:F]=[L:K][K:F].\] In particular, both $[K:F]$ and $[L:K]$ divide $[L:F]$. 
%\\

%\textbf{Reminder:} A field $F$ of characteristc $p>0$ is perfect if the Frobenius map $\varphi:x\mapsto x^p$ is surjective. \\

\begin{enumerate}
%\item Let $K/L/F$ be a tower of algebraic extensions. Show that $K/F$ is separable if and only if $K/L$ and $L/F$ are separable. \\
%\item Let $K/F$ be a finite separable extension. Show that there is a finite number of fields $L$ such that $F\subseteq L\subseteq K$. \\
%\item Let $F$ be a field of $\ch(F)=p>0.$ Show that $F$ admits a finite inseparable extension $K/F$ if and only if $F$ is not perfect. \\
\item In this problem you will need the following two definitions. \\
\textbf{Definition 1:} Let $L/F$ be a finite separable extension and let $\overline{F}$ be an algebraic closure of $F$ containing $L$. A subfield $L'$ of $\overline{F}$ is called \textbf{conjugate to $L$ over $F$} if $L'=\sigma(L)$ for some $F$-embedding $\sigma:L\to\overline{F}$. (Note: $L/F$ is Galois if and only if the only conjugate to $L$ over $F$ is itself.) \\
\textbf{Definition 2:} A finite extension $K/F$ is called a \textbf{$p$-extension} if $K/F$ is \textbf{Galois} and $\Gal(K/F)$ is a $p$-group. 
\begin{enumerate}
\item Let $L/F$ be a separable extension of degree $n$ and let $K$ be the Galois closure of $L$ over $F$. Prove that $K$ can be written as a compositum $L_1 L_2\cdots L_n$, where $L_1,\ldots, L_n$ are (not necessarily distinct) conjugates of $L$ over $F$. 
\item Let $K/F$ and $L/F$ be finite $p$-extensions. Prove that $KL/F$ is also a $p$-extension.  
\item Suppose that $K/L$ and $L/F$ are both $p$-extensions, and let $M$ be the Galois closure of $K$ over $F$ (note: we do not know whether $K/F$ is Galois or not). Prove that $M/F$ is also a $p$-extension. 
\item Now assume only that $L/F$ is a separable extension with $[L:F]=p^r$, for some $r\geq 1$. Let $M$ be the Galois closure of $L$ over $F$. Prove that $[M:F]$ need not be a power of $p$. 
\end{enumerate} 
\medskip 
\medskip 
\item Let $f(x)$ and $g(x)$ be irreducible polynomials in $\F_p[x]$ of the same degree. Let $F=\F_p[x]/(f(x))$. Prove that $g(x)$ splits completely over $F$. 
\\
\item Consider the polynomial $f(x)=x^4-2x^2-5\in\Q[x]$.
\begin{enumerate}
\item Determine the Galois group $G$ of the splitting field $K$ of $f(x)$ over $\Q$.
\item Find all subgroups of $G$ and their corresponding fixed fields. Which of those are normal extensions of $\Q$?  
\end{enumerate} 
\medskip
\medskip
\item Let $p$ and $q$ be distinct primes with $q>p$, and let $K/F$ be a Galois extension of degree $pq$. Prove the following: 
 \begin{enumerate}
[(a)]\item There exists a field $L$ with $F\subset L\subset K$ and $[L:F]=q$. 
\item There exists a \textbf{unique} field $M$ with $F\subset M\subset K$ and $[M:F]=p$. 
\end{enumerate} 
\medskip
\medskip 
\item Prove the following analogue of Kummer's theorem for abelian extensions: Let $n\in\N$ and let $F$ be a field containing a primitive $n^{th}$ root of unity. 
\begin{enumerate}
\item  Let $K/F$ be a finite Galois extension such that $G=\Gal(K/F)$ is abelian of exponent $n$. Then there exists $a_1,\ldots,a_t\in F$ such that $K=F(\sqrt[n]{a_1},\ldots,\sqrt[n]{a_t})$. More precisely, there exists $\alpha_1,\ldots,\alpha_t\in K$ such that $K=F(\alpha_1,\ldots,\alpha_t)$ and $\alpha_i^n\in F$ for all $i$. 
\item Conversely, suppose that $K=F(\sqrt[n]{a_1},\ldots,\sqrt[n]{a_t})$ for some $a_1,\ldots, a_t\in F$. Prove that $K/F$ is Galois and $G=\Gal(K/F)$ is abelian of exponent $n$. \textbf{Hint:} For part (b) use one of the problems from the previous homework. 
\end{enumerate}
\medskip
\medskip
\item Let $F$ be a field containing a primitive $n^{th}$ root of unity. Let $a,b\in F$ be such that the polynomials $f(x)=x^n-a$, and $g(x)=x^n-b$ are both irreducible over $F$. Consider the Kummer extensions $F(\alpha)$, $F(\beta)$, where $\alpha$ is a root of $f(x)$ and $\beta$ is a root of $g(x)$. Prove that 
$F(\alpha)=F(\beta)$ if and only if $\beta=c\alpha^r$, for some $c\in F$ and some integer $r$ which is coprime to $n$ (equivalently, if and only if $b=c^n a^r$, for some $c\in F$ and some $(r,n)=1)$. 
\\
%\item Let $n\geq 1$. Consider the finite field $\F_{p^n}=\{\alpha\in\overline{\F}_p: \alpha^{p^n}=\alpha\}$. Consider the Frobenius endomorphism $\varphi:\F_{p^n}\to\F_{p^n}$ given by $x\mapsto x^{p}$.  
%\begin{enumerate}
%\item Prove that $\varphi$ is an isomorphism of $\F_{p^n}$ to itself (and hence an automorphism). 
%\item  Determine the rational canonical form of the $\F_p$-linear transformation $\varphi$. 
%\end{enumerate}
%\medskip
%\item Let $F$ be a field and $p(x)\in F[x]$ be a non-constant polynomial. Show that $p(x)$ is separable if and only if $\gcd(p(x),p'(x))=1$.\\
%\item  The purpose of this problem is to prove that the group $\Aut_\Q(\R)$ is trivial by following the suggested steps. Let $\sigma:\R\to\R$ be an automorphism of $\R$.
%\begin{enumerate}
%\item  Show that $\sigma$ is strictly increasing. 
%\item Use the density of $\Q$ in $\R$ to show that $\sigma$ is continuous at $x=0$. 
%\item Deduce that $\sigma$ is continuous on $\R$, and hence $\sigma(x)=x$.  
%\end{enumerate}
%
%\medskip
%\item (\textbf{Bonus}) Finish the proof we started in class to show that the Galois group $\Gal(\overline{\F}_p/\F_p)$ is uncountable.  
%\end{enumerate} 
%\medskip
%\medskip
%\item Dummit and Foote, Problems 17 \& 18 on pages 582-583. 




\end{enumerate}





\end{document}
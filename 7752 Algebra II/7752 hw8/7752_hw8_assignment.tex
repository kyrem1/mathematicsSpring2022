\documentclass[12pt,
psamsfonts]{amsart}

%-------Packages---------
\usepackage{amssymb,amsfonts,amsmath}
\usepackage[all,arc]{xy}
\usepackage{enumerate}
\usepackage{mathrsfs}
\usepackage{fullpage}
\usepackage{xspace}
\usepackage[margin=1.0in]{geometry}
\usepackage{tcolorbox}
\usepackage{tikz-cd}
\usepackage{color}
\usepackage{aliascnt}
\usepackage[foot]{amsaddr}
\usepackage{hyperref}


%--------Theorem Environments--------
%theoremstyle{plain} --- default
\newtheorem{thm}{Theorem}[section]

%----Theorem
\newaliascnt{theo}{thm}
\newtheorem{theo}[theo]{Theorem}
\aliascntresetthe{theo}
\newcommand{\theoautorefname}{Theorem}
%----Corollary
\newaliascnt{cor}{thm}
\newtheorem{cor}[cor]{Corollary}
\aliascntresetthe{cor}
\newcommand{\corautorefname}{Corollary}
%----Proposition
\newaliascnt{prop}{thm}
\newtheorem{prop}[prop]{Proposition}
\aliascntresetthe{prop}
\newcommand{\propautorefname}{Proposition}
%----Lemma
\newaliascnt{lem}{thm}
\newtheorem{lem}[lem]{Lemma}
\aliascntresetthe{lem}
\newcommand{\lemautorefname}{Lemma}
%----Conjecture
\newaliascnt{conj}{thm}
\newtheorem{conj}[conj]{Conjecture}
\aliascntresetthe{conj}
\newcommand{\conjautorefname}{Conjecture}
%----Question
\newaliascnt{que}{thm}
\newtheorem{que}[que]{Question}
\aliascntresetthe{que}
\newcommand{\queautorefname}{Question}
%----Assumption
\newaliascnt{ass}{thm}
\newtheorem{ass}[ass]{Assumption}
\aliascntresetthe{ass}
\newcommand{\assautorefname}{Assumption}
%----Definition
\newaliascnt{defn}{thm}
\newtheorem{defn}[defn]{Definition}
\aliascntresetthe{defn}
\newcommand{\defnautorefname}{Definition}




%Style
\theoremstyle{remark}
%----Remark
\newaliascnt{rem}{thm}
\newtheorem{rem}[rem]{Remark}
\aliascntresetthe{rem}
\newcommand{\remautorefname}{Remark}

\newtheorem*{ack}{Acknowledgements}




\newtheorem{Proof}{Proof}

\theoremstyle{definition}
%\newtheorem{defn}[thm]{Definition}
\newtheorem{defns}[thm]{Definitions}
\newtheorem{con}[thm]{Construction}
\newtheorem{exmp}[thm]{Example}
\newtheorem{exmps}[thm]{Examples}
\newtheorem{notn}[thm]{Notation}
\newtheorem{notns}[thm]{Notations}
\newtheorem{addm}[thm]{Addendum}
\newtheorem{exer}[thm]{Exercise}
\newtheorem{conv}[thm]{Convention}

\newtheorem{case}[thm]{Case}


\newtheorem{rems}[thm]{Remarks}
\newtheorem{warn}[thm]{Warning}
%\newtheorem{sch}[thm]{Scholium}
\newtheorem{notation}[thm]{Notation}
\newtheorem{ex}[thm]{Examples}
\newtheorem{note}[thm]{Note}



\newcommand{\N}{\mathbb{N}\xspace}
\newcommand{\I}{\mathbb{I}\xspace}
\newcommand{\R}{\mathbb{R}\xspace}
\newcommand{\Z}{\mathbb{Z}\xspace}
\newcommand{\Q}{\mathbb{Q}\xspace}
\newcommand{\C}{\mathbb{C}\xspace}
\newcommand{\G}{\mathbb{G}\xspace}
\newcommand{\F}{\mathbb{F}\xspace}
\DeclareMathOperator{\Spec}{Spec}
\DeclareMathOperator{\res}{res}
\DeclareMathOperator{\Tr}{Tr}
\DeclareMathOperator{\ord}{ord}
\DeclareMathOperator{\Sym}{Sym}
\DeclareMathOperator{\dv}{div}
\DeclareMathOperator{\alb}{alb}
\DeclareMathOperator{\img}{Im}
\DeclareMathOperator{\et}{et}
\DeclareMathOperator{\ck}{coker}
\DeclareMathOperator{\Reg}{Reg}
\DeclareMathOperator{\Cor}{Cor}
\DeclareMathOperator{\ch}{char}
\DeclareMathOperator{\supp}{supp}
\DeclareMathOperator{\Hom}{Hom}
\DeclareMathOperator{\Aut}{Aut}
\DeclareMathOperator{\Gal}{Gal}
\DeclareMathOperator{\fc}{frac}
\DeclareMathOperator{\Ann}{Ann}
\DeclareMathOperator{\Mod}{Mod}
\DeclareMathOperator{\Cone}{Cone}
\DeclareMathOperator{\FI}{FI}
\DeclareMathOperator{\End}{End}
\DeclareMathOperator{\rk}{rk}
\DeclareMathOperator{\Ext}{Ext}
\DeclareMathOperator{\ab}{ab}

\DeclareMathOperator{\coker}{coker}
\DeclareMathOperator{\fr}{frac}
\makeatletter
\let\c@equation\c@theo
\makeatother
\numberwithin{equation}{section}

\bibliographystyle{plain}
%\newcommand{\textlatin }




%--------Meta Data: Fill in your info------
\title{Math 7752 - Homework 8\\
Due Friday 04/01/22}

\begin{document}

\maketitle

%\textbf{Reminders:} If $K/F$ and $L/K$ are finite extensions, then 
%$L/F$ is finite and \[[L:F]=[L:K][K:F].\] In particular, both $[K:F]$ and $[L:K]$ divide $[L:F]$. 
%\\

\textbf{Note:} Let $K/F$ be a field extension. Consider the set $\Aut_F(K)$ consisting of all $F$-isomorphisms $\sigma:K\to K$. Such isomorphisms are called $F$-\textit{automorphisms} of $K$. Note that $\Aut_F(K)$ forms a group under composition. 
\\
\begin{enumerate}
\item  
\begin{enumerate}
\item Let $F$ be a field and let $\overline{F}$ be an algebraic closure of $F$. Let $\sigma:\overline{F}\to\overline{F}$ be an $F$-embedding. Prove that $\sigma(\overline{F})=\overline{F}$, and thus $\sigma$ is an automorphism of $\overline{F}$.  
\item Prove that for any field $F$ any two algebraic closures of $F$ are $F$-isomorphic. \textbf{Hint:} Use the Main Extension Lemma. 
\end{enumerate} 
\medskip 
\item For each of the following polynomials $f(x)\in\Q[x]$ let $K\subset \C$ be the splitting field of $f$ over $\Q$. 
 \begin{enumerate}
[(i)]\item $f(x)=x^n-1$, $n\geq 2$.
\item $f(x)=x^4+3x^3+4x^2+3x+3$.
\item  $f(x)=x^4-2$.
%\item Suppose that the integers $n_1,\ldots, n_r$ are square-free and pairwise relatively prime. Prove that $[\Q_r:\Q]=2^r$. Conclude that the extension $L=\Q(T)$, where $T=\{\sqrt{n}:n\in\N, n \text{ square free}\}$ is an infinite algebraic extension of $\Q$. 
\end{enumerate} Find the degree $[K:\Q]$ and express $K$ in the form $\Q(\alpha)$, or $\Q(\alpha,\beta)$. \textbf{Hint:} For (i) you can take a look at [DF. Section 13.6].
\medskip
\item 
\begin{enumerate}
%\item Let $R$ be a domain. Show that the characteristic of $R$ is either $0$ or a prime $p$.
\item Let $F$ be a field. Prove that if $\ch(F)=0$, then there is an embedding $\Q\hookrightarrow F$, while if $\ch(F)=p$, then there exists an embedding $\F_p\hookrightarrow F$. \\
\textbf{Conclusion:} The fields $\Q,\F_p$ are the smallest in each characteristic and we call them \textit{prime fields}. 
\item Let $F$ be a field. Prove that every automorphism $\sigma:F\to F$ of $F$ is an $F_0$-automorphism, where $F_0$ is the prime subfield of $F$. 
\end{enumerate}
\medskip
%\item Let $n\geq 1$. Consider the finite field $\F_{p^n}=\{\alpha\in\overline{\F}_p: \alpha^{p^n}=\alpha\}$. Consider the Frobenius endomorphism $\varphi:\F_{p^n}\to\F_{p^n}$ given by $x\mapsto x^{p}$.  
%\begin{enumerate}
%\item Prove that $\varphi$ is an isomorphism of $\F_{p^n}$ to itself (and hence an automorphism). 
%\item  Determine the rational canonical form of the $\F_p$-linear transformation $\varphi$. 
%\end{enumerate}
%\medskip

\item  The purpose of this problem is to prove that the group $\Aut_\Q(\R)$ is trivial by following the suggested steps. Let $\sigma:\R\to\R$ be an automorphism of $\R$.
\begin{enumerate}
\item  Show that $\sigma$ is strictly increasing. 
\item Use the density of $\Q$ in $\R$ to show that $\sigma$ is continuous at $x=0$. 
\item Deduce that $\sigma$ is continuous on $\R$, and hence $\sigma(x)=x$.  
\end{enumerate}

\medskip
\item \begin{enumerate}
\item Let $K/F$ be an algebraic extension. Prove that $K/F$ is normal if and only if for any algebraic extension $L/K$ and any $F$-automorphism $\sigma\in\Aut_F(L)$ we have $\sigma(K)=K$. 
\item Let $K/F$ be a field extension, and let $K_1$ and $K_2$ be subfields of $K$ containing $F$ such that the extensions $K_1/F$ and $K_2/F$ are normal. Prove that the extensions $K_1K_2/F$ and $K_1\cap K_2/F$ are also normal. \\
\end{enumerate} 
\item \textbf{(Optional only for practice)} Let $F$ be a field and $p(x)\in F[x]$ be a non-constant polynomial. Show that $p(x)$ is separable if and only if $\gcd(p(x),p'(x))=1$.\\




\end{enumerate}





\end{document}
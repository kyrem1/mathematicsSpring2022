\documentclass[12pt,letterpaper]{article}

%--------Packages--------
\usepackage{amsmath, amsthm, amssymb}
\usepackage{xspace}
\usepackage{graphicx}
\usepackage{amssymb}
\usepackage{array}
\usepackage{braket}
\usepackage{multicol}
\usepackage{mathtools}
\usepackage{enumerate}
\usepackage{delarray}
\usepackage{mathtools}
\usepackage{fullpage}
\usepackage{faktor} % For quotients
\usepackage{mathrsfs}
% \usepackage{quiver}
% \usepackage{tikz}

% \usepackage{quiver}
\usepackage[linguistics]{forest}




%--------Page Setup--------

\pagestyle{empty}%

\setlength{\hoffset}{-1.54cm}
\setlength{\voffset}{-1.54cm}

\setlength{\topmargin}{0pt}
\setlength{\headsep}{0pt}
\setlength{\headheight}{0pt}

\setlength{\oddsidemargin}{0pt}

\setlength{\textwidth}{195mm}
\setlength{\textheight}{250mm}


%--------Macros--------

\newcommand{\ilm}[1]{\begin{psmallmatrix} #1 \end{psmallmatrix}}
\newcommand{\ilmb}[1]{\boxed{\begin{smallmatrix} #1 \end{smallmatrix}}}

\newcommand{\sub}{\subseteq}
\newcommand{\lcm}{\text{lcm}}
\newcommand{\ms}[1]{\mathscr{#1}}
\newcommand{\mc}[1]{\mathcal{#1}}
\newcommand{\mf}[1]{\mathfrak{#1}}
\newcommand{\sO}{\mathcal{O}}
\newcommand{\cyclic}[1]{\langle#1\rangle}
\newcommand{\units}[1]{#1 ^{\times}}
\newcommand{\la}{\langle}
\newcommand{\ra}{\rangle}
\newcommand{\lr}[1]{\left(#1\right)}
%----Switch phi and varphi
\let\temp\phi
\let\phi\varphi
\let\varphi\temp

\newcommand{\C}{\mathbb{C}}
\newcommand{\F}{\mathbb{F}}
\newcommand{\N}{\mathbb{N}\xspace}
\newcommand{\I}{\mathbb{I}\xspace}
\newcommand{\R}{\mathbb{R}\xspace}
\newcommand{\Z}{\mathbb{Z}\xspace}
\newcommand{\Q}{\mathbb{Q}\xspace}
\newcommand{\G}{\mathbb{G}\xspace}
\DeclareMathOperator{\Spec}{Spec}
\DeclareMathOperator{\res}{res}
\DeclareMathOperator{\Tr}{Tr}
\DeclareMathOperator{\ord}{ord}
\DeclareMathOperator{\Sym}{Sym}
\DeclareMathOperator{\dv}{div}
\DeclareMathOperator{\alb}{alb}
\let\Im\relax
\DeclareMathOperator{\Im}{Im}
\DeclareMathOperator{\et}{et}
\DeclareMathOperator{\ck}{coker}
\DeclareMathOperator{\Reg}{Reg}
\DeclareMathOperator{\Cor}{Cor}
\DeclareMathOperator{\Ac}{at}
\DeclareMathOperator{\supp}{supp}
\DeclareMathOperator{\Hom}{Hom}
\DeclareMathOperator{\Pic}{Pic}
\DeclareMathOperator{\Gal}{Gal}
\DeclareMathOperator{\fc}{frac}
\DeclareMathOperator{\Ann}{Ann}
\DeclareMathOperator{\Mod}{Mod}
\DeclareMathOperator{\Cone}{Cone}
\DeclareMathOperator{\FI}{FI}
\DeclareMathOperator{\End}{End}
\DeclareMathOperator{\Alb}{Alb}
\DeclareMathOperator{\Ext}{Ext}
\DeclareMathOperator{\ab}{ab}
\DeclareMathOperator{\Jac}{Jac}
\DeclareMathOperator{\coker}{coker}
\DeclareMathOperator{\fr}{frac}
\DeclareMathOperator{\spn}{span}
\DeclareMathOperator{\im}{im}
\DeclareMathOperator{\rk}{rk}
\DeclareMathOperator{\GL}{GL}


%----Analysis
\newcommand{\dd}[2][]{\frac{\partial^{#1}}{\partial {#2}^{#1}}}
\newcommand{\summ}{\sum\limits}
\newcommand{\norm}[1]{\left \vert \left \vert #1 \right \vert \right \vert}
\newcommand{\thicc}{\bigg}
\newcommand{\eps}{\varepsilon}
\newcommand*\cls[1]{\overline{#1}}


%--------Theorem environments--------
\newtheorem{definition}{Definition}[]
\newtheorem{lemma}{Lemma}[]
\newtheorem{corollary}{Corollary}[]
\newtheorem{theorem}{Theorem}[]
\theoremstyle{remark}
\newtheorem*{claim}{Claim}


\newenvironment{solution}
{\begin{proof}[Solution]}
{\end{proof}}


\makeatletter
\newcolumntype{"}{@{\hskip\tabcolsep\vrule width 1pt\hskip\tabcolsep}}
\makeatother

% --------Problem environment--------
\setlength\parindent{0pt}
\setcounter{secnumdepth}{0}
\newcounter{partCounter}
\newcounter{homeworkProblemCounter}
\setcounter{homeworkProblemCounter}{1}


\newenvironment{homeworkProblem}[1][-1]{
    \ifnum#1>0
        \setcounter{homeworkProblemCounter}{#1}
    \fi
    \section{Problem \arabic{homeworkProblemCounter}}
    \setcounter{partCounter}{1}
    \stepcounter{homeworkProblemCounter}
}


%--------Metadata--------
\title{MATH 7752 Homework 7}
\author{James Harbour}


\begin{document}
\maketitle

\begin{homeworkProblem}
  \textbf{(a)}: Consider the field $K=\Q(\sqrt{2},\sqrt{3})$. Prove that $[K:\Q]=4$. \\

  \textbf{(b)}: Let $L=\Q(\sqrt{2}+\sqrt{3})$. Show that $L=K$. \\
\end{homeworkProblem}

\begin{homeworkProblem}
  Let $S=\{n_1,\ldots, n_r\}$ be a finite set of positive integers with $n_i\geq 2$. For each $j\in\{1,\ldots,r\}$ let $\Q_j=\Q(\sqrt{n_1},\ldots,\sqrt{n_j})$. Moreover, set $\Q_0=\Q$. \\

  \textbf{(a)}: Prove that $[\Q_r:\Q]=2^m$ for some integer $0\leq m\leq r$. Moreover, show that the following set spans $\Q_r$ over $\Q$, \[P(S)=\{1\}\cup\{\sqrt{n}:n\text{ is a product of distinct elements from }S\}.\]

  \textbf{(b)}: Prove that $[\Q_r:\Q]<2^r$ if and only if $n_1$ is a complete square, or there exists $2\leq j\leq r$ such that $\sqrt{n_j}=\alpha+\beta\sqrt{n_{j-1}}$, for some $\alpha,\beta\in\Q_{j-2}$.\\

  \textbf{(c)}: Suppose that the integers $n_1,\ldots, n_r$ are square-free and pairwise relatively prime. Prove that $[\Q_r:\Q]=2^r$. Conclude that the extension $L=\Q(T)$, where $T=\{\sqrt{n}:n\in\N, n \text{ square free}\}$ is an infinite algebraic extension of $\Q$. \\
\end{homeworkProblem}


\begin{homeworkProblem}
  Let $F$ be a field and $\alpha$ an algebraic element of odd degree over $F$ (i.e. the degree $[F(\alpha):F]$ is odd). Show that $F(\alpha^2)=F(\alpha)$. \\
\end{homeworkProblem}

\begin{homeworkProblem}
  Let $K/F$ be an algebraic extension. \\

  \textbf{(a)}: Let $F\subset R\subset K$ where $R$ is a subring of $K$. Prove that $R$ must be a subfield. \\

  \textbf{(b)}: Show that (a) would be false if we dropped the assumption that $K/F$ is algebraic. \\
\end{homeworkProblem}

\begin{homeworkProblem}
  Let $K/F$ be a finite field extension, $n=[K:F]$, and fix some basis $\Omega=\{\alpha_1,\ldots,\alpha_n\}$ of $K$ over $F$. For any $\alpha\in K$ define $T_\alpha:K\rightarrow K$ by $\beta\mapsto\alpha\beta$. Note that $T_\alpha\in\End_F(K)$.  Let $A_\alpha=[T_\alpha]_\Omega\in M_n(F)$ be the matrix of $T_\alpha$ with respect to $\Omega$.  \\

  \textbf{(a)}: Prove that the map $K\xrightarrow{\rho} M_n(F)$ given by $\alpha\mapsto A_\alpha$ is an injective ring homomorphism. \\

  \textbf{(b)}: Prove that the minimal polynomial of $\alpha$ over $F$ and the minimal polynomial of $A_\alpha$ coincide.
\end{homeworkProblem}

\begin{homeworkProblem}
  Let $K/F$ be an extension of fields and let $F\subseteq K_1\subseteq K$ and $F\subseteq K_2\subseteq K$ be two subextensions of $K/F$. The \textit{compositum} of $K_1$ and $K_2$ is the smallest subfield of $K$ that contains both $K_1$ and $K_2$. \textbf{Notation:} We denote the compositum by $K_1 K_2$. \\

  \textbf{(a)}: Consider the $F$-algebra $K_1\otimes_F K_2$. Show that there exists a unique $F$-algebra homomorphism $\Phi:K_1\otimes_F K_2\rightarrow K_1 K_2$ such that $\Phi(a\otimes b)=ab$. Conclude that $[K_1 K_2:F]\leq [K_1:F][K_2:F]$. \\

  \textbf{(b)}: Show that $K_1\otimes_F K_2$ is a field if and only if the above $\leq$ becomes an equality.\\
  
  \textbf{(c)}: Suppose that $K_1\cap K_2\neq F$. Prove that $K_1\otimes_F K_2$ is not a field. \\
\end{homeworkProblem}


\end{document}

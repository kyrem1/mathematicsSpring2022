\documentclass[12pt,
psamsfonts]{amsart}

%-------Packages---------
\usepackage{amssymb,amsfonts,amsmath}
\usepackage[all,arc]{xy}
\usepackage{enumerate}
\usepackage{mathrsfs}
\usepackage{fullpage}
\usepackage{xspace}
\usepackage[margin=1.0in]{geometry}
\usepackage{tcolorbox}
\usepackage{tikz-cd}
\usepackage{color}
\usepackage{aliascnt}
\usepackage[foot]{amsaddr}
\usepackage{hyperref}


%--------Theorem Environments--------
%theoremstyle{plain} --- default
\newtheorem{thm}{Theorem}[section]

%----Theorem
\newaliascnt{theo}{thm}
\newtheorem{theo}[theo]{Theorem}
\aliascntresetthe{theo}
\newcommand{\theoautorefname}{Theorem}
%----Corollary
\newaliascnt{cor}{thm}
\newtheorem{cor}[cor]{Corollary}
\aliascntresetthe{cor}
\newcommand{\corautorefname}{Corollary}
%----Proposition
\newaliascnt{prop}{thm}
\newtheorem{prop}[prop]{Proposition}
\aliascntresetthe{prop}
\newcommand{\propautorefname}{Proposition}
%----Lemma
\newaliascnt{lem}{thm}
\newtheorem{lem}[lem]{Lemma}
\aliascntresetthe{lem}
\newcommand{\lemautorefname}{Lemma}
%----Conjecture
\newaliascnt{conj}{thm}
\newtheorem{conj}[conj]{Conjecture}
\aliascntresetthe{conj}
\newcommand{\conjautorefname}{Conjecture}
%----Question
\newaliascnt{que}{thm}
\newtheorem{que}[que]{Question}
\aliascntresetthe{que}
\newcommand{\queautorefname}{Question}
%----Assumption
\newaliascnt{ass}{thm}
\newtheorem{ass}[ass]{Assumption}
\aliascntresetthe{ass}
\newcommand{\assautorefname}{Assumption}
%----Definition
\newaliascnt{defn}{thm}
\newtheorem{defn}[defn]{Definition}
\aliascntresetthe{defn}
\newcommand{\defnautorefname}{Definition}




%Style
\theoremstyle{remark}
%----Remark
\newaliascnt{rem}{thm}
\newtheorem{rem}[rem]{Remark}
\aliascntresetthe{rem}
\newcommand{\remautorefname}{Remark}

\newtheorem*{ack}{Acknowledgements}




\newtheorem{Proof}{Proof}

\theoremstyle{definition}
%\newtheorem{defn}[thm]{Definition}
\newtheorem{defns}[thm]{Definitions}
\newtheorem{con}[thm]{Construction}
\newtheorem{exmp}[thm]{Example}
\newtheorem{exmps}[thm]{Examples}
\newtheorem{notn}[thm]{Notation}
\newtheorem{notns}[thm]{Notations}
\newtheorem{addm}[thm]{Addendum}
\newtheorem{exer}[thm]{Exercise}
\newtheorem{conv}[thm]{Convention}

\newtheorem{case}[thm]{Case}


\newtheorem{rems}[thm]{Remarks}
\newtheorem{warn}[thm]{Warning}
%\newtheorem{sch}[thm]{Scholium}
\newtheorem{notation}[thm]{Notation}
\newtheorem{ex}[thm]{Examples}
\newtheorem{note}[thm]{Note}



\newcommand{\N}{\mathbb{N}\xspace}
\newcommand{\I}{\mathbb{I}\xspace}
\newcommand{\R}{\mathbb{R}\xspace}
\newcommand{\Z}{\mathbb{Z}\xspace}
\newcommand{\Q}{\mathbb{Q}\xspace}
\newcommand{\C}{\mathbb{C}\xspace}
\newcommand{\G}{\mathbb{G}\xspace}
\DeclareMathOperator{\Spec}{Spec}
\DeclareMathOperator{\res}{res}
\DeclareMathOperator{\Tr}{Tr}
\DeclareMathOperator{\ord}{ord}
\DeclareMathOperator{\Sym}{Sym}
\DeclareMathOperator{\dv}{div}
\DeclareMathOperator{\alb}{alb}
\DeclareMathOperator{\img}{Im}
\DeclareMathOperator{\et}{et}
\DeclareMathOperator{\ck}{coker}
\DeclareMathOperator{\Reg}{Reg}
\DeclareMathOperator{\Cor}{Cor}
\DeclareMathOperator{\Ac}{at}
\DeclareMathOperator{\supp}{supp}
\DeclareMathOperator{\Hom}{Hom}
\DeclareMathOperator{\Pic}{Pic}
\DeclareMathOperator{\Gal}{Gal}
\DeclareMathOperator{\fc}{frac}
\DeclareMathOperator{\Ann}{Ann}
\DeclareMathOperator{\Mod}{Mod}
\DeclareMathOperator{\Cone}{Cone}
\DeclareMathOperator{\FI}{FI}
\DeclareMathOperator{\End}{End}
\DeclareMathOperator{\rk}{rk}
\DeclareMathOperator{\Ext}{Ext}
\DeclareMathOperator{\ab}{ab}

\DeclareMathOperator{\coker}{coker}
\DeclareMathOperator{\fr}{frac}
\makeatletter
\let\c@equation\c@theo
\makeatother
\numberwithin{equation}{section}

\bibliographystyle{plain}
%\newcommand{\textlatin }




%--------Meta Data: Fill in your info------
\title{Math 7752 - Homework 7\\
Due Friday 03/25/22}

\begin{document}

\maketitle

%\textbf{Reminders:} If $K/F$ and $L/K$ are finite extensions, then
%$L/F$ is finite and \[[L:F]=[L:K][K:F].\] In particular, both $[K:F]$ and $[L:K]$ divide $[L:F]$.
%\\

\begin{enumerate}
\item
\begin{enumerate}
\item Consider the field $K=\Q(\sqrt{2},\sqrt{3})$. Prove that $[K:\Q]=4$.
\item Let $L=\Q(\sqrt{2}+\sqrt{3})$. Show that $L=K$.
\end{enumerate}
\medskip
\item Let $S=\{n_1,\ldots, n_r\}$ be a finite set of positive integers with $n_i\geq 2$. For each $j\in\{1,\ldots,r\}$ let $\Q_j=\Q(\sqrt{n_1},\ldots,\sqrt{n_j})$. Moreover, set $\Q_0=\Q$.
 \begin{enumerate}
[(a)]\item Prove that $[\Q_r:\Q]=2^m$ for some integer $0\leq m\leq r$. Moreover, show that the following set spans $\Q_r$ over $\Q$, \[P(S)=\{1\}\cup\{\sqrt{n}:n\text{ is a product of distinct elements from }S\}.\]
\item Prove that $[\Q_r:\Q]<2^r$ if and only if $n_1$ is a complete square, or there exists $2\leq j\leq r$ such that $\sqrt{n_j}=\alpha+\beta\sqrt{n_{j-1}}$, for some $\alpha,\beta\in\Q_{j-2}$.
\item Suppose that the integers $n_1,\ldots, n_r$ are square-free and pairwise relatively prime. Prove that $[\Q_r:\Q]=2^r$. Conclude that the extension $L=\Q(T)$, where $T=\{\sqrt{n}:n\in\N, n \text{ square free}\}$ is an infinite algebraic extension of $\Q$.
\end{enumerate}
\medskip
\item Let $F$ be a field and $\alpha$ an algebraic element of odd degree over $F$ (i.e. the degree $[F(\alpha):F]$ is odd). Show that $F(\alpha^2)=F(\alpha)$. \\

\item  Let $K/F$ be an algebraic extension.
\begin{enumerate}
\item Let $F\subset R\subset K$ where $R$ is a subring of $K$. Prove that $R$ must be a subfield.
\item Show that (a) would be false if we dropped the assumption that $K/F$ is algebraic.
\end{enumerate}
\medskip
\item Let $K/F$ be a finite field extension, $n=[K:F]$, and fix some basis $\Omega=\{\alpha_1,\ldots,\alpha_n\}$ of $K$ over $F$. For any $\alpha\in K$ define $T_\alpha:K\rightarrow K$ by $\beta\mapsto\alpha\beta$. Note that $T_\alpha\in\End_F(K)$.  Let $A_\alpha=[T_\alpha]_\Omega\in M_n(F)$ be the matrix of $T_\alpha$ with respect to $\Omega$.
\begin{enumerate}
[(a)]\item Prove that the map $K\xrightarrow{\rho} M_n(F)$ given by $\alpha\mapsto A_\alpha$ is an injective ring homomorphism.
\item Prove that the minimal polynomial of $\alpha$ over $F$ and the minimal polynomial of $A_\alpha$ coincide.
%\item (Optional) Find the minimal polynomial of the matrix
%\[A=\left(\begin{array}{cccc}
%0 & 3 & 5 & 0\\
%1 & 0 & 0 & 5\\
%1 & 0 & 0 & 3 \\
%0 & 1 & 1 & 0
%\end{array}\right)\] without doing extensive computations. \textbf{Hint:} Find an extension $K/\Q$ of degree $4$, a basis for $K$ over $\Q$ and an element $\alpha\in K$ such that $A=A_\alpha$.
\end{enumerate}
\medskip
\item Let $K/F$ be an extension of fields and let $F\subseteq K_1\subseteq K$ and $F\subseteq K_2\subseteq K$ be two subextensions of $K/F$. The \textit{compositum} of $K_1$ and $K_2$ is the smallest subfield of $K$ that contains both $K_1$ and $K_2$. \textbf{Notation:} We denote the compositum by $K_1 K_2$.
\begin{enumerate}
[(a)]\item Consider the $F$-algebra $K_1\otimes_F K_2$. Show that there exists a unique $F$-algebra homomorphism $\Phi:K_1\otimes_F K_2\rightarrow K_1 K_2$ such that $\Phi(a\otimes b)=ab$. Conclude that $[K_1 K_2:F]\leq [K_1:F][K_2:F]$. 
\item Show that $K_1\otimes_F K_2$ is a field if and only if the above $\leq$ becomes an equality.
\item Suppose that $K_1\cap K_2\neq F$. Prove that $K_1\otimes_F K_2$ is not a field.
\end{enumerate}



\end{enumerate}





\end{document}

\documentclass[12pt,letterpaper]{article}

%--------Packages--------
\usepackage{amsmath, amsthm, amssymb}
\usepackage{xspace}
\usepackage{graphicx}
\usepackage{hhline}
\usepackage{amssymb}
\usepackage{array}
\usepackage{braket}
\usepackage{multicol}
\usepackage{mathtools}
\usepackage{enumerate}
\usepackage{delarray}
\usepackage{mathtools}
\usepackage{fullpage}
\usepackage{faktor} % For quotients
\usepackage{mathrsfs}

\usepackage[italicdiff]{physics} % For differentials
\usepackage{bbm} % For indicator

% \usepackage{quiver}
\usepackage[linguistics]{forest}




%--------Page Setup--------

\pagestyle{empty}%

\setlength{\hoffset}{-1.54cm}
\setlength{\voffset}{-1.54cm}

\setlength{\topmargin}{0pt}
\setlength{\headsep}{0pt}
\setlength{\headheight}{0pt}

\setlength{\oddsidemargin}{0pt}

\setlength{\textwidth}{195mm}
\setlength{\textheight}{250mm}


%--------Macros--------

\newcommand{\sub}{\subseteq}
\newcommand{\lcm}{\text{lcm}}
\newcommand{\mc}[1]{\mathcal{#1}}
\newcommand{\mf}[1]{\mathfrak{#1}}
\newcommand{\ms}[1]{\mathscr{#1}}
\newcommand{\sO}{\mathcal{O}}
\newcommand{\cyclic}[1]{\langle#1\rangle}
\newcommand{\units}[1]{#1 ^{\times}}
\newcommand{\la}{\langle}
\newcommand{\ra}{\rangle}
\newcommand{\lr}[1]{\left(#1\right)}

\DeclarePairedDelimiterX{\inp}[2]{\langle}{\rangle}{#1, #2}

%----Switch phi and varphi
% \let\temp\phi
% \let\phi\varphi
% \let\varphi\temp

\newcommand{\C}{\mathbb{C}}
\newcommand{\F}{\mathbb{F}}
\newcommand{\E}{\mathbb{E}}
\newcommand{\N}{\mathbb{N}\xspace}
\newcommand{\I}{\mathbb{I}\xspace}
\newcommand{\R}{\mathbb{R}\xspace}
\newcommand{\Z}{\mathbb{Z}\xspace}
\newcommand{\Q}{\mathbb{Q}\xspace}
\newcommand{\G}{\mathbb{G}\xspace}

\DeclareMathOperator{\Spec}{Spec}
\DeclareMathOperator{\res}{res}
% \DeclareMathOperator{\Tr}{Tr}
\DeclareMathOperator{\ord}{ord}
\DeclareMathOperator{\Sym}{Sym}
% \DeclareMathOperator{\dv}{div}
\DeclareMathOperator{\alb}{alb}
\DeclareMathOperator{\img}{Im}
\DeclareMathOperator{\et}{et}
\DeclareMathOperator{\ck}{coker}
\DeclareMathOperator{\Reg}{Reg}
\DeclareMathOperator{\Cor}{Cor}
\DeclareMathOperator{\Ac}{at}
\DeclareMathOperator{\supp}{supp}
\DeclareMathOperator{\Hom}{Hom}
\DeclareMathOperator{\Pic}{Pic}
\DeclareMathOperator{\Gal}{Gal}
\DeclareMathOperator{\fc}{frac}
\DeclareMathOperator{\Ann}{Ann}
\DeclareMathOperator{\Mod}{Mod}
\DeclareMathOperator{\Cone}{Cone}
\DeclareMathOperator{\FI}{FI}
\DeclareMathOperator{\End}{End}
\DeclareMathOperator{\Alb}{Alb}
\DeclareMathOperator{\Ext}{Ext}
\DeclareMathOperator{\ab}{ab}
\DeclareMathOperator{\Jac}{Jac}
\DeclareMathOperator{\coker}{coker}
\DeclareMathOperator{\fr}{frac}
\DeclareMathOperator{\Int}{Int}
\let\Span\relax
\DeclareMathOperator{\Span}{Span}
\DeclareMathOperator{\Ran}{Ran}



%----Analysis
\newcommand{\summ}{\sum\limits}
% \newcommand{\norm}[1]{\left\lVert#1\right\rVert}
\newcommand{\thicc}{\bigg}
\newcommand{\eps}{\varepsilon}
\newcommand*\cls[1]{\overline{#1}}
\newcommand{\ind}{\mathbbm{1}}


%--------Theorem environments--------
\newtheorem{definition}{Definition}[]
\newtheorem{lemma}{Lemma}[]
\newtheorem{corollary}{Corollary}[]
\newtheorem{theorem}{Theorem}[]
\theoremstyle{remark}
\newtheorem*{claim}{Claim}


\newenvironment{solution}
{\begin{proof}[Solution]}
{\end{proof}}


\makeatletter
\newcommand{\thickhline}{%
    \noalign {\ifnum 0=`}\fi \hrule height 1pt
    \futurelet \reserved@a \@xhline
}
\newcolumntype{"}{@{\hskip\tabcolsep\vrule width 1pt\hskip\tabcolsep}}
\makeatother

% --------Problem environment--------
\setlength\parindent{0pt}
\setcounter{secnumdepth}{0}
\newcounter{partCounter}
\newcounter{homeworkProblemCounter}
\setcounter{homeworkProblemCounter}{1}


\newenvironment{homeworkProblem}[1][-1]{
    \ifnum#1>0
        \setcounter{homeworkProblemCounter}{#1}
    \fi
    \section{Problem \arabic{homeworkProblemCounter}}
    \setcounter{partCounter}{1}
    \stepcounter{homeworkProblemCounter}
}


%--------Metadata--------
\title{MATH 7310 Homework 9}
\author{James Harbour}

\begin{document}
\maketitle


\begin{homeworkProblem}[3]
  \textbf{(a)}: Let $(X,\Sigma)$ be a measurable space. Let $M(\Sigma)$ be the vector space of complex measures on $\Sigma$ with the total variation norm $\norm{\mu} = |\mu|(X)$. Show that $M(\Sigma)$ is a Banach space. \\
  \hspace{10pt}Suggestion: it may be helpful to use that for $\mu\in M(\Sigma)$ we ahve
  \[
    \sum_{n=1}^{\infty}|\mu(E_n)|\leq\norm{\mu}
  \]
  where $(E_n)_{n=1}^
  \infty$ is a sequence of pairwise disjoint elements of $\Sigma$ (this is a consequence of a prior problem on this homework).\\

  \textbf{(b)}: Fix a positive, $\sigma$-finite measure $\mu$ on $\Sigma$. Show that the map $J : L^1(X,\mu)\to M(\Sigma)$ given by $J(f) = f\dd{\mu}$ is a linear isometry with closed image.

  \begin{proof}
    Let  We wish to show that for $f,g\in L^1(X,\mu)$ and $\alpha\in\C$, $J(\alpha f+g) = \alpha J(f)+J(g)$, after which showing that $\norm{J(f)}_{M(\Sigma)} = \norm{f}_{L^1(\mu)}$ would imply that $J$ is a linear isometry. \\

    % TODO Linearity

    Let $f\in L^1(X,\mu)$. We compute
    \[
      \norm{J(f)}_{M(\Sigma)} = |J(f)|(X) = J(f)(X) = \int_{X}f\dd{\mu} = \norm{f}_{L^1(\mu)}.
    \]

    % TODO image is closed.

    Suppose that $(J(f_n))_{n=1}^{\infty}$ converges to $\nu$ in $M(\Sigma)$ where $(f_n)_{n=1}^{\infty}$ is in $L^1(X,\mu)$. So
    \[
      \norm{J(f_n)-\nu}_{M(\Sigma)}\xrightarrow{n\to \infty}0
    \]
    Suppose that $E\in\Sigma$ is null. Then as $J(f_n)\ll \mu$, $|J(f_n)|(E) = 0$ for all $n\in\N$. So by problem 2,
    \[
      |\nu|(E) \leq |J(f_n)(E-\nu(E))| + |J(f_n)(E)| \leq \norm{J(f_n)-\nu}_{M(\Sigma)}\xrightarrow{n\to \infty}0
    \]
    whence $|\nu|(E) = 0$ i.e. $E$ is null for $\nu$. Thus $\nu\ll \mu$. By the Lebesgue-Radon-Nikodym theorem, there exists some $f\in L^1(X,\mu)$ such that $\nu = f\dd{\mu} = J(f)$, so $\nu$ is in the image of $J$.
  \end{proof}

  \textbf{(c)}: Suppose that $\mu,\nu\in M(\Sigma)$, and let $\dd{\nu} = f\dd{\mu}+\dd{\lambda}$ with $\lambda\perp\mu$ be the Lebesgue-Radon-Nikodym decomposition. Show that
  \[
    \norm{\mu-\nu} = \norm{1-f}_{L^1(\mu)}+\norm{\lambda}.
  \]

  \begin{proof}
    Observe that
  \end{proof}
\end{homeworkProblem}

\begin{homeworkProblem}
  If $E$ is a Borel set in $\R^n$, the density $D_E(x)$ of $E$ at $x$ is defined as
  \[
    D_E(x) = \lim_{r\to0}\frac{m(E\cap B_r(x))}{m(B_r(x))},
  \]
  whenever the limit exists.\\

  \textbf{(a)}: Show that $D_E(x) = 1$ for a.e. $x\in E$ and $D_E(x) = 0$ for a.e. $x\in E^c$.

  \begin{proof}
    Define a new Borel measure $\nu$ by $\nu(F) = \mu(E\cap F)$ for Borel $F$. Then $\nu\ll m$ and observe that, for Borel sets $F$,
    \[
      \nu(F) = \int \ind_{E\cap F}\dd{m} = \int \ind_E \ind_F\dd{m} = \int_F \ind_E\dd{m},
    \]
    so $\frac{\dd{\nu}}{\dd{m}} = \ind_E$ $m$-a.e. by uniqueness in Lebesgue-Radon-Nikodym theorem. Moreover, as $\nu(F)\leq m(F)$ for all Borel $F$, it follows that $\nu$ is finite on compacts and thus regular, so by the Lebesgue differentiation theorem, the following limit exists for $m$-a.e. $x$ and is equal to
    \[
      D_E(x) = \lim_{r\to 0}\frac{\nu(B_r(x))}{m(B_r(x))} = \ind_E(x),
    \]
    whence the claim follows.
  \end{proof}

  \textbf{(b)}: Find examples of $E$ and $x$ such that $D_E(x)$ is a given number $\alpha\in (0,1)$, or such that $D_E(x)$ does not exist. \\

\end{homeworkProblem}


\begin{homeworkProblem}[5]
  Let $\psi:\R\to\R$ be given as $\psi = \ind_{[0,1/2)}-\ind_{[1/2,1]}$. For $n,k\in\Z$ define $h_{n,k}(t) = 2^{n/2}\psi(2^n t-k)$. Show that $\mc{E} = \{ 1\}\cup\{h_{n,k}:n\in \N\cup\{ 0\{,\, 0\leq k<2^n \}$ is an orthonormal basis for $L^2([0,1])$.
\end{homeworkProblem}

\end{document}

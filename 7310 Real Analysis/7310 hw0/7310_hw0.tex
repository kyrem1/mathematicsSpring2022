\documentclass[12pt,letterpaper]{article}

%%%%%%%%%%%%%%%%%%%%%%%% Packages %%%%%%%%%%%%%%%%%%%%%%%%
\usepackage{tikz,amsmath, amsthm, amssymb}
\usetikzlibrary{calc}%
\usepackage{graphicx}
\usepackage{hhline}
\usepackage{amssymb}
\usepackage{array}
\usepackage{multicol}
\usepackage{tikz}
\usepackage{mathtools}
\usepackage{enumerate}
\usepackage{delarray}
\usepackage{mathtools}
\usepackage{tikz-cd}
\usepackage{fullpage}
\usepackage{faktor} % For quotients

% \usepackage{quiver}

% Special Algorthm Packages
\usepackage[plain]{algorithm}
\usepackage{algpseudocode}
\usepackage{amsfonts}
\newcommand{\alg}[1]{\textsc{\bfseries \footnotesize #1}}

% \usepackage{quiver}
\usetikzlibrary{cd}
\usepackage[linguistics]{forest}
%%
\pagestyle{empty}%

\setlength{\hoffset}{-1.54cm}
\setlength{\voffset}{-1.54cm}

\setlength{\topmargin}{0pt}
\setlength{\headsep}{0pt}
\setlength{\headheight}{0pt}

\setlength{\oddsidemargin}{0pt}

\setlength{\textwidth}{195mm}
\setlength{\textheight}{250mm}


%%%%%%%%%%%%%%%%%%%%%%%%% Macros %%%%%%%%%%%%%%%%%%%%%%%%%

\newcommand{\sub}{\subseteq}
\newcommand{\bigzero}{\mbox{\normalfont\Large\bfseries 0}} % For block matrices


\newcommand{\RR}{\mathbb{R}}
\newcommand{\QQ}{\mathbb{Q}}
\newcommand{\NN}{\mathbb{N}}
\newcommand{\CC}{\mathbb{C}}
\newcommand{\ZZ}{\mathbb{Z}}
\newcommand{\FF}{\mathbb{F}}
\newcommand{\PP}{\mathbb{F}}
\newcommand{\lcm}{\text{lcm}}
\newcommand{\mc}[1]{\mathcal{#1}}
\newcommand{\mf}[1]{\mathfrak{#1}}
\newcommand{\sO}{\mathcal{O}}
\newcommand{\cyclic}[1]{\langle#1\rangle}
\newcommand{\units}[1]{#1 ^{\times}}
\newcommand{\la}{\langle}
\newcommand{\ra}{\rangle}
\DeclareMathOperator{\Hom}{Hom}
\DeclareMathOperator{\Ann}{Ann}
\DeclareMathOperator{\End}{End}
\DeclareMathOperator{\Aut}{Aut}
\DeclareMathOperator{\Char}{char}
\DeclareMathOperator{\Gal}{Gal}
\DeclareMathOperator{\id}{id}
\DeclareMathOperator{\Irr}{Irr}


\DeclareMathOperator{\SL}{SL}
\DeclareMathOperator{\GL}{GL}

\DeclareMathOperator{\U}{U}
\DeclareMathOperator{\SU}{SU}

% \newcommand{\GL}[0]{\text{GL}}
% \newcommand{\SL}[2]{\text{SL}_{#1}(#2)}
\newcommand{\tr}[0]{\text{Tr}}
\DeclareMathOperator{\im}{im}
\DeclareMathOperator{\Ext}{Ext}
\DeclareMathOperator{\Tor}{Tor}
\newcommand{\coker}{\text{coker}}


\newcommand{\dd}[2][]{\frac{\partial^{#1}}{\partial {#2}^{#1}}}
\newcommand{\summ}{\sum\limits}


\newtheorem{definition}{Definition}[]
\newtheorem{lemma}{Lemma}[]
\newtheorem{corollary}{Corollary}[]
\newtheorem{theorem}{Theorem}[]

\newenvironment{solution}
  {\begin{proof}[Solution]}
  {\end{proof}}


\newcommand{\norm}[1]{\left \vert \left \vert #1 \right \vert \right \vert}
\newcommand{\thicc}{\bigg}
\newcommand{\overbar}[1]{\mkern 1.5mu\overline{\mkern-1.5mu#1\mkern-1.5mu}\mkern 1.5mu}
\newcommand{\eps}{\varepsilon}

%%%%%%%%%%%%%%%%%%%%%%%%%%%%%%%%%%%%%%%%%%%%%%%%%%%%%%%%%%

\makeatletter
\newcommand{\thickhline}{%
    \noalign {\ifnum 0=`}\fi \hrule height 1pt
    \futurelet \reserved@a \@xhline
}
\newcolumntype{"}{@{\hskip\tabcolsep\vrule width 1pt\hskip\tabcolsep}}
\makeatother

%%%%%%%%%%%%%%%%% Problem environment %%%%%%%%%%%%%%%%%%%%
\setlength\parindent{0pt}
\setcounter{secnumdepth}{0}
\newcounter{partCounter}
\newcounter{homeworkProblemCounter}
\setcounter{homeworkProblemCounter}{1}
% \nobreak\extramarks{Problem \arabic{homeworkProblemCounter}}{}\nobreak{}
%
% Homework Problem Environment
%
% This environment takes an optional argument. When given, it will adjust the
% problem counter. This is useful for when the problems given for your
% assignment aren't sequential.

\newenvironment{homeworkProblem}[1][-1]{
    \ifnum#1>0
        \setcounter{homeworkProblemCounter}{#1}
    \fi
    \section{Problem \arabic{homeworkProblemCounter}}
    \setcounter{partCounter}{1}
    \stepcounter{homeworkProblemCounter}
}

\title{MATH 7310 Homework 0}
\author{James Harbour}


\begin{document}
\maketitle

\begin{homeworkProblem}
  Let $J$ be an infinite set, and $(t_j)_{j\in J}$ nonnegative real numbers. We define $\sum_{j\in J} t_j = \sup_F \sum_{j\in F} t_j$ where the supremum is over all finite subsets of $J$, and is equal to $\infty$ if $\left\{ \sum_{j\in F}t_j : F\sub J\ \textrm{is finite}\right\}$ is not bounded above. \\

  \textbf{(i)} Suppose that $\sum_{j\in J}t_j < \infty$. Prove that for every $\eps > 0$, there is a finite $F\sub J$ so that $\sum_{j\in J\setminus F}t_j < \eps$. \\

  \begin{proof}
    Let $\eps > 0$ an $J_0 = \{ j\in J : t_j > 0\}$. By proposition 0.20, as $\sum_{j\in J} t_j < \infty$, $J_0$ is countably infinite. Moreover, letting $g:\NN \to J_0$ be a bijection, proposition 0.20 gives that
    \[
      \sum_{n=1}^{\infty} t_{g(n)} = \sum_{j\in J} t_j < \infty
    \]
    As this sum is nonnegative and converges, there exists a $k\in \NN$ such that
    \[\sum_{n=k+1}^{\infty} t_{g(n)} < \eps.\]
    Letting $F = g(\{ 1,2,\ldots, k\})$, it follows that
    \[\sum_{j\in J\setminus F} t_j= \sum_{n=k+1}^{\infty} t_{g(n)} < \eps.\]
  \end{proof}

  \textbf{(ii)} Suppose that $(\alpha_j)_{j\in J}$ are complex numbers and $\sum_{j\in J} |\alpha_j| < \infty$. Suppose further that $J_0 = \{ j\in J : \alpha_j \neq 0\}$ is infinite. Suppose that $\phi: \NN \to J_0,\ \psi:\NN\to J_0$ are two bijections. Prove that
  \[
    \sum_{n=1}^{\infty} \alpha_{\phi(n)} = \sum_{n=1}^{\infty} \alpha_{\psi(n)}.
  \]


\end{homeworkProblem}


\begin{homeworkProblem}
  It follows from Problem 1 that if $(\alpha_j)_{j\in J}$ are complex numbers and $\sum_{j\in J} |\alpha_j| < \infty$, we may define $\sum_{j\in J}$ as follows: let $J_0 = \{ j:\alpha_j \neq 0\}$. If $J_0$ is finite, then $\sum_{j\in J} = \sum_{j\in J_0}$. If $J_0$ is infinite, choose a bijection $\phi: \NN \to J_0$, and define
  \[
    \sum_{j\in J}\alpha_j = \sum_{n=1}^{\infty}\alpha_{\phi(n)}.
  \]
  Suppose that $(\alpha_j)_{j\in J}$ are complex numbers and $\sum_{j\in J}|\alpha_j| < \infty$. Show that $\sum_{j\in J}|\alpha_j|$ is the unique complex number $s$ satisfying the following property. For every $\eps>0$, there is a finite set $F\sub J$ so that if $F\sub E\sub J$ and $E$ is finite, then
  \[
    \left|s-\sum_{j\in E}\alpha_j \right| < \eps.
  \]
\end{homeworkProblem}


\begin{homeworkProblem}
  Suppose that $I, J$ are sets, and $(a_{ij})_{i\in I}{j\in J}$ are nonnegative real numbers. Prove that
  \[
    \sum_{j\in J}\left(\sum_{i\in I} a_{ij}\right) = \sum_{(i,j)\in I\times J} a_{ij} = \sum_{i\in I} \left(\sum_{j\in J} a_{ij}\right)
  \]
\end{homeworkProblem}






\end{document}

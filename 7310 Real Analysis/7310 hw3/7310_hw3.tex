\documentclass[12pt,letterpaper]{article}

%--------Packages--------
\usepackage{amsmath, amsthm, amssymb}
\usepackage{xspace}
\usepackage{graphicx}
\usepackage{hhline}
\usepackage{amssymb}
\usepackage{array}
\usepackage{braket}
\usepackage{multicol}
\usepackage{mathtools}
\usepackage{enumerate}
\usepackage{delarray}
\usepackage{mathtools}
\usepackage{fullpage}
\usepackage{faktor} % For quotients
\usepackage{mathrsfs}

% \usepackage{quiver}
\usepackage[linguistics]{forest}




%--------Page Setup--------

\pagestyle{empty}%

\setlength{\hoffset}{-1.54cm}
\setlength{\voffset}{-1.54cm}

\setlength{\topmargin}{0pt}
\setlength{\headsep}{0pt}
\setlength{\headheight}{0pt}

\setlength{\oddsidemargin}{0pt}

\setlength{\textwidth}{195mm}
\setlength{\textheight}{250mm}


%--------Macros--------

\newcommand{\sub}{\subseteq}
\newcommand{\lcm}{\text{lcm}}
\newcommand{\mc}[1]{\mathcal{#1}}
\newcommand{\mf}[1]{\mathfrak{#1}}
\newcommand{\ms}[1]{\mathscr{#1}}
\newcommand{\sO}{\mathcal{O}}
\newcommand{\cyclic}[1]{\langle#1\rangle}
\newcommand{\units}[1]{#1 ^{\times}}
\newcommand{\la}{\langle}
\newcommand{\ra}{\rangle}
\newcommand{\lr}[1]{\left(#1\right)}
%----Switch phi and varphi
\let\temp\phi
\let\phi\varphi
\let\varphi\temp

\newcommand{\C}{\mathbb{C}}
\newcommand{\F}{\mathbb{F}}
\newcommand{\N}{\mathbb{N}\xspace}
\newcommand{\I}{\mathbb{I}\xspace}
\newcommand{\R}{\mathbb{R}\xspace}
\newcommand{\Z}{\mathbb{Z}\xspace}
\newcommand{\Q}{\mathbb{Q}\xspace}
\newcommand{\G}{\mathbb{G}\xspace}
\DeclareMathOperator{\Spec}{Spec}
\DeclareMathOperator{\res}{res}
\DeclareMathOperator{\Tr}{Tr}
\DeclareMathOperator{\ord}{ord}
\DeclareMathOperator{\Sym}{Sym}
\DeclareMathOperator{\dv}{div}
\DeclareMathOperator{\alb}{alb}
\DeclareMathOperator{\img}{Im}
\DeclareMathOperator{\et}{et}
\DeclareMathOperator{\ck}{coker}
\DeclareMathOperator{\Reg}{Reg}
\DeclareMathOperator{\Cor}{Cor}
\DeclareMathOperator{\Ac}{at}
\DeclareMathOperator{\supp}{supp}
\DeclareMathOperator{\Hom}{Hom}
\DeclareMathOperator{\Pic}{Pic}
\DeclareMathOperator{\Gal}{Gal}
\DeclareMathOperator{\fc}{frac}
\DeclareMathOperator{\Ann}{Ann}
\DeclareMathOperator{\Mod}{Mod}
\DeclareMathOperator{\Cone}{Cone}
\DeclareMathOperator{\FI}{FI}
\DeclareMathOperator{\End}{End}
\DeclareMathOperator{\Alb}{Alb}
\DeclareMathOperator{\Ext}{Ext}
\DeclareMathOperator{\ab}{ab}
\DeclareMathOperator{\Jac}{Jac}
\DeclareMathOperator{\coker}{coker}
\DeclareMathOperator{\fr}{frac}
\DeclareMathOperator{\Int}{Int}


%----Analysis
\newcommand{\dd}[2][]{\frac{\partial^{#1}}{\partial {#2}^{#1}}}
\newcommand{\summ}{\sum\limits}
\newcommand{\norm}[1]{\left \vert \left \vert #1 \right \vert \right \vert}
\newcommand{\thicc}{\bigg}
\newcommand{\eps}{\varepsilon}
\newcommand*\cls[1]{\overline{#1}}


%--------Theorem environments--------
\newtheorem{definition}{Definition}[]
\newtheorem{lemma}{Lemma}[]
\newtheorem{corollary}{Corollary}[]
\newtheorem{theorem}{Theorem}[]
\theoremstyle{remark}
\newtheorem*{claim}{Claim}


\newenvironment{solution}
{\begin{proof}[Solution]}
{\end{proof}}


\makeatletter
\newcommand{\thickhline}{%
    \noalign {\ifnum 0=`}\fi \hrule height 1pt
    \futurelet \reserved@a \@xhline
}
\newcolumntype{"}{@{\hskip\tabcolsep\vrule width 1pt\hskip\tabcolsep}}
\makeatother

% --------Problem environment--------
\setlength\parindent{0pt}
\setcounter{secnumdepth}{0}
\newcounter{partCounter}
\newcounter{homeworkProblemCounter}
\setcounter{homeworkProblemCounter}{1}


\newenvironment{homeworkProblem}[1][-1]{
    \ifnum#1>0
        \setcounter{homeworkProblemCounter}{#1}
    \fi
    \section{Problem \arabic{homeworkProblemCounter}}
    \setcounter{partCounter}{1}
    \stepcounter{homeworkProblemCounter}
}


%--------Metadata--------
\title{MATH 7310 Homework 3}
\author{James Harbour}

\begin{document}
\maketitle

\begin{homeworkProblem}
  Let $F$ be increasing and right continuous, and let $\mu_F$ be the associated measure. Then $\mu_F(\{a\}) = F(a)-F(a-)$, $\mu_F([a,b)) = F(b-)- F(a-)$, $\mu_F([a,b]) = F(b)- F(a-)$, and $\mu_F((a,b)) = F(b-)-F(a)$.

  \begin{proof}
    Let $a,b\in\R$ with $a<b$. Then, take $x<a$. Then by definition, as $(x,a]$ is an h-interval,
    \[
      \mu_F(\{a\})\leq F(a)-F(x) \implies F(x)\leq F(a)-\mu_F(\{a\}).
    \]
    Hence, taking the infimum over all such $x<a$, we have that
    \[
      F(a-)\leq F(a)-\mu_F(\{a\}) \implies \mu_F(\{a\}) \leq F(a)-F(a-).
    \]
    On the other hand, note that
    \[
      \mu_F(\{a\}) = \inf\lr{\sum_{i=1}^{\infty}\mu_F((a_i,b_i]): a\in\bigcup_{i=1}^{\infty}(a_i,b_i]} \leq \inf\lr{\sum_{i=1}^{\infty}\mu_F((a_i,a]): a\in\bigcup_{i=1}^{\infty}(a_i,a]} = \inf\{\mu_F((a_i,a]): a_i<a\} = F(a)-F(a-).
    \]


    Also, observe that
    \begin{align*}
      F(b)-F(a) = \mu_F((a,b]) &= \mu_F(([a,b)\setminus\{a\})\sqcup\{b\}) = \mu_F([a,b))-\mu_F(\{a\})+\mu_F(\{b\}) \\ &= \mu_F([a,b)) - F(a)+F(a-)+F(b)-F(b-) \implies \mu_F([a,b)) = F(b-)- F(a-)
    \end{align*}

    \begin{align*}
      F(b)-F(a) = \mu_F((a,b]) &= \mu_F(([a,b]\setminus\{a\}) = \mu_F([a,b]) - F(a)+F(a-) \implies \mu_F([a,b]) = F(b)-F(a-)
    \end{align*}

    \begin{align*}
      \mu_F((a,b)) = \mu((a,b]\setminus\{b\}) = F(b)-F(a) - F(b)+F(b-) = F(b-) - F(a).
    \end{align*}
  \end{proof}
\end{homeworkProblem}

\begin{homeworkProblem}[2]
  Let $(X,\Sigma,\mu)$ be a measure space. We say that $E\sub X$ is \emph{an atom} if
  \begin{itemize}
    \item $E\in\Sigma$,
    \item $\mu(E)>0$,
    \item $\{\mu(F) : F\sub E, F\in \Sigma\} = \{ 0, \mu(E)\}$.
  \end{itemize}
  We say the $\mu$ is \emph{diffuse} if it has no atoms.\\

  \textbf{(a)} Let $(X,d,\mu)$ be a metric measure space. Assume that $\mu$ is outer regular, and that
  \[
    \mu(E) = \sup\{\mu(K):K\sub E\text{ compact}\}\text{ for all Borel } E\sub X.
  \]
  If $\mu(\{ p\}) = 0$ for all $p\in X$, show that $\mu$ is diffuse.

  \begin{proof}
    Note that, for all $p\in X$, by outer regularity we have that
    \[
      0 = \mu(\{ p\}) = \inf\{ \mu(U): U\supset \{p\} \text{ open}\}.
    \]

    Suppose, for the sake of contradiction, that $\mu$ is not diffuse. Then there exists an atom $E\sub X$. As $0<\mu(E) = \sup\{\mu(K):K\sub E\text{ compact}\}$ and $\{\mu(K):K\sub E\text{ compact}\} = \{0,\mu(E)\}$, there exists a $K\sub E$ compact such that $\mu(K) = \mu(E) > 0$. \\

    For each $p\in K$, as $0 = \inf\{ \mu(U): U\supset \{p\} \text{ open}\}$, there exists an open $U_p$ such that $\mu(U_p)<\mu(K)$. Then $\{U_p: p\in K\}$ is an open cover for $K$, so there exist $p_1,\ldots, p_n\in K$ such that $\{ U_{p_1},\ldots,U_{p_n}\}$ covers $K$. Then, as $E$ is an atom, $\mu(U_p\cap K)\leq \mu(U_p)<\mu(K)\implies \mu(U_p\cap K) = 0$. But then, as $K = \bigcup_{i=1}^{n}U_{p_i}\cap K$,
    \[
      \mu(K) \leq \mu\lr{\bigcup_{i=1}^{n}U_{p_i}\cap K} \leq \sum_{i=1}^{n}\mu(U_{p_i}\cap K) = 0,
    \]
    contradicting that $\mu(K) > 0$.
  \end{proof}

  \textbf{(b)} Let $F:\R\to\R$ be an increasing, right-continuous function. Show that for $p\in\R$ we have that $\{ p\}$ is an atom of $\mu_F$ if and only if $F$ is discontinuous at $p$. Show that $\mu_F$ is diffuse if and only if $F$ is continuous.

  \begin{proof}
    Suppose that $\{ p\}$ is an atom of $\mu_F$. Then $\mu({p})>0$. By problem (1), $0 < \mu(\{ p\}) = F(p)-F(p-) \implies F(p) > F(p-)$, whence $F(p)\neq F(p-)$ so $F$ is discontinuous at $p$. \\

    Coversely, suppose that $F$ is discontinuous at $p$. As $F$ is already right continous, it follows that $F(p) \neq F(p-)$, as otherwise $F$ would be continous at $p$. Then $\mu({p}) = F(p)-F(p-) > 0$ as \emph{a priori} $F(p)\geq F(p-)$, so $\{ p\}$ is an atom. \\

    Now suppose that $\mu_F$ is diffuse. Then, for $p\in \R$, $\{ p\}$ is not an atom whence $\mu(\{ p\}) = 0$ so part (a) implies that $0 = \mu(\{ p\}) = F(p)-F(p-)$. Thus $F(p+)=F(p)=F(p-)$, so $F$ is continuous. \\

    Conversely, suppose that $F$ is continuous. Then for all $p\in \R$, $\mu(\{ p\}) = F(p)-F(p-) = F(p)-F(p) = 0$. As $\mu_F$ is outer regular and inner regular with respect to compacts, by part (a) $\mu_F$ we have that is diffuse.
  \end{proof}
\end{homeworkProblem}


\begin{homeworkProblem}[3]
  Let $(X,\Sigma,\mu)$ be a $\sigma$-finite measure space.

  \textbf{(i)} Suppose  that $(E_j)_{j\in J}$ is a collection of sets with $E_j\in\Sigma$ for all $j\in J$ and with $\mu(E_j)>0$ for all $j\in J$, and so that $\mu(E_j\cap E_k) = 0$ for all $j\neq k$ in $J$. Show that $J$ is countable.

  \begin{proof}
    Without loss of generality, assume that $X=\bigsqcup_{n=1}^{\infty}X_n$ where $X_n\in\Sigma$, $\mu(X_n>0)$ for all $n\in\N$, and $X_i\cap X_j = \emptyset$ for $i\neq j$. \\

    Suppose, for the sake of contradiction, that $J$ is uncountable. For $j\in J$, note that
    \[
      0<\mu(E_j) = \sum_{n=1}^{\infty}\mu(E_j\cap X_n),
    \]
    whence there exists an $n_j\in\N$ such that $\mu(E_j\cap X_n)>0$. As there can only be countably many such $n_j$'s and there are uncountably many $E_j$'s, there exists a $k\in\N$ and $J_0\sub J$ uncountable such that $\mu(E_j\cap X_{k})>0$ for all $j\in J_0$. By a pigeonhole argument, there exists a $b>0$ and an infinite $J_0'\sub J_0$ such that $\mu(E_j\cap X_k) > b$ for all $j\in J_0'$. \\

    Choose a countable sequence $(j_l)_{l=1}^{\infty}$ in $J_0$ such that $j_{l}\neq j_{s}$ for $l\neq s$. For $n\in\N$, set $F_n = E_j\cap X_k$. Note that, for $l\neq s$, we have that $\mu(F_l\cap F_s) = 0$. \\


    We claim that $\mu(\sum_{l=1}^{n}F_l) = \sum_{l=1}^{n}\mu(F_l)$ for all $n\in\N$ by induction. Observe that
    \[
      \mu(F_1\cup F_2) = \mu(F_1)+\mu(F_2)-\mu(F_1\cap F_2) = \mu(F_1)+\mu(F_2).
    \]
    Now fix $n>2$ and suppose that $\mu(\sum_{l=1}^{n-1}F_l) = \sum_{l=1}^{n-1}\mu(F_l)$. Observe that
    \[
      \mu\lr{F_n \cap \bigcup_{j=1}^{n-1}F_j} = \mu\lr{\bigcup_{j=1}^{n-1}F_n\cap F_j} \leq \sum_{j=1}^{n-1}\mu(F_n\cap F_j) = 0,
    \]
    so
    \[
      \mu\lr{\bigcup_{j=1}^{n}F_j} = \mu\lr{F_n\cup\bigcup_{j=1}^{n-1}F_j} = \mu(F_n)+\sum_{j=1}^{n-1}\mu(F_j) - \mu\lr{F_n\cap\bigcup_{j=1}^{n-1}F_j} = \sum_{j=1}^{n}\mu(F_j).
    \]
    By induction, the claim holds for all $n\in\N$. \\

    Observe that, for all $n\in\N$,

    \[
      \mu\lr{\bigcup_{j=1}^{\infty}F_j} \geq \mu\lr{\bigcup_{j=1}^{n}F_j} = \sum_{j=1}^{n}\mu(F_j) \geq n\cdot b.
    \]
    Hence $\mu\lr{\bigcup_{j=1}^{\infty}F_j} = +\infty$, contradicting the fact that $\mu(X_k)<+\infty$.

  \end{proof}

  \textbf{(ii)} Let $(\Omega, \rho)$ be the metric space defined in Problem 12 of Chapter 1 of Folland. For $E\in\Sigma$, let $[E]$ be its equivalence class in $\Omega$. Show that
  \[
    \{[E]:E\sub X \text{ is an atom}\},
  \]
  is countable.

  \begin{proof}
    Let $\sim$ be the equivalence relation $E\sim F\iff \mu(E\Delta F) = 0$. Let $\mc{E} = \{ E:E\sub X \text{ is an atom}\}$. Let $\pi:\mc{E}\to\mc{E}/\sim$ be the canonical surjection. By the axiom of choice, there exists a section $s:\mc{E}/\sim\to \mc{E}$ such that $\pi\circ s = id_{\mc{E}/\sim}$. Take $E\neq F\in s(\mc{E}/\sim)$. Then, as $s$ is injective, $[E] \neq [F]$, so $\mu(E_i\Delta E_j)>0$. \\

    Suppose, without loss of generality, that $\mu(E_i\setminus E_j)>0$. Then, as $E_i$ is an atom, $\mu(E_i\setminus E_j) = \mu(E_i)$. Hence
    \[
      \mu(E_i) = \mu(E_i\setminus E_j) + \mu(E_i\cap E_j) = \mu(E_i) + \mu(E_i\cap E_j) \implies \mu(E_i\cap E_j) = 0
    \]
    Hence, $s(\mc{E}/\sim)$ has the properties of the collection in part (i), so $s(\mc{E}/\sim)$ is countable whence injectivity implies that $\mc{E}/\sim$ is countable.
  \end{proof}

\end{homeworkProblem}

\begin{homeworkProblem}[4]
  Let $(X,\Sigma,\mu)$ be a diffuse $\sigma$-finite measure space. For $A\in\Sigma$, show that:
  \[
    \{ \mu(B): B\sub A, B\in\Sigma\} = [0,\mu(A)].
  \]
  Suggestions: Reduce to the finite case. It might be helpful to first show that for every $E\in\Sigma$ with $\mu(E)>0$, we have $0 = \inf\{ \mu(B): B\sub E\text{ and }\mu(B)>0\}$.

  \begin{proof}\ \\
    (\emph{reduction to finite case}): Suppose we have shown the claim for finite measure spaces. Write $X = \bigcup_{i=1}^{\infty}X_i$ where $X_i\in\Sigma$ and $\mu(X_i) <+\infty$, and without loss of generality the $X_i$'s are pairwise disjoint.  \\

    If $\mu(E)<+\infty$, then we are done as we assumed that we have already shown the finite case. So, suppose that $\mu(E) = +\infty$. Then $E\sub E$ is a witness for $\mu(E) = +\infty$, so take $b\in (0,+\infty)$.

    As
    \[
      +\infty = \mu(E) = \sum_{i=1}^{\infty}\mu(E\cap X_i),
    \]
    there exist $k,l\in\N$ such that $\sum_{i=k}^{l}\mu(E\cap X_i) > b$. Noting that $\mu(\bigsqcup_{i=k}^{l}E\cap X_i) = \sum_{i=k}^{l}\mu(E\cap X_i)<+\infty$, by the finite case there exists a $B\sub \bigsqcup_{i=k}^{l}E\cap X_i$ with $B\in\Sigma$ such that $\mu(B) = b$. \\

    \vspace{0.5cm}
    (\emph{finite case}): Suppose that $E\in \Sigma$ with $\mu(E)>0$. Since $\mu$ is diffuse, there exists a $B_1\sub E$ such that $B_1\in \Sigma$ and $0<\mu(B_1)<\mu(E)$. Note that either $\mu(B_1)$ or $\mu(E\setminus B_1)$ is less than $2^{-1}\mu(E)$, so without loss of generality assume that $\mu(B_1)<2^{-1}\mu(E)$. Now, again as $\mu$ is diffuse, there exists a $B_2\sub B_1$ such that $B_2\in\Sigma$ and $0<\mu(B_2)<\mu(B_1) <\mu(E)$. Again, we may assume without loss of generality that $\mu(B_2)<2^{-1}\mu(B_1)<2^{-2}\mu(E)$. Continuing as such, we obtain a decreasing sequence of sets $E\supset B_1\supset B_2 \supset\cdots$ such that $0<\mu(B_n)<2^{-n}\mu(E)$. It follows that
    \begin{equation}\label{eq:smallsets}
      0 = \inf\{ \mu(B): B\sub E\text{ and }\mu(B)>0\}.
    \end{equation}

    Suppose, for the sake of contradiction, that the claim is false. Then there exists an $A\in\Sigma\setminus\{\emptyset\}$ and $b\in (0,\mu(A))$ such that $\mu(B)\neq b$ for all $B\sub A$ with $B\in\Sigma$.

    We proceed via transfinite recursion.\\


    First, note that we may choose $B_0$ such that $0<\mu(B_0)\leq b$. If $\mu(B_0) = b$ then stop; otherwise, we have that $0<\mu(B_0)<b$. Suppose now that $\alpha$ is an ordinal and we have constructed $(B_\eta)_{\eta<\alpha}$ pairwise disjoint elements of $\Sigma$ which are subsets of $A$ such that $\mu(B_\eta)>0$ for all $\eta\in\alpha$,
    % $\bigsqcup_{\eta\in\alpha}B_\eta \in \Sigma$,
    and $b - \sum_{\eta\in\alpha}\mu(B_\eta)>0$. By \eqref{eq:smallsets}, there exists a $B_\alpha\in \Sigma$ with $B_\alpha\sub A\setminus \bigsqcup_{\eta\in\alpha}B_\eta$ such that
    \[
      0<\mu(B_\alpha)\leq b - \sum_{\eta\in\alpha}\mu(B_\eta).
    \]
    If $\mu(B_\alpha) = b - \sum_{\eta\in\alpha}\mu(B_\eta)$, stop; otherwise, we have $0<\sum_{\eta\leq\alpha}\mu(B_\eta) < b$.\\


    We claim that this recursion halts at some countable ordinal. Suppose, for the sake of contradiction, that this recursion does not halt at some countable ordinal. Then we reach $\omega_1$, so we have pairwise disjoint subsets $(B_\eta)_{\eta<\omega_1}$ of $A$ in $\Sigma$ such that $\mu(B_\eta)>0$ for all $\eta<\omega_1$. Noting that each $\mu(B_\eta)$ is finite, we have that
    \[
      \{\eta<\omega_1 : \mu(B_\eta) > 0\} = \bigcup_{n=1}^{\infty}\{\eta<\omega_1 : \frac{1}{n}\leq\mu(B_\eta)<\frac{1}{n-1}\}
    \]
    where $1/0:=+\infty$. By uncountability of $\omega_1$, there exists an $n\in\N$ such that $\{\eta<\omega_1 : \frac{1}{n}\leq\mu(B_\eta)<\frac{1}{n-1}\}$ is infinite. Take a countable sequence $\eta_1,\eta_2,\ldots$  in $\{\eta<\omega_1 : \frac{1}{n}\leq\mu(B_\eta)<\frac{1}{n-1}\}$ with $\eta_i\neq\eta_j$ for $i\neq j$. Then $\bigsqcup_{i=1}^{\infty}B_{\eta_i}\in \Sigma$, whence for all $N\in \N$,
    \[
      \mu\lr{\bigsqcup_{i=1}^{\infty}B_{\eta_i}} = \sum_{i=1}^{\infty}\mu(B_{\eta_i}) \geq \sum_{i=1}^{N}\mu(B_{\eta_i}) \geq \frac{N}{n}
    \]
    so $\mu(\bigsqcup_{i=1}^{\infty}B_{\eta_i})=+\infty$, contradicting that $\mu(A)<+\infty$. \\

    Hence, the recursion halts at some countable ordinal $\alpha$. Then $\sum_{\eta\in\alpha}\mu(B_\eta) = \sum_{\eta<\alpha}\mu(B_\eta) = b$. Let $\phi:\N\to\alpha$ be a bijection. By nonnegativity, $b = \sum_{\eta\in\alpha}\mu_{B_\eta}=\sum_{i=1}^{\infty}\mu(B_{\phi(i)})$. Moreover, $\bigsqcup_{i=1}^{\infty}B_{\phi(i)}\in \Sigma$, so
    \[
      b = \sum_{i=1}^{\infty}\mu(B_{\phi(i)}) = \mu\lr{\bigsqcup_{i=1}^{\infty}B_{\phi(i)}}
    \]
    as desired.
  \end{proof}
\end{homeworkProblem}

\begin{homeworkProblem}[5]
  Let $E$ be a Lebesgue measurable set.\\

  % Definition of $N$: \\
  % 
  % Define an equivalence relation $\sim$ on $[0,1)$ by $x\sim y$ if and only if $x-y\in \Q$. Set $Y=[0,1]/\sim$ and let $\pi:[0,1]\to Y$ be the canonical projection. By the axiom of choice, there exists a section $s:Y\to[0,1]$ s.t. $\pi\circ s = id_{Y}$, so $[s([x])] =[x]$ for $[x]\in Y$. Moreover, $s$ is injective. Take $N = s(Y)$. For $a\in\Q$, as $[a] = [0,1]\cap \Q$, there exists a $q\in\Q$ such that $s([a]) = q$ for all $a\in$
  % \[ x\sim s([x]) \implies s([x]) - x \in \Q\]
  % Suppose $x\in N+q\cap N+q'$ with $q,q'\in\Q\cap [-1,1]$. Then $x-q\in N, x-q'\in N$, so there exist $c,c'\in Y$ such that $x-q = s(c)$ and $x-q'=s(c')$. Write $c = [a]$, $c'=[a']$ for some $a,b\in [0,1]$. The  $[x-q] = [s(c)] = c = [a]$ and\\
  % \[[0,1] = \bigsqcup_{x\in N}[x],\quad \]



  \textbf{(a)} Let $E\sub N$ where $N$ is the nonmeasurable set described in section 1.1. Prove that $m(E) = 0$. \\

  \begin{proof}
    As in class, we have that $\bigsqcup_{q\in [-1,1]\cap\Q}E+q \sub \bigsqcup_{q\in [-1,1]\cap\Q}N+q\sub [-2,3]$, so
    \[
      \sum_{q\in [-1,1]\cap\Q} m(E) = \sum_{q\in [-1,1]\cap\Q} m(E+q)  = m\lr{\bigsqcup_{q\in [-1,1]\cap\Q}E+q} \leq 5
    \]
    so $\sum_{q\in [-1,1]\cap\Q} m(E)$ is whence we must have that  $m(E) = 0$.
  \end{proof}

  \textbf{(b)} Prove that if $m(E)>0$, then $E$ contains a nonmeasurable set. \\

  \begin{proof}
    As $m(E)>0$, $E\neq\emptyset$ so there exists a $k\in\Z$ such $(E+k)\cap [0,1] \neq \emptyset$. \\

    Performing the same construction of the Vitali set on $(E+k)\cap [0,1]$, we obtain a set $V\sub (E+k)\cap [0,1]$ which is non measurable. Hence, by translation invariance, $V-k$ is non-measurable.
  \end{proof}

\end{homeworkProblem}


\begin{homeworkProblem}
  \textbf{(a)} Let $\mc{E}_q$ be the family of $h$-intervals in $\R$ with rational endpoints. Show that $\mc{E}_q$ is an elementary family and that the $\sigma$-algebra generated by this elementary family is all Borel subsets of $\R$. \\

  \begin{proof}
    We have shown in previous homework that $\mc{E}_q$ is indeed an elementary family. \\

    Take $(a,b]\in \mc{E}_q$. As $(a,b] = \bigcap_{n=1}^{\infty}(a, b+\frac{1}{n})$, $(a,b]\in \mc{B}_\R$. Thus $\Sigma(\mc{E}_q)\sub\mc{B}_\R$.\\

    On the other hand, any  $(\alpha,\beta)\in\mc{B}_\R$ can be written as a countable union of open rational intervals, so $(\alpha,\beta)\in\Sigma(\mc{E}_q)$. Hence, as open intervals generate $\mc{B}_\R$, it follows that $\mc{B}_\R\sub \Sigma(\mc{E}_q)$.
  \end{proof}

  \textbf{(b)} Suppose that $\mu:\mc{B}_\R\to[0,\infty]$ is a measure such that $\mu(E+x) = \mu(E)$ for all $E\in\mc{B}_\R, x\in \R$. Assume that $0<\mu((0,1])<+\infty$. Show that $\mu(E)=\mu((0,1])m(E)$ for all $E\in\mc{B}_\R$.

  \begin{proof}
    First, suppose that $n\in \N$. Observe that
    \[
      \mu((0,1]) = \mu\lr{\bigsqcup_{j=0}^{n-1}(\frac{j}{n},\frac{j+1}{n}]} = \sum_{j=0}^{n-1}\mu((\frac{j}{n},\frac{j+1}{n}]) = n\cdot\mu((0,\frac{1}{n}])\implies \mu((0,\frac{1}{n}]) = \frac{1}{n}\mu((0,1]) = \mu((0,1])m((0,\frac{1}{n}])
    \]
    Now let $q=\frac{k}{n}\in\Q$. Then
    \[
      \mu((0,q]) = \sum_{j=0}^{k}\mu\lr{(\frac{k}{n},\frac{k+1}{n}]} = \sum_{j=0}^{k}\mu\lr{(0,\frac{1}{n}]} = k\cdot\mu\lr{(0,\frac{1}{n}]} = k\cdot\mu((0,1])\frac{1}{n} = \mu((0,1])m((0,q]).
    \]
    Consider the algebra $\mc{A}=\{\text{finite disjoint unions of elements of $E_q$}\}$. Note that, by part (a), $\Sigma(\mc{A}) = \mc{B}_\R$. Consider the measure $\nu:\mc{B}_{\R}\to[0,+\infty]$ given by $\nu(E)=\mu((0,1])\cdot m(E)$ for $E\in\mc{B}_\R$. \\

    We have shown that $\mu\vert_{\mc{A}}=\nu\vert_{\mc{A}}$, whence $\mu = \nu$, as desired.
  \end{proof}
\end{homeworkProblem}

\end{document}

\documentclass[12pt,
psamsfonts]{amsart}

%-------Packages---------
\usepackage{amssymb,amsfonts,amsmath}
\usepackage[all,arc]{xy}
\usepackage{enumerate}
\usepackage{mathrsfs}
\usepackage{fullpage}
\usepackage{xspace}
\usepackage[margin=1.0in]{geometry}
\usepackage{tcolorbox}
\usepackage{tikz-cd}
\usepackage{color}
\usepackage{aliascnt}
\usepackage[foot]{amsaddr}
\usepackage{hyperref}


%--------Theorem Environments--------
%theoremstyle{plain} --- default
\newtheorem{thm}{Theorem}[section]

%----Theorem
\newaliascnt{theo}{thm}
\newtheorem{theo}[theo]{Theorem}
\aliascntresetthe{theo}
\newcommand{\theoautorefname}{Theorem}
%----Corollary
\newaliascnt{cor}{thm}
\newtheorem{cor}[cor]{Corollary}
\aliascntresetthe{cor}
\newcommand{\corautorefname}{Corollary}
%----Proposition
\newaliascnt{prop}{thm}
\newtheorem{prop}[prop]{Proposition}
\aliascntresetthe{prop}
\newcommand{\propautorefname}{Proposition}
%----Lemma
\newaliascnt{lem}{thm}
\newtheorem{lem}[lem]{Lemma}
\aliascntresetthe{lem}
\newcommand{\lemautorefname}{Lemma}
%----Conjecture
\newaliascnt{conj}{thm}
\newtheorem{conj}[conj]{Conjecture}
\aliascntresetthe{conj}
\newcommand{\conjautorefname}{Conjecture}
%----Question
\newaliascnt{que}{thm}
\newtheorem{que}[que]{Question}
\aliascntresetthe{que}
\newcommand{\queautorefname}{Question}
%----Assumption
\newaliascnt{ass}{thm}
\newtheorem{ass}[ass]{Assumption}
\aliascntresetthe{ass}
\newcommand{\assautorefname}{Assumption}
%----Definition
\newaliascnt{defn}{thm}
\newtheorem{defn}[defn]{Definition}
\aliascntresetthe{defn}
\newcommand{\defnautorefname}{Definition}




%Style
\theoremstyle{remark}
%----Remark
\newaliascnt{rem}{thm}
\newtheorem{rem}[rem]{Remark}
\aliascntresetthe{rem}
\newcommand{\remautorefname}{Remark}

\newtheorem*{ack}{Acknowledgements}




\newtheorem{Proof}{Proof}

\theoremstyle{definition}
%\newtheorem{defn}[thm]{Definition}
\newtheorem{defns}[thm]{Definitions}
\newtheorem{con}[thm]{Construction}
\newtheorem{exmp}[thm]{Example}
\newtheorem{exmps}[thm]{Examples}
\newtheorem{notn}[thm]{Notation}
\newtheorem{notns}[thm]{Notations}
\newtheorem{addm}[thm]{Addendum}
\newtheorem{exer}[thm]{Exercise}
\newtheorem{conv}[thm]{Convention}

\newtheorem{case}[thm]{Case}


\newtheorem{rems}[thm]{Remarks}
\newtheorem{warn}[thm]{Warning}
%\newtheorem{sch}[thm]{Scholium}
\newtheorem{notation}[thm]{Notation}
\newtheorem{ex}[thm]{Examples}
\newtheorem{note}[thm]{Note}



\newcommand{\N}{\mathbb{N}\xspace}
\newcommand{\I}{\mathbb{I}\xspace}
\newcommand{\R}{\mathbb{R}\xspace}
\newcommand{\Z}{\mathbb{Z}\xspace}
\newcommand{\Q}{\mathbb{Q}\xspace}
\newcommand{\C}{\mathbb{C}\xspace}
\newcommand{\G}{\mathbb{G}\xspace}
\DeclareMathOperator{\Spec}{Spec}
\DeclareMathOperator{\res}{res}
\DeclareMathOperator{\Tr}{Tr}
\DeclareMathOperator{\ord}{ord}
\DeclareMathOperator{\Sym}{Sym}
\DeclareMathOperator{\dv}{div}
\DeclareMathOperator{\alb}{alb}
\DeclareMathOperator{\img}{Im}
\DeclareMathOperator{\et}{et}
\DeclareMathOperator{\ck}{coker}
\DeclareMathOperator{\Reg}{Reg}
\DeclareMathOperator{\Cor}{Cor}
\DeclareMathOperator{\Ac}{at}
\DeclareMathOperator{\supp}{supp}
\DeclareMathOperator{\Hom}{Hom}
\DeclareMathOperator{\Pic}{Pic}
\DeclareMathOperator{\Gal}{Gal}
\DeclareMathOperator{\fc}{frac}
\DeclareMathOperator{\Ann}{Ann}
\DeclareMathOperator{\Mod}{Mod}
\DeclareMathOperator{\Cone}{Cone}
\DeclareMathOperator{\FI}{FI}
\DeclareMathOperator{\End}{End}
\DeclareMathOperator{\rk}{rk}
\DeclareMathOperator{\Ext}{Ext}
\DeclareMathOperator{\ab}{ab}
\DeclareMathOperator{\Jac}{Jac}
\DeclareMathOperator{\coker}{coker}
\DeclareMathOperator{\fr}{frac}
\makeatletter
\let\c@equation\c@theo
\makeatother
\numberwithin{equation}{section}

\bibliographystyle{plain}
%\newcommand{\textlatin }




%--------Meta Data: Fill in your info------
\title{Math 7752 - Homework 6\\
Due Friday 03/18/22 at 1 p.m.}

\begin{document}

\maketitle

\begin{enumerate}
\item \begin{enumerate}
[(a)]\item Prove that two $3\times 3$ matrices over some field $F$ are similar if and only if they have the same minimal and characteristic polynomials. Is the same true for $4\times 4$ matrices? 
\item A matrix $A$ is called idempotent if $A^2=A$. Prove that two idempotent $n\times n$ matrices are similar if and only if they have the same rank. \textbf{Hint:} What is the minimal polynomial of an idempotent matrix? How does rank relate to eigenvalue $0$? 
\end{enumerate}
\medskip
\medskip
%\item (\textbf{Optional for extra practice in case you haven't seen this before}) Solve Problem 6 from p. 488 of Dummit-Foote. \\
%\item Solve Problem 9 from p. 489 of Dummit-Foote. It is enough to find the RCF of only the 3rd matrix. \\
%\item\begin{enumerate}[(a)]
%\item Determine the number of possible RCF's of $8\times 8$ matrices $A$ over $\Q$ with $\chi_A(x)=x^8-x^4$. Explain your argument in detail. 

\item Let $F$ be an algebraically closed field and $V$ a finite dimensional $F$-vector space. 
\begin{enumerate}[(a)]
\item  Let $S,T\in\mathcal{L}(V)$ such that $ST=TS$. Let $\lambda$ be an eigenvalue of $S$ and $E_\lambda(S)\leq V$ be the corresponding eigenspace of $S$. Prove that $E_\lambda(S)$ is a $T$-invariant subspace. \\
\item  Assume that $T\in\mathcal{L}(V)$ is diagonalizable and let $W\leq V$ be a $T$-invariant subspace. Prove that $T|_W\in\mathcal{L}(W)$ is also diagonalizable. \\
\item  Assume again that $S,T\in\mathcal{L}(V)$ such that $ST=TS$. Prove that there exists a basis $\Omega$ of $V$ such that $[T]_\Omega$, and $[S]_\Omega$ are both diagonal. \\
\item  Give an example of a vector space $V$ with $\dim_F(V)\geq 3$ and two commuting linear transformations $S,T\in\mathcal{L}(V)$ such that NO basis $\Omega$ of $V$ exists such that both $[T]_\Omega$, and $[S]_\Omega$ are in JCF. 
\end{enumerate}
%\item  Determine the number of possible RCF's of $10\times 10$ matrices over $\Q$ with $\mu_A(x)=x^6-x^2$. Do the same for matrices over $\C$. 
%\end{enumerate} 
\medskip
\medskip
\item  Find the number of distinct conjugacy classes in the group $GL_3(\Z/2\Z)$, and specify one element in each conjugacy class. \\

\item Let $V$ be an $n$-dimensional vector space over an algebraically closed field and $T\in\mathcal{L}(V)$. Assume that $T$ has just one eigenvalue $\lambda$ and just one Jordan block. Let $S=T-\lambda I$. 
\begin{enumerate}
[(a)]\item Prove that $\rk(S^k)=n-k$, for all $0\leq k\leq n$. Deduce that $\img(S^k)=\ker(S^{n-k})$, for all $0\leq k\leq n$. 
\item Let $v\in V$ be any vector which lies outside of $\img(S)=\ker(S^{n-1})$. Prove that $\{S^{n-1}v,\ldots, Sv, v\}$ is a Jordan basis for $T$. 
\end{enumerate}
\medskip
\medskip
\item Assume again that $V$ is an $n$-dimensional vector space over an algebraically closed field $F$ and $T\in\mathcal{L}(V)$. 
\begin{enumerate}
[(a)]\item Assume that $T$ has unique eigenvalue $0$ and two Jordan blocks: a $1\times 1$ block and a $2\times 2$ block (so $n=3$ in this case). Justify the following algorithm for computing a Jordan basis for $T$: Take any $v\in V\setminus \ker(T)$ and choose $w\in\ker(T)$ such that $\{w, Tv\}$ is a basis for $\ker(T)$ (why is this possible?); then $\{w, Tv, v\}$ is a Jordan basis for $T$. 
\item Assume that $T$ has unique eigenvalue $0$ and two Jordan blocks, both of which are $2\times 2$ (so $n=4$). State an algorithm for finding a Jordan basis similar to the one in (a). 
\item Assume that for each $\lambda\in\Spec(T)$ there is only one Jordan $\lambda$-block in $JCF(T)$. Describe an algorithm for computing a Jordan basis of $T$. \textbf{Hint:} You just need a minor generalization of the algorithm in the previous problem. 
\end{enumerate}
\medskip
\medskip
\item Compute the Jordan canonical form and a Jordan basis for each of the following matrices over $\Q$: 

(a) $\left(\begin{array}{ccc}
-1 & 3 & 0\\
0 & 2 & 0 \\
2 & 1 & -1
\end{array}\right)$ \hspace{100 pt} (b)  $\left(\begin{array}{ccc}
1 & -1 & 1\\
1 & -1 & 1 \\
1 & -1 & 0
\end{array}\right)$. 
\\
\\
\item Let $F=\mathbb{F}_3$ be the field with $3$ elements and let $A\in M_{12}(\mathbb{F}_3)$. Suppose that $A$ satisfies all the following assumptions: 
\begin{itemize}
\item $\rk(A)=10$, \hspace{20pt} $\rk(A^2)=9$, \hspace{20pt} $\rk(A^3)=9$. 
\item $\rk(A-I)=12$. 
\item $\rk(A-2I)=9$, \hspace{20pt} $\rk((A-2I)^2)=7$, \hspace{20pt} $\rk((A-2I)^3)=6$. 
\end{itemize} 
\begin{enumerate}
\item Assume in addition that the characteristic polynomial $\chi_A(x)$ splits completely over $F$ (i.e. it splits into linear factors in $F[x]$). Find the Jordan canonical form of $A$. 
\item Find all possible RCF's of matrices $A$ satisfying all the bullet assumptions, but not necessarily the extra assumption in (a). 
\end{enumerate}
\medskip 
\medskip
\item \textbf{(Optional)} Let $V=C^\infty(\R)$ be the space of all infinitely differentiable functions on the real line. (Observe that $V$ is an infinite dimensional $\R$-vector space). Consider the linear transformation $T=\frac{d}{dx}:V\rightarrow V$ that sends a function $f$ to its derivative. Find all eigenvalues of $T$ and the corresponding \textbf{generalized eigenspaces}. \textbf{Hint:} Start by computing the generalized eigenspaces corresponding to the eigenvalue $\lambda=0$.  







\end{enumerate}





\end{document}
\documentclass[12pt,letterpaper]{article}

%--------Packages--------
\usepackage{amsmath, amsthm, amssymb}
\usepackage{xspace}
\usepackage{graphicx}
\usepackage{amssymb}
\usepackage{array}
\usepackage{braket}
\usepackage{multicol}
\usepackage{mathtools}
\usepackage{enumerate}
\usepackage{delarray}
\usepackage{mathtools}
\usepackage{fullpage}
\usepackage{faktor} % For quotients
\usepackage{mathrsfs}
\usepackage{quiver}
% \usepackage{tikz}

% \usepackage{quiver}
\usepackage[linguistics]{forest}




%--------Page Setup--------

\pagestyle{empty}%

\setlength{\hoffset}{-1.54cm}
\setlength{\voffset}{-1.54cm}

\setlength{\topmargin}{0pt}
\setlength{\headsep}{0pt}
\setlength{\headheight}{0pt}

\setlength{\oddsidemargin}{0pt}

\setlength{\textwidth}{195mm}
\setlength{\textheight}{250mm}


%--------Macros--------

\newcommand{\sub}{\subseteq}
\newcommand{\lcm}{\text{lcm}}
\newcommand{\ms}[1]{\mathscr{#1}}
\newcommand{\mc}[1]{\mathcal{#1}}
\newcommand{\mf}[1]{\mathfrak{#1}}
\newcommand{\sO}{\mathcal{O}}
\newcommand{\cyclic}[1]{\langle#1\rangle}
\newcommand{\units}[1]{#1 ^{\times}}
\newcommand{\la}{\langle}
\newcommand{\ra}{\rangle}
\newcommand{\lr}[1]{\left(#1\right)}
%----Switch phi and varphi
\let\temp\phi
\let\phi\varphi
\let\varphi\temp

\newcommand{\C}{\mathbb{C}}
\newcommand{\F}{\mathbb{F}}
\newcommand{\N}{\mathbb{N}\xspace}
\newcommand{\I}{\mathbb{I}\xspace}
\newcommand{\R}{\mathbb{R}\xspace}
\newcommand{\Z}{\mathbb{Z}\xspace}
\newcommand{\Q}{\mathbb{Q}\xspace}
\newcommand{\G}{\mathbb{G}\xspace}
\DeclareMathOperator{\Spec}{Spec}
\DeclareMathOperator{\res}{res}
\DeclareMathOperator{\Tr}{Tr}
\DeclareMathOperator{\ord}{ord}
\DeclareMathOperator{\Sym}{Sym}
\DeclareMathOperator{\dv}{div}
\DeclareMathOperator{\alb}{alb}
\DeclareMathOperator{\img}{Im}
\DeclareMathOperator{\et}{et}
\DeclareMathOperator{\ck}{coker}
\DeclareMathOperator{\Reg}{Reg}
\DeclareMathOperator{\Cor}{Cor}
\DeclareMathOperator{\Ac}{at}
\DeclareMathOperator{\supp}{supp}
\DeclareMathOperator{\Hom}{Hom}
\DeclareMathOperator{\Pic}{Pic}
\DeclareMathOperator{\Gal}{Gal}
\DeclareMathOperator{\fc}{frac}
\DeclareMathOperator{\Ann}{Ann}
\DeclareMathOperator{\Mod}{Mod}
\DeclareMathOperator{\Cone}{Cone}
\DeclareMathOperator{\FI}{FI}
\DeclareMathOperator{\End}{End}
\DeclareMathOperator{\Alb}{Alb}
\DeclareMathOperator{\Ext}{Ext}
\DeclareMathOperator{\ab}{ab}
\DeclareMathOperator{\Jac}{Jac}
\DeclareMathOperator{\coker}{coker}
\DeclareMathOperator{\fr}{frac}
\DeclareMathOperator{\spn}{span}
\DeclareMathOperator{\im}{im}
\DeclareMathOperator{\rk}{rk}


%----Analysis
\newcommand{\dd}[2][]{\frac{\partial^{#1}}{\partial {#2}^{#1}}}
\newcommand{\summ}{\sum\limits}
\newcommand{\norm}[1]{\left \vert \left \vert #1 \right \vert \right \vert}
\newcommand{\thicc}{\bigg}
\newcommand{\eps}{\varepsilon}


%--------Theorem environments--------
\newtheorem{definition}{Definition}[]
\newtheorem{lemma}{Lemma}[]
\newtheorem{corollary}{Corollary}[]
\newtheorem{theorem}{Theorem}[]
\theoremstyle{remark}
\newtheorem*{claim}{Claim}


\newenvironment{solution}
{\begin{proof}[Solution]}
{\end{proof}}


\makeatletter
\newcolumntype{"}{@{\hskip\tabcolsep\vrule width 1pt\hskip\tabcolsep}}
\makeatother

% --------Problem environment--------
\setlength\parindent{0pt}
\setcounter{secnumdepth}{0}
\newcounter{partCounter}
\newcounter{homeworkProblemCounter}
\setcounter{homeworkProblemCounter}{1}


\newenvironment{homeworkProblem}[1][-1]{
    \ifnum#1>0
        \setcounter{homeworkProblemCounter}{#1}
    \fi
    \section{Problem \arabic{homeworkProblemCounter}}
    \setcounter{partCounter}{1}
    \stepcounter{homeworkProblemCounter}
}


%--------Metadata--------
\title{MATH 7752 Homework 5}
\author{James Harbour}


\begin{document}
\maketitle

\begin{homeworkProblem}
  Let $F=\Z^3$ be the free $\Z$-module of rank $3$. Let $N$ be the submodule of $F$ generated by $v_1=(1,2,3), (5,4,6)$, and $(7,8,9)$.\\


  \textbf{(1)} Find compatible bases for $F$ and $N$, that is, bases satisfying the submodule theorem 1.\\

  \begin{proof}
    \[\begin{bmatrix} 1&2&3 \\ 4&5&6 \\ 7&8&9 \end{bmatrix} \xrightarrow{\mc{E}_{21}(-4),\, \mc{E}_{31}(-7)}
    \begin{bmatrix}1&2&3 \\ 0&-3&-6 \\ 0&-6&-12 \end{bmatrix} \xrightarrow{\mc{E}_{32}(-2)}
    \begin{bmatrix}1&2&3 \\ 0&-3&-6 \\ 0&0&0\end{bmatrix}
    \xrightarrow{\mc{E}_{21}'(-2),\;\mc{E}_{31}'(-3)}
    \begin{bmatrix}1&0&0 \\ 0&-3&-6 \\ 0&0&0\end{bmatrix}
    \xrightarrow{\mc{E}_{32}'(-2)}
    \begin{bmatrix}1&0&0 \\ 0&-3&0 \\ 0&0&0\end{bmatrix}\]
    Then the desired matrix $B$ has the form
    \[
      B=E_{23}(-2)^{-1}E_{12}(-2)^{-1}E_{13}(-3)^{-1}=E_{23}(2)E_{12}(2)E_{13}(3) = \begin{bmatrix}1&2&3 \\ 0&1&2 \\ 0&0&1\end{bmatrix}
    \]
    so our new basis $\{y_1,y_2,y_3\}$ of $F$ is given by
    \[
      \begin{bmatrix}y_1\\y_2\\y_3\end{bmatrix} = \begin{bmatrix}1&2&3 \\ 0&1&2 \\ 0&0&1\end{bmatrix} \begin{bmatrix}e_1\\e_2\\e_3\end{bmatrix}
    \]
    and our new basis of $N$ is $\{y_1, -3y_2\}$.
  \end{proof}

  \textbf{(2)} Describe the quotient $F/N$ in the IF form.\\

  \begin{proof}
    The quotient is given by
    \[
      F/N\cong (y_1\Z \oplus y_2\Z \oplus y_3\Z)/(y_1\Z\oplus -3y_2\Z) \cong \Z/3\Z \oplus \Z
    \]
  \end{proof}

  \textbf{(3)} Describe in IF form the abelian group given by the presentation
  \[\langle a,b,c\;|\;a+2b+3c=0,5a+4b+6c=0, 7a+8b+9c=0\rangle.\]


\end{homeworkProblem}

\begin{homeworkProblem}
  Let $R$ be a PID. For an $R$-module $M$ define $\rk(M)$ to be the minimal size of a generating set of $M$.

  \textbf{(a)} Let $M$ be a finitely generated $R$-module and $R/a_1R\oplus\cdots\oplus R/a_mR\oplus R^s$ be its invariant factor decomposition. That is, $s\geq 0$ and the elements $a_1,\ldots, a_m$ are non-zero, non-units such that $a_1|a_2|\cdots|a_m$. Prove that $\rk(M)=m+s$. \textbf{Warning:} It is not true in general that $\rk(P\oplus Q)=\rk(P)\oplus\rk(Q)$.\\

  \begin{proof}
    As $M$ has a generating set of size $m+s$, we have that $n=rk(M)\leq m+s$. Then there exists a surjective $R$-module homomorphism $\phi:R^n\to M$ such that $\phi(e_i)=x_i$. Letting $K:= \ker(\phi)$, as $\{e_1,\ldots,e_n\}$ is a basis for $R^n$, there exist nonzero $b_1,\ldots,b_k\in R$ with $k\leq n$ and $b_1\vert\cdots\vert b_k$ such that $\{b_1 e_1,\ldots, b_n e_n\}$ is a basis for $K$. Suppose that $1\leq l\leq k$ is such that $b_1\cdots b_l$ are units and $b_{l+1}, \ldots, b_k$ are non-units. Then
    \begin{align*}
      M\cong \lr{\bigoplus_{i=1}^{n}e_i R}/\lr{\bigoplus_{i=1}^{k}b_i e_i R} &\cong R/b_1 R\oplus \cdots \oplus R/b_k R \oplus R^{n-k}\\
      &\cong R/b_{l+1} R\oplus \cdots \oplus R/b_k R \oplus R^{n-k}
    \end{align*}
    which is in invariant factor form, so $n-k = s$ and $k-l = m$, so
    \[
      n = k+s = l+m+s \geq m+s
    \]
    and thus $rk(M) = m+s$.
  \end{proof}


  \textbf{(b)}: Let $F$ be a free $R$-module of rank $n$ with basis $e_1,\ldots e_n$. Let $N$ be the submodule of $F$ generated by some elements $v_1,\ldots,v_n\in F$. Let $A\in Mat_n(R)$ be the matrix such that
  \[\left(\begin{array}{ccc}
  v_1\\
  \vdots\\
  v_n
  \end{array}\right)=A\left(\begin{array}{ccc}
  e_1\\
  \vdots\\
  e_n
  \end{array}\right).\] Find a simple condition on the entries of $A$ which holds if and only if $\rk(F/N)=n$.

\end{homeworkProblem}

\begin{homeworkProblem}
  In this problem $R$ will be a commutative domain. An $R$-module $P$ is called \textit{projective} if it is a direct summand of a free $R$-module. That is, if there exist a free $R$-module $F$ and a submodule $Q$ of $F$ such that $F=P\oplus Q$. \\


  \textbf{(1)} Let $P,M,N$ be $R$-modules and suppose $f:M\rightarrow N$ is a surjective $R$-module homomorphism. The map $f$ induces a homomorphism of $R$-modules,
  \begin{eqnarray*}
  f_\star:&&\Hom_R(P,M)\rightarrow\Hom_R(P,N)\\
  &&[\phi:P\rightarrow M]\mapsto [f\circ\phi:P\rightarrow N].
  \end{eqnarray*} Prove that if $P$ is finitely generated and projective, then $f_\star$ is surjective.\\ \textbf{Hint:} The universal property of free $R$-modules will be useful.

  \begin{proof}

\[\begin{tikzcd}
	&& F && {F=P\oplus Q} & {} & P \\
	\\
	M && N && M && N
	\arrow["\phi", from=1-3, to=3-3]
	\arrow["f", two heads, from=3-1, to=3-3]
	\arrow["\exists\psi"', dashed, from=1-3, to=3-1]
	\arrow["f", two heads, from=3-5, to=3-7]
	\arrow[""{name=0, anchor=center, inner sep=0}, "\pi", shift left=1, from=1-5, to=1-7]
	\arrow["\exists\psi"', dotted, from=1-5, to=3-5]
	\arrow["\iota", shift left=1, from=1-7, to=1-5]
	\arrow["\phi", from=1-7, to=3-7]
	\arrow[shorten <=0pt, Rightarrow, from=0, to=1-6]
\end{tikzcd}\]

  We first show that such $f_\star$ is surjective when $P=F$ is a free $R$-module. Suppose that $\phi\in\Hom_R(F,N)$. Let $F$ be free over some subset $X\sub F$. By surjectivity of $f$, for all $x\in X$, there exists an $m_x\in M$ such that $f(m_x) = \phi(x)$. Then, by the universal property of free modules, there exists a unique $\psi\in\Hom_R(P,M)$ such that $\psi(x) = m_x$ for all $x\in X$. It follows then that, for $x\in X$,
  \[
    f(\psi(x)) = f(m_x) = \phi(x)
  \]
  whence by linearity $f_\star(\psi) = f\circ \psi = \phi$.\\

  Now we treat the general case. Let $P$ be a finitely generated projective $R$-module. Then by definition there exists a free module $F$ and a submodule $Q\sub F$ such that $F=P\oplus Q$. Take $\pi:F\to P$ to be the natural projection and $\iota:P\to F$ the natural inclusion. Then, appealing to the previous case, there exists an $R$-module homomorphism $\psi:F\to M$ such that $f\circ \psi = \phi\circ\pi$. Deine a new $\widetilde\psi\in \Hom_R(P,M)$ by $\widetilde\psi:= \psi\circ\iota$. Now, for $p\in P$, we have that
  \[
    (f\circ\widetilde\psi)(p) = (f\circ\psi)((p,0))) = (\phi\circ\pi)((p,0)) = \phi(p)
  \]
  so $\phi = f\circ \widetilde\psi = f_\star(\widetilde\psi)$.
  \end{proof}

  \textbf{(2)} Show that if $R$ is a PID and $P$ is finitely generated, then $P$ is projective if and only if $P$ is free.

  \begin{proof}\ \\
    The reverse direction follows from the fact that $P = P\oplus 0$, so it suffices to show the forward direction. Let $P$ be a finitely generated projective module. Then there exists a surjective $R$-module homomorphism $f:R^n\to P$ for some $n\in\N$. Consider the identity map $1_P\in \Hom_R(P,P)$. By part (1), there exists a $\psi\in\Hom_R(P,R^n)$ such that $f\circ \psi = f_\star(\psi) = 1_P$. Then $\psi(P)$ is a submodule of a finitely generated free module and is thus free (as $R$ is a PID). Moreover, as $\psi$ has a left inverse, it is injective whence $P\cong\psi(P)$ is free.
  \end{proof}



\end{homeworkProblem}

\begin{homeworkProblem}
  Determine the number of possible RCF's of $8\times 8$ matrices $A$ over $\Q$ with $\chi_A(x)=x^8-x^4$. Explain your argument in detail.
\end{homeworkProblem}

\end{document}

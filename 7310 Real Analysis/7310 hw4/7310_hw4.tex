\documentclass[12pt,letterpaper]{article}

%--------Packages--------
\usepackage{amsmath, amsthm, amssymb}
\usepackage{xspace}
\usepackage{graphicx}
\usepackage{hhline}
\usepackage{amssymb}
\usepackage{array}
\usepackage{braket}
\usepackage{multicol}
\usepackage{mathtools}
\usepackage{enumerate}
\usepackage{delarray}
\usepackage{mathtools}
\usepackage{fullpage}
\usepackage{faktor} % For quotients
\usepackage{mathrsfs}

\usepackage[italicdiff]{physics} % For differentials
\usepackage{bbm} % For indicator

% \usepackage{quiver}
\usepackage[linguistics]{forest}




%--------Page Setup--------

\pagestyle{empty}%

\setlength{\hoffset}{-1.54cm}
\setlength{\voffset}{-1.54cm}

\setlength{\topmargin}{0pt}
\setlength{\headsep}{0pt}
\setlength{\headheight}{0pt}

\setlength{\oddsidemargin}{0pt}

\setlength{\textwidth}{195mm}
\setlength{\textheight}{250mm}


%--------Macros--------

\newcommand{\sub}{\subseteq}
\newcommand{\lcm}{\text{lcm}}
\newcommand{\mc}[1]{\mathcal{#1}}
\newcommand{\mf}[1]{\mathfrak{#1}}
\newcommand{\ms}[1]{\mathscr{#1}}
\newcommand{\sO}{\mathcal{O}}
\newcommand{\cyclic}[1]{\langle#1\rangle}
\newcommand{\units}[1]{#1 ^{\times}}
\newcommand{\la}{\langle}
\newcommand{\ra}{\rangle}
\newcommand{\lr}[1]{\left(#1\right)}
%----Switch phi and varphi
% \let\temp\phi
% \let\phi\varphi
% \let\varphi\temp

\newcommand{\C}{\mathbb{C}}
\newcommand{\F}{\mathbb{F}}
\newcommand{\N}{\mathbb{N}\xspace}
\newcommand{\I}{\mathbb{I}\xspace}
\newcommand{\R}{\mathbb{R}\xspace}
\newcommand{\Z}{\mathbb{Z}\xspace}
\newcommand{\Q}{\mathbb{Q}\xspace}
\newcommand{\G}{\mathbb{G}\xspace}
\DeclareMathOperator{\Spec}{Spec}
\DeclareMathOperator{\res}{res}
% \DeclareMathOperator{\Tr}{Tr}
\DeclareMathOperator{\ord}{ord}
\DeclareMathOperator{\Sym}{Sym}
% \DeclareMathOperator{\dv}{div}
\DeclareMathOperator{\alb}{alb}
\DeclareMathOperator{\img}{Im}
\DeclareMathOperator{\et}{et}
\DeclareMathOperator{\ck}{coker}
\DeclareMathOperator{\Reg}{Reg}
\DeclareMathOperator{\Cor}{Cor}
\DeclareMathOperator{\Ac}{at}
\DeclareMathOperator{\supp}{supp}
\DeclareMathOperator{\Hom}{Hom}
\DeclareMathOperator{\Pic}{Pic}
\DeclareMathOperator{\Gal}{Gal}
\DeclareMathOperator{\fc}{frac}
\DeclareMathOperator{\Ann}{Ann}
\DeclareMathOperator{\Mod}{Mod}
\DeclareMathOperator{\Cone}{Cone}
\DeclareMathOperator{\FI}{FI}
\DeclareMathOperator{\End}{End}
\DeclareMathOperator{\Alb}{Alb}
\DeclareMathOperator{\Ext}{Ext}
\DeclareMathOperator{\ab}{ab}
\DeclareMathOperator{\Jac}{Jac}
\DeclareMathOperator{\coker}{coker}
\DeclareMathOperator{\fr}{frac}
\DeclareMathOperator{\Int}{Int}



%----Analysis
% \newcommand{\dd}[2][]{\frac{\partial^{#1}}{\partial {#2}^{#1}}}
\newcommand{\summ}{\sum\limits}
% \newcommand{\norm}[1]{\left \vert \left \vert #1 \right \vert \right \vert}
\newcommand{\thicc}{\bigg}
\newcommand{\eps}{\varepsilon}
\newcommand*\cls[1]{\overline{#1}}
\newcommand{\ind}{\mathbbm{1}}


%--------Theorem environments--------
\newtheorem{definition}{Definition}[]
\newtheorem{lemma}{Lemma}[]
\newtheorem{corollary}{Corollary}[]
\newtheorem{theorem}{Theorem}[]
\theoremstyle{remark}
\newtheorem*{claim}{Claim}


\newenvironment{solution}
{\begin{proof}[Solution]}
{\end{proof}}


\makeatletter
\newcommand{\thickhline}{%
    \noalign {\ifnum 0=`}\fi \hrule height 1pt
    \futurelet \reserved@a \@xhline
}
\newcolumntype{"}{@{\hskip\tabcolsep\vrule width 1pt\hskip\tabcolsep}}
\makeatother

% --------Problem environment--------
\setlength\parindent{0pt}
\setcounter{secnumdepth}{0}
\newcounter{partCounter}
\newcounter{homeworkProblemCounter}
\setcounter{homeworkProblemCounter}{1}


\newenvironment{homeworkProblem}[1][-1]{
    \ifnum#1>0
        \setcounter{homeworkProblemCounter}{#1}
    \fi
    \section{Problem \arabic{homeworkProblemCounter}}
    \setcounter{partCounter}{1}
    \stepcounter{homeworkProblemCounter}
}


%--------Metadata--------
\title{MATH 7310 Homework 4}
\author{James Harbour}

\begin{document}
\maketitle

\begin{homeworkProblem}
  \textbf{(i)}: Let $(X,\Sigma), (Y,\mc{F})$ be two measurable spaces and let $\phi:X\to Y$ be measurable. Given a measure $\nu$ on $\Sigma$, define $\phi_* (\nu):\mc{F}\to[0,+\infty]$ by $\phi_*(\nu)(E) = \nu(\phi^{-1}(E))$. Prove that $\phi_*(\nu)$ is a measure.\\

  \textbf{(ii)}: If $x\in[0,1]$, a \emph{binary expansion} for $x$ is a sequence $(a_n)_{n=1}^{\infty}\in\{ 0,1\}^{\N}$ so that $x = \sum_{n=1}^{\infty}a_n 2^{-n}$. Let $N$ be the set of $x\in [0,1]$ whose binary expansion is not unique. Show that $N$ is a Borel set of measure $0$.

  \begin{proof}
    The set of all points in $[0,1]$ with nonunique binary expansion is precisely the set of all points of the form $2^{-n}$ for $n\in\N\cup\{0\}$. Thus, $N = \bigcup_{n=0}^{\infty}\{ 2^{-n}\}$ is Borel as singletons are Borel. As $N$ is a countable set, it follows that $m(N) = 0$.
  \end{proof}

  \textbf{(iii)}: Let $C\sub [0,1]$ be the middle thirds Cantor set. For $k\in \N$, define \[\phi_k,\phi:[0,1]\setminus N\to \R\] by $\phi_k(\sum_{n=1}^{\infty}a_n 2^{-n}) = \sum_{n=1}^{k}2a_n 3^{-n}$ and $\phi(\sum_{n=1}^{\infty}a_n 2^{-n}) = \sum_{n=1}^{\infty}2a_n 3^{-n}$ for all $(a_n)_{n=1}^{\infty}\in\{ 0,1\}^{\N}$. Show that $\phi_k$, $\phi$ are Borel and that $\phi_k([0,1]\setminus N)$ and $\phi([0,1]\setminus N)$ are subsets of $C$.

  \begin{proof}
    Noting that $\phi = \sup_{k}\phi_k$, it suffices to show that each $\phi_k$ is Borel. We claim that $\phi_k$ is in fact a finite linear combination of step functions. For $(a_1,\ldots,a_k)\in \{0,1\}^{k}$ consider the set
    \[
      I_{(a_1,\ldots, a_k)} = \Set{ x\in [0,1]\setminus N: x=\sum_{n=1}^{k}a_k 2^{-k} + \sum_{j=k+1}^{\infty} b_j 2^{-j} \text{ where }b_j\in\{0,1\}}
    \]
    Note that $\phi_k$ is constant on each $I_a$ for and is equal to $\sum_{n=1}^{k}2a_n 3^{-n}$ for each $a\in \{ 0,1\}^{k}$, and each $x\in [0,1]\setminus N$ lies in some $I_k$, so
    \[
      \phi_k = \sum_{(a_1,\ldots, a_k)\in \{0,1\}^{k}} \lr{\sum_{n=1}^{k}a_n 2^{-n}}\ind_{I_{(a_1,\ldots, a_k)}}.
    \]

    Note that each element of the images of $\phi$ and $\phi_k$ have ternary decompositions with only 0s or 2s, so they clearly lie in the Cantor set.
  \end{proof}

  \textbf{(iv)}: Set $\mu=\phi_*(m)$, where $m$ is the Lebesgue measure on $[0,1]$. Show that $\mu(C^{c}) = 0$ and that there is a unique, increasing continuous function $f:[0,1]\to[0,1]$ so that $f(0)= 0$ and $\mu([a,b])=f(b)-f(a)$ for all $0\leq a< b\leq 1$. (In particular, $f(1) = 1$).

  \begin{proof}
    Note that $\mu(C^c) = m(\phi^{-1}(C^c)) = m(\emptyset) = 0$. Moreover, $\mu$ is a Borel measure which is finite on compact sets, so there exists an increasing right continuous function $f:[0,1]\to[0,1]$ such that $\mu = \mu_f$, namely, $f(x) = \mu((0,x])$. Then $f(0)=0$. As such $f$ is determined up to a constant and $f(0)$ has been specified, $f$ is unique. That $f$ is continuous follows from the previous homework and the fact that $\mu$ is diffuse.
  \end{proof}

  \textbf{(v)}: Show that $f(2\sum_{n=1}^{k}a_n 3^{-n}) = \sum_{n=1}^{k}a_n 2^{-n}$ for all $k\in\N$ and all $(a_n)_{n=1}^{k}\in\{ 0,1\} ^{k}$. If $(a,b)$ is an open interval disjoint from $C$, show that $f(b) = f(a)$.

  \begin{proof}
    Observe that
    \[
      f\lr{2\sum_{n=1}^{k}a_n 3^{-n}} = m(\phi^{-1}(\left(0,\ 2\sum_{n=1}^{k}a_n 3^{-n}\right])) = m((\left(0,\ \sum_{n=1}^{k}a_n 2^{-n}\right])) = \sum_{n=1}^{k}a_n 2^{-n}.
    \]

    Suppose that $(a,b)\sub C^c$. Then $f(b)-f(a) = \mu((a,b))\leq \mu(C^c) = 0$ whence $f(a) = f(b)$.
  \end{proof}
\end{homeworkProblem}

\begin{homeworkProblem}[2]
  Let $f:[0,1]\to[0,1]$ be the Cantor function, and let $g(x) = f(x)+x$.\\

  \textbf{(a)}: Prove that $g$ is a bijection from $[0,1]$ to $[0,2]$ and $h=g^{-1}$ is a continuous map from $[0,2]$ to $[0,1]$.

  \begin{proof}
    Note that $g$ is strictly monotone increasing and continuous, so $g$ is injective. As $g(1) = 2$, by the intermediate value theorem $g$ is surjective. Morover, as $g$ is a strictly monotone bijection between intervals, $g^{-1}$ is continous.
  \end{proof}

  \textbf{(b)}: If $C$ is the Cantor set, $m(g(C))=1$.

  \begin{proof}
    Write $C^c = \bigcup_{j=1}^{\infty}I_j$ where $I_j = (a_j,b_j)$ for some $a_j<b_j$. Note that, as $(a_j,b_j)\sub C^{c}$, so $f$ is constant on $(a_j,b_j)$. Moreover,
    \[
      g\vert_{I_j}(x) = f(x)+x = a_j+x \implies g(I_j) = I_j
    \]
    \[
      m(g(C)) = 2 - m(g^(C^c)) = 2-\sum_{j=1}^{\infty} m(g(I_j)) = 2-\sum_{j=1}^{\infty} m(I_j) = 2-m(C^c) = 2-1 = 1
    \]

  \end{proof}

  \textbf{(c)}: By exercise $29$ of chapter 1, $g(C)$ contains a Lebesgue nonmeasurable set $A$. Let $B = g^{-1}(A)$. Then $B$ is Lebesgue measurable but not Borel.

  \begin{proof}
    To see that $B$ is Lebesgue measurable, note that $B = g^{-1}(A)\sub g^{-1}(g(C)) = C$ and $m(C) = 0$, so by completeness, $B$ is Lebesgue measurable. As $g$ is a homeomorphism and $A = g(B)$, if $B$ were Borel then so would $A$ be, which is absurd as $A$ is non-measurable.
  \end{proof}

  \textbf{(d)}: There exist a Lebesgue measurable function $F$ and a continuous function $G$ on $\R$ such that $F\circ G$ is not Lebesgue measurable.

  \begin{proof}
    Take $F = \ind_B$. As $B$ is Lebesgue measurable, so is $\ind_B$. Take $G = g$. Then $(F\circ G)^{-1}(\{ 1\}) = (G^{-1}\circ F^{-1})(\{ 1\}) = g^{-1}(B) = A$, which is not measurable, so $F\circ G$ is not Lebesgue measurable.
  \end{proof}
\end{homeworkProblem}

\begin{homeworkProblem}[3]
  Prove that the following hold if and only if the measure $\mu$ is complete:\\

  \textbf{(a)}: If $f$ is measurable and $f=g$ $\mu$-a.e., then $g$ is measurable.

  \begin{proof}\ \\
    \underline{$\implies$}: Suppose that $\mu$ is complete. Let $f$ be measurable and suppose that $f=g$ almost everywhere. Let $N=\{ x: f(x)\neq g(x)\}$. If $E\sub \C$ is measurable, then
    \begin{align*}
      g^{-1}(E)  = \{x: g(x)\in E\text{ and }f(x) = g(x)\} \cup \{x:g(x)\in E\text{ and }f(x) \neq g(x)\} = f^{-1}(E)\cup(g^{-1}(E)\cap N),
    \end{align*}
    which is measurable by completeness of $\mu$ and measurability of $f$. \\

    \underline{$\impliedby$}: Let $N\in \Sigma$ with $\mu(N)=0$ and suppose that $F\sub N$. Then $\ind_{N} = \ind_{F}$ on $X\setminus N$, so $\ind_{N} = \ind_{F}$ $\mu$-a.e. whence by hypothesis $\ind_F$ is measurable so $F\in\Sigma$.
  \end{proof}

  \textbf{(b)}: If $f_n$ is measurable for $n\in\N$ and $f_n\to f$ $\mu$-a.e., then $f$ is measurable.

  \begin{proof}\ \\
    \underline{$\implies$}: Suppose $\mu$ is complete. Let $f_n$ is measurable for $n\in\N$ and $f_n\to f$ $\mu$-a.e. Let $N = \{ x: f_n(x)\not\to f(x)\}$. Then $f_n\ind_{X\setminus N}\to f\ind_{X\setminus N}$ pointwise. Thus, $f\ind_{X\setminus N}$ is measurable. But $f\ind_{X\setminus N} = f$ $\mu$-a.e., so by part (a), $f$ is measurable.\\

    \underline{$\impliedby$}: Let $N\in\Sigma$ such that $\mu(N)=0$. Suppose that $E\in\Sigma$. Then there exist simple functions $\{\phi_n\}_{n=1}^{\infty}$ with $0\leq|\phi_1|\leq|\phi_2|\leq\cdots\leq |\ind_{N}|$ and $\phi_n\to \ind_N$ pointwise. On $X\setminus N$, $\phi_n \ind_N = \phi_n \ind_F$, so $\phi_n\ind_N = \phi_n \ind_F$ almost everywhere whence by part (a) $\phi_n \ind_F$ is measurable. On $X\setminus N$, $\phi_n \ind_F \to \ind_N\ind_F = \ind_N$, so $\phi_n\to \ind_F$ almost everywhere, whence by assumption $\ind_F$ is measurable i.e. $F$ is measurable.

  \end{proof}


\end{homeworkProblem}

\begin{homeworkProblem}[4]
  If $f\in L^+$ and $\int f \dd{\mu} < +\infty$, show that $\{ x:f(x) = \infty\}$ is a null set and that $\{ x: f(x)>0\}$ is $\sigma$-finite.

  \begin{proof}
    Suppose, for the sake of contradiction, that $\mc{N}:=\{ x:f(x)=\infty\} = f^{-1}(\{\infty\})\in\Sigma$ has positive measure. Let $ \{\phi_n\}_{n\in\N}$ be a sequence of simple functions (valued in $[0,+\infty]$) with $0\leq\phi_1\leq \phi_2 \leq \cdots \leq f$ such that $\phi_n\to f$ pointwise. For $n\in \N$, define a new simple function $\phi_n'$ by
    \[
      \phi_n' = \phi_n \ind_{X\setminus \mc{N}} + n\cdot \ind_{\mc{N}}.
    \]
  Note that, as $\phi_n \equiv  \phi_n'$ on $X\setminus \mc{N}$ and $\phi_n' \leq f$ on $
  \mc{N}$, it follows that $0\leq \phi_1'\leq \phi_2'\leq \cdots \leq f$ as well. Moreover, for $n\in \N$, as $\phi_n'\geq n\cdot\ind_{\mc{N}}$, we have that
  \[
    \int f\dd{\mu} \geq \int \phi_n'\dd{\mu} \geq \int n\cdot \ind_{\mc{N}}\dd{\mu} = n\cdot \mu(\mc{N})\to \infty \text{ as }n\to\infty.
  \]
  Thus, $\int f \dd{\mu} = +\infty$, contradicting the assumption.\\

  Let $X= \{x:f(x)>0\}$ and consider the sets $\{A_n\}_{n=0}^{\infty}$ given by $A_0 = f^{-1}(\{\infty\})$, $A_n = f^{-1}([\frac{1}{n}, \frac{1}{n-1}))$ for $n\geq1$. Then
  \[
    X = \bigsqcup_{n=0}^{\infty}A_n
  \]
  Suppose, for the sake of contradiction, that $X$ is not $\sigma$-finite. Then, as $\mu(A_0)=0$, some $A_k$ for $k\geq1$ must have infinite measure. As $f\geq f\cdot \ind_{A_k} \geq \frac{1}{n}\ind_{A_k}$, it follows that
  \[
    \int f\dd{\mu} \geq \int f\cdot \ind_{A_k}\dd{\mu} \geq \int  \frac{1}{n}\ind_{A_k} \dd{\mu} = \frac{1}{n}\mu(A_k) = \infty,
  \]
  contradicting the assumption that $\int f \dd{\mu} <\infty$.
\end{proof}

\end{homeworkProblem}

\begin{homeworkProblem}
  If $f\in L^{+}$, let $\lambda(E) = \int_E f\dd{\mu}$ for $E\in\Sigma$. Prove that $\lambda$ is a measure on $\Sigma$, and that for any $g\in L^{+}$, $\int g\dd{\lambda} = \int fg\dd{\mu}$.

  \begin{proof}
    We first show that $\lambda$ is a measure. Note that $\ind_{\emptyset}$ is the zero function on $X$, so $\lambda(\emptyset) = \int_\emptyset f\dd{\mu} = \int f\ind_{\emptyset}\dd{\mu} = 0$. If $E,F\in \Sigma$ are such that $E\sub F$, then $\ind_E\leq \ind_F \implies f\ind_E \leq f\ind_F$, so by monotonicity,
    \[
      \lambda(E) = \int f\ind_E\dd{\mu} \leq \int f\ind_F\dd{\mu} = \lambda{F}.
    \]
    Lastly, suppose that $\{ A_n\}_{n\in\N}$ is a sequence of disjoint elements of $\Sigma$. Set $A=\bigsqcup_{i=1}^{\infty}A_i$. Let $f_n = f\cdot \ind_{\bigsqcup_{i=1}^{n}A_i}$. Then $0\leq f_1\leq f_2\leq \cdots \leq f\cdot \ind_{A}$ and $f_n\to f \ind_{A}$ pointwise. By the monotone convergence theorem,
    \[
      \lambda(A) = \int f\ind_{A}\dd{\mu} = \lim_{n\to\infty} \int f\ind_{\bigsqcup_{i=1}^{n}A_i}\dd{\mu} = \lim_{n\to\infty} \sum_{i=1}^{n}\int f\ind_{A_i}\dd{\mu} = \sum_{i=1}^{\infty}\lambda(A_i).
    \]

    Suppose that $g$ is a simple function. Write $g = \sum_{i=1}^{n}c_i\ind_{E_i}$ where $E_i\in\Sigma$ and $c_i\in[0,\infty)$. By definition,
    \[
      \int g\dd{\lambda} = \sum_{i=1}^{n}c_i \lambda(E_i) = \sum_{i=1}^{n}c_i \int f\ind_{E_i}\dd{\mu} = \int f\lr{\sum_{i=1}^{n} c_i\ind_{E_i}}\dd{\mu} = \int fg \dd{\mu}.
    \]
    Now suppose that $g\in L^+$ is arbitrary. Then there exist a sequence of simple functions $0\leq \phi_1\leq \phi_2\leq \cdots\leq g$ such that $\phi_n\to g$ pointwise. Then $0\leq f\phi_1\leq f\phi_2\leq \cdots \leq fg$ and $f\phi_n\to fg$ pointwise. By applying the monotone convergence theorem twice, we see that
    \[
      \int g\dd{\lambda} = \lim_{n\to\infty}\int \phi_n \dd{\lambda} = \lim_{n\to\infty}\int f\phi_n \dd{\mu} = \int fg \dd{\mu}.
    \]
  \end{proof}
\end{homeworkProblem}

\begin{homeworkProblem}[6]
  If $f\in L^+$ and $\int f \dd{\mu}<\infty$, show that for every $\eps > 0$ there exists an $E\in\Sigma$ such that $\mu(E)<\infty$ and $\int_E f\dd{\mu} > (\int f \dd{\mu}) - \eps$.

  \begin{proof}
    Let $\eps > 0$. By definition, there exists a simple $\phi$ with $0\leq \phi \leq f$ such that $\int \phi\dd{\mu} > (\int f\dd{\mu}) -\eps $. Write $\phi$ as $\phi = \sum_{i=1}^{n} c_i \ind_{E_i}$ for some $E_i\in\Sigma$ and $c_i\in[0,\infty)$. Note that, as $\sum_{i=1}^{n}c_i\mu(E_i) = \int \phi \dd{\mu} \leq \int f \dd{\mu} < \infty$, we have that $\mu(E_i) < \infty$ for all $i$. Set $E= \bigcup_{i=1}^{n}E_i$. \\

    Noting that $\phi = \phi \ind_{E} \leq f \ind_{E}$, it follows that
    \[
      \int_E f\dd{\mu}\geq \int f \ind_{E} \dd{\mu} \geq\int \phi\dd{\mu} > (\int f\dd{\mu}) -\eps
    \]
    with $\mu(E)\leq \sum_{i=1}^{n}\mu(E_i)<\infty$ as desired.
  \end{proof}

\end{homeworkProblem}

\end{document}

\documentclass[12pt,letterpaper]{article}

%--------Packages--------
\usepackage{amsmath, amsthm, amssymb}
\usepackage{xspace}
\usepackage{graphicx}
\usepackage{hhline}
\usepackage{amssymb}
\usepackage{array}
\usepackage{braket}
\usepackage{multicol}
\usepackage{mathtools}
\usepackage{enumerate}
\usepackage{delarray}
\usepackage{mathtools}
\usepackage{fullpage}
\usepackage{faktor} % For quotients


% \usepackage{quiver}
\usepackage[linguistics]{forest}




%--------Page Setup--------

\pagestyle{empty}%

\setlength{\hoffset}{-1.54cm}
\setlength{\voffset}{-1.54cm}

\setlength{\topmargin}{0pt}
\setlength{\headsep}{0pt}
\setlength{\headheight}{0pt}

\setlength{\oddsidemargin}{0pt}

\setlength{\textwidth}{195mm}
\setlength{\textheight}{250mm}


%--------Macros--------

\newcommand{\sub}{\subseteq}
\newcommand{\lcm}{\text{lcm}}
\newcommand{\mc}[1]{\mathcal{#1}}
\newcommand{\mf}[1]{\mathfrak{#1}}
\newcommand{\sO}{\mathcal{O}}
\newcommand{\cyclic}[1]{\langle#1\rangle}
\newcommand{\units}[1]{#1 ^{\times}}
\newcommand{\la}{\langle}
\newcommand{\ra}{\rangle}
%----Switch phi and varphi
\let\temp\phi
\let\phi\varphi
\let\varphi\temp

\newcommand{\C}{\mathbb{C}}
\newcommand{\F}{\mathbb{F}}
\newcommand{\N}{\mathbb{N}\xspace}
\newcommand{\I}{\mathbb{I}\xspace}
\newcommand{\R}{\mathbb{R}\xspace}
\newcommand{\Z}{\mathbb{Z}\xspace}
\newcommand{\Q}{\mathbb{Q}\xspace}
\newcommand{\G}{\mathbb{G}\xspace}
\DeclareMathOperator{\Spec}{Spec}
\DeclareMathOperator{\res}{res}
\DeclareMathOperator{\Tr}{Tr}
\DeclareMathOperator{\ord}{ord}
\DeclareMathOperator{\Sym}{Sym}
\DeclareMathOperator{\dv}{div}
\DeclareMathOperator{\alb}{alb}
\DeclareMathOperator{\img}{Im}
\DeclareMathOperator{\et}{et}
\DeclareMathOperator{\ck}{coker}
\DeclareMathOperator{\Reg}{Reg}
\DeclareMathOperator{\Cor}{Cor}
\DeclareMathOperator{\Ac}{at}
\DeclareMathOperator{\supp}{supp}
\DeclareMathOperator{\Hom}{Hom}
\DeclareMathOperator{\Pic}{Pic}
\DeclareMathOperator{\Gal}{Gal}
\DeclareMathOperator{\fc}{frac}
\DeclareMathOperator{\Ann}{Ann}
\DeclareMathOperator{\Mod}{Mod}
\DeclareMathOperator{\Cone}{Cone}
\DeclareMathOperator{\FI}{FI}
\DeclareMathOperator{\End}{End}
\DeclareMathOperator{\Alb}{Alb}
\DeclareMathOperator{\Ext}{Ext}
\DeclareMathOperator{\ab}{ab}
\DeclareMathOperator{\Jac}{Jac}
\DeclareMathOperator{\coker}{coker}
\DeclareMathOperator{\fr}{frac}


%----Analysis
\newcommand{\dd}[2][]{\frac{\partial^{#1}}{\partial {#2}^{#1}}}
\newcommand{\summ}{\sum\limits}
\newcommand{\norm}[1]{\left \vert \left \vert #1 \right \vert \right \vert}
\newcommand{\thicc}{\bigg}
\newcommand{\eps}{\varepsilon}


%--------Theorem environments--------
\newtheorem{definition}{Definition}[]
\newtheorem{lemma}{Lemma}[]
\newtheorem{corollary}{Corollary}[]
\newtheorem{theorem}{Theorem}[]
\theoremstyle{remark}
\newtheorem*{claim}{Claim}


\newenvironment{solution}
{\begin{proof}[Solution]}
{\end{proof}}


\makeatletter
\newcommand{\thickhline}{%
    \noalign {\ifnum 0=`}\fi \hrule height 1pt
    \futurelet \reserved@a \@xhline
}
\newcolumntype{"}{@{\hskip\tabcolsep\vrule width 1pt\hskip\tabcolsep}}
\makeatother

% --------Problem environment--------
\setlength\parindent{0pt}
\setcounter{secnumdepth}{0}
\newcounter{partCounter}
\newcounter{homeworkProblemCounter}
\setcounter{homeworkProblemCounter}{1}


\newenvironment{homeworkProblem}[1][-1]{
    \ifnum#1>0
        \setcounter{homeworkProblemCounter}{#1}
    \fi
    \section{Problem \arabic{homeworkProblemCounter}}
    \setcounter{partCounter}{1}
    \stepcounter{homeworkProblemCounter}
}


%--------Metadata--------
\title{MATH 7310 Homework 1}
\author{James Harbour}


\begin{document}
\maketitle

\begin{homeworkProblem}
    A family of sets $\mc{R}\sub\mc{P}(X)$ is called a \emph{ring} if it is closed under finite unions and differences. A ring that is closed under countable unions is called a \emph{$\sigma$-ring}.

    \textbf{(a)} Rings (resp. $\sigma$-rings) are closed under finite (resp. countable) intersections.
    \begin{proof}
      Let $\mc{R}$ be ring (resp. $\sigma$-ring). Let $(E_i)_{i\in I}$ be a finite (resp. countable) family of elements of $\mc{R}$ with finite (resp. countable) indexing set $I$. Let $j\in I$ be arbitrary. For any set $F\sub X$, note that
      \[
        E_j \setminus (E_j \setminus F) = (E_j\cap F) \cup (E_j\setminus E_j) = E_j \cap F
      \]
      Hence, utilizing De Morgan's laws, we write
      \[
        \bigcap_{i\in I} E_i = E_j \cap \bigcap_{i\in I\setminus\{ j\}} E_i = E_j \setminus \left(E_j \setminus \left(\bigcap_{i\in I\setminus\{ j\}} E_i \right)\right) = E_j \setminus \left(\bigcup_{i\in I\setminus\{ j\}}( E_j \setminus E_i )\right) \in \mc{R}
      \]
      as desired.
    \end{proof}

    \textbf{(b)} If $\mc{R}$ is a ring (resp. $\sigma$-ring), then $\mc{R}$ is an algebra (resp. $\sigma$-algebra) if and only if $X\in \mc{R}$.

    \begin{proof}\ \\
      \underline{$\implies$}: Let $E\in \mc{R}$. Then $\emptyset = E\setminus E \in \mc{R}$. As $\mc{R}$ is an algebra (resp. $\sigma$-algebra), it follows that $X= \emptyset^c \in \mc{R}$.

      \underline{$\impliedby$}: Suppose that $X\in\mc{R}$. As $\mc{R}$ is already a ring (resp. $\sigma$-algebra), to show $\mc{R}$ is an algebra (resp.$\sigma$-algebra) it suffices to show that $\mc{R}$ is closed under complements. Suppose $E\in \mc{R}$. As $X\in \mc{R}$ and $\mc{R}$ is closed under differences, it follows that $E^c = X\setminus E \in \mc{R}$.
    \end{proof}

    \textbf{(c)} If $\mc{R}$ is a $\sigma$-ring, then $\{ E\sub X : E\in \mc{R} \text{ or } E^c\in \mc{R}\}$ is a $\sigma$-algebra.

    \begin{proof}
      Let $\Sigma = \{ E\sub X : E\in \mc{R} \text{ or } E^c\in \mc{R}\}$. By construction, $\Sigma$ is closed under complements. Suppose that $(E_i)_{i\in I}$ is a countable family of elements of $\Sigma$, where $I$ is an indexing set. Let $I_0 = \{ i\in I : E_i \in \mc{R}\}$. Note that $\bigcup_{i\in I_0} E_i \in \mc{R}\sub \Sigma$. Moreover
      \[
        \left(\bigcup_{j\in I\setminus I_0} E_j \right)^c =  \bigcap_{j\in I\setminus I_0} E_j ^c \in \mc{R},
      \]
      so $\bigcap_{j\in I\setminus I_0} E_j \in \Sigma$. Hence
      \[
        \bigcup_{i\in I} E_i = \bigcup_{i\in I_0} E_i \cup \bigcup_{j\in I\setminus I_0} E_j.
      \]
      Thus, it suffices to show that, if $E,F \in \Sigma$ such that $E\in \mc{R}$ and $F^c\in\mc{R}$, then $E\cup F\in \Sigma$. Observe that
      \[
        E\cup F = E\cup E \cup F = E\cup (E\setminus F^c) \in \mc{R},
      \]
      so $\Sigma$ is a $\sigma$-algebra.
    \end{proof}

    \textbf{(d)} If $\mc{R}$ is a $\sigma$-ring, then $\{ E\sub X : E\cap F\in \mc{R} \text{ for all } F\in\mc{R}\}$ is a $\sigma$-algebra.

    \begin{proof}
      Let $\Sigma = \{ E\sub X : E\cap F\in \mc{R} \text{ for all } F\in\mc{R}\}$. Suppose $(E_i)_{i\in I}$ is a countable family of elements of $\Sigma$. For any $F\in \mc{R}$,
      \[
        F\cap \bigcup_{i\in I} E_i = \bigcup_{i\in I} F\cap E_i \in \mc{R}
      \]
      by definition, so $\bigcup_{i\in I} E_i \in \Sigma$.
      Now suppose $E\in \Sigma$. Let $F\in \mc{R}$. Observe that, as $E\cap F \in \mc{R}$ and $\mc{R}$ is closed under differences,
      \[
        E^c\cap F = (X\setminus E) \cap F = X \cap(F\setminus E) = F\setminus E = (F\setminus E) \cup (F\setminus F ) = F\setminus (E\cap F)
      \]



      Thus, $E^c \in \Sigma$, so $\Sigma$ is a $\sigma$-algebra.
    \end{proof}
\end{homeworkProblem}

\begin{homeworkProblem}
  An algebra $\mc{A}$ is a $\sigma$-algebra if and only if $\mc{A}$ is closed under countable increasing unions.

  \begin{proof}
    The forward direction follows \emph{a fortiori} by the definition of a $\sigma$-algebra. For the reverse direction, suppose that $\mc{A}$ is closed under countable increasing unions. As $\mc{A}$ is already an algebra, it suffices to show that $\mc{A}$ is closed under arbitrary countable unions. Let $(E_i)_{i=1}^{\infty}$ be a countable family of elements of $\mc{A}$. Define a new sequence $(E_j')_{j=1}^{\infty}$ by $E_j' := E_1 \cup \cdots \cup E_j$. Note that, as $\mc{A}$ is closed under finite unions, $E_j' \in \mc{A}$ for all $j\in \N$. Moreover, $(E_j')_{j=1}^{\infty}$ is an increasing sequence of sets by construction, so
    \[
      \bigcup_{n=1}^{\infty}E_n = \bigcup_{n=1}^{\infty} \bigcup_{i=1}^{n} E_i = \bigcup_{n=1}^{\infty} E_n' \in \mc{A}.
    \]
  \end{proof}
\end{homeworkProblem}

\begin{homeworkProblem}[3]
  Let $X,Y$ be sets and $f:X\to Y$.

  \textbf{(i)} If $\Sigma \sub \mc{P}(X)$ is a $\sigma$-algebra, show that \[ \{ E\sub Y : f^{-1}(E)\in \Sigma\}\]
  is a $\sigma$-algebra.

  \begin{proof}
    Let $\mf{A} = \{ E\sub Y : f^{-1}(E)\in \Sigma\}$. Suppose $(E_i)_{i\in I}$ is a countable collection of elements of $\mf{A}$ where $I$ is an indexing set. Then
    \[
      f^{-1}\left(\bigcup_{i\in I}E_i\right) = \bigcup_{i\in I}f^{-1}(E_i) \in \Sigma
    \]
    by definition, so $\bigcup_{i\in I}E_i \in \mf{A}$. Now let $E\in \mf{A}$.
    \[
      f^{-1}(Y\setminus E) = f^{-1}(Y)\setminus f^{-1}(E) = X\setminus f^{-1}(E) \in \Sigma,
    \]
    so $E^c = Y\setminus E \in \mf{A}$.
  \end{proof}

  \textbf{(ii)} If $\Sigma \sub \mc{P}(X)$ is a $\sigma$-algebra, show that \[ \{ E\sub X : E=f^{-1}(F) \text{ for some } F\in \Sigma\}\]
  is a $\sigma$-algebra.

  \begin{proof}
    Let $\mf{A} = \{ E\sub X : E=f^{-1}(F) \text{ for some } F\in \Sigma\}$. Suppose that $(E_i)_{i\in I}$ is a countable collection of elements of $\mf{A}$. Then there exists a countable collection $(F_i)_{i\in I}$ of elements of $\Sigma$ such that $E_i = f^{-1}(F_i)$ for all $i\in I$. Observe that
    \[
      \bigcup_{i\in I} E_i = \bigcup_{i\in I} f^{-1}(F_i) = f^{-1}\left(\bigcup_{i\in I} F_i\right),
    \]
    so $\bigcup_{i\in I}E_i\in \mf{A}$. Now let $E\in \mf{A}$. There exists an $F\in \Sigma$ such that $E = f^{-1}(F)$. Then
    \[
      X\setminus E = X\setminus f^{-1}(F) = f^{-1}(Y)\setminus f^{-1}(F) = f^{-1}(Y\setminus F),
    \]
    so $E^c = X\setminus E \in \mf{A}$.
  \end{proof}

  \textbf{(iii)} If $Y$ is a countable, show that \[ \{ E\sub X : E=f^{-1}(F) \text{ for some } F\sub Y\}\] is the $\sigma$-algebra generated by $\{ f^{-1}(\{ y\}) : y\in Y\}$.

  \begin{proof}

    Let $\mf{A} = \{ E\sub X : E=f^{-1}(F) \text{ for some } F\sub Y\}$ and $\mc{S} = \{ f^{-1}(\{ y\}) : y\in Y\}$. Note that, by part (ii) with $\Sigma = \mc{P}(X)$, $\mf{A}$ is a $\sigma$-algebra. Clearly $\mc{S}\sub \mf{A}$, so by minimality $\Sigma(\mc{S})\sub \mf{A}$.\\

    On the other hand, suppose $E\in \mf{A}$. Then there exists some $F\sub Y$ such that $E = f^{-1}(F)$. Observe that then,
    \[
      E = f^{-1}(F) = f^{-1}\left(\bigcup_{y\in F} \{ y\}\right) = \bigcup_{y\in F} f^{-1}(\{ y\}) \in \Sigma(\mc{S})
    \]
    since $F\sub Y$ is countable. Hence $\mf{A} \sub \Sigma(\mc{S})$.
  \end{proof}
\end{homeworkProblem}

\begin{homeworkProblem}
  \textbf{(i)} Suppose that $X$ is a set. Let $J$ be a countable set and suppose that $X=\bigsqcup_{j\in J} A_j$ with $A_j \neq \emptyset$ for all $j\in J$. Show that the $\sigma$-algebra generated by $(A_j)_{j\in J}$ is $\{\bigcup_{j\in J_0} A_j : J_0\sub J\}$.

  \begin{proof}
    Let $\mf{A} = \{\bigcup_{j\in J_0} A_j : J_0\sub J\}$ and $\mc{A} = \{ A_j : j\in J\}$. We show first that $\mf{A}$ is indeed a $\sigma$-algebra.

    Suppose that $(J_n)_{n\in \N}$ is a sequence of subsets of $J$ and let $I = \bigcup_{n=1}^{\infty}J_n \sub J$. Then
    \[
      \bigcup_{n=1}^{\infty}\bigcup_{j\in J_n} A_j = \bigcup_{j\in I} A_j \in \mf{A}.
    \]
    Now let $J_0 \sub J$. Then
    \[
      \left(\bigsqcup_{j\in J_0} A_j\right) = X\setminus\bigsqcup_{j\in J_0} A_j = \bigsqcup_{j\in J\setminus J_0} A_j \in \mf{A},
    \]
    so $\mf{A}$ is a $\sigma$-algebra. Clearly $\mc{A}\sub\mf{A}$, so by minimality $\Sigma(\mc{A})\sub\mf{A}$. \\

    On the other hand, as $J$ is countable, each element of $\mf{A}$ is an at most countable union of elements of $\mc{A}$, so $\mf{A} \sub \Sigma(\mc{A})$.
  \end{proof}

  \textbf{(ii)} Show that the $\sigma$-algebra generated by a finite collection of sets is finite.

  \begin{proof}
    Let $\mc{A} = \{ A_1, \ldots, A_n\}\sub \mc{P}(X)$ be a finite collection of sets. Define a new collection $\mc{A'}$ as follows:
    \begin{center}
      For all $(x_1, \ldots, x_n) \in (\Z/2\Z)^n$, define a set $A_{(x_1,\ldots,x_n)}' \in \mc{A'}$ by
      \[
        A_{(x_1,\ldots,x_n)}' = \left(\bigcap_{\substack{1\leq i\leq n \\ \text{s.t. } x_i = 1}} A_i\right)\setminus \left(\bigcup_{\substack{1\leq j\leq n \\ \text{s.t. } x_j = 0}} A_j\right).
      \]
    \end{center}

    Note that
    \[
      A_i = \bigsqcup_{\substack{(x_1,\ldots,x_n)\in (\Z/2\Z)^n \\ x_i = 1}} A_{(x_1, \ldots, x_n)}',
    \]
    so $\mc{A}\sub \Sigma(\mc{A'}) \implies \Sigma(\mc{A})\sub\Sigma(\mc{A'})$. As each $A'\in \mc{A'}$ is a finite intersection, complement, and union of elements of $\mc{A}$, $\mc{A'}\sub \Sigma(\mc{A}) \implies \Sigma(\mc{A'})\sub \Sigma(\mc{A})$. Hence, $\Sigma(\mc{A}) = \Sigma(\mc{A'})$. \\

    The elements of $\mc{A'}$ are pairwise disjoint by constuction, and there are at most $2^n$ of them. Label the elements of $\mc{A'}$ as $A_1',\ldots, A_l'$ where $l\leq 2^n$. Hence, by part (i), $\Sigma(\mc{A'}) = \{\bigcup_{j\in J_0} A_j : J_0\sub \{1,\ldots, k\}\}$ is finite of order at most $2^k \leq 2^{2^n}$.


  \end{proof}

\end{homeworkProblem}


\begin{homeworkProblem}
  Let $(X,d_X)$ be a metric space. A collection $\mc{B}$ of open sets in $X$ is said to be a \emph{basis} if for every open $U\sub X$ and every $x\in U$, there is a $B\in\mc{B}$ so that $x\in B\sub U$. If $\mc{B}$ is a \emph{countable} basis of open sets in $X$, show that the $\sigma$-algebra generated by $\mc{B}$ coincides with the Borel sets in $X$.

  \begin{proof}
    Let $\mf{B}$ be the collection of Borel sets in $X$. This collection is a $\sigma$-algebra by definition. As $\mc{B}$ is a collection of open sets in $X$, $\mc{B}\sub \mf{B}$. Hence, by minimality, $\Sigma(\mc{B}) \sub \mf{B}$.\\

    Suppose that $U\sub X$ is open. For all $x\in U$, there exists a $B_x\in \mc{B}$ such such that $x\in B_x \sub U$. As $\mc{B}$ is countable, the union $U = \bigcup_{x\in U} B_x$ is in fact countable, so $U = \bigcup_{x\in U} B_x \in \Sigma(\mc{B})$. Thus, $\Sigma(\mc{B})$ contains all open sets in $X$, whence by minimality of $\mf{B}$, $\Sigma(\mc{B})\sub \mf{B}$.
  \end{proof}
\end{homeworkProblem}

\end{document}

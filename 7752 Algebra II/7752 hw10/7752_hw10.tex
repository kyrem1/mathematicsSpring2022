\documentclass[12pt,letterpaper]{article}

%--------Packages--------
\usepackage{amsmath, amsthm, amssymb}
\usepackage{xspace}
\usepackage{graphicx}
\usepackage{amssymb}
\usepackage{array}
\usepackage{braket}
\usepackage{multicol}
\usepackage{mathtools}
\usepackage{enumerate}
\usepackage{delarray}
\usepackage{mathtools}
\usepackage{fullpage}
\usepackage{faktor} % For quotients
\usepackage{mathrsfs}
\usepackage{quiver}
\usepackage{tikz}

\usepackage[linguistics]{forest}




%--------Page Setup--------

\pagestyle{empty}%

\setlength{\hoffset}{-1.54cm}
\setlength{\voffset}{-1.54cm}

\setlength{\topmargin}{0pt}
\setlength{\headsep}{0pt}
\setlength{\headheight}{0pt}

\setlength{\oddsidemargin}{0pt}

\setlength{\textwidth}{195mm}
\setlength{\textheight}{250mm}


%--------Macros--------

\newcommand{\ilm}[1]{\begin{psmallmatrix} #1 \end{psmallmatrix}}
\newcommand{\ilmb}[1]{\boxed{\begin{smallmatrix} #1 \end{smallmatrix}}}

\newcommand{\sub}{\subseteq}
\newcommand{\lcm}{\text{lcm}}
\newcommand{\ms}[1]{\mathscr{#1}}
\newcommand{\mc}[1]{\mathcal{#1}}
\newcommand{\mf}[1]{\mathfrak{#1}}
\newcommand{\sO}{\mathcal{O}}
\newcommand{\cyclic}[1]{\langle#1\rangle}
\newcommand{\units}[1]{#1 ^{\times}}
\newcommand{\la}{\langle}
\newcommand{\ra}{\rangle}
\newcommand{\lr}[1]{\left(#1\right)}
\newcommand{\divides}{\bigm|}
%----Switch phi and varphi
\let\temp\phi
\let\phi\varphi
\let\varphi\temp

\newcommand{\C}{\mathbb{C}}
\newcommand{\F}{\mathbb{F}}
\newcommand{\N}{\mathbb{N}\xspace}
\newcommand{\I}{\mathbb{I}\xspace}
\newcommand{\R}{\mathbb{R}\xspace}
\newcommand{\Z}{\mathbb{Z}\xspace}
\newcommand{\Q}{\mathbb{Q}\xspace}
\newcommand{\G}{\mathbb{G}\xspace}
\DeclareMathOperator{\Spec}{Spec}
\DeclareMathOperator{\res}{res}
\DeclareMathOperator{\Tr}{Tr}
\DeclareMathOperator{\ord}{ord}
\DeclareMathOperator{\Sym}{Sym}
\DeclareMathOperator{\dv}{div}
\DeclareMathOperator{\alb}{alb}
\let\Im\relax
\DeclareMathOperator{\Im}{Im}
\DeclareMathOperator{\et}{et}
\DeclareMathOperator{\ck}{coker}
\DeclareMathOperator{\Reg}{Reg}
\DeclareMathOperator{\Cor}{Cor}
\DeclareMathOperator{\Ac}{at}
\DeclareMathOperator{\supp}{supp}
\DeclareMathOperator{\Hom}{Hom}
\DeclareMathOperator{\Pic}{Pic}
\DeclareMathOperator{\Gal}{Gal}
\DeclareMathOperator{\fc}{frac}
\DeclareMathOperator{\Ann}{Ann}
\DeclareMathOperator{\Mod}{Mod}
\DeclareMathOperator{\Cone}{Cone}
\DeclareMathOperator{\FI}{FI}
\DeclareMathOperator{\End}{End}
\DeclareMathOperator{\Alb}{Alb}
\DeclareMathOperator{\Ext}{Ext}
\DeclareMathOperator{\ab}{ab}
\DeclareMathOperator{\Jac}{Jac}
\DeclareMathOperator{\coker}{coker}
\DeclareMathOperator{\fr}{frac}
\DeclareMathOperator{\spn}{span}
\DeclareMathOperator{\im}{im}
\DeclareMathOperator{\rk}{rk}
\DeclareMathOperator{\GL}{GL}
\DeclareMathOperator{\Aut}{Aut}
\DeclareMathOperator{\ch}{char}
\DeclareMathOperator{\Fix}{Fix}


%----Analysis
\newcommand{\dd}[2][]{\frac{\partial^{#1}}{\partial {#2}^{#1}}}
\newcommand{\summ}{\sum\limits}
\newcommand{\norm}[1]{\left \vert \left \vert #1 \right \vert \right \vert}
\newcommand{\thicc}{\bigg}
\newcommand{\eps}{\varepsilon}
\newcommand*\cls[1]{\overline{#1}}


%--------Theorem environments--------
\newtheorem{definition}{Definition}[]
\newtheorem{lemma}{Lemma}[]
\newtheorem{corollary}{Corollary}[]
\newtheorem{theorem}{Theorem}[]
\theoremstyle{remark}
\newtheorem*{claim}{Claim}


\newenvironment{solution}
{\begin{proof}[Solution]}
{\end{proof}}


\makeatletter
\newcolumntype{"}{@{\hskip\tabcolsep\vrule width 1pt\hskip\tabcolsep}}
\makeatother

% --------Problem environment--------
\setlength\parindent{0pt}
\setcounter{secnumdepth}{0}
\newcounter{partCounter}
\newcounter{homeworkProblemCounter}
\setcounter{homeworkProblemCounter}{1}


\newenvironment{homeworkProblem}[1][-1]{
    \ifnum#1>0
        \setcounter{homeworkProblemCounter}{#1}
    \fi
    \section{Problem \arabic{homeworkProblemCounter}}
    \setcounter{partCounter}{1}
    \stepcounter{homeworkProblemCounter}
}


%--------Metadata--------
\title{MATH 7752 Homework 10}
\author{James Harbour}


\begin{document}
\maketitle

\begin{homeworkProblem}
  Let $F$ be a field, $f(x)\in F[x]$ be an irreducible separable polynomial over $F$ of degree $n$ and let $K$ be a splitting field of $f(x)$.\\

  \textbf{(a)}: Prove that $|\Gal(K/F)|$ is a multiple of $n$ and divides $n!$.

  \begin{proof}
    By separability, $[K:F] = |\Gal(K/F)|$. Let $\alpha\in K$ be a root of $f$. By irreducibility of $f$, $f = \mu_{\alpha,F}$ whence
    \[
      |\Gal(K/F)| = [K:F] = [K: F(\alpha)][F(\alpha): F] = [K:F(\alpha)]\deg(\mu_{\alpha,F}) = [K:F(\alpha)]\cdot n,
    \]
    so $n\divides |\Gal(K/F)|$.\\

    To see that $|Gal(K/F)|$ divides $n!$, note as $K$ is the splitting field of irreducible $f$ of degree $n$, we have an injective group homomorphism $\rho:\Gal(K/F)\hookrightarrow S_n$ induced by a labeling of the roots of $f$.
  \end{proof}

  \textbf{(b)}: Let $n=3$. Prove that $\Gal(K/F)$ is isomorphic to either $\Z/3\Z$ or $S_3$.

  \begin{proof}
    Let $G = \Gal(K/F)$. By part (a), $3\divides |G| = |\rho(G)|$ so by Sylow's existence theorem there is some subgroup $H\sub \rho(G)$ such that $|H| = 3$. Hence, there is some $3$-cycle $\sigma\in H$ such that $H = \cyclic{\sigma}$. \\


    If $\rho(G) = H$, then $G\cong \Z/3\Z$ and we would be done. \\

    Otherwise, suppose that $\rho(G)\neq H$. Then there exists some $\tau\in \rho(G)\setminus H$. This $\tau$ must be a $2$-cycle by noting that $H$ contains the identity and all of the $3$-cycles in $S_3$. Noting that $\sigma$ is a $3$-cycle, $\tau$ is a transposition, and $3$ is prime, it follows by elementary group theory that $S_3$ is generated by $\sigma$ and $\tau$. Thus $\rho(G) = S_3$, so $G\cong S_3$.
  \end{proof}

  \textbf{(c)}: Let $n=4$ and assume that $|\Gal(K/F)|=8$. Determine the isomorphism class of $\Gal(K/F)$.

  \begin{proof}
    Noting that $|\rho(G)|=8= 2^3$, it follows that $\rho(G)$ is a Sylow $2$-subgroup of $S_4$. As all Sylow subgroups for a fixed prime are isomorphic via conjugation (by Sylow's conjugate theorem), it suffices to determine the isomorphism class of one Sylow $2$-subgroup of $S_4$. Noting that the natural embedding of $D_8$ into $S_4$ induced by a labeling of the vertices of the square upon which $D_8$ acts, it follows that the identified copy of $D_8$ inside $S_4$ is a Sylow $2$-subgroup of $S_4$. Thus, all Sylow $2$-subgroups of $S_4$ are isomorphic to $D_8$, whence $\Gal(K/F)\cong D_8$.
  \end{proof}

\end{homeworkProblem}


\begin{homeworkProblem}
  Let $f(x)\in\Q[x]$ be an irreducible polynomial of degree $n$, and let $K$ be a splitting field of $f(x)$ contained in $\C$. Label the roots of $f(x)$ by $\alpha_1,\ldots, \alpha_n$ (in some order), and let $\rho:\Gal(K/\Q)\hookrightarrow S_n$ be the associated embedding.

  \textbf{(a)}: Assume that $f(x)$ has at least one non-real root. Prove that the complex conjugation gives an element $\tau$ of $\Gal(K/\Q)$ of order $2$. What can you say about $\tau$ if $f(x)$ has precisely two non-real roots?

  \begin{proof}
    Consider the $\Q$-embedding $\tau:K\hookrightarrow\cls{Q}\sub\C$ given by complex conjugation. By normality of $K/\Q$, it follows that $\tau(K) = K$, so $\tau\in \Gal(K/\Q)$. As $f$ has a non-real root, it follows that $\tau\neq id_K$. Thus, noting that $\tau^2 = id_K$, $\tau$ has order 2.\\

    If $f$ has precisely two non-real roots, as $\Gal(K/\Q)$ acts transitively on the set of roots of $f$ and fixes all of the real roots it follows that the two non-real roots say $\alpha_1$ and $\alpha_2$ are complex conjugates of each other. In this case, $\rho(\tau)$ is in fact a single transposition.
  \end{proof}

  \textbf{(b)}: Suppose that the degree $n$ of $f(x)$ is a  prime number, and that $f(x)$ has precisely two non-real roots. Prove that $\Gal(K/\Q)$ is isomorphic to $S_n$. \textbf{Hint:} You might need to recall some facts from Algebra I about generators of $S_n$.

  \begin{proof}
    Let $p = n$ be a prime number. Let $G = \Gal(K/\Q)$. By problem 1 part (a), $p\divides |\rho(G)| \divides p!$, whence by Sylow's theorem there exists a subgroup $H\sub\rho(G)$ such that $|H| = p$. This subgroup is cyclic of order $p$, so there exists a $p$-cycle $\sigma\in H$ such that $H=\cyclic{\sigma}$. Let $\tau\in\rho(G)$ be as in part (a). Then $\tau\not\in H$ as all elements of $H$ are either the identity or have order $p$. \\

    As $\tau$ is a transposition, $\sigma$ is a $p$-cycle, and $p$ is prime, it follows that $\S_p = \cyclic{\sigma,\tau}\sub \rho(G)$ whence $\rho(G) = S_p$, so $G\cong S_p$.
  \end{proof}
\end{homeworkProblem}

\begin{homeworkProblem}
  Let $K$ be the splitting field of $f(x)=x^4-2\in\Q[x]$.\\

  \textbf{(a)}:  Choose an order on the set of roots of $f(x)$ and describe the associated embedding $\Gal(K/\Q)\hookrightarrow S_4$. (You can use the information you obtained in Homework 8).

  \begin{proof}
    Let $\zeta = e^{2\pi i/4} = i$ and $\alpha_j = \zeta^{j-1}\sqrt[4]{2}$ for $1\leq j \leq 4$. Then $f(x) = (x-\alpha_1)\cdots(x-\alpha_4)$ and $K = F(\alpha_1,\ldots,\alpha_4) = F(\sqrt[4]{2},i)$.\\

    As $\deg_{\Q(\sqrt[4]{2})}(i) = \deg_{\Q}(i)$, it follows by the explicit description of Galois groups that there exist $\sigma,\tau\in\Gal(K/\Q)$ such that
    \begin{align*}
      \sigma(\sqrt[4]{2}) &= \zeta \sqrt[4]{2} &\tau(\sqrt[4]{2}) &= \sqrt[4]{2}\\
      \sigma(\zeta) &= \zeta &\tau(\zeta) &= \zeta^3.
    \end{align*}
    Letting $\rho:\Gal(K/\Q)\hookrightarrow S_4$ be the embedding associated to the aforementioned labeling of the roots of $f$, it follows that $\rho(\sigma) = (1234)$ and $\rho(\tau) = (24)$. The subgroup of $S_4$ generated by these two elements is an isomorphic copy of $D_8$, so it follows from the fact that $[K:\Q] = 8$ that $\Gal(K/\Q) = \cyclic{\sigma,\tau}$, whence these values of $\rho$ completely determine the embedding.
  \end{proof}

  \textbf{(b)}: Describe all subgroups of $\Gal(K/\Q)$ and the corresponding subfields of $K$.

  \begin{proof}
    For the subfield diagram, we have
    \[\begin{tikzcd}
    	&& {\langle \sigma,\tau\rangle} \\
    	& {\langle \sigma^2,\sigma\tau\rangle} & {\langle \sigma\rangle} & {\langle \sigma^2,\tau\rangle} \\
    	{\langle \sigma\tau\rangle} & {\langle \sigma^3\tau\rangle} & {\langle \sigma^2\rangle} & {\langle \sigma^2\tau\rangle} & {\langle \tau\rangle} \\
    	&& {\{e\}}
    	\arrow[no head, from=3-1, to=4-3]
    	\arrow[no head, from=3-2, to=4-3]
    	\arrow[no head, from=3-3, to=4-3]
    	\arrow[no head, from=3-4, to=4-3]
    	\arrow[no head, from=3-5, to=4-3]
    	\arrow[no head, from=2-2, to=3-2]
    	\arrow[no head, from=2-2, to=3-1]
    	\arrow[no head, from=2-2, to=3-3]
    	\arrow[no head, from=2-3, to=3-3]
    	\arrow[no head, from=2-4, to=3-3]
    	\arrow[no head, from=2-4, to=3-4]
    	\arrow[no head, from=2-4, to=3-5]
    	\arrow[no head, from=1-3, to=2-3]
    	\arrow[no head, from=1-3, to=2-2]
    	\arrow[no head, from=1-3, to=2-4]
    \end{tikzcd}\]

    with its corresponding subfield diagram

    \[\begin{tikzcd}
    	&& {\mathbb{Q}(\sqrt[4]{2},i)} \\
    	{\mathbb{Q}((1+i)\sqrt[4]{2})} & {\mathbb{Q}((1-i)\sqrt[4]{2})} & {\mathbb{Q}(\sqrt{2},i)} & {\mathbb{Q}(i\sqrt[4]{2})} & {\mathbb{Q}(\sqrt[4]{2})} \\
    	& {\mathbb{Q}(i\sqrt{2})} & {\mathbb{Q}(i)} & {\mathbb{Q}(\sqrt{2})} \\
    	&& {\mathbb{Q}}
    	\arrow[no head, from=2-1, to=3-2]
    	\arrow[no head, from=2-2, to=3-2]
    	\arrow[no head, from=2-3, to=3-3]
    	\arrow[no head, from=2-3, to=3-2]
    	\arrow[no head, from=2-3, to=3-4]
    	\arrow[no head, from=2-4, to=3-4]
    	\arrow[no head, from=2-5, to=3-4]
    	\arrow[no head, from=3-2, to=4-3]
    	\arrow[no head, from=3-3, to=4-3]
    	\arrow[no head, from=3-4, to=4-3]
    	\arrow[no head, from=2-1, to=1-3]
    	\arrow[no head, from=2-2, to=1-3]
    	\arrow[no head, from=1-3, to=2-3]
    	\arrow[no head, from=1-3, to=2-4]
    	\arrow[no head, from=1-3, to=2-5]
    \end{tikzcd}\]

    where correspondence between subgroups and subfields is depicted via vertically flipping one of the diagrams. The majority of these fields were found via direct computation of fixed fields; however, those corresponding to $\cyclic{\sigma\tau}$ and $\cyclic{\sigma^3\tau}$ require more work.\\

    First, we compute that $\rho(\sigma\tau) = (12)(34)$ and $\rho(\sigma^3\tau) = (14)(23)$. Thus, $\alpha_1+\alpha_2\in K^{\cyclic{\sigma\tau}}$ and $\alpha_1+\alpha_4\in K^{\cyclic{\sigma^3\tau}}$. We claim that in fact $K^{\cyclic{\sigma\tau}} = \Q(\alpha_1+\alpha_2)$ and $K^{\cyclic{\sigma^3\tau}} = \Q(\alpha_1+\alpha_4)$. Noting that $\alpha_1+\alpha_2$ and $\alpha_1+\alpha_4$ are both roots of the irreducible polynomial $x^4+8$, it follows that
    \[
      [K:\Q(\alpha_1+\alpha_2)] = 2 = |\cyclic{\sigma\tau}| = [K:K^{\cyclic{\sigma\tau}}]
    \]
    and
    \[
      [K:\Q(\alpha_1+\alpha_4)] = 2 = |\cyclic{\sigma^3\tau}| = [K:K^{\cyclic{\sigma^3\tau}}]
    \]
    whence the claim follows by the inclusions.

  \end{proof}

\end{homeworkProblem}


\begin{homeworkProblem}
  Let $K/F$ and $L/F$ be field extensions.
  \begin{enumerate}
    \item Assume that $L/F$ is finite Galois. Show that $KL/K$ is also  Galois.
    \item Suppose that both $K/F$ and $L/F$ are Galois extensions.
    \begin{enumerate}
      \item Prove that the extension $KL/F$ is also Galois and there is a natural embedding $\iota:\Gal(KL/F)\rightarrow\Gal(K/F)\times \Gal(L/F)$.
      \item Assume now that $K/F$ and $L/F$ are both finite. Prove that the map $\iota$ in part (i) is an isomorphism if and only if $K\cap L=F$.
    \end{enumerate}
  \end{enumerate}
\end{homeworkProblem}


\end{document}

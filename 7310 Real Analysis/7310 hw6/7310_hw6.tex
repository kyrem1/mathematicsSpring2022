\documentclass[12pt,letterpaper]{article}

%--------Packages--------
\usepackage{amsmath, amsthm, amssymb}
\usepackage{xspace}
\usepackage{graphicx}
\usepackage{hhline}
\usepackage{amssymb}
\usepackage{array}
\usepackage{braket}
\usepackage{multicol}
\usepackage{mathtools}
\usepackage{enumerate}
\usepackage{delarray}
\usepackage{mathtools}
\usepackage{fullpage}
\usepackage{faktor} % For quotients
\usepackage{mathrsfs}

\usepackage[italicdiff]{physics} % For differentials
\usepackage{bbm} % For indicator

% \usepackage{quiver}
\usepackage[linguistics]{forest}




%--------Page Setup--------

\pagestyle{empty}%

\setlength{\hoffset}{-1.54cm}
\setlength{\voffset}{-1.54cm}

\setlength{\topmargin}{0pt}
\setlength{\headsep}{0pt}
\setlength{\headheight}{0pt}

\setlength{\oddsidemargin}{0pt}

\setlength{\textwidth}{195mm}
\setlength{\textheight}{250mm}


%--------Macros--------

\newcommand{\sub}{\subseteq}
\newcommand{\lcm}{\text{lcm}}
\newcommand{\mc}[1]{\mathcal{#1}}
\newcommand{\mf}[1]{\mathfrak{#1}}
\newcommand{\ms}[1]{\mathscr{#1}}
\newcommand{\sO}{\mathcal{O}}
\newcommand{\cyclic}[1]{\langle#1\rangle}
\newcommand{\units}[1]{#1 ^{\times}}
\newcommand{\la}{\langle}
\newcommand{\ra}{\rangle}
\newcommand{\lr}[1]{\left(#1\right)}
%----Switch phi and varphi
% \let\temp\phi
% \let\phi\varphi
% \let\varphi\temp

\newcommand{\C}{\mathbb{C}}
\newcommand{\F}{\mathbb{F}}
\newcommand{\N}{\mathbb{N}\xspace}
\newcommand{\I}{\mathbb{I}\xspace}
\newcommand{\R}{\mathbb{R}\xspace}
\newcommand{\Z}{\mathbb{Z}\xspace}
\newcommand{\Q}{\mathbb{Q}\xspace}
\newcommand{\G}{\mathbb{G}\xspace}
\DeclareMathOperator{\Spec}{Spec}
\DeclareMathOperator{\res}{res}
% \DeclareMathOperator{\Tr}{Tr}
\DeclareMathOperator{\ord}{ord}
\DeclareMathOperator{\Sym}{Sym}
% \DeclareMathOperator{\dv}{div}
\DeclareMathOperator{\alb}{alb}
\DeclareMathOperator{\img}{Im}
\DeclareMathOperator{\et}{et}
\DeclareMathOperator{\ck}{coker}
\DeclareMathOperator{\Reg}{Reg}
\DeclareMathOperator{\Cor}{Cor}
\DeclareMathOperator{\Ac}{at}
\DeclareMathOperator{\supp}{supp}
\DeclareMathOperator{\Hom}{Hom}
\DeclareMathOperator{\Pic}{Pic}
\DeclareMathOperator{\Gal}{Gal}
\DeclareMathOperator{\fc}{frac}
\DeclareMathOperator{\Ann}{Ann}
\DeclareMathOperator{\Mod}{Mod}
\DeclareMathOperator{\Cone}{Cone}
\DeclareMathOperator{\FI}{FI}
\DeclareMathOperator{\End}{End}
\DeclareMathOperator{\Alb}{Alb}
\DeclareMathOperator{\Ext}{Ext}
\DeclareMathOperator{\ab}{ab}
\DeclareMathOperator{\Jac}{Jac}
\DeclareMathOperator{\coker}{coker}
\DeclareMathOperator{\fr}{frac}
\DeclareMathOperator{\Int}{Int}



%----Analysis
\newcommand{\summ}{\sum\limits}
% \newcommand{\norm}[1]{\left\lVert#1\right\rVert}
\newcommand{\thicc}{\bigg}
\newcommand{\eps}{\varepsilon}
\newcommand*\cls[1]{\overline{#1}}
\newcommand{\ind}{\mathbbm{1}}


%--------Theorem environments--------
\newtheorem{definition}{Definition}[]
\newtheorem{lemma}{Lemma}[]
\newtheorem{corollary}{Corollary}[]
\newtheorem{theorem}{Theorem}[]
\theoremstyle{remark}
\newtheorem*{claim}{Claim}


\newenvironment{solution}
{\begin{proof}[Solution]}
{\end{proof}}


\makeatletter
\newcommand{\thickhline}{%
    \noalign {\ifnum 0=`}\fi \hrule height 1pt
    \futurelet \reserved@a \@xhline
}
\newcolumntype{"}{@{\hskip\tabcolsep\vrule width 1pt\hskip\tabcolsep}}
\makeatother

% --------Problem environment--------
\setlength\parindent{0pt}
\setcounter{secnumdepth}{0}
\newcounter{partCounter}
\newcounter{homeworkProblemCounter}
\setcounter{homeworkProblemCounter}{1}


\newenvironment{homeworkProblem}[1][-1]{
    \ifnum#1>0
        \setcounter{homeworkProblemCounter}{#1}
    \fi
    \section{Problem \arabic{homeworkProblemCounter}}
    \setcounter{partCounter}{1}
    \stepcounter{homeworkProblemCounter}
}


%--------Metadata--------
\title{MATH 7310 Homework 6}
\author{James Harbour}

\begin{document}
\maketitle
\begin{homeworkProblem}
  Let $X=Y$ be an uncountable linearly ordered set such that for each $x\in X$, $\{ y\in X: y<x\}$ is  countable. Let $\mc{M}=\mc{N}$ be the $\sigma$-algebra of countable or co-countable sets, and let $\mu=\nu$ be defined on $\mc{M}$ by $\mu(A)=0$ if $A$ is countable and $\mu(A)=1$ if $A$ is co-countable. Let $E = \{ (x,y)\in X\times X: y<x\}$. Prove that $E_x$ and $E^y$ are measurable for all $x,y$, and that $\int\int \ind_E \dd{\mu}\dd{\nu}$ and $\int\int\ind_E \dd{\nu}\dd{\mu}$ exist but are not equal.

  \begin{proof}
    For $x\in X$, define the set $S(x)=\{ y\in X: y<x\}$. Observe that, for $x\in X$,
    $E_x = \{ y\in X: (x,y)\in E\} = \{ y\in X: y < x\} = S(x)$ which is countable by assumption so $E_x$ is measurable. On the other hand, for $y\in X$, since the ordering on $X$ is total,
    \[
      X\setminus E^y= \{ x\in X : y\not\in S(x)\} = \{ x\in X : x = y \text{ or } x < y\} = \{y\}\cup S(y)
    \]
    which is countable by assumption, so $E^y$ is cocountable and thus measurable.\\

    Thus, for $x,y\in X$, the $x$ and $y$-sections of $\ind_E$, i.e. $(\ind_E)^y = \ind_{E^y}$ and $(\ind)_x = \ind_{E_x}$, are measurable. Thus, the inner integrals in each of the iterated integrals exist. To see that both of the whole iterated integrals exist, we compute for fixed $y\in X$
    \[
      \int \ind_E (x,y)\dd{\mu(x)} = \int \ind_{E^y}(x)\dd{\mu(x)} = \mu(E^y) = 1
    \]
    and for fixed $x\in X$
    \[
      \int \ind_E (x,y)\dd{\nu(y)} = \int \ind_{E_x}(y)\dd{\nu(y)} = \nu(E_x) = 0
    \]
    which are both measurable functions as they are constant functions. Hence, both of the interated integrals exist and we compute on one hand that
    \[
      \int\int \ind_E(x,y) \dd{\mu(x)}\dd{\nu(y)} = \int \mu(E^y)\dd{\nu(y)} = \int 1\dd{\nu(y)} = \nu(X) = 1
    \]
    and on the other hand that
    \[
      \int\int \ind_E(x,y) \dd{\nu(y)}\dd{\mu(x)} = \int \nu(E_x)\dd{\mu(x)} = \int 0\dd{\mu(x)} = 0.
    \]
    Thus $\int\int \ind_E \dd{\mu}\dd{\nu}$ and $\int\int\ind_E \dd{\nu}\dd{\mu}$ exist but are not equal.
  \end{proof}
\end{homeworkProblem}


\begin{homeworkProblem}
  Prove Theorem 2.39 by using Theorem 2.37 and proposition 2.12 together with the following lemmas:
  \begin{enumerate}
    [(a)]\item If $E\in \mc{M}\otimes\mc{N}$ and $\mu\otimes\nu(E) = 0$, then $\nu(E_x) = \mu(E^y) = 0$ for a.e. $x$ and $y$.
    \item If $f$ is $\ms{L}$-measurable and $f=0$ $\lambda$-a.e., then $f_x$ and $f^y$ are integrable for a.e. $x$ and $y$, and $\int f_x \dd{\nu}$=0 and $\int f^y \dd{\mu}=0$ for a.e. $x$ and $y$. (This uses completeness of $\mu$ and $\nu$.)
  \end{enumerate}

  Suppose that $(X,\mc{M},\mu)$ and $(Y,\mc{N},\nu)$ are complete, $\sigma$-finite measure spaces, and let $(X\times Y,\ms{L}, \lambda)$ be the completion of $(X\times Y,\mc{M}\otimes\mc{N},\mu\otimes\nu)$. If $f$ is $\ms{L}$-measurable and either (a) $f\geq 0$ or (b) $f\in L^1(\lambda)$, then $f_x$ is $\mc{N}$-measurable for a.e. $x$ and $f^y$ is $\mc{M}$-measurable for a.e. $y$, and in case (b) $f_x$ and $f^y$ are also integrable for a.e. $x$ and $y$. Moreover, $x\mapsto \int f_x \dd{\nu}$ and $y\mapsto \int f^y\dd{\mu}$ are measurable, and in case (b) also integrable, and
  \[
    \int f\dd{\lambda} = \int\int f(x,y) \dd{\mu(x)}\dd{\nu(y)} = \int\int f(x,y)\dd{\nu(y)}\dd{\mu(x)}.
  \]

  \begin{proof}[Proof of Lemma (a)]
    By theorem 2.36, the functions $x\mapsto \nu(E_x)$ and $y\mapsto \mu(E^y)$ are measurable and
    \[
      0 = \mu\otimes\nu(E) = \int \nu(E_x)\dd{\mu(x)} = \int\mu(E^y)\dd{\nu(y)},
    \]
    whence by nonnegativity $\nu(E_x) = 0$ for a.e. $x$ and $\mu(E^y) = 0$ for a.e. $y$.
  \end{proof}

  \begin{proof}[Proof of Lemma (b)]
    Consider the set $E = \{(x,y): f(x,y)\neq 0\}$. Then, by definition, there exists an $F\in\mc{M}\otimes\mc{N}$ such that $E\sub F$ and $\mu\otimes\nu(F) = 0$. Then, for all $x\in X$ and $y\in Y$, $F_x\in\mc{N}$ and $F^y\in\mc{M}$. As $E_x\sub F_x$ and $E^y\in F^y$, the completeness of $\mu$ and $\nu$ gives that $E_x\in \mc{N}$ and $E^y\in \mc{M}$. Combining this with lemma (a), we find that $\nu(E_x) = 0$ and $\mu(E^y) = 0$. Now, for $x\in X$, $f_x = 0$ on $Y\setminus E_x$ whence $f_x = 0$ $\nu$-a.e. and for $y\in Y$, $f^y = 0$ on $X\setminus E^y$ whence $f^y = 0$ $\mu$-a.e. Thus $f^y$ and $f_x$ are integrable and both integrate to zero.
  \end{proof}

  \begin{proof}[Proof of Theorem 2.39]
    Suppose that $f$ is $\ms{L}$-measurable. By proposition 2.12, there exists an $\mc{M}\otimes\mc{N}$-measurable function $g$ such that $f = g$ $\lambda$-a.e. As $\ms{L}\supseteq \mc{M}\otimes\mc{N}$, it follows that $g$ is $\ms{L}$-measurable so $f-g$ is $\ms{L}$-measurable and equal to $0$ $\lambda$-a.e.\\

    By lemma (b), $f_x-g_x$ and $f^y-g^y$ are integrable for a.e. $x$ and $y$ and $\int f_x-g_x\dd{\nu} = 0$ and $\int f^y-g^y\dd{\mu} = 0$ for a.e. $x$ and $y$. Now the claims for parts (a) and (b) of the theorem follow from their corresponding parts in Theorem 2.37.
  \end{proof}
\end{homeworkProblem}


\begin{homeworkProblem}[3]
    \textbf{(a)}: Suppose $(X,\Sigma,\mu)$ is a $\sigma$-finite measure space and $f\in L^+(X)$. Let
    \[
      G_f = \{(x,y)\in X\times[0,+\infty] : y\leq f(x)\}.
    \]
    Show that $G_f$ is $\Sigma\times \mc{B}_\R$-measurable and $\mu\times m(G_f) = \int f\dd{\mu}$. Show also that the same is true if the inequality in the definition of $G_f$ is made strict.

    \begin{proof}
      Let $\tilde f:X\times[0,+\infty]\to X\times [0,+\infty]$ be given by $(x,y)\mapsto(f(x),y)$ and $S:X\times [0,+\infty]\to [-\infty,+\infty]$ be given by $S(z,y) = z-y$ if $z,y$ not both $\pm\infty$ and $S(z,y) = 0$ if $z=y=\infty$. Then $S$ is measurable, and as $\pi_1 \circ \tilde f$ and $\pi_2\circ\tilde f$ are measurable, so is $\tilde f$. Hence, as intermediate codomain and domain match for the corresponding measure spaces, $S\circ\tilde f$ is measurable. Noting that $G_f = (S\circ \tilde f)^{-1}([0,+\infty])$, measurability of $S\circ \tilde f$ implies that $G_f$ is $\Sigma\times \mc{B}_\R$-measurable.\\

      Observe that, for $x\in X$, $m((G_f)_x) = m([0,f(x)]) = f(x)$. As $G_f$ is measurable, by Theorem 2.36 in Folland, the function $x\mapsto m((G_f)_x)$ is measurable and
      \[
        \mu\times m(G_f) = \int m((G_f)_x)\dd{\mu(x)} = \int f(x)\dd{\mu(x)}.
      \]
    \end{proof}

    \textbf{(b)}: Let $(X,\mu)$ be a $\sigma$-finite measure space. Fix $p\in[1,+\infty)$. Show that if $f\in L^p(X,\mu)$, then
    \[
      \norm{f}_p^{p} = p\int_{0}^{\infty} t^{p-1}\mu(\{x:|f(x)|>t\})\dd{t}.
    \]

    \begin{proof}
      Observe that, by part (a),
      \[
        \norm{f}_p^{p} = \int_X |f|^p\dd{\mu} = (\mu\times m)(G_{|f|^p}) = \int_{X\times [0,+\infty]} \ind_{G_{|f|^p}}(x,t) \dd{(\mu\times m)}(x,t)
      \]

      As $\norm{f}_p^p < +\infty$, it follows that $\ind_{G_{|f|^p}}\in L^1(X\times[0,+\infty], \mu\times m)$ whence by Fubini's theorem $(\ind_{G_{|f|^p}})^t\in L^1 (X,\mu)$ for almost every $t\in[0,+\infty]$, the a.e. defined function $\int (\ind_{G_{|f|^p}})^t \dd{\mu}\in L^1([0,+\infty], m)$, and
      \begin{align*}
        \norm{f}_p^{p} = \int_{X\times [0,+\infty]}\ind_{G_{|f|^p}} \dd{(\mu\times m)} &= \int_{0}^{\infty}\left[\int_X (\ind_{G_{|f|^p}})^t(x)\dd{\mu(x)}\right]\dd{t} \\
        &= \int_{0}^{\infty}\left[\int_X (\ind_{(G_{|f|^p})^t})(x)\dd{\mu(x)}\right]\dd{t} = \int_{0}^{\infty}\mu((G_{|f|^p})^t)\dd{t}\\
        &= \int_{0}^{\infty}\mu(\{ x: |f(x)|^p<t\})\dd{t}
      \end{align*}

      Consider the functions $F:[0,+\infty]\to [0,\infty]$ and $\phi:[0,+\infty]\to[0,+\infty]$ given by $F(t) = \mu(\{ x: |f(x)|^p>t\})$ and $\phi(t) = t^p$. For $t$ nonnegative, observe that $\{ x:|f(x)|^p > t^p\} = \{x:|f(x)| > t\}$, so $(F\circ \phi)(t) = \mu(\{ x: |f(x)|^p>t\}) = \mu(\{ x:|f(x)| > t\})$. Lastly, noting that $F$ is measurable and $\phi$ is a $C^1$-diffeomorphism, it follows that
      \[
        \norm{f}_p^{p} = \int_{0}^{\infty} F(t)\dd{t} = \int_{0}^{\infty}(F\circ\phi)(t)|\det D_t\phi|\dd{t} = p\int_{0}^{\infty} t^{p-1}\mu(\{x:|f(x)|>t\})\dd{t}.
      \]

    \end{proof}

    \textbf{(c)}: Let $(X,\mu)$ be a $\sigma$-finite measure space. Show that if $f,g\in L^1(X,\mu)$ with $0\leq f,g$ a.e., then
    \[
      \norm{f-g}_1 = \int_0^{\infty} \mu(\{x:f(x)>t\}\Delta\{x:g(x)>t\})\dd{t}.
    \]
    Suggestion: it might be helpful to first show that for $a,b\in[0,+\infty)$ we have
    \[
      |a-b| = \int_0^{\infty} |\ind_{(t,\infty)}(a) - \ind_{(t,\infty)}(b)|\dd{t}
    \]

    \begin{proof}
      % TODO Prove lemma descirbed in the suggestion.
      % TODO Justify change of order of integration.
      \begin{align*}
        \norm{f-g}_1 = \int_X |f-g|\dd{\mu} &= \int_X \int_0^{\infty}|\ind_{(t,+\infty)}(f(x)) - \ind_{(t,+\infty)}(g(x))|\dd{t}\dd{\mu(x)} \\
        &=\int_0^{\infty}\left[\int_X |\ind_{f^{-1}((t,+\infty))}(x) - \ind_{g^{-1}((t,+\infty))}(x)|\right]\dd{\mu(x)}\dd{t}\\
        &=\int_0^{\infty} \left[\int_X \ind_{f^{-1}((t,+\infty))\Delta g^{-1}((t,+\infty))}(x)\right]\dd{\mu(x)}\dd{t} \\&= \int_0^{\infty}\mu(\{x:f(x)>t\}\Delta\{x:g(x)>t\})\dd{t}
      \end{align*}
    \end{proof}
\end{homeworkProblem}

\begin{homeworkProblem}
  If $f$ is Lebesgue integrable on $(0,a)$ and $g(x) = \int_{x}^a t^{-1}f(t)\dd{t}$, then $g$ is integrable on $(0,a)$ and $\int_0 ^{a} g(x)\dd{x} = \int_0 ^{a} f(x)\dd{x}$.

  \begin{proof}
    Define a set $E = \{(x,t)\in (0,a)^2:x<t\}$. This set is measurable. Then, for fixed $x\in(0,a)$, $\ind_{(x,a)}(t)=\ind_{E_x}(t) = (\ind_E)_x(t)$. Then we compute,
    \begin{align*}
      \int_{(0,a)}|g(x)|\dd{x} &= \int_{(0,a)}\left|\int_{(0,a)}t^{-1}f(t)\ind_E(x,t)\dd{t}\right|\dd{x} \leq \int_{(0,a)}\int_{(0,a)}| t^{-1}f(t)|\ind_{E^t}(x)\dd{x}\dd{t}\\ &= \int_{(0,a)} t^{-1}|f(t)| m(E^t)\dd{t} = \int_{(0,a)}|f(t)|\dd{t}<+\infty
    \end{align*}
    whence by Tonelli's theorem $g$ is measurable. So, we may apply Fubini's theorem.
    \begin{align*}
      \int_{(0,a)}g(x)\dd{x} &= \int_{(0,a)}\int_{(0,a)}t^{-1}f(t)\ind_E(x,t)\dd{t}\dd{x} = \int_{(0,a)}\int_{(0,a)} t^{-1}f(t)\ind_{E^t}(x)\dd{x}\dd{t}\\ &= \int_{(0,a)} t^{-1}f(t) m(E^t)\dd{t} = \int_{(0,a)}f(t)\dd{t}
    \end{align*}
    as desired.
  \end{proof}
\end{homeworkProblem}


\begin{homeworkProblem}
  Let $\mc{E}_q$ be the set of products of the form $\Pi_{j=1}^d I_j$ where each $I_j$ is an $h$-interval with the property that all of its finite endpoints are rational. \\

  \textbf{(a)}: Show that $\mc{E}_q$ is an elementary family which generates the Borel sets.

  \begin{proof}
    Intersections of rational intervals are rational intervals and intersections of products of sets are componentwise intersections, so $\mc{E}_q$ is closed under intersections.\\

    The complement of an element of $\mc{E}_q$ is a a disjoint union of finitely many boxes. Thus, $\mc{E}_q$ is an elementary family.\\

    Any $h$-box in $\R^d$ can be written as a countable union of elements of $\mc{E}_q$, so $\mc{E}_q$ generates the Borel sets in $\R^d$.
  \end{proof}

  \textbf{(b)}: Suppose that $\mu$ is a Borel measure on $\R^d$ with $0<\mu((0,1]^d)<+\infty$. If $\mu(E+x) = \mu(E)$ for every $x\in\R^d$, show that $\mu(E) = \mu((0,1])^d m(E)$ for every Borel $E\sub\R^d$.

  \begin{proof}
    
  \end{proof}
\end{homeworkProblem}


\begin{homeworkProblem}
  Fix $d\in\N$.\\

  \textbf{(a)}: Let $s:\R^d\times \R^d \to\R^d$ be the map $s(x,y) = x+y$. Let $\mu,\nu$ be finite, Borel measures on $\R^d$. Define $\mu*\nu = s_*(\mu\otimes\nu)$. Show that for every Borel $E\sub\R^d$ we have
  \[
    \mu*\nu(E) = \int\int \ind_E(x+y)\dd{\mu(x)}\dd{\nu(y)}
  \]
  and
  \[
    \int\mu(E-y)\dd{\nu(y)} = \mu*\nu(E) = \int\nu(E-x)\dd{\mu(x)}.
  \]
  Show as a consequence that
  \[
    \mu*\nu(X) = \mu(X)\nu(X).
  \]

  \begin{proof}
    On one hand, by finiteness of the measures and measurability of $E$, we may apply Fubini's theorem to see that
    \begin{align*}
      \int\int\ind_E(s(x,y))\dd{\mu(x)}\dd{\nu(y)} &= \int\int\ind_{s^{-1}(E)}(x,y)\dd{\mu(x)}\dd{\nu(y)} = \int\int\ind_{s^{-1}(E)}\dd{(\mu\otimes\nu)} = \mu*\nu(E).
    \end{align*}
    Moreover, noting that $(s^{-1}(E))^y = E-y$ and $(s^{-1}(E))_x = E-x$, theorem 2.36 gives that
    \[
      \mu*\nu(E) = \mu\otimes\nu(s^{-1}(E)) = \int \nu(E-x)\dd{\mu(x)}
    \]
    and
    \[
      \mu*\nu(E) = \mu\otimes\nu(s^{-1}(E)) = \int \mu(E-y)\dd{\nu(y)}
    \]
    It follows that
    \[
      \mu*\nu(\R^d) = \int \mu(\R^d-y)\dd{\nu(y)} = \int \mu(R^d)\dd{\nu(y)} = \mu(R^d)\nu(R^d).
    \]
  \end{proof}

  \textbf{(b)}: Show that for finite, Borel measures $\mu,\nu,\eta$ on $\R^d$ we have
  \[
    (\mu*\nu)*\eta = \mu*(\nu*\eta).
  \]

  \begin{proof}
    Let $E\in\mc{B}_{\R^d}$. By the finiteness of the measures, we may apply Fubini freely, whence
    \begin{align*}
      ((\mu*\nu)*\eta)(E) &= \int (\mu*\nu)(E-z)\dd{\eta(z)} = \int\left[\int\nu(E-z-x)\dd{\mu(x)}\right]\dd{\eta(z)}\\
      &=\int \left[\int\nu(E-x-z)\dd{\eta(z)}\right]\dd{\mu(x)} = \int (\nu*\eta)(E-x)\dd{\mu(x)} = (\mu*(\nu*\eta))(E).
    \end{align*}
  \end{proof}

  \textbf{(c)}: For $f,g\in L^1(\R^d)$ show that
  \[
    \int_{\R^d}\int_{\R^d} |f(y)g(x-y)|\dd{x}\dd{y} = \norm{f}_1 \norm{g}_1.
  \]
  Explain why this implies that $y\mapsto f(y)g(x-y)$ is in $L^1(\R^d)$ for almost every $x\in\R^d$ and why if we set $f*g(x) = \int_{\R^d}f(y)g(x-y)\dd{y}$ then we have that $f*g\in L^1(\R^d)$ and
  \[
    \norm{f*g}_1 \leq \norm{f}_1 \norm{g}_1.
  \]

  \begin{proof}
    \begin{align*}
      \int_{\R^d}\int_{\R^d} |f(y)g(x-y)|\dd{x}\dd{y} = \int_{\R^d}|f(y)|\left[\int_{\R^d} |g(x-y)|\dd{x}\right]\dd{y} = \int_{\R^d}|f(y)|\left[\int_{\R^d} |g(x)|\dd{x}\right]\dd{y} = \norm{f}_1\norm{g}_1
    \end{align*}
    As $f,g\in L^1(\R^d)$, $\norm{f}_1 \norm{g}_1<+\infty$ the above equality and Fubini's theorem give that the function $x\mapsto \int|f(y)g(x-y)|\dd{y}$ is $L^1(\R^d)$ for almost every $x\in\R^d$. Hence, $\{x: \int|f(y)g(x-y)|\dd{y} = +\infty\}$ is a null set, so $y\mapsto f(y)g(x-y)$ is in $L^1(\R^d)$ for almost every $x\in \R^d$. Thus,
    \[
      \norm{f*g}_1 = \int \left|\int f(y)g(x-y)\dd{y}\right|\dd{x}\leq \int\int |f(y)g(x-y)|\dd{y}\dd{x} = \norm{f}_1 \norm{g}_1.
    \]

  \end{proof}

  \textbf{(d)}: Adopt notation as in Problem 1 of HW5. Show that if $f,g\in L^1(R^d)$ are nonnegative than $(f\dd{m})*(g\dd{m}) = f*g\dd{m}$ with $m$ being the Lebesgue measure.

  \begin{proof}
    Let $E\sub \R^d$ be measurable and let $\dd{\mu} = f\dd{m}$, $\dd{\nu} = g\dd{m}$, and $\dd{\lambda} = f*g\dd{m}$. On one hand, we compute that
    \begin{align*}
      \lambda(E) = \int_E (f*g)(y)\dd{y} = \int_{\R^d}\ind_E(y)\left[\int_{\R^d}f(x)g(y-x)\dd{x}\right]\dd{y}.
    \end{align*}
    On the other hand, by part (a), we have that
    \begin{align*}
      \mu*\nu(E) &= \int_{\R^d}\nu(E-x)\dd{\mu(x)} = \int_{\R^d}f(x)\left[\int_{E-x}g(y)\dd{y}\right]\dd{x} \\
      &= \int_{\R^d}\left[\int_{E}f(x)g(y-x)\dd{y}\right]\dd{x} =\int_{\R^d}f(x)\int_{\R^d}\ind_{E}(y)f(x)g(y-x)\dd{y}\dd{x},
    \end{align*}
    which by nonnegativity and applying Tonelli's theorem, it follows that these two computed quantities are in fact equal.
  \end{proof}

  \textbf{(e)}: Show that for $f,g,k\in L^1(\R^d)$ we have that
  \[(f*g)*k = f*(g*k) \text{ almost everywhere.}\]

  \begin{proof}
    Applying associativity from part (b) and expanding using part (d), we observe that
    \begin{align*}
      (f*g)*k\dd{m} &= (f*g\dd{m})*k\dd{m} = (f\dd{m}*g\dd{m})*k\dd{m} = f\dd{m}*(g\dd{m}*k\dd{m}) \\
      &= f\dd{m}*(g*k)\dd{m} = f*(g*k)\dd{m}.
    \end{align*}
    By uniqueness of Radon-Nikodym derivatives, it follows that $(f*g)*k = f*(g*k)$ almost everywhere.
  \end{proof}
\end{homeworkProblem}



\end{document}

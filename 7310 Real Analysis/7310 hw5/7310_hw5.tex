\documentclass[12pt,letterpaper]{article}

%--------Packages--------
\usepackage{amsmath, amsthm, amssymb}
\usepackage{xspace}
\usepackage{graphicx}
\usepackage{hhline}
\usepackage{amssymb}
\usepackage{array}
\usepackage{braket}
\usepackage{multicol}
\usepackage{mathtools}
\usepackage{enumerate}
\usepackage{delarray}
\usepackage{mathtools}
\usepackage{fullpage}
\usepackage{faktor} % For quotients
\usepackage{mathrsfs}

\usepackage[italicdiff]{physics} % For differentials
\usepackage{bbm} % For indicator

% \usepackage{quiver}
\usepackage[linguistics]{forest}




%--------Page Setup--------

\pagestyle{empty}%

\setlength{\hoffset}{-1.54cm}
\setlength{\voffset}{-1.54cm}

\setlength{\topmargin}{0pt}
\setlength{\headsep}{0pt}
\setlength{\headheight}{0pt}

\setlength{\oddsidemargin}{0pt}

\setlength{\textwidth}{195mm}
\setlength{\textheight}{250mm}


%--------Macros--------

\newcommand{\sub}{\subseteq}
\newcommand{\lcm}{\text{lcm}}
\newcommand{\mc}[1]{\mathcal{#1}}
\newcommand{\mf}[1]{\mathfrak{#1}}
\newcommand{\ms}[1]{\mathscr{#1}}
\newcommand{\sO}{\mathcal{O}}
\newcommand{\cyclic}[1]{\langle#1\rangle}
\newcommand{\units}[1]{#1 ^{\times}}
\newcommand{\la}{\langle}
\newcommand{\ra}{\rangle}
\newcommand{\lr}[1]{\left(#1\right)}
%----Switch phi and varphi
% \let\temp\phi
% \let\phi\varphi
% \let\varphi\temp

\newcommand{\C}{\mathbb{C}}
\newcommand{\F}{\mathbb{F}}
\newcommand{\N}{\mathbb{N}\xspace}
\newcommand{\I}{\mathbb{I}\xspace}
\newcommand{\R}{\mathbb{R}\xspace}
\newcommand{\Z}{\mathbb{Z}\xspace}
\newcommand{\Q}{\mathbb{Q}\xspace}
\newcommand{\G}{\mathbb{G}\xspace}
\DeclareMathOperator{\Spec}{Spec}
\DeclareMathOperator{\res}{res}
% \DeclareMathOperator{\Tr}{Tr}
\DeclareMathOperator{\ord}{ord}
\DeclareMathOperator{\Sym}{Sym}
% \DeclareMathOperator{\dv}{div}
\DeclareMathOperator{\alb}{alb}
\DeclareMathOperator{\img}{Im}
\DeclareMathOperator{\et}{et}
\DeclareMathOperator{\ck}{coker}
\DeclareMathOperator{\Reg}{Reg}
\DeclareMathOperator{\Cor}{Cor}
\DeclareMathOperator{\Ac}{at}
\DeclareMathOperator{\supp}{supp}
\DeclareMathOperator{\Hom}{Hom}
\DeclareMathOperator{\Pic}{Pic}
\DeclareMathOperator{\Gal}{Gal}
\DeclareMathOperator{\fc}{frac}
\DeclareMathOperator{\Ann}{Ann}
\DeclareMathOperator{\Mod}{Mod}
\DeclareMathOperator{\Cone}{Cone}
\DeclareMathOperator{\FI}{FI}
\DeclareMathOperator{\End}{End}
\DeclareMathOperator{\Alb}{Alb}
\DeclareMathOperator{\Ext}{Ext}
\DeclareMathOperator{\ab}{ab}
\DeclareMathOperator{\Jac}{Jac}
\DeclareMathOperator{\coker}{coker}
\DeclareMathOperator{\fr}{frac}
\DeclareMathOperator{\Int}{Int}



%----Analysis
\newcommand{\summ}{\sum\limits}
% \newcommand{\norm}[1]{\left \vert \left \vert #1 \right \vert \right \vert}
\newcommand{\thicc}{\bigg}
\newcommand{\eps}{\varepsilon}
\newcommand*\cls[1]{\overline{#1}}
\newcommand{\ind}{\mathbbm{1}}


%--------Theorem environments--------
\newtheorem{definition}{Definition}[]
\newtheorem{lemma}{Lemma}[]
\newtheorem{corollary}{Corollary}[]
\newtheorem{theorem}{Theorem}[]
\theoremstyle{remark}
\newtheorem*{claim}{Claim}


\newenvironment{solution}
{\begin{proof}[Solution]}
{\end{proof}}


\makeatletter
\newcommand{\thickhline}{%
    \noalign {\ifnum 0=`}\fi \hrule height 1pt
    \futurelet \reserved@a \@xhline
}
\newcolumntype{"}{@{\hskip\tabcolsep\vrule width 1pt\hskip\tabcolsep}}
\makeatother

% --------Problem environment--------
\setlength\parindent{0pt}
\setcounter{secnumdepth}{0}
\newcounter{partCounter}
\newcounter{homeworkProblemCounter}
\setcounter{homeworkProblemCounter}{1}


\newenvironment{homeworkProblem}[1][-1]{
    \ifnum#1>0
        \setcounter{homeworkProblemCounter}{#1}
    \fi
    \section{Problem \arabic{homeworkProblemCounter}}
    \setcounter{partCounter}{1}
    \stepcounter{homeworkProblemCounter}
}


%--------Metadata--------
\title{MATH 7310 Homework 5}
\author{James Harbour}

\begin{document}

\begin{homeworkProblem}
  \textbf{(a)} Let $(X,\mu)$ be a measure space. For $f:X\to[0,+\infty]$ measurable, we define a measure $\nu$ by $\nu(E) = \int_E f\dd{\mu}$ where $E\sub X$ is measurable. If $g:X\to\C$ is measurable, show that $g\in L^{1}(X,\nu)$ if and only if $gf\in L^1(X,\mu)$ and that $\int g\dd{\nu} = \int fg\dd{\mu}$ for all $g\in L^1(X,\nu)$.

  \begin{proof}\ \\
    \underline{$\implies$}: Suppose $g\in L^1(X,\nu)$, so $\int |g| \dd{\nu}<+\infty$. Thus $|g|\in L^+(X,\nu)$, so by problem 5 on homework 4, $\int |g|\dd{\nu} = \int |g|f \dd{\mu} = \int |gf| \dd{\mu}$, so $gf\in L^1(X,\mu)$. \\

    \underline{$\implies$}: Suppose $gf\in L^1(X,\mu)$. So $\int |g|\dd{\nu} = \int |g|f\dd{\mu}  = \int |gf|\dd{\mu}<+\infty$, whence $g\in L^{1}(X,\nu)$. \\

    Now let $g\in L^1(X,\nu)$ and write $g = u + iv$ where $u = \Re(g)$ and $v = \Im(g)$. Let $u^+, u^-, v^+, v^-$ be the positive and negative parts of $u$ and $v$ respectively. As $g\in L^1(X,\nu)$, each of these functions are in $L^+(X,\nu)$. Then, using nonnegativity of these functions and problem 5 of homework 4,
    \begin{align*}
      \int g \dd{\nu} = \int u \dd{\nu} + i\int v \dd{\nu} &= \int u^+\dd{\nu} - \int u^- \dd{\nu} + i\int v \dd{\nu} - i\int v\dd{\nu}\\
      &= \int u^+ f\dd{\mu} - \int u^- f\dd{\mu} + i\int v^+ f\dd{\mu} - i\int v^- f\dd{\mu} = \int g f\dd{\mu}.
    \end{align*}
  \end{proof}


  \textbf{(b)} Let $(X,\Sigma),(Y,\mc{F})$ be measurable spaces and let $\mu:\Sigma\to[0,+\infty]$ be a measure. Let $\phi:X\to Y$ be measurable. If $f:Y\to\C$ is measurable, show that $f\in L^1(Y,\phi_*(\mu))$ if and only if $f\circ \phi\in L^{1}(X,\mu)$ and that $\int f\dd{(\phi_*(\mu))} = \int f\circ \phi\dd{\mu}$ for all $f\in L^1(Y,\phi_*(\mu))$.

  \begin{proof}\ \\
    \underline{$\implies$}: Suppose that $f\in L^1(Y,\phi_*(\mu))$. Then \[\int |f|\dd{(\phi_*(\mu))} < +\infty\]

  \end{proof}
\end{homeworkProblem}


\begin{homeworkProblem}
  Let $f(x) = x^{-1/2}$ if $0<x<1$, $f(x) = 0$ otherwise. Let $(r_n)_{n=1}^{\infty}$ be an enumeration of the rationals, and set $g(x) = \sum_{n=1}^{\infty}2^{-n}f(x-r_n)$. \\

  \textbf{(a)} Show that $g\in L^{1}(m)$, and in particular that $g<\infty$ a.e. \\

  \begin{proof}

  \end{proof}

  \textbf{(b)} Prove that $g$ is discontinuous at every point and unbounded on every interval, and it remains so after any modification on a Lebesgue null set.\\

  \textbf{(c)} Prove that $g^2<\infty$ almost everywhere, but $g^2$ is not integrable on any interval.
\end{homeworkProblem}

\begin{homeworkProblem}
  Compute the following limits and justify the calculations:\\

  \textbf{(a)} $\lim_{n\to\infty}\int_{0}^{\infty}(1+(x/n))^{-n}\sin(x/n)\dd{x}$.

  \begin{proof}
    Let $f_n(x) = (1+(x/n))^{-n}\sin(x/n)$ for $x\in [0,+\infty)$. Then for all $x\in[0,+\infty)$, $f(x):=\lim_{n\to\infty}f_n(x) = 0/e^{x} = 0$, so $\int f(x) \dd{x} = 0$. On the other hand, we estimate via cherrypicking terms in the binomial expansion that for $n\in\N\setminus\{ 1\}$,
    \[
      |f_n| = \frac{|\sin(\frac{x}{n})|}{(1+\frac{x}{n})^n} \leq \frac{1}{(1+\frac{x}{n})^n} \leq \frac{1}{1+\binom{n}{2}x^2}\leq \frac{1}{1+x^2}
    \]
    which is in $L^1$. Hence, by the dominated convergence theorem, $\lim_{n\to\infty}\int_{0}^{\infty}f_n(x)\dd{x} = \int_{0}^{\infty}f(x)\dd{x} = 0$.
  \end{proof}

  \textbf{(b)} $\lim_{n\to\infty}\int_{0}^{1}(1+nx^2)(1+x^2)^{-n}\dd{x}$.

  \begin{proof}
    Let $f_n(x) = (1+nx^2)(1+x^2)^{-n}$ on $[0,1]$. Let $f(x) = \lim_{n\to\infty}(1+nx^2)(1+x^2)^{-n} = $ By Bernoulli's inequality, for $n\in \N$

    \[
      |f_n| \leq (1+x^2)^{n}(1+x^2)^{-n} = 1
    \]
    which is in $L^{1}([0,1],m)$. Thus, by the dominated convergence theorem
    \[
      \lim_{n\to\infty}\int_{0}^{1}f_n(x)\dd{x} = \int_{0}^{1}\lim_{n\to\infty}(1+nx^2)(1+x^2)^{-n} = \int_{0}^{1}\lim_{n\to\infty}\frac{x^2}{(1+x^2)^n \ln{(1+x^2)}2x}\dd{x} = 0.
    \]
  \end{proof}

  \textbf{(c)} $\lim_{n\to\infty}\int_{0}^{\infty}n\sin(x/n)[x(1+x^2)]^{-1}\dd{x}$\\

  \begin{proof}
    Let $f_n(x) = n\sin(x/n)[x(1+x^2)]^{-1}$ and $f(x) = \lim_{n\to\infty}f_n(x) = \lim_{n\to\infty}\frac{\cos(x/n)}{1+x^2} = \frac{1}{1+x^2}$. For $n\in \N$, note that
    \[
      |f_n|\leq \frac{1}{1+x^2}
    \]
    which is in $L^+$, so by the dominated convergence theorem $\lim_{n\to\infty}\int_{0}^{\infty}n\sin(x/n)[x(1+x^2)]^{-1}\dd{x} = \int_{0}^{\infty}\frac{1}{1+x^2}\dd{x} = \pi/2$.
  \end{proof}

  \textbf{(d)} $\lim_{n\to\infty}\int_{a}^{\infty}n(1+n^2 x^2)^{-1}\dd{x}$.

  \begin{proof}
    We compute
    \begin{align*}
      \int_{a}^{\infty}n(1+n^2 x^2)^{-1}\dd{x} = \int_{na}^{\infty} \frac{1}{1+x^2}\dd{x} = \frac{\pi}{2} - \arctan(na).
    \end{align*}
    If $a>0$, $\lim_{n\to\infty}\frac{\pi}{2} - \arctan(na) = 0$.\\
    If $a=0$, $\lim_{n\to\infty}\frac{\pi}{2} = \frac{\pi}{2}$.\\
    If $a<0$, $\lim_{n\to\infty}\frac{\pi}{2} - \arctan(na) = \frac{\pi}{2} - (-\frac{\pi}{2}) = \pi$.

  \end{proof}

\end{homeworkProblem}

\begin{homeworkProblem}
  \textbf{(a)} Suppose $\mu(X)<\infty$. If $f$ and $g$ are complex-valued measurable functions on $X$, define
  \[
    \rho(f,g) = \int \frac{|f-g|}{1+|f-g|}\dd{\mu}.
  \]
  Then $\rho$ is a metric on the space of measurable functions if we identify functions that are equal a.e., and $f_n\to f$ with respect to this metric if and only if $f_n\to f$ in measure.\\

  \textbf{(b)} Suppose $(X,\mu)$ is a finite measure space. Let $\rho$ be the metric in (a). Show that a sequence of measurable functions $f_n:X\to \C$ is Cauchy in measure if and only if it is Cauchy with respect to $\rho$.
\end{homeworkProblem}


\begin{homeworkProblem}
    Suppose that $|f_n|\leq g\in L^1 $ and $f_n\to f$ in measure.\\

    \textbf{(a)} Prove that $\int f\dd{\mu} = \lim_{n\to\infty}\int f_n\dd{\mu}$.\\

    \textbf{(b)} Prove that $f_n\to f$ in $L^1$.
\end{homeworkProblem}

\begin{homeworkProblem}
  If $f:[a,b]\to \C$ is Lebesgue measurable and $\eps>0$, there is a compact set $E\sub [a,b]$ such that $\mu(E^c)<\eps$ and $f\vert_E$ is continuous. (\emph{Hint}: Use Egoroff's theorem and Theorem 2.26)
\end{homeworkProblem}

\end{document}

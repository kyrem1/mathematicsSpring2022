\documentclass[12pt,letterpaper]{article}

%--------Packages--------
\usepackage{amsmath, amsthm, amssymb}
\usepackage{xspace}
\usepackage{graphicx}
\usepackage{hhline}
\usepackage{amssymb}
\usepackage{array}
\usepackage{braket}
\usepackage{multicol}
\usepackage{mathtools}
\usepackage{enumerate}
\usepackage{delarray}
\usepackage{mathtools}
\usepackage{fullpage}
\usepackage{faktor} % For quotients
\usepackage{mathrsfs}

\usepackage[italicdiff]{physics} % For differentials
\usepackage{bbm} % For indicator

% \usepackage{quiver}
\usepackage[linguistics]{forest}




%--------Page Setup--------

\pagestyle{empty}%

\setlength{\hoffset}{-1.54cm}
\setlength{\voffset}{-1.54cm}

\setlength{\topmargin}{0pt}
\setlength{\headsep}{0pt}
\setlength{\headheight}{0pt}

\setlength{\oddsidemargin}{0pt}

\setlength{\textwidth}{195mm}
\setlength{\textheight}{250mm}


%--------Macros--------

\newcommand{\sub}{\subseteq}
\newcommand{\lcm}{\text{lcm}}
\newcommand{\mc}[1]{\mathcal{#1}}
\newcommand{\mf}[1]{\mathfrak{#1}}
\newcommand{\ms}[1]{\mathscr{#1}}
\newcommand{\sO}{\mathcal{O}}
\newcommand{\cyclic}[1]{\langle#1\rangle}
\newcommand{\units}[1]{#1 ^{\times}}
\newcommand{\la}{\langle}
\newcommand{\ra}{\rangle}
\newcommand{\lr}[1]{\left(#1\right)}
\newcommand{\lrvert}[1]{\left\lvert#1\right\rvert}

\DeclarePairedDelimiterX{\inp}[2]{\langle}{\rangle}{#1, #2}

%----Switch phi and varphi
% \let\temp\phi
% \let\phi\varphi
% \let\varphi\temp

\newcommand{\C}{\mathbb{C}}
\newcommand{\F}{\mathbb{F}}
\newcommand{\E}{\mathbb{E}}
\newcommand{\N}{\mathbb{N}\xspace}
\newcommand{\I}{\mathbb{I}\xspace}
\newcommand{\R}{\mathbb{R}\xspace}
\newcommand{\Z}{\mathbb{Z}\xspace}
\newcommand{\Q}{\mathbb{Q}\xspace}
\newcommand{\G}{\mathbb{G}\xspace}

\DeclareMathOperator{\Spec}{Spec}
\DeclareMathOperator{\res}{res}
% \DeclareMathOperator{\Tr}{Tr}
\DeclareMathOperator{\ord}{ord}
\DeclareMathOperator{\Sym}{Sym}
% \DeclareMathOperator{\dv}{div}
\DeclareMathOperator{\alb}{alb}
\DeclareMathOperator{\img}{Im}
\DeclareMathOperator{\et}{et}
\DeclareMathOperator{\ck}{coker}
\DeclareMathOperator{\Reg}{Reg}
\DeclareMathOperator{\Cor}{Cor}
\DeclareMathOperator{\Ac}{at}
\DeclareMathOperator{\supp}{supp}
\DeclareMathOperator{\Hom}{Hom}
\DeclareMathOperator{\Pic}{Pic}
\DeclareMathOperator{\Gal}{Gal}
\DeclareMathOperator{\fc}{frac}
\DeclareMathOperator{\Ann}{Ann}
\DeclareMathOperator{\Mod}{Mod}
\DeclareMathOperator{\Cone}{Cone}
\DeclareMathOperator{\FI}{FI}
\DeclareMathOperator{\End}{End}
\DeclareMathOperator{\Alb}{Alb}
\DeclareMathOperator{\Ext}{Ext}
\DeclareMathOperator{\ab}{ab}
\DeclareMathOperator{\Jac}{Jac}
\DeclareMathOperator{\coker}{coker}
\DeclareMathOperator{\fr}{frac}
\DeclareMathOperator{\Int}{Int}
\let\Span\relax
\DeclareMathOperator{\Span}{Span}
\DeclareMathOperator{\Ran}{Ran}



%----Analysis
\newcommand{\summ}{\sum\limits}
% \newcommand{\norm}[1]{\left\lVert#1\right\rVert}
\newcommand{\thicc}{\bigg}
\newcommand{\eps}{\varepsilon}
\newcommand*\cls[1]{\overline{#1}}
\newcommand{\ind}{\mathbbm{1}}
\DeclareMathOperator{\sgn}{sgn}


%--------Theorem environments--------
\newtheorem{definition}{Definition}[]
\newtheorem{lemma}{Lemma}[]
\newtheorem{corollary}{Corollary}[]
\newtheorem{theorem}{Theorem}[]
\theoremstyle{remark}
\newtheorem*{claim}{Claim}


\newenvironment{solution}
{\begin{proof}[Solution]}
{\end{proof}}


\makeatletter
\newcommand{\thickhline}{%
    \noalign {\ifnum 0=`}\fi \hrule height 1pt
    \futurelet \reserved@a \@xhline
}
\newcolumntype{"}{@{\hskip\tabcolsep\vrule width 1pt\hskip\tabcolsep}}
\makeatother

% --------Problem environment--------
\setlength\parindent{0pt}
\setcounter{secnumdepth}{0}
\newcounter{partCounter}
\newcounter{homeworkProblemCounter}
\setcounter{homeworkProblemCounter}{1}


\newenvironment{homeworkProblem}[1][-1]{
    \ifnum#1>0
        \setcounter{homeworkProblemCounter}{#1}
    \fi
    \section{Problem \arabic{homeworkProblemCounter}}
    \setcounter{partCounter}{1}
    \stepcounter{homeworkProblemCounter}
}


%--------Metadata--------
\title{MATH 7310 Homework 10}
\author{James Harbour}

\begin{document}
\maketitle

\begin{homeworkProblem}
  If $F\in NBV$, let $G(x)=|\mu_F|((-\infty,x])$, Prove that $|\mu_F|=\mu_{T_F}$ by showing that $G=T_F$ via the following steps.\\

  \textbf{(a)}: From the definition of $T_F$, show that $T_F\leq G$.

  \begin{proof}
    Let $x\in \R$. Then for $x_0<x_1<\cdots<x_n=x$, observe that
    \[
      \sum_{j=1}^{\infty} |F(x_j)-F(x_{j-1})| = \sum_{j=1}^{\infty} |\mu_F((x_{j-1},x_j])|\leq \sum_{j=1}^{\infty} |\mu_F|((x_{j-1},x_j]) = |\mu_F|((x_0,x]),
    \]
    whence $T_F(x)\leq \sup_{x_0<x}|\mu_F|((x_0,x]) = |\mu_F|((-\infty,x]) = G(x)$.
  \end{proof}

  \textbf{(b)}: $|\mu_F(E)|\leq \mu_{T_F}(E)$ when $E$ is an interval, and hence when $E$ is a Borel set.

  \begin{proof}
    Let $I=(a,b)$ be an interval. Then
    \begin{align*}
      |\mu_F(I)| = |F(b)-F(a)| \leq T_F(b)-T_F(a) = \mu_{T_F}(I).
    \end{align*}

    Now suppose $U\sub\R$ is open. Then there exist countably many disjoint intervals $(a_j,b_j)\sub \R$ such that $U = \bigsqcup_{j=1}^{\infty}(a_j,b_j)$. Hence,
    \[
      |\mu_F(U)| \leq \sum_{j=1}^{\infty} |\mu_F((a_j,b_j))| \leq \sum_{j=1}^{\infty} \mu_{T_F}((a_j,b_j)) = \mu_{T_F}(U).
    \]

    The case where $E$ is Borel follows by outer regularity.
  \end{proof}

  \textbf{(c)}: $|\mu_F|\leq \mu_{T_F}$, and hence $G\leq T_F$ (Use exercise 21).

  \begin{proof}
    Observe that, by part (b),
    \begin{align*}
      |\mu_F|(E) &= \sup\left\{\sum_1^\infty |\mu_F(E_j)|:E_1,E_2,\ldots \text{ disjoint},\,E=\bigsqcup_1^\infty E_j\right\}\\
      &\leq \sup\left\{\sum_1^\infty \mu_{T_F}(E_j):E_1,E_2,\ldots \text{ disjoint},\,E=\bigsqcup_1^\infty E_j\right\} = \mu_{T_F}(E),
    \end{align*}
    so $T_F(x) = \mu_{T_F}((-\infty,x]) \geq |\mu_F|((-\infty,x]) = G(x)$.

  \end{proof}
\end{homeworkProblem}


\begin{homeworkProblem}
  Let $G$ be a continuous increasing function on $[a,b]$ and let $G(a)=c$, $G(b)=d$.\\

  \textbf{(a)}: If $E\sub [c,d]$ is a Borel set, then $m(E) = \mu_G(G^{-1}(E))$. (First consider the case where $E$ is an interval.)

  \begin{proof}
    Suppose first that $(e,f]\sub[c,d]$. Then by elementary analysis there exist some $p<q\in[c,d]$ such that $G^{-1}((e,f]) = (p,q]$, $G(p) = e$, $G(q) =f$. So,
    \[
      \mu_{G}(G^{-1}((e,f])) = G(q)-G(p) = e-f = m((e,f]).
    \]
    Now suppose that $E\sub [c,d]$ is Borel. As $G$ is real-valued and increasing, we may apply outer regularity of $\mu_G$ combined with the fact that $G^{-1}(E)\sub \bigcup I_j$ if and only if $E\sub \bigcup G(I_j)$ to conclude that $m(E) = \mu_G(G^{-1}(E))$.
  \end{proof}

  \textbf{(b)}: If $f$ is a Borel measurable and integrable function on $[c,d]$, then $\int_c^d f(y)\dd{y} = \int_a^b f(G(x))\dd{G(x)}$. In particular, $\int_c^d f(y)\dd{y} = \int_a^b f(G(x))G'(x)\dd{x}$ if $G$ is absolutely continuous.

  \begin{proof}
    By linearity of the statement, it suffices to prove for real-valued nonnegative functions $f$, as then we would apply to positive and negative parts of real and imaginary parts. Thus, assume $f\geq 0$. Choose simple functions $0\leq \phi_1\leq \phi_2\leq \cdots\leq f$ such that $\phi_n\to f$ pointwise. Write $\phi_n = \sum a_j \ind_{E_j}$ for some $a_j\geq0$ and $E_j\sub[c,d]$ Borel. Then we compute by part (a) that
    \begin{align*}
      \int_{[c,d]}\phi_n\dd{m} = \sum a_j m(E_j) = \sum a_j \mu_G(G^{-1}(E_j)) = \int_{[a,b]}\phi_n\circ G\dd{\mu_G},
    \end{align*}
    whence by the monotone convergence theorem $\int_{[c,d]} f\dd{m} = \int_{[a,b]}f\circ G\dd{\mu_G}$.
  \end{proof}

  \textbf{(c)}: The validity of (b) may fail if $G$ is merely right continuous rather than continuous.

  \begin{proof}
    Consider the function $G = \ind_{[1/2,1]}$ and $f(x) = x$. Then $\int_{0}^1 f(x)\dd{x} = 1$ but \[\int_{0}^1 (f\circ G)(x)\dd{G(x)} = \int_{0}^1 \ind_{[1/2,1]}(x)\dd{G(x)} = \mu_G([1/2,1]) = 0. \]


  \end{proof}
\end{homeworkProblem}


\begin{homeworkProblem}
  Suppose $F:\R\to \C$. Prove that $F$ is Lipschitz with constant $M$ if and only if $F$ is absolutely continuous and $|F'|\leq M$ a.e.

  \begin{proof}\ \\
    \underline{$\implies$}: Suppose that $M>0$ is such that $|F(x)-F(y)|\leq M|x-y|$ for all $x,y\in\R$. Let $\eps >0$. Choose $\delta = \eps/M$. Then, for any finite set of disjoint intervals $(a_1,b_1),\ldots,(a_N,b_N)$ with $\sum_{j=1}^{N}(b_j-a_j)<\delta$, we have
    \[
      \sum_{j=1}^{N} |F(b_j)-F(a_j)|\leq M\sum_{j=1}^{N} (b_j-a_j)< M\cdot\frac{\eps}{M} = \eps,
    \]
    so $F$ is absolutely continuous. Thus, $F$ is differentiable almost everywhere. If $x\neq y$, then $|F(x)-F(y)|/|x-y|\leq M$, so for a.e. $x\in \R$ we have that $|F'|\leq M$.\\

    \underline{$\impliedby$}: Suppose that $F$ is absolutely continuous and $|F'|\leq M$ a.e. Let $x,y\in\R$ and without loss of generality suppose that $x<y$. Then by the FTC for Lebesgue integrals,
    \[
      |F(y)-F(x)| = \lrvert{\int_{x}^{y} F'\dd{t}} \leq \int_{x}^{y}|F'| \dd{t} \leq M|y-x|.
    \]
  \end{proof}
\end{homeworkProblem}



\begin{homeworkProblem}
  Let $A\sub [0,1]$ be a Borel set such that $0<m(A\cap I)<m(I)$ for every subinterval $I$ of $[0,1]$.\\

  \textbf{(a)}: Let $F(x) = m([0,x]\cap A)$. Show that $F$ is absolutely continuous and strictly increasing on $[0,1]$, but $F'=0$ on a set of positive measure.

  \begin{proof}
    If $y>x$, then $F(y) = m([0,y]\cap A) = m([0,x]\cap A) + m((x,y]\cap A) > F(x)$ by assumption, so $F$ is strictly increasing. Now fix $\eps>0$ and set $\delta = \eps$. Then, for any finite set of disjoint intervals $(a_1,b_1),\ldots,(a_N,b_N)$ with $\sum_{j=1}^{N}(b_j-a_j)<\delta$,
    \[
      \sum_{j=1}^{N} |F(b_j)-F(a_j)|\leq \sum_{j=1}^{N} m((a_j,b_j]\cap A)< \sum_{j=1}^{N}(b_j-a_j) <\eps,
    \]
    so $F$ is absolutely continuous. Let $\mu_F$ be the unique Borel measure such that $F(x) = \mu_F([0,x])$. Then the absolute continuity of $F$ implies that $\mu_F\ll m$ and $\dd{\mu_F} = F'\dd{m}$. On the other hand, for $a<b$ in $[0,1]$,
    \[
      \mu_F((a,b]) = F(b)-F(a) = m([0,b]\cap A) - m([0,a]\cap A) = m([a,b]\cap A) = \int_a^b \ind_A \dd{m},
    \]
    whence $\dd{\mu_F} = \ind_A\dd{m}$. By the uniqueness of Radon-Nikodym derivatives, $F' = \ind_A$ almost everywhere, whence $F' = 0$ on $[0,1]\setminus A$. Moreover $m([0,1]\setminus A) = m([0,1])-m(A) = m([0,1])-m([0,1]\cap A) > 0$ by assumption.
  \end{proof}

  \textbf{(b)}: Let $G(x) = m([0,x]\cap A) - m([0,x]\setminus A)$. Show that $G$ is absolutely continuous on $[0,1]$, but $G$ is not monotone on any subinterval of $[0,1]$.

  \begin{proof}
    Let $\eps>0$. Again set $\delta = \eps$. Then, for any finite set of disjoint intervals $(a_1,b_1),\ldots,(a_N,b_N)$ with $\sum_{j=1}^{N}(b_j-a_j)<\delta$,
    \[
      \sum_{j=1}^{N} |G(b_j)-G(a_j)| = \sum_{j=1}^{N} m((a_j,b_j]\cap A) - m((a_j,b_j]\setminus A)< \sum_{j=1}^{N}m((a_j,b_j]) <\eps,
    \]
    so $G$ is absolutely continuous. Let $\mu$ be the measure given by $\dd{\mu} = G'\dd{m}$. Then for $a<b$,
    \[
      \int_{a}^{b}G' \dd{m}= m((a,b]\cap A) -m((a,b]\setminus A) = \int_{a}^{b}\ind_A-\ind_{[0,1]\setminus A}\dd{m}.
    \]
    Then as intervals generate all borel subsets of $[0,1]$, $G'\dd{m} = \dd{\mu} = (\ind_A-\ind_{[0,1]\setminus A})\dd{m}$, so by uniqueness of Radon-Nikodym derivatives $G' = \ind_A-\ind_{[0,1]\setminus A}$ almost everywhere. Let $I\sub [0,1]$ be an interval. Then $G' = 1$ on $I\cap A$ and $G' = -1$ on $I\setminus A$, but $m(I\cap A),m(I\setminus A)>0$, so we must have that $G$ is not monotonic on $I$.
  \end{proof}
\end{homeworkProblem}


\begin{homeworkProblem}

\end{homeworkProblem}


\begin{homeworkProblem}
  Let $a<b$ be real numbers and let $1\leq p \leq +\infty$. Let $X$ be the set of functions $f:[a,b]\to \C$ which are absolutely continuous and such that $f'\in L^p([a,b])$. Fix $x_0\in[a,b]$. For $f\in X$, define
  \[
    \norm{f} = |f(x_0)| + \norm{f'}_p.
  \]
  Show that $\norm{\cdot}$ is a norm which turns $X$ into a Banach space.

  \begin{proof}
    % TODO prove is a norm
    That $\norm{\cdot}$ is a norm follows from the fact that $|\cdot|$ and $\norm{\cdot}_p$ are norms.\\

    Suppose that $(f_n)_{n=1}^{\infty}$ is a Cauchy sequence in $(X,\norm{\cdot})$. Then
    \[
      |f_n(x_0)-f_m(x_0)| + \norm{f_n'-f_m'}_p = \norm{f_n-f_m} \xrightarrow{n,m\to\infty}0
    \]
    whence $|f_n(x_0)-f_m(x_0)|,\norm{f_n'-f_m'}_p \xrightarrow{n,m\to\infty}0$. By completeness of $\C$ and $L^p([a,b])$, there exists some $y_0\in\C$ and $F\in L^p([a,b])$ such that $f_n(x_0)\to y_0$ and $f_n'\xrightarrow{L^p} F$. As $\norm{F}_1\leq m([a,b])^{1-\frac{1}{p}}\norm{F}_p < +\infty$, $F\in L^1([a,b])$. Define $f:[a,b]\to\C$ by
    \[
      f(x) = y_0 + \int_{a}^x F(t)\dd{t}.
    \]
    Then $f(x)-f(x_0) = \int_{a}^x F(t)\dd{t}$, whence by Folland 3.35(a) $f$ is absolutely continuous. Moreover, by Folland 3.35(c), $f$ is differentiable almost everywhere, $f'\in L^1([a,b])$, and $f(x)-f(x_0) = \int_{a}^x f'(t)\dd{t}$. So
    \[
      \int_a^x f'(t)\dd{t} = \int_a^x F(t)\dd{t}
    \]
    for all $x\in[a,b]$, whence by uniqueness in premesures give measures $f'(t)\dd{t} = F(t)\dd{t}$. However then by uniqueness of Radon-Nikodym derivatives, it follows that $f' = F$ almost everywhere, so $f'\in L^p([a,b])$. Thus $f\in X$.\\

    As $f' = F$, we have that
    \[
      \norm{f-f_n} = |f(x_0)-f_n(x_0)| + \norm{f'-f_n'}_p = |y_0-f_n(x_0)| + \norm{F-f_n'}_p \xrightarrow{n\to\infty}0.
    \]


  \end{proof}
\end{homeworkProblem}
\end{document}

\documentclass[12pt,letterpaper]{article}

%--------Packages--------
\usepackage{amsmath, amsthm, amssymb}
\usepackage{xspace}
\usepackage{graphicx}
\usepackage{hhline}
\usepackage{amssymb}
\usepackage{array}
\usepackage{braket}
\usepackage{multicol}
\usepackage{mathtools}
\usepackage{enumerate}
\usepackage{delarray}
\usepackage{mathtools}
\usepackage{fullpage}
\usepackage{faktor} % For quotients
\usepackage{mathrsfs}

\usepackage[italicdiff]{physics} % For differentials
\usepackage{bbm} % For indicator

% \usepackage{quiver}
\usepackage[linguistics]{forest}




%--------Page Setup--------

\pagestyle{empty}%

\setlength{\hoffset}{-1.54cm}
\setlength{\voffset}{-1.54cm}

\setlength{\topmargin}{0pt}
\setlength{\headsep}{0pt}
\setlength{\headheight}{0pt}

\setlength{\oddsidemargin}{0pt}

\setlength{\textwidth}{195mm}
\setlength{\textheight}{250mm}


%--------Macros--------

\newcommand{\sub}{\subseteq}
\newcommand{\lcm}{\text{lcm}}
\newcommand{\mc}[1]{\mathcal{#1}}
\newcommand{\mf}[1]{\mathfrak{#1}}
\newcommand{\ms}[1]{\mathscr{#1}}
\newcommand{\sO}{\mathcal{O}}
\newcommand{\cyclic}[1]{\langle#1\rangle}
\newcommand{\units}[1]{#1 ^{\times}}
\newcommand{\la}{\langle}
\newcommand{\ra}{\rangle}
\newcommand{\lr}[1]{\left(#1\right)}
%----Switch phi and varphi
% \let\temp\phi
% \let\phi\varphi
% \let\varphi\temp

\newcommand{\C}{\mathbb{C}}
\newcommand{\F}{\mathbb{F}}
\newcommand{\N}{\mathbb{N}\xspace}
\newcommand{\I}{\mathbb{I}\xspace}
\newcommand{\R}{\mathbb{R}\xspace}
\newcommand{\Z}{\mathbb{Z}\xspace}
\newcommand{\Q}{\mathbb{Q}\xspace}
\newcommand{\G}{\mathbb{G}\xspace}
\DeclareMathOperator{\Spec}{Spec}
\DeclareMathOperator{\res}{res}
% \DeclareMathOperator{\Tr}{Tr}
\DeclareMathOperator{\ord}{ord}
\DeclareMathOperator{\Sym}{Sym}
% \DeclareMathOperator{\dv}{div}
\DeclareMathOperator{\alb}{alb}
\DeclareMathOperator{\img}{Im}
\DeclareMathOperator{\et}{et}
\DeclareMathOperator{\ck}{coker}
\DeclareMathOperator{\Reg}{Reg}
\DeclareMathOperator{\Cor}{Cor}
\DeclareMathOperator{\Ac}{at}
\DeclareMathOperator{\supp}{supp}
\DeclareMathOperator{\Hom}{Hom}
\DeclareMathOperator{\Pic}{Pic}
\DeclareMathOperator{\Gal}{Gal}
\DeclareMathOperator{\fc}{frac}
\DeclareMathOperator{\Ann}{Ann}
\DeclareMathOperator{\Mod}{Mod}
\DeclareMathOperator{\Cone}{Cone}
\DeclareMathOperator{\FI}{FI}
\DeclareMathOperator{\End}{End}
\DeclareMathOperator{\Alb}{Alb}
\DeclareMathOperator{\Ext}{Ext}
\DeclareMathOperator{\ab}{ab}
\DeclareMathOperator{\Jac}{Jac}
\DeclareMathOperator{\coker}{coker}
\DeclareMathOperator{\fr}{frac}
\DeclareMathOperator{\Int}{Int}



%----Analysis
\newcommand{\summ}{\sum\limits}
% \newcommand{\norm}[1]{\left\lVert#1\right\rVert}
\newcommand{\thicc}{\bigg}
\newcommand{\eps}{\varepsilon}
\newcommand*\cls[1]{\overline{#1}}
\newcommand{\ind}{\mathbbm{1}}
\DeclareMathOperator{\Span}{Span}

%--------Theorem environments--------
\newtheorem{definition}{Definition}[]
\newtheorem{lemma}{Lemma}[]
\newtheorem{corollary}{Corollary}[]
\newtheorem{theorem}{Theorem}[]
\theoremstyle{remark}
\newtheorem*{claim}{Claim}


\newenvironment{solution}
{\begin{proof}[Solution]}
{\end{proof}}


\makeatletter
\newcommand{\thickhline}{%
    \noalign {\ifnum 0=`}\fi \hrule height 1pt
    \futurelet \reserved@a \@xhline
}
\newcolumntype{"}{@{\hskip\tabcolsep\vrule width 1pt\hskip\tabcolsep}}
\makeatother

% --------Problem environment--------
\setlength\parindent{0pt}
\setcounter{secnumdepth}{0}
\newcounter{partCounter}
\newcounter{homeworkProblemCounter}
\setcounter{homeworkProblemCounter}{1}


\newenvironment{homeworkProblem}[1][-1]{
    \ifnum#1>0
        \setcounter{homeworkProblemCounter}{#1}
    \fi
    \section{Problem \arabic{homeworkProblemCounter}}
    \setcounter{partCounter}{1}
    \stepcounter{homeworkProblemCounter}
}


%--------Metadata--------
\title{MATH 7310 Homework 7}
\author{James Harbour}

\begin{document}

\maketitle

\begin{homeworkProblem}
  Let $(X,\Sigma,\mu)$ be a measure space. \\

  \textbf{(i)}: Prove that if $\mu(E_n)<+\infty$ for $n\in\N$ and $\ind_{E_n}\to f$ in $L^1$, then $f$ is (a.e. equal to) the characteristic function of a measurable set.

  \begin{proof}
    For $m\in\N$, let
    \[
      F_m = \{x: \min\{|1-f(x)|,|f(x)|\}>\frac{1}{m}\}.
    \]
    Then $(F_m)_{m=1}^{\infty}$ is an increasing sequence of sets with $F = \{ x:f(x)\not\in\{ 0,1\}\} = \bigcup_{n=1}^{\infty}A_n$.\\

    Observe that, for fixed $m\in \N$,
    \[
      \norm{\ind_{E_n}-f}_1 \geq \int_{F_m}|\ind_{E_n}-f|\dd{\mu}\geq \int_{F_m}\frac{1}{m}\dd{\mu} = \frac{1}{m}\mu(F_m)
    \]
    for all $n\in \N$, whence sending $n\to\infty$ it follows that $\mu(F_m) = 0$. Thus, it follows that $\mu(F) = 0$. Thus, $f = \ind_{f^{-1}(\{1\})}$ almost everywhere.
  \end{proof}

  \textbf{(ii)}: Let $\Sigma_f = \{ E\in\Sigma: \mu(E)<+\infty\}$. Define an equivalence relation on $\Sigma_f$ by $E\sim F$ if $\mu(E\Delta F)=0$. Let $\Omega = \Sigma_f/\sim$, and define a metric $\rho$ on $\Omega$ be $\rho([E],[F]) = \mu(E\Delta F)$. Show that the map $\iota:\Omega\to L^1(X,\mu)$ given by $\iota([E]) = \ind_E$ is an isometry with closed image.

  \begin{proof}
    Observe that, if $E,F\in \Sigma_f$, then
    \[
      \rho(\iota(E),\iota(F)) = \mu(E\Delta F) = \int \ind_{E\Delta F}\dd{\mu} = \int |\ind_E - \ind_F|\dd{\mu},
    \]
    so $\iota$ is an isometry. Now suppose that $(f_n)_{n=1}^{\infty}$ is in $\iota(\Omega)$ and $f\in L^1(X,\mu)$ with $\norm{f_n-f}_1\xrightarrow{n\to\infty}0$. Then for $n\in\N$, there are $E_n\in\Sigma_f$ such that $f_n = \ind_{E_n}$, whence by part (i) there is some measurable $E\sub X$ such that $f = \ind_{E}$. As $\ind_E = f\in L^1(\mu)$, it follows that $\mu(E)<+\infty$ whence $[E]\in\Omega$ and thus $f = \iota([E])\in\iota(\Omega)$.
  \end{proof}

  \textbf{(iii)}: Show that $(\Omega,\rho)$ is a complete metric space.

  \begin{proof}
    Let $([E_n])_{n=1}^{\infty}$ be a Cauchy sequence in $(\Omega,\rho)$. Then as $\iota$ is an isometry, it follows that $(\ind_{E_n})_{n=1}^{\infty}$ is a Cauchy sequence in $L^1(X,,\mu)$. By completeness of $L^1(X,\mu)$, there exists some $f\in L^1(X,\mu)$ such that $\norm{\ind_{E_n}-f}_1\xrightarrow{n\to\infty}0$. As the image of $\iota$ is closed, it follows that there is some $E\sub X$ with $\mu(E)<+\infty$ such that $f = \ind_{E} = \iota([E])$ almost everywhere. Then, $\iota$ being an isometry implies that $\rho([E_n],[E])\xrightarrow{n\to\infty}0$.
  \end{proof}
\end{homeworkProblem}

\begin{homeworkProblem}[2]
  If $X,Y$ are sets, and $f:X\to\C$, $g:Y\to\C$, we define $f\otimes g:X\times Y\to\C$ by $(f\otimes g)(x,y) = f(x)g(y)$. Fix $1\leq p<+\infty$.\\

  \textbf{(a)}: Let $(X,\Sigma,\mu),(Y,\mc{F},\nu)$ be $\sigma$-finite measure spaces. Show that if $f\in L^p(X,\mu),g\in L^p(Y,\nu)$, then $\norm{f\otimes g}_p = \norm{f}_p \norm{g}_p$.

  \begin{proof}
    By Tonelli's theorem,
    \begin{align*}
      \norm{f\otimes g}_p^p &= \int_{X\times Y} |f\otimes g|^p\dd{\mu\otimes\nu} = \int_{Y}\int_{X} |f(x)|^p |g(y)|^p \dd{\mu(x)}\dd{\nu(y)} \\&= \int_{Y}|g(y)|^p\int_{X}|f(x)|^p \dd{\mu(x)}\dd{\nu(y)} = \norm{f}_p^p \int_{Y}|g(y)|^p \dd{\nu(y)} = \norm{f}_p^p \norm{g}_p^p.
    \end{align*}
  \end{proof}

  \textbf{(b)}: Let $(Z,\mc{O},\zeta)$ be a finite measure space. Suppose that $\mc{A}\sub\mc{O}$ is an algebra which generates the $\sigma$-algebra of $\mc{O}$. Use the monotone class lemma to show that $\{\ind_A:A\in\mc{A}\}$ is dense in $\{\ind_E:E\in\mc{O}\}$ in the $L^p$-norm for all $1\leq p <+\infty$.

  \begin{proof}
    By the monotone class lemma, $\mc{O} = \Sigma(\mc{A}) = M(\mc{A})$. Let $E\in\mc{O}$. Let
    \[
      \ms{C} = \left\{E\in\mc{O}: \ind_E \in \cls{\{\ind_A : A\in\mc{A}\}}^{\norm{\cdot}_p}\right\}.
    \]
    We will show that $\ms{C}$ is a monotone class as then $\mc{A}\sub \ms{C}$ would imply that $\mc{O} = M(\mc{A})\sub\ms{C}$ and we would be done.\\

    First, we compute that for any $A,B\in\ms{O}$, by problem 1(ii)
    \[
      \norm{\ind_A-\ind_B}_p^p = \int|\ind_A-\ind_B|^p \dd{\zeta} = \int \ind_{A\Delta B}\dd{\zeta} = \norm{\ind_A-\ind_B}_1 = \rho([A],[B]) = \zeta(A\Delta B).
    \]

    Suppose that $E_1\sub E_2\sub \cdots$ is an increasing sequence of sets in $\ms{C}$ and let $E = \bigcup_{j\in\N} E_j$. Then, the finiteness of $\zeta$ and continuity from below,
    \[
        \norm{\ind_E - \ind_{E_n}}_p^p = \zeta(E\Delta E_n) = \zeta(E\setminus E_n) = \zeta(E)-\zeta(E_n)\xrightarrow{n\to\infty}0,
    \]
    whence $\ind_E\in\ms{C}$.\\

    On the other hand, suppose that $F_1\supseteq F_2\supseteq \cdots$ is a decreasing sequence of sets in $\ms{C}$ and let $F = \bigcup_{j\in\N} F_j$. Then by finiteness of $\zeta$ and continuity from below,
    \[
      \norm{\ind_F - \ind_{F_n}}_p^p = \zeta(F_n\Delta F) = \zeta(F_n\setminus F) = \zeta(F_n)-\zeta(F)\xrightarrow{n\to\infty}0,
    \]
    whence $\ind_F\in\ms{C}$.
  \end{proof}

  \textbf{(c)}: Let $(X,\Sigma,\mu),(Y,\mc{F},\nu)$ be finite measure spaces. Use the previous part to show that $\{\ind_E : E\in \Sigma\otimes\mc{F}\}\sub\cls{\Span}^{\norm{\cdot}_p}\{\ind_E\otimes\ind_F:E\in\Sigma,F\in\mc{F}\}$. Use this to show that $\cls{\Span}^{\norm{\cdot}_p}\{\ind_E\otimes\ind_F:E\in\Sigma,F\in\mc{F}\} = L^p(X\times Y,\mu\otimes\nu)$.

  \begin{proof}
    Let $\mc{A}\sub \Sigma\otimes\mc{F}$ be the algebra of finite disjoint unions of rectangles in $X\times Y$. Note that any $A\in \mc{A}$ has the form $A = \bigsqcup_{j=1}^{n}E_j\times F_j$ for some $E_j\in\Sigma$ and $F_j\in \mc{F}$, whence $\ind_A = \sum_{j=1}^{n}\ind_{E_j}\otimes\ind_{F_j}$, so
    \[
      \{\ind_A : A\in \mc{A}\} \sub \Span\{\ind_E\otimes\ind_F:E\in\Sigma,F\in\mc{F}\}.
    \]

    By definition of the product sigma algebra, $\Sigma(\mc{A}) = \Sigma\otimes\mc{F}$, so part (b) implies that
    \[
      \{\ind_E : E\in \Sigma\otimes\mc{F}\} = \cls{\{\ind_A : A\in \mc{A}\}}^{\norm{\cdot}_p} \sub \cls{\Span}^{\norm{\cdot}_p}\{\ind_E\otimes\ind_F:E\in\Sigma,F\in\mc{F}\}.
    \]
    As the left hand side is a linear subspace,
    \[
      \Span\{\ind_E : E\in \Sigma\otimes\mc{F}\}\sub\cls{\Span}^{\norm{\cdot}_p}\{\ind_E\otimes\ind_F:E\in\Sigma,F\in\mc{F}\}
    \]
    whence by density of simple functions in $L^p$ and the finiteness of $\mu\otimes\nu$,
    \[
      L^p(X\times Y,\mu\otimes\nu)\sub \cls{\Span}^{\norm{\cdot}_p} \{\ind_E : E\in \Sigma\otimes\mc{F}\} \sub \cls{\Span}^{\norm{\cdot}_p}\{\ind_E\otimes\ind_F:E\in\Sigma,F\in\mc{F}\} \sub L^p(X\times Y,\mu\otimes\nu).
    \]

  \end{proof}

  \textbf{(d)}: Let $(X,\Sigma,\mu),(Y,\mc{F},\nu)$ be $\sigma$-finite measure spaces. Suppose that $D_X\sub L^p(X,\mu)$, $D_Y\sub L^p(Y,\nu)$ and that
  \[
    \cls{\Span}^{\norm{\cdot}_p}(D_X) = L^p(X,\mu),\hspace{10pt}\cls{\Span}^{\norm{\cdot}_p}(D_Y) = L^p(Y,\nu).
  \]
  Show that $\cls{\Span}^{\norm{\cdot}_p}(\{f\otimes g: f\in D_X,g\in D_Y\}) = L^p(X\times Y,\mu\otimes\nu)$.

  \begin{proof}
    First suppose that $X,Y$ are finite measure spaces. By part (c), it suffices to show that
    \[
      \{\ind_E\otimes\ind_F : E\in\Sigma, F\in\mc{F}\} \sub\cls{\Span}^{\norm{\cdot}_p}\{f\otimes g : f\in D_X, g\in D_Y\} = L^p(X\times Y,\mu\otimes\nu).
    \]
    To this end, take $E\in\Sigma$ and $F\in \mc{F}$. By finiteness, $\ind_E\in L^p(X,\mu)$ and $\ind_F\in L^p(Y,\nu)$. Then by assumption, there are sequences $(f_n)_{n=1}^{\infty}$ in $\Span(D_X)$ and $(g_n)_{n=1}^{\infty}$ in $\Span(D_Y)$ such that
    \[
      \norm{f_n-\ind_E}_p,\norm{g_n-\ind_F}_p \xrightarrow{n\to\infty}0.
    \]
    Then, we compute using Tonelli's theorem that
    \begin{align*}
      \norm{f_n\otimes g_n-\ind_E\otimes\ind_F}_p^p &=\int|f_n\otimes g_n-\ind_E\otimes\ind_F|^p\dd{(\mu\otimes\nu)} \\&=\int|f_n\otimes g_n-\ind_E\otimes g_n+\ind_E\otimes g_n-\ind_E\otimes\ind_F|^p\dd{(\mu\otimes\nu)} \\
      &\leq \int|f_n\otimes g_n-\ind_E\otimes g_n|^p+\ind_E\otimes g_n-\ind_E\otimes\ind_F|^p\dd{(\mu\otimes\nu)} \\ &= \norm{f_n-\ind_E}_p^p\norm{g_n}_p^p + \mu(E)\norm{g_n-\ind_F}_p^p \xrightarrow{n\to\infty}0,
    \end{align*}
    as desired. Thus, the claim holds for finite measure spaces. Now suppose $X,Y$ are $\sigma$-finite and let $X_1\sub X_2\sub \cdots$ and $Y_1\sub Y_2\sub\cdots$ be measurable of finite measure such that $X = \bigcup X_i$ and $Y= \bigcup Y_j$. By part (c), for all $n\in\N$
    \[
      \cls{\Span}^{\norm{\cdot}_p}\{\ind_{X_n\times Y_n}(f\otimes g) : f\in D_X, g\in D_Y\} = L^p(X_n\times Y_n,\mu\otimes\nu)
    \]
    whence
    \[
      L^p(X\times Y) = \bigcup_{n=1}^\infty \cls{\Span}^{\norm{\cdot}_p}\{\ind_{X_n\times Y_n}(f\otimes g) : f\in D_X, g\in D_Y\}.
    \]
    As such, it suffices to show that
    \[
      \{\ind_{X_n\times Y_n}(f\otimes g): f\in D_X, g\in D_Y\}\sub \cls{\Span}^{\norm{\cdot}_p}\{f\otimes g : f\in D_X, g\in D_Y\}
    \]
    for all $n\in \N$. Hence, let $n\in\N$, $f\in D_X$, and $g\in D_Y$. Then there are sequences $(f_k)_{k=1}^{\infty}$ in $\Span(D_X)$ and $(g_k)_{k=1}^{\infty}$ in $\Span(D_Y)$ such that
    \[
      \norm{f_k-\ind_{X_n}f}_p,\norm{g_k-\ind_{Y_n}g}_p\xrightarrow{n\to\infty}0.
    \]
    Then we compute
    \begin{align*}
      \norm{f_k\otimes g_k-\ind_{X_n}f\otimes\ind_{Y_n}g}_p^p &=\int|f_k\otimes g_k-\ind_{X_n}f\otimes\ind_{Y_n}g|^p\dd{(\mu\otimes\nu)} \\&=\int|f_k\otimes g_k-\ind_{X_n}f\otimes g_k+\ind_{X_n}f\otimes g_k-\ind_{X_n}f\otimes\ind_{Y_n}g|^p\dd{(\mu\otimes\nu)} \\
      &\leq \int|f_k\otimes g_k-\ind_{X_n}f\otimes g_k|^p+\ind_{X_n}f\otimes g_k-\ind_{X_n}f\otimes\ind_{Y_n}g|^p\dd{(\mu\otimes\nu)} \\ &= \norm{f_k-\ind_{X_n}f}_p^p\norm{g_k}_p^p + \norm{\ind_{X_n}f}_p^p\norm{g_k-\ind_{Y_n}g}_p^p \xrightarrow{k\to\infty}0,
    \end{align*}
    so $\ind_{X_n\times Y_n}(f\otimes g)\in\cls{\Span}^{\norm{\cdot}_p}\{f\otimes g : f\in D_X, g\in D_Y\}$.
  \end{proof}
\end{homeworkProblem}


\begin{homeworkProblem}
  Suppose that $f\in L^p\, \cap\, L^\infty$ for some $p<+\infty$ so that $f\in L^q$ for all $q>p$. Prove that then $\norm{f}_\infty = \lim_{q\to\infty}\norm{f}_q$.

  \begin{proof}
    By Folland Proposition 6.10, we have that
    \[
      \norm{f}_q^{\frac{p}{q}}\leq \norm{f}_p^{\frac{p}{q}}\norm{f}_{\infty}^{\frac{p}{q}}
    \]
    whence $\limsup_{q\to\infty}\norm{f}_q \leq \norm{f}_{\infty}$. On the other hand, for $n\in\N$ let $E_n = \{x : |f(x)|>\norm{f}_\infty-\frac{1}{n}\}$. Then $(E_n)_{n=1}^{\infty}$ is a decreasing sequence of measurable sets with $E = \bigcap_{n=1}^{\infty}E_n = \{x: |f(x)|\geq \norm{f}_\infty\}$ having $\mu(E) = 0$ by definition of the $L^\infty$-norm. Observe that, for $n\in\N$ and $q>p$,
    \[
      \norm{f}_q \geq \lr{\int_{E_n}|f|^q\dd{\mu}}^{\frac{1}{q}} > (\norm{f}_\infty-\frac{1}{n})\mu(E_n)^{\frac{1}{q}}
    \]
    whence
    \[
      \liminf_{q\to\infty} \norm{f}_q \geq \norm{f}_\infty-\frac{1}{n}.
    \]
    As this holds for all $n\in \N$, it follows that $\liminf_{q\to\infty} \norm{f}_q \geq \norm{f}_\infty$ as desired.
  \end{proof}
\end{homeworkProblem}


\begin{homeworkProblem}
  If $f$ is a measurable function on $X$, define the \emph{essential range} $R_f$ of $f$ to be the set of all $z\in\C$ such that $\{x: |f(x)-z|<\eps\}$ has positive measure for all $\eps > 0$.\\

  \textbf{(a)}: Prove that $R_f$ is closed.

  \begin{proof}
    Let $z\in \cls{R_f}$. Then there exists a sequence $(z_n)_{n=1}^{\infty}$ in $R_f$ such that $z_n\to z$. Fix $\eps>0$. There is some $N\in\N$ such that $n\geq N\implies B_{\eps/2}(z_n)\sub B_\eps(z)$. Then $f^{-1}(B_{\eps/2}(z_n))\sub f^{-1}(B_\eps(z))$, whence $0<\mu(f^{-1}(B_{\eps/2}(z_n)))\leq \mu(f^{-1}(B_\eps(z)))$. Hence $z\in R_f$, so $R_f$ is closed.
  \end{proof}

  \textbf{(b)}: Prove that if $f\in L^\infty$, then $R_f$ is compact and $\norm{f}_\infty = \max\{|z|: z\in R_f\}$.

  \begin{proof}
    Fix $z\in R$ and let $M>0$ be such that $\mu(f^{-1}(X\setminus \cls{B_M(0))}) = 0$. Suppose, for the sake of contradiction, that $|z| > M$. Then we may choose $\eps>0$ such that $B_{\eps}(z)\sub X\setminus B_M(0)$. Then $f^{-1}(B_{\eps}(z))\sub f^{-1}(X\setminus B_M(0))$, whence $\mu(f^{-1}(B_{\eps}(z))) = 0$ contradicting that $z\in R_f$. Thus $|z|\leq M$. As $M>0$ was arbitrary for its condition, it follows that $|z|\leq \norm{f}_{\infty}$. As $z\in R_f$ was arbitrary, it follows that $\sup_{z\in R_f}|z|\leq \norm{f}_\infty < +\infty$. Hence $R_f$ is compact by part (a) and Heine-Borel. Let $z_{max}\in R_f$ such that $|z_{\max}| = \max_{z\in R_f}|z| = \sup_{z\in R_f}|z|$. \\

    We show that in fact $\mu(f^{-1}(\C\setminus B_{|z_{\max}|}(0))) = 0$, whence it would follow that $\norm{f}_{\infty}\leq |z_{\max}|$ as desired. Let $N\in\N$ such that for all $n\geq N$ we have $|z_{\max}|<|z_{\max}|+\frac{1}{n}<\norm{f}_{\infty}$. Then for $n\geq N$, set $U_n = (\C\setminus \cls{B_{|z_{\max}|+\frac{1}{n}}(0)})\setminus (\C\setminus B_{\norm{f}_{\infty}}(0))$. We may cover $U_n$ by countably many balls $B_{r_j}(z_j)$ where $B_{r_j}(z_j)\sub U$, i.e. $U = \bigcup_{j=1}^{\infty}B_{r_j}(z_j)$. It follows then that $\mu(f^{-1}(B_{r_j}(z_j))) = 0$, whence $\mu(U_n) = 0$. As the $U_n$'s are a decreasing sequence of measure zero sets with intersection $(\C\setminus B_{|z_{\max}|(0)})\setminus(\C\setminus B_{\norm{f}_{\infty}}(0)))$, it follows that $\mu(f^{-1}((\C\setminus B_{|z_{\max}|(0)})\setminus(\C\setminus B_{\norm{f}_{\infty}}(0)))))=0$. Thus $\mu(f^{-1}(\C\setminus B_{|z_{\max}|}(0))) = 0$ as desired.
  \end{proof}


\end{homeworkProblem}


\begin{homeworkProblem}
  Suppose that $1\leq p < +\infty$ and $(f_n)_{n=1}^{\infty}$ in $L^p$. Prove that $(f_n)_{n=1}^{\infty}$ is Cauchy in the $L^p$-norm if and only if the following three conditions hold:
  \begin{enumerate}
    \item $(f_n)$ is Cauchy in measure;
    \item the sequence $(|f_n|^p)_{n=1}^{\infty}$ is uniformly integrable
    \item for every $\eps>0$ there exists $E\sub X$ such that $\mu(E) < +\infty$ and $\int_{E^c}|f_n|^p\dd{\mu}<\eps$ for all $n\in\N$.
  \end{enumerate}

  \begin{lemma}
    Any finite subset $\{f_k\}_{k=1}^{n}\sub L^1(\mu)$ is uniformly integrable.
  \end{lemma}
  \begin{proof}[Proof of Lemma 1]
    We show first that $f\in L^1(\mu)$ is uniformly integrable. Note that, if $f\in L^1(\mu)$, then $|f|\ind_{\{|f|> m\}}\searrow 0$ pointwise a.e. as $\{|f| = +\infty\} = \bigcap_{M\in\N}\{|f|> M\}$ implies that $\lim_{M\to\infty}\mu({|f|>M})=\mu(\{|f| = +\infty\}) = 0$. Moreover, for all $M\in\N$, $|f\ind_{|f|>M}|\leq |f|\in L^1(\mu)$, so by the dominated convergence theorem
    \begin{equation}
      \lim_{M\to\infty}\int_{\{|f|>M\}} |f|\dd{\mu} = 0.
    \end{equation}
    For any $E\sub X$ measurable and $M\in\N$, we have that
    \begin{equation}
      \int_E |f|\dd{\mu} = \int_{E\,\cap\{ |f|\leq M\}} |f|\dd{\mu} + \int_{E\,\cap\{ |f|> M\}} |f|\dd{\mu} \leq M\cdot\mu(E) + \int_{\{ |f|> M\}} |f|\dd{\mu}.
    \end{equation}
    Fix $\eps > 0$. By (1), there exists some $N\in \N$ such that $\int_{\{|f|>N\}} |f|\dd{\mu}<\frac{\eps}{2}$. Choose $\delta = \frac{\eps}{2N}$. Then, for any $E\sub X$ measurable such that $\mu(E)<\delta$, we have by (2) that
    \[
      \left\lvert\int_E f\dd{\mu}\right\rvert\leq \int_E |f|\dd{\mu} < N\cdot\delta + \frac{\eps}{2} = \eps.
    \]

    Now suppose that $\{f_k\}_{k=1}^{n}\sub L^1(\mu)$ is a finite subset of $L^1(\mu)$. Fix $\eps > 0$. By uniform integrability of each of the singletons, for each $k\in \{1,\ldots,n\}$ there exists a $\delta_k>0$ such that $\mu(E)<\delta_k\implies |\int_{E} f_k|<\eps$. Choosing $\delta = \min\{\delta_1,\ldots,\delta_n\}>0$, the claim follows.
  \end{proof}

  \begin{lemma}
    Suppose $(f_n)_{n=1}^{\infty}$ is a sequence in $L^1(\mu)$ and $f\in L^1(\mu)$ such that $\norm{f_n-f}_1 \xrightarrow{n\to\infty} 0$. Then $\{f_n\}_{n=1}^{\infty}$ is uniformly integrable.
  \end{lemma}
  \begin{proof}[Proof of Lemma 2]
    Observe that, for any measurable $E\sub X$ and $n\in \N$,
    \[
      \int_E |f_n|\dd{\mu} \leq \int_E |f|\dd{\mu} + \int_E |f_n - f|\dd{\mu} \leq \int_E |f|\dd{\mu} + \norm{f_n-f}_1.
    \]
    Fix $\eps > 0$ and choose $N\in\N$ such that for $n\geq N$ we have $\norm{f_n-f}_1 < \frac{\eps}{2}$. By Lemma 1, $\{ f\}$ is uniformly integrable, so there is some $\delta'>0$ such that $\mu(E)<\delta'$ implies that $\int_E |f|\dd{\mu} < \frac{\eps}{2}$. \\

    Again by Lemma 1, $\{ f_k\}_{k=1}^{N-1}$ is uniformly integrable, so there is some $\delta''>0$ such that $\mu(E)<\delta''$ implies $\int_E |f_k|\dd{\mu} < \eps$ for all $k\in\{1,\ldots,N-1\}$. Setting $\delta = \min\{\delta',\delta''\}$, the claim follows.
  \end{proof}

  \begin{lemma}[Lemma 3]
    Condition (3) holds for any finite subset $\{f_k\}_{k=1}^{n}\sub L^1(\mu)$.
  \end{lemma}

  \begin{proof}[Proof of Lemma 3]
    We show first that the claim holds for just one function $f\in L^1(\mu)$. Suppose first that $f$ is nonnegative and let $\eps>0$. Then by definition of the integral, there exists a simple function $0\leq g\leq f$ such that
    \[
      \int f\dd{\mu} - \int g \dd{\mu} < \eps.
    \]
    By monotonicity of the integral, $\int g\leq \int f <+\infty$, whence it follows that the set $E = \{x : g(x)>0\}$ has finite measure (as $g$ takes finitely many values). Then
    \[
      \int_{E^c} f\dd{\mu} = \int_{E^c}f-g \dd{\mu} \leq \int f-g \dd{\mu} < \eps.
    \]

    Now suppose that $f$ is real-valued and let $f^\pm$ be the positive and negative parts of $f$. Fix $\eps>0$. Then by the previous case there exist measurable $E^\pm\sub X$ with $\mu(E^\pm)<+\infty$ and $\int_{(E^\pm)^c}f^\pm \dd{\mu} <\frac{\eps}{2}$. Letting $E = E^+ \cup E^-$, it follows that $\mu(E)<+\infty$ and
    \[
      \int_{E^c} |f|\dd{\mu} = \int_{(E^+)^c\cap (E^-)^c}f^+ + f^-\dd{\mu}\leq \int_{(E^+)^c}f^+ \dd{\mu} + \int_{(E^-)^c}f^- \dd{\mu} < \eps.
    \]
    Finally, suppose that $f$ is complex-valued. Let $u = \Re(f)$ and $v = \Im(f)$. The claim then follows from applying the previous case to $u,v$ and using the inequality $|f|\leq |u|+|v|$.\\

    Now suppose that we have a finite subset $\{f_k\}_{k=1}^{n}\sub L^1(\mu)$ and fix $\eps>0$. Then for $1\leq k\leq n$ there exists $E_k\sub X$ with $\mu(E_k)<+\infty$ and $\int_{E_k^c}|f_k|\dd{\mu}<\eps$. Let $E = E_1\cup\cdots\cup E_n$. Then $\mu(E)<+\infty$ and for $1\leq k\leq n$ we have
    \[
      \int_{E^c} |f_k|\dd{\mu} = \int_{\bigcap_{j=1}^{n}E_j^c}|f_k|\dd{\mu} \leq \int_{E_k^c}|f_k|\dd{\mu} < \eps.
    \]
  \end{proof}

  \begin{lemma}[Lemma 4]
    Suppose $(f_n)_{n=1}^{\infty}$ is a sequence in $L^1(\mu)$ and $f\in L^1(\mu)$ such that $\norm{f_n-f}_1 \xrightarrow{n\to\infty} 0$. Then $\{f_n\}_{n=1}^{\infty}$ satisfies condition (3).
  \end{lemma}

  \begin{proof}[Proof of Lemma 4]
    As in the proof of Lemma 2, we utilize that for any measurable $E\sub X$ and $n\in\N$,
    \[
      \int_{E^c}|f_n|\dd{\mu}\leq \int_{E^c}|f|\dd{\mu} + \norm{f_n-f}_1.
    \]
    Fix $\eps>0$ and choose $N\in\N$ such that for $n\geq N$ we have $\norm{f_n-f}_1 < \frac{\eps}{2}$. By Lemma 3, $\{f_k\}_{k=1}^{N-1}$ satisfies condition (3), so there is some $E_1\sub X$ with $\mu(E_1)<+\infty$ such that $\int_{E_1^c}|f_k|\dd{\mu}<\eps$ for all $k\in\{1,\ldots,N-1\}$. The singleton $\{f\}$ also satisfies condition 3, so there is some $E_2\sub X$ with $\mu(E_2)<+\infty$ and $\int_{E_2^c}|f|\dd{\mu}< \frac{\eps}{2}$. Setting $E = E_1 \cup E_2$, it follows that $\mu(E) < +\infty$, $\int_{E^c}|f_k|\dd{\mu}\leq\int_{E_1^c}|f_k|\dd{\mu}<\eps$ for $k\in\{1,\ldots,N-1\}$, and for all $n\geq N$
    \[
      \int_{E^c}|f_n|\dd{\mu}\leq \int_{E^c}|f|\dd{\mu} + \norm{f_n-f}_1 < \int_{E_2^c}|f|\dd{\mu} +\frac{\eps}{2} < \eps.
    \]
  \end{proof}

  \begin{proof}[Proof of Theorem]\ \\
    \underline{$\implies$}: Suppose that $(f_n)_{n=1}^{\infty}$ is Cauchy in the $L^p$-norm. Then by completeness, there is some $f\in L^p(\mu)$ such $\norm{f-f_n}_p \xrightarrow{n\to\infty} 0$. For $\eps>0$, noting that $\{ |f_n-f|\geq\eps\} = \{|f_n-f|^p / \eps^p\geq 1\}$, we have that
    \[
      \mu(\{|f_n-f|\geq\eps\}) = \int_{\{|f_n-f|\geq\eps\}}\frac{|f_n-f|^p}{\eps^p}\dd{\mu} \leq\frac{1}{\eps^p}\norm{f_n-f}_p^p \xrightarrow{n\to\infty} 0.
    \]

    Thus $f_n\to f$ in measure, whence $(f_n)_{n=1}^{\infty}$ is Cauchy in measure. \\

    By an inequality obtained from Gennady, $\norm{|f_n|^p}_1\xrightarrow{n\to\infty}\norm{|f|^p}_1\in L^1(\mu)$. Now by Lemma 2, $(|f_n|^p)_{n=1}^{\infty}$ is uniformly integrable. Also by Lemma 4, condition (3) holds for $(|f_n|^p)_{n=1}^{\infty}$.\\

    \underline{$\impliedby$}: Suppose that $(f_n)_{n=1}^{\infty}$ in $L^p(\mu)$ satisfies the three listed conditions. Fix $\eps>0$ and let $E\sub X$ be as in condition (3). Set $A_{mn} = \{x: |f_m(x)-f_n(x)|\geq\eps\}$ and let $\delta >0$ be as in condition (2). \\

    By construction, observe that
    \[
      \int_{E\setminus A_{mn}} |f_m-f_n|^p\dd{\mu} \leq \int_{E\setminus A_{mn}} \eps^p \dd{\mu} \leq \mu(E)\eps^p
    \]
    As $(f_n)_{n=1}^{\infty}$ is Cauchy in measure, there exists some $N\in\N$ such that for $m,n\geq N$, we have $\mu(A_{mn})<\delta$. It follows by condition (2) that for $m,n\geq N$,
    \[
      \int_{A_{mn}} |f_m-f_n|^p\dd{\mu} \leq \int_{A_{mn}}2^{p-1}(|f_m|^p+|f_n|^p)\dd{\mu} <2^p\eps.
    \]
    Lastly, by condition (3), for all $m,n\in\N$,
    \[
      \int_{E^c} |f_m-f_n|^p\dd{\mu} \leq \int_{E^c}2^{p-1}(|f_m|^p+|f_n|^p)\dd{\mu} <2^p\eps.
    \]
    So, for $m,n\geq N$, we have that
    \[
      \norm{f_m-f_n}_p^p\leq \mu(E)\eps^p + 2^p\eps + 2^p\eps,
    \]
    so $(f_n)_{n=1}^{\infty}$ is Cauchy in the $L^p$-norm.



  \end{proof}
\end{homeworkProblem}


\begin{homeworkProblem}
    Prove that if $E$ is a subset of a Hilbert space $\mc{H}$, then $(E^\perp)^\perp$ is the smallest closed subspace of $\mc{H}$ containing $E$.

    \begin{claim}
      If $M$ is a closed linear subspace of $\mc{H}$, then $(M^\perp)^\perp = M$.
    \end{claim}

    \begin{proof}[Proof of Claim]
      Note that we have $\mc{H} = M\oplus M^\perp$. Let $y\in (M^\perp)^\perp$. Then there exist unique $x\in M$, $x^\perp\in    M^\perp$ such that $y = x + x^\perp$. Noting that $M\sub (M^\perp)^\perp$, we have that $x^\perp = y-x \in M^\perp\cap (M^\perp)^\perp = \{0\}$, whence $x^\perp = 0$ and $y=x\in M$. Thus $M = (M^\perp)^\perp$.
    \end{proof}

    \begin{proof}
      On one hand, note that $E\sub \cls{\Span(E)}\implies (E^\perp)^\perp\sub (\cls{\Span(E)}^\perp)^\perp \overset{\text{claim}}{=}\cls{\Span(E)}$. On the other hand, by the continuity and linearity of the inner product, $(E^\perp)^\perp$ is a closed linear subspace of $\mc{H}$. Thus, as $E\sub(E^\perp)^\perp$, it follows that $\cls{\Span(E)}\sub (E^\perp)^\perp$.
    \end{proof}

    % TODO Show E^\perp is a closed linear subspace of $\mc{H}$.
\end{homeworkProblem}


\end{document}

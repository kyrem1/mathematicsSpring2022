\documentclass[12pt,letterpaper]{article}

%--------Packages--------
\usepackage{tikz,amsmath, amsthm, amssymb}
\usetikzlibrary{calc}%
\usepackage{xspace}
\usepackage{graphicx}
\usepackage{hhline}
\usepackage{amssymb}
\usepackage{array}
\usepackage{multicol}
\usepackage{tikz}
\usepackage{mathtools}
\usepackage{enumerate}
\usepackage{delarray}
\usepackage{mathtools}
\usepackage{tikz-cd}
\usepackage{fullpage}
\usepackage{faktor} % For quotients


% \usepackage{quiver}
\usetikzlibrary{cd}
\usepackage[linguistics]{forest}

\pagestyle{empty}%

\setlength{\hoffset}{-1.54cm}
\setlength{\voffset}{-1.54cm}

\setlength{\topmargin}{0pt}
\setlength{\headsep}{0pt}
\setlength{\headheight}{0pt}

\setlength{\oddsidemargin}{0pt}

\setlength{\textwidth}{195mm}
\setlength{\textheight}{250mm}


%--------Macros--------

\newcommand{\sub}{\subseteq}

\newcommand{\lcm}{\text{lcm}}
\newcommand{\mc}[1]{\mathcal{#1}}
\newcommand{\mf}[1]{\mathfrak{#1}}
\newcommand{\sO}{\mathcal{O}}
\newcommand{\cyclic}[1]{\langle#1\rangle}
\newcommand{\units}[1]{#1 ^{\times}}
\newcommand{\la}{\langle}
\newcommand{\ra}{\rangle}
%----Switch phi and varphi
\let\temp\phi
\let\phi\varphi
\let\varphi\temp

\newcommand{\C}{\mathbb{C}}
\newcommand{\F}{\mathbb{F}}
\newcommand{\N}{\mathbb{N}\xspace}
\newcommand{\I}{\mathbb{I}\xspace}
\newcommand{\R}{\mathbb{R}\xspace}
\newcommand{\Z}{\mathbb{Z}\xspace}
\newcommand{\Q}{\mathbb{Q}\xspace}
\newcommand{\G}{\mathbb{G}\xspace}
\DeclareMathOperator{\Spec}{Spec}
\DeclareMathOperator{\res}{res}
\DeclareMathOperator{\Tr}{Tr}
\DeclareMathOperator{\ord}{ord}
\DeclareMathOperator{\Sym}{Sym}
\DeclareMathOperator{\dv}{div}
\DeclareMathOperator{\alb}{alb}
\DeclareMathOperator{\img}{Im}
\DeclareMathOperator{\et}{et}
\DeclareMathOperator{\ck}{coker}
\DeclareMathOperator{\Reg}{Reg}
\DeclareMathOperator{\Cor}{Cor}
\DeclareMathOperator{\Ac}{at}
\DeclareMathOperator{\supp}{supp}
\DeclareMathOperator{\Hom}{Hom}
\DeclareMathOperator{\Pic}{Pic}
\DeclareMathOperator{\Gal}{Gal}
\DeclareMathOperator{\fc}{frac}
\DeclareMathOperator{\Ann}{Ann}
\DeclareMathOperator{\Mod}{Mod}
\DeclareMathOperator{\Cone}{Cone}
\DeclareMathOperator{\FI}{FI}
\DeclareMathOperator{\End}{End}
\DeclareMathOperator{\Alb}{Alb}
\DeclareMathOperator{\Ext}{Ext}
\DeclareMathOperator{\ab}{ab}
\DeclareMathOperator{\Jac}{Jac}
\DeclareMathOperator{\coker}{coker}
\DeclareMathOperator{\fr}{frac}


%----Analysis
\newcommand{\dd}[2][]{\frac{\partial^{#1}}{\partial {#2}^{#1}}}
\newcommand{\summ}{\sum\limits}
\newcommand{\norm}[1]{\left \vert \left \vert #1 \right \vert \right \vert}
\newcommand{\thicc}{\bigg}
\newcommand{\overbar}[1]{\mkern 1.5mu\overline{\mkern-1.5mu#1\mkern-1.5mu}\mkern 1.5mu}
\newcommand{\eps}{\varepsilon}


%--------Theorem environments--------
\newtheorem{definition}{Definition}[]
\newtheorem{lemma}{Lemma}[]
\newtheorem{corollary}{Corollary}[]
\newtheorem{theorem}{Theorem}[]
\newtheorem*{claim}{Claim}


\newenvironment{solution}
{\begin{proof}[Solution]}
{\end{proof}}



%%%%%%%%%%%%%%%%%%%%%%%%%%%%%%%%%%%%%%%%%%%%%%%%%%%%%%%%%%

\makeatletter
\newcommand{\thickhline}{%
    \noalign {\ifnum 0=`}\fi \hrule height 1pt
    \futurelet \reserved@a \@xhline
}
\newcolumntype{"}{@{\hskip\tabcolsep\vrule width 1pt\hskip\tabcolsep}}
\makeatother

% --------Problem environment--------
\setlength\parindent{0pt}
\setcounter{secnumdepth}{0}
\newcounter{partCounter}
\newcounter{homeworkProblemCounter}
\setcounter{homeworkProblemCounter}{1}


\newenvironment{homeworkProblem}[1][-1]{
    \ifnum#1>0
        \setcounter{homeworkProblemCounter}{#1}
    \fi
    \section{Problem \arabic{homeworkProblemCounter}}
    \setcounter{partCounter}{1}
    \stepcounter{homeworkProblemCounter}
}


%--------Metadata--------
\title{MATH 7752 Homework 1}
\author{James Harbour}


\begin{document}
\maketitle

\begin{homeworkProblem}
  Let $R$ be a ring and $M$ an $R$-module. \\

  \textbf{(a)} Prove that for every $m\in M$, the map $r\mapsto rm$ from $R$ to $M$ is a homomorphism of $R$-modules. \\

  \textbf{(b)} Assume that $R$ is commutative and $M$ an $R$-module. Prove that there is an isomorphism  $\Hom_R(R,M)\simeq M$ as $R$-modules.
\end{homeworkProblem}

\begin{homeworkProblem}
  Give an explicit example of a map $f:A\to B$ with the following properties:
  \begin{itemize}
    \item $A,B$ are $R$-modules.
    \item $f$ is a group homomorphism.
    \item $f$ is not an $R$-module homomorphism.
  \end{itemize}


\end{homeworkProblem}

\begin{homeworkProblem}
  Let $R$ be a ring and $M$ an $R$-module. \\

  \textbf{(a)} Let $N$ be a subset of $M$. The \textit{annihilator} of $N$ is defined to be the set \[\Ann_R(N):=\{r\in R: rn=0, \text{ for all }n\in N\}.\] Prove that $\Ann_R(N)$ is a left ideal of $R$.

  \begin{proof}
    Let $x,y\in I$ and $r\in R$. Fix $n\in N$. Noting that $xn = 0 = yn$, it follows that
    \[
      (x+ry)n = xn + (ry)n = xn + r(yn) = 0.
    \]
    Thus $x+ry\in \Ann_R(N)$. Since all elements chosen were arbitrary, $\Ann_R(N)$ is a left ideal of $R$.
  \end{proof}

  \textbf{(b)} Show that if $N$ is an $R$-submodule of $M$, then $\Ann_R(N)$ is an ideal of $R$ (i.e. it is two-sided ideal). \\

  \begin{proof}
    By part (a), it suffices to show that $\Ann_R(N)$ is a right ideal of $R$. Moreover, part (a) shows \textit{a fortiori} that $\Ann_R(N)$ is already an abelian group, so we need only address its multiplicative structure. Let $y\in \Ann_R(N)$ and $r\in R$. Fix $n\in N$. As $N$ is an $R$-submodule of $M$, $yn\in N$, whence $(yr)n = y(rn) = 0$ by definition. Hence $\Ann_R(N)$ is a two-sided ideal of $R$.
  \end{proof}

  \textbf{(c)} For a subset $I$ of $R$ the \textit{annihilator} of $I$ in $M$ is defined to be the set,
  \[
    \Ann_M(I):=\{m\in M:xm=0, \text{ for all }x\in I\}.
  \]
  Find a natural condition on $I$ that guarantees that $\Ann_M(I)$ is a submodule of $M$. \\

  \begin{claim}
     $\Ann_M(I)$ is an $R$-submodule of $M$ if $I$ is a right ideal of $R$.
  \end{claim}

  \begin{proof}
    Suppose $I$ is a right ideal of $R$. As $x\cdot 0 = 0$ for all $x\in I$, $\Ann_M(I) \neq \emptyset$. Suppose $m,n \in \Ann_M(I)$ and $r\in R$. Fix $x\in I$. By definition $x\cdot m = 0$. As $I$ is a right ideal, $xr \in I$, so $x\cdot (m + r\cdot n) = x\cdot m + (xr)\cdot n = 0$. Thus $\Ann_M(I)$ is an $R$-submodule of $M$.
  \end{proof}


  \textbf{(d)} Let $R$ be an integral domain. Prove that every finitely generated torsion $R$-module has a nonzero annihilator.

  \begin{proof}
    Let $M$ be a finitely generated torsion $R$-module. Taking a generating set $m_1,\ldots, m_n\in M$ of $M$, for each $k\in \{ 1,\ldots, n\}$ there exists an $x_k\in \units{R} = R\setminus \{ 0\}$ such that $x_k m_k = 0$. As $\units{R}$ is closed under multiplication, $r := x_1 \cdots x_n \in \units{R}$ whence $r\neq 0$.

    Now suppose that $m\in M$. Then there exist $r_1, \ldots, r_n \in R$ such that $m = r_1 m_1 + \cdots + r_n m_n$. Observe that, by the commutativity of $R$,
    \[
      rm = (x_1 \cdots x_n)(r_1 m_1 + \cdots + r_n m_n) = \sum_{k=1}^{n} \left(\prod_{i\neq k}x_i\right) (x_k m_k) = 0.
    \]
    Thus $0\neq r\in \Ann_R(M)$, so $M$ has nonzero annihilator.
  \end{proof}
\end{homeworkProblem}

\begin{homeworkProblem}
  In class we obtained a simple characterization of $R$-modules when $R=\Z$, and $R=F[x]$, with $F$ a field. Imitate the method to find similar characterizations for $R$-modules in the following cases: \\

  \textbf{(a)} $R=\Z/n\Z$, for some $n\geq 2$.\\

  \textbf{(b)} $R=\Z[x]$. \\

  \textbf{(c)} $R=F[x,y]$.
\end{homeworkProblem}

\begin{homeworkProblem}
  An $R$-module $M$ is called \textit{simple} (or \textit{irreducible}) if its only submodules are $\{0\}$ and $M$. An $R$-module $M$ is called \textit{indecomposable} if $M$ is not isomorphic to $N\oplus Q$ for some non-zero submodules $N,Q$. Show that every simple $R$-module is indecomposable, but the converse is not true.
\end{homeworkProblem}

\begin{homeworkProblem}
  Let $R$ be a ring. An $R$-module $M$ is called \textit{cyclic} if it is generated as an $R$-module by a single element. \\

  \textbf{(a)} Prove that every cyclic $R$-module is of the form $R/I$ for some left ideal $I$ of $R$.

  \begin{proof}
    Let $M$ be a cyclic $R$-module. Then there exists an $m\in M$ such that $M = Rm$. Consider the map $\phi:R\to M$ given by $\phi(r)= rm$ for $r\in R$. By problem 1 part (a), $\phi$ is an $R$-module homomorphism; moreover, $\phi$ is surjective since $m$ generates $M$. Let $I = \ker(\phi)$, a left ideal of $R$ (actually two-sided, but we are identifying $R$ with its left regular module over itself so a priori $I$ is just a left R-submodule). Then, by the first isomorphism theorem, $M = \phi(R)\cong R/\ker(\phi) = R/I$.
  \end{proof}

  \textbf{(b)} Show that the simple $R$-modules are precisely the ones which are isomorphic to $R/\mathfrak{m}$ for some maximal left ideal $\mathfrak{m}$.

  \begin{proof}
    On one hand, $\mf{m}$ be a maximal left ideal of $R$. By the correspondence theorem applied to the natural projection, the only $R$-submodules of $R/\mf{m}$ are $\{ 0\}$ and $R/\mf{m}$, so $R/\mf{m}$ is simple (and so is every $R$-module isomorphic to it). \\

    On the other hand, suppose $M$ is a nonzero simple $R$-module. Take $m\in M\setminus\{ 0\}$. Then by the simplicity of $M$, $Rm = M$ i.e. $M$ is a cyclic module generated by $m$. Part (a) implies that there is some left ideal $\mf{m}$ of $R$ such that $M\cong R/\mf{m}$. Suppose that $I$ is a proper left ideal of $R$ such that $\mf{m}\sub I\subsetneq R$. Applying the natural projection, we see that $0 \sub I/\mf{m} \subsetneq R/\mf{m}$, whence simplicity of $R/\mf{m}$ implies that $I/\mf{m}$ is trivial i.e. $I = \mf{m}$. Thus by definition $\mf{m}$ is a maximal left ideal.
  \end{proof}

  \textbf{(c)} Show that any non-zero homomorphism of simple $R$-modules is an isomorphism. Deduce that if $M$ is simple, its endomorphism ring $\End_R(M):=\Hom_R(M,M)$ is a division ring. This result is known as \textit{Schur's Lemma}. \\
\end{homeworkProblem}

\begin{homeworkProblem}
  Show that $\Q$ is not a free $\Z$-module, that is $\Q$ is not isomorphic to a direct sum of the form $\displaystyle\bigoplus_I \Z$, for any index set $I$. More generally, let $R$ be a PID which is not a field and $K=\fr(R)$ be its fraction field. Show that $K$ is not a free $R$-module.
\end{homeworkProblem}

\begin{homeworkProblem}
  Let $R$ be a commutative ring. Recall that an ideal $I$ of $R$ is called \textit{nilpotent} if there exists some $n\in\N$ such that $I^n=0$. \\

  \textbf{(a)} Let $i\in I$. Show that the element $r=1-i$ is invertible in $R$. \\

  \textbf{(b)} Let $M,N$ be $R$-modules and let $\phi:M\rightarrow N$ be an $R$-module homomorphism. Show that $\phi$ induces an $R$-module homomorphism, $\overline{\phi}:M/IM\rightarrow N/IN$. \\

  \textbf{(c)} Prove that if $\overline{\phi}$ is sujective, then $\phi$ is itself surjective.
\end{homeworkProblem}

\begin{homeworkProblem}
  Let $G$ be a finite group and $k$ a field. Consider the group ring $k[G]$. \\

  \textbf{(a)} Let $M$ be a $k$-vector space with a $G$-action. Show that $M$ becomes a $k[G]$-module. Conversely, if $M$ is a $k[G]$-module, show that $M$ is a $G$-set. \\

  \textbf{(b)} Let $M,N$ be two $k[G]$-modules. Show that $\Hom_{k}(M,N)$ becomes a $k[G]$-module with the following $G$-action: For $g\in G$ and $\phi:M\to N$ a $k[G]$-homomorphism define
  \[
    (g\cdot\phi)(m):=g\phi(g^{-1}m),\text{ for }m\in M.
  \]
\end{homeworkProblem}


\end{document}

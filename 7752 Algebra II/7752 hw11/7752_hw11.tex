\documentclass[12pt,letterpaper]{article}

%--------Packages--------
\usepackage{amsmath, amsthm, amssymb}
\usepackage{xspace}
\usepackage{graphicx}
\usepackage{amssymb}
\usepackage{array}
\usepackage{braket}
\usepackage{multicol}
\usepackage{mathtools}
\usepackage{enumerate}
\usepackage{delarray}
\usepackage{mathtools}
\usepackage{fullpage}
\usepackage{faktor} % For quotients
\usepackage{mathrsfs}
\usepackage{quiver}
\usepackage{tikz}

\usepackage[linguistics]{forest}




%--------Page Setup--------

\pagestyle{empty}%

\setlength{\hoffset}{-1.54cm}
\setlength{\voffset}{-1.54cm}

\setlength{\topmargin}{0pt}
\setlength{\headsep}{0pt}
\setlength{\headheight}{0pt}

\setlength{\oddsidemargin}{0pt}

\setlength{\textwidth}{195mm}
\setlength{\textheight}{250mm}


%--------Macros--------

\newcommand{\ilm}[1]{\begin{psmallmatrix} #1 \end{psmallmatrix}}
\newcommand{\ilmb}[1]{\boxed{\begin{smallmatrix} #1 \end{smallmatrix}}}

\newcommand{\sub}{\subseteq}
\newcommand{\lcm}{\text{lcm}}
\newcommand{\ms}[1]{\mathscr{#1}}
\newcommand{\mc}[1]{\mathcal{#1}}
\newcommand{\mf}[1]{\mathfrak{#1}}
\newcommand{\sO}{\mathcal{O}}
\newcommand{\cyclic}[1]{\langle#1\rangle}
\newcommand{\units}[1]{#1 ^{\times}}
\newcommand{\la}{\langle}
\newcommand{\ra}{\rangle}
\newcommand{\lr}[1]{\left(#1\right)}
\newcommand{\divides}{\bigm|}
%----Switch phi and varphi
\let\temp\phi
\let\phi\varphi
\let\varphi\temp

\newcommand{\C}{\mathbb{C}}
\newcommand{\F}{\mathbb{F}}
\newcommand{\N}{\mathbb{N}\xspace}
\newcommand{\I}{\mathbb{I}\xspace}
\newcommand{\R}{\mathbb{R}\xspace}
\newcommand{\Z}{\mathbb{Z}\xspace}
\newcommand{\Q}{\mathbb{Q}\xspace}
\newcommand{\G}{\mathbb{G}\xspace}
\DeclareMathOperator{\Spec}{Spec}
\DeclareMathOperator{\res}{res}
\DeclareMathOperator{\Tr}{Tr}
\DeclareMathOperator{\ord}{ord}
\DeclareMathOperator{\Sym}{Sym}
\DeclareMathOperator{\dv}{div}
\DeclareMathOperator{\alb}{alb}
\let\Im\relax
\DeclareMathOperator{\Im}{Im}
\DeclareMathOperator{\et}{et}
\DeclareMathOperator{\ck}{coker}
\DeclareMathOperator{\Reg}{Reg}
\DeclareMathOperator{\Cor}{Cor}
\DeclareMathOperator{\Ac}{at}
\DeclareMathOperator{\supp}{supp}
\DeclareMathOperator{\Hom}{Hom}
\DeclareMathOperator{\Pic}{Pic}
\DeclareMathOperator{\Gal}{Gal}
\DeclareMathOperator{\fc}{frac}
\DeclareMathOperator{\Ann}{Ann}
\DeclareMathOperator{\Mod}{Mod}
\DeclareMathOperator{\Cone}{Cone}
\DeclareMathOperator{\FI}{FI}
\DeclareMathOperator{\End}{End}
\DeclareMathOperator{\Alb}{Alb}
\DeclareMathOperator{\Ext}{Ext}
\DeclareMathOperator{\ab}{ab}
\DeclareMathOperator{\Jac}{Jac}
\DeclareMathOperator{\coker}{coker}
\DeclareMathOperator{\fr}{frac}
\DeclareMathOperator{\spn}{span}
\DeclareMathOperator{\im}{im}
\DeclareMathOperator{\rk}{rk}
\DeclareMathOperator{\GL}{GL}
\DeclareMathOperator{\Aut}{Aut}
\DeclareMathOperator{\ch}{char}
\DeclareMathOperator{\Fix}{Fix}


%----Analysis
\newcommand{\dd}[2][]{\frac{\partial^{#1}}{\partial {#2}^{#1}}}
\newcommand{\summ}{\sum\limits}
\newcommand{\norm}[1]{\left \vert \left \vert #1 \right \vert \right \vert}
\newcommand{\thicc}{\bigg}
\newcommand{\eps}{\varepsilon}
\newcommand*\cls[1]{\overline{#1}}


%--------Theorem environments--------
\newtheorem{definition}{Definition}[]
\newtheorem{lemma}{Lemma}[]
\newtheorem{corollary}{Corollary}[]
\newtheorem{theorem}{Theorem}[]
\theoremstyle{remark}
\newtheorem*{claim}{Claim}


\newenvironment{solution}
{\begin{proof}[Solution]}
{\end{proof}}


\makeatletter
\newcolumntype{"}{@{\hskip\tabcolsep\vrule width 1pt\hskip\tabcolsep}}
\makeatother

% --------Problem environment--------
\setlength\parindent{0pt}
\setcounter{secnumdepth}{0}
\newcounter{partCounter}
\newcounter{homeworkProblemCounter}
\setcounter{homeworkProblemCounter}{1}


\newenvironment{homeworkProblem}[1][-1]{
    \ifnum#1>0
        \setcounter{homeworkProblemCounter}{#1}
    \fi
    \section{Problem \arabic{homeworkProblemCounter}}
    \setcounter{partCounter}{1}
    \stepcounter{homeworkProblemCounter}
}


%--------Metadata--------
\title{MATH 7752 Homework 11}
\author{James Harbour}


\begin{document}
\maketitle


\begin{homeworkProblem}
  In this problem you will need the following two definitions. \\

  \textbf{Definition 1:} Let $L/F$ be a finite separable extension and let $\overline{F}$ be an algebraic closure of $F$ containing $L$. A subfield $L'$ of $\overline{F}$ is called \textbf{conjugate to $L$ over $F$} if $L'=\sigma(L)$ for some $F$-embedding $\sigma:L\to\overline{F}$. (Note: $L/F$ is Galois if and only if the only conjugate to $L$ over $F$ is itself.) \\

  \textbf{Definition 2:} A finite extension $K/F$ is called a \textbf{$p$-extension} if $K/F$ is \textbf{Galois} and $\Gal(K/F)$ is a $p$-group.

  \begin{enumerate}
    \item Let $L/F$ be a separable extension of degree $n$ and let $K$ be the Galois closure of $L$ over $F$. Prove that $K$ can be written as a compositum $L_1 L_2\cdots L_n$, where $L_1,\ldots, L_n$ are (not necessarily distinct) conjugates of $L$ over $F$.
    \item Let $K/F$ and $L/F$ be finite $p$-extensions. Prove that $KL/F$ is also a $p$-extension.
    \item Suppose that $K/L$ and $L/F$ are both $p$-extensions, and let $M$ be the Galois closure of $K$ over $F$ (note: we do not know whether $K/F$ is Galois or not). Prove that $M/F$ is also a $p$-extension.
    \item Now assume only that $L/F$ is a separable extension with $[L:F]=p^r$, for some $r\geq 1$. Let $M$ be the Galois closure of $L$ over $F$. Prove that $[M:F]$ need not be a power of $p$.
  \end{enumerate}
\end{homeworkProblem}



\begin{homeworkProblem}
  Let $f(x)$ and $g(x)$ be irreducible polynomials in $\F_p[x]$ of the same degree. Let $F=\F_p[x]/(f(x))$. Prove that $g(x)$ splits completely over $F$.

  \begin{proof}
    By a vector space counting argument, $|F| = p^n$. By uniqueness of splitting fields, $F$ is $\F_p$-isomorphic to $\F_{p^n}$ which is $\F_p$-isomorphic to $\F_p[x]/(q(x))$ which contains a root of $q(x)$. Thus, $F$ contains a root of $q(x)$ whence by normality of the extensions $F/\F_p$, $q(x)$ splits over $F$.
  \end{proof}
\end{homeworkProblem}


\begin{homeworkProblem}
  Consider the polynomial $f(x)=x^4-2x^2-5\in\Q[x]$.\\

  \textbf{(a)}: Determine the Galois group $G$ of the splitting field $K$ of $f(x)$ over $\Q$.

  \begin{proof}
    Let $\alpha = \sqrt{1+\sqrt{6}}$ and $\beta = \sqrt{1-\sqrt{6}}$. Then $f(x) = (x-\alpha)(x+\alpha)(x-\beta)(x+\beta)$ and $K = \Q(\alpha,\beta)$. Noting that $\alpha^2+\beta^2 = 2$, it follows that $\mu_{\beta,\Q(\alpha)} = x^2 +(\alpha^2-2)$ and thus $[K:\Q(\alpha)] = 2$. Note that $f(x)$ is irreducible as  none of the choices of pairs of linear factors provide a polynomial in $\Q[x]$ by appealing to Vieta's formulae and the fact that $\alpha^2,\beta^2,\alpha\pm\beta\not\in\Q$.

    Thus $\G$ is an order $8$ subgroup of $S_4$, whence its isomorphism class is $D_8$.
  \end{proof}

  \textbf{(b)}: Find all subgroups of $G$ and their corresponding fixed fields. Which of those are normal extensions of $\Q$?

\end{homeworkProblem}



\begin{homeworkProblem}
  Let $p$ and $q$ be distinct primes with $q>p$, and let $K/F$ be a Galois extension of degree $pq$. Prove the following:\\

  \textbf{(a)}: There exists a field $L$ with $F\subset L\subset K$ and $[L:F]=q$.

  \begin{proof}
    Let $G = \Gal(K/F)$. Then $|G| = pq$, whence by Sylow's existence theorem there is some subgroup $H\sub G$ such that $|H| = p$. Setting $L = K^{H}$, by the fundamental theorem of Galois theory, $p = |H| = [K:K^H]$ whence $[K^H: F] = q$ as desired.
  \end{proof}

  \textbf{(b)}: There exists a \textbf{unique} field $M$ with $F\subset M\subset K$ and $[M:F]=p$.

  \begin{proof}
    Let $G = \Gal(K/F)$. Let $n_q$ denote the number of Sylow $q$-subgroups of $G$. Then as $n_q \divides p$ and $n_q\equiv 1 mod\ q$, the restriction that $q>p$ forces  $n_q = 1$. Thus there is a unique subgroup of $Q$ of $G$ of order $q$, whence by the fundamental theorem of Galois theory there is a unique intermediate subfield $M = K^Q$ of $K/F$ with $[K:M] = q$ or equivalently $[M:F] = p$.
  \end{proof}

\end{homeworkProblem}


\begin{homeworkProblem}
  Prove the following analogue of Kummer's theorem for abelian extensions: Let $n\in\N$ and let $F$ be a field containing a primitive $n^{th}$ root of unity.\\

  \textbf{(a)}: Let $K/F$ be a finite Galois extension such that $G=\Gal(K/F)$ is abelian of exponent $n$. Then there exists $a_1,\ldots,a_t\in F$ such that $K=F(\sqrt[n]{a_1},\ldots,\sqrt[n]{a_t})$. More precisely, there exists $\alpha_1,\ldots,\alpha_t\in K$ such that $K=F(\alpha_1,\ldots,\alpha_t)$ and $\alpha_i^n\in F$ for all $i$. \\

  \textbf{(b)}: Conversely, suppose that $K=F(\sqrt[n]{a_1},\ldots,\sqrt[n]{a_t})$ for some $a_1,\ldots, a_t\in F$. Prove that $K/F$ is Galois and $G=\Gal(K/F)$ is abelian of exponent $n$. \textbf{Hint:} For part (b) use one of the problems from the previous homework. \\
\end{homeworkProblem}


\begin{homeworkProblem}
  Let $F$ be a field containing a primitive $n^{th}$ root of unity. Let $a,b\in F$ be such that the polynomials $f(x)=x^n-a$, and $g(x)=x^n-b$ are both irreducible over $F$. Consider the Kummer extensions $F(\alpha)$, $F(\beta)$, where $\alpha$ is a root of $f(x)$ and $\beta$ is a root of $g(x)$. Prove that
 $F(\alpha)=F(\beta)$ if and only if $\beta=c\alpha^r$, for some $c\in F$ and some integer $r$ which is coprime to $n$ (equivalently, if and only if $b=c^n a^r$, for some $c\in F$ and some $(r,n)=1)$.
\end{homeworkProblem}




\end{document}

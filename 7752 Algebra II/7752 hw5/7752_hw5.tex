\documentclass[12pt,letterpaper]{article}

%--------Packages--------
\usepackage{amsmath, amsthm, amssymb}
\usepackage{xspace}
\usepackage{graphicx}
\usepackage{amssymb}
\usepackage{array}
\usepackage{braket}
\usepackage{multicol}
\usepackage{mathtools}
\usepackage{enumerate}
\usepackage{delarray}
\usepackage{mathtools}
\usepackage{fullpage}
\usepackage{faktor} % For quotients
\usepackage{mathrsfs}

% \usepackage{quiver}
\usepackage[linguistics]{forest}




%--------Page Setup--------

\pagestyle{empty}%

\setlength{\hoffset}{-1.54cm}
\setlength{\voffset}{-1.54cm}

\setlength{\topmargin}{0pt}
\setlength{\headsep}{0pt}
\setlength{\headheight}{0pt}

\setlength{\oddsidemargin}{0pt}

\setlength{\textwidth}{195mm}
\setlength{\textheight}{250mm}


%--------Macros--------

\newcommand{\sub}{\subseteq}
\newcommand{\lcm}{\text{lcm}}
\newcommand{\ms}[1]{\mathscr{#1}}
\newcommand{\mc}[1]{\mathcal{#1}}
\newcommand{\mf}[1]{\mathfrak{#1}}
\newcommand{\sO}{\mathcal{O}}
\newcommand{\cyclic}[1]{\langle#1\rangle}
\newcommand{\units}[1]{#1 ^{\times}}
\newcommand{\la}{\langle}
\newcommand{\ra}{\rangle}
\newcommand{\lr}[1]{\left(#1\right)}
%----Switch phi and varphi
\let\temp\phi
\let\phi\varphi
\let\varphi\temp

\newcommand{\C}{\mathbb{C}}
\newcommand{\F}{\mathbb{F}}
\newcommand{\N}{\mathbb{N}\xspace}
\newcommand{\I}{\mathbb{I}\xspace}
\newcommand{\R}{\mathbb{R}\xspace}
\newcommand{\Z}{\mathbb{Z}\xspace}
\newcommand{\Q}{\mathbb{Q}\xspace}
\newcommand{\G}{\mathbb{G}\xspace}
\DeclareMathOperator{\Spec}{Spec}
\DeclareMathOperator{\res}{res}
\DeclareMathOperator{\Tr}{Tr}
\DeclareMathOperator{\ord}{ord}
\DeclareMathOperator{\Sym}{Sym}
\DeclareMathOperator{\dv}{div}
\DeclareMathOperator{\alb}{alb}
\DeclareMathOperator{\img}{Im}
\DeclareMathOperator{\et}{et}
\DeclareMathOperator{\ck}{coker}
\DeclareMathOperator{\Reg}{Reg}
\DeclareMathOperator{\Cor}{Cor}
\DeclareMathOperator{\Ac}{at}
\DeclareMathOperator{\supp}{supp}
\DeclareMathOperator{\Hom}{Hom}
\DeclareMathOperator{\Pic}{Pic}
\DeclareMathOperator{\Gal}{Gal}
\DeclareMathOperator{\fc}{frac}
\DeclareMathOperator{\Ann}{Ann}
\DeclareMathOperator{\Mod}{Mod}
\DeclareMathOperator{\Cone}{Cone}
\DeclareMathOperator{\FI}{FI}
\DeclareMathOperator{\End}{End}
\DeclareMathOperator{\Alb}{Alb}
\DeclareMathOperator{\Ext}{Ext}
\DeclareMathOperator{\ab}{ab}
\DeclareMathOperator{\Jac}{Jac}
\DeclareMathOperator{\coker}{coker}
\DeclareMathOperator{\fr}{frac}
\DeclareMathOperator{\spn}{span}
\DeclareMathOperator{\im}{im}
\DeclareMathOperator{\rk}{rk}


%----Analysis
\newcommand{\dd}[2][]{\frac{\partial^{#1}}{\partial {#2}^{#1}}}
\newcommand{\summ}{\sum\limits}
\newcommand{\norm}[1]{\left \vert \left \vert #1 \right \vert \right \vert}
\newcommand{\thicc}{\bigg}
\newcommand{\eps}{\varepsilon}


%--------Theorem environments--------
\newtheorem{definition}{Definition}[]
\newtheorem{lemma}{Lemma}[]
\newtheorem{corollary}{Corollary}[]
\newtheorem{theorem}{Theorem}[]
\theoremstyle{remark}
\newtheorem*{claim}{Claim}


\newenvironment{solution}
{\begin{proof}[Solution]}
{\end{proof}}


\makeatletter
\newcolumntype{"}{@{\hskip\tabcolsep\vrule width 1pt\hskip\tabcolsep}}
\makeatother

% --------Problem environment--------
\setlength\parindent{0pt}
\setcounter{secnumdepth}{0}
\newcounter{partCounter}
\newcounter{homeworkProblemCounter}
\setcounter{homeworkProblemCounter}{1}


\newenvironment{homeworkProblem}[1][-1]{
    \ifnum#1>0
        \setcounter{homeworkProblemCounter}{#1}
    \fi
    \section{Problem \arabic{homeworkProblemCounter}}
    \setcounter{partCounter}{1}
    \stepcounter{homeworkProblemCounter}
}


%--------Metadata--------
\title{MATH 7752 Homework 5}
\author{James Harbour}


\begin{document}
\maketitle

\begin{homeworkProblem}
  Let $F=\Z^3$ be the free $\Z$-module of rank $3$. Let $N$ be the submodule of $F$ generated by $v_1=(1,2,3), (5,4,6)$, and $(7,8,9)$.
  \begin{enumerate}
  \item Find compatible bases for $F$ and $N$, that is, bases satisfying the submodule theorem 1.
  \item Describe the quotient $F/N$ in the IF form.
  \item Describe in IF form the abelian group given by the presentation
  \[\langle a,b,c\;|\;a+2b+3c=0,5a+4b+6c=0, 7a+8b+9c=0\rangle.\]
  \end{enumerate}

\end{homeworkProblem}

\begin{homeworkProblem}
  Let $R$ be a PID. For an $R$-module $M$ define $\rk(M)$ to be the minimal size of a generating set of $M$.  \begin{enumerate}
  [(a)]\item Let $M$ be a finitely generated $R$-module and $R/a_1R\oplus\cdots\oplus R/a_mR\oplus R^s$ be its invariant factor decomposition. That is, $s\geq 0$ and the elements $a_1,\ldots, a_m$ are non-zero, non-units such that $a_1|a_2|\cdots|a_m$. Prove that $\rk(M)=m+s$. \textbf{Warning:} It is not true in general that $\rk(P\oplus Q)=\rk(P)\oplus\rk(Q)$.
  \item Let $F$ be a free $R$-module of rank $n$ with basis $e_1,\ldots e_n$. Let $N$ be the submodule of $F$ generated by some elements $v_1,\ldots,v_n\in F$. Let $A\in Mat_n(F)$ be the matrix such that
  \[\left(\begin{array}{ccc}
  v_1\\
  \vdots\\
  v_n
  \end{array}\right)=A\left(\begin{array}{ccc}
  e_1\\
  \vdots\\
  e_n
  \end{array}\right).\] Find a simple condition on the entries of $A$ which holds if and only if $\rk(F/N)=n$.
  \end{enumerate}
\end{homeworkProblem}

\begin{homeworkProblem}
  In this problem $R$ will be a commutative domain. An $R$-module $P$ is called \textit{projective} if it is a direct summand of a free $R$-module. That is, if there exist a free $R$-module $F$ and a submodule $Q$ of $F$ such that $F=P\oplus Q$. \\

  \begin{enumerate}
  \item Let $P,M,N$ be $R$-modules and suppose $f:M\rightarrow N$ is a surjective $R$-module homomorphism. The map $f$ induces a homomorphism of $R$-modules,
  \begin{eqnarray*}
  f_\star:&&\Hom_R(P,M)\rightarrow\Hom_R(P,N)\\
  &&[\phi:P\rightarrow M]\mapsto [f\circ\phi:P\rightarrow N].
  \end{eqnarray*} Prove that if $P$ is finitely generated and projective, then $f_\star$ is surjective.\\ \textbf{Hint:} The universal property of free $R$-modules will be useful.
  \item Show that if $R$ is a PID and $P$ is finitely generated, then $P$ is projective if and only if $P$ is free.
  \\
  \end{enumerate}
\end{homeworkProblem}

\begin{homeworkProblem}
  Determine the number of possible RCF's of $8\times 8$ matrices $A$ over $\Q$ with $\chi_A(x)=x^8-x^4$. Explain your argument in detail.
\end{homeworkProblem}

\end{document}

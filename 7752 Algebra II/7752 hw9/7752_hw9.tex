\documentclass[12pt,letterpaper]{article}

%--------Packages--------
\usepackage{amsmath, amsthm, amssymb}
\usepackage{xspace}
\usepackage{graphicx}
\usepackage{amssymb}
\usepackage{array}
\usepackage{braket}
\usepackage{multicol}
\usepackage{mathtools}
\usepackage{enumerate}
\usepackage{delarray}
\usepackage{mathtools}
\usepackage{fullpage}
\usepackage{faktor} % For quotients
\usepackage{mathrsfs}
% \usepackage{quiver}
% \usepackage{tikz}

% \usepackage{quiver}
\usepackage[linguistics]{forest}




%--------Page Setup--------

\pagestyle{empty}%

\setlength{\hoffset}{-1.54cm}
\setlength{\voffset}{-1.54cm}

\setlength{\topmargin}{0pt}
\setlength{\headsep}{0pt}
\setlength{\headheight}{0pt}

\setlength{\oddsidemargin}{0pt}

\setlength{\textwidth}{195mm}
\setlength{\textheight}{250mm}


%--------Macros--------

\newcommand{\ilm}[1]{\begin{psmallmatrix} #1 \end{psmallmatrix}}
\newcommand{\ilmb}[1]{\boxed{\begin{smallmatrix} #1 \end{smallmatrix}}}

\newcommand{\sub}{\subseteq}
\newcommand{\lcm}{\text{lcm}}
\newcommand{\ms}[1]{\mathscr{#1}}
\newcommand{\mc}[1]{\mathcal{#1}}
\newcommand{\mf}[1]{\mathfrak{#1}}
\newcommand{\sO}{\mathcal{O}}
\newcommand{\cyclic}[1]{\langle#1\rangle}
\newcommand{\units}[1]{#1 ^{\times}}
\newcommand{\la}{\langle}
\newcommand{\ra}{\rangle}
\newcommand{\lr}[1]{\left(#1\right)}
%----Switch phi and varphi
\let\temp\phi
\let\phi\varphi
\let\varphi\temp

\newcommand{\C}{\mathbb{C}}
\newcommand{\F}{\mathbb{F}}
\newcommand{\N}{\mathbb{N}\xspace}
\newcommand{\I}{\mathbb{I}\xspace}
\newcommand{\R}{\mathbb{R}\xspace}
\newcommand{\Z}{\mathbb{Z}\xspace}
\newcommand{\Q}{\mathbb{Q}\xspace}
\newcommand{\G}{\mathbb{G}\xspace}
\DeclareMathOperator{\Spec}{Spec}
\DeclareMathOperator{\res}{res}
\DeclareMathOperator{\Tr}{Tr}
\DeclareMathOperator{\ord}{ord}
\DeclareMathOperator{\Sym}{Sym}
\DeclareMathOperator{\dv}{div}
\DeclareMathOperator{\alb}{alb}
\let\Im\relax
\DeclareMathOperator{\Im}{Im}
\DeclareMathOperator{\et}{et}
\DeclareMathOperator{\ck}{coker}
\DeclareMathOperator{\Reg}{Reg}
\DeclareMathOperator{\Cor}{Cor}
\DeclareMathOperator{\Ac}{at}
\DeclareMathOperator{\supp}{supp}
\DeclareMathOperator{\Hom}{Hom}
\DeclareMathOperator{\Pic}{Pic}
\DeclareMathOperator{\Gal}{Gal}
\DeclareMathOperator{\fc}{frac}
\DeclareMathOperator{\Ann}{Ann}
\DeclareMathOperator{\Mod}{Mod}
\DeclareMathOperator{\Cone}{Cone}
\DeclareMathOperator{\FI}{FI}
\DeclareMathOperator{\End}{End}
\DeclareMathOperator{\Alb}{Alb}
\DeclareMathOperator{\Ext}{Ext}
\DeclareMathOperator{\ab}{ab}
\DeclareMathOperator{\Jac}{Jac}
\DeclareMathOperator{\coker}{coker}
\DeclareMathOperator{\fr}{frac}
\DeclareMathOperator{\spn}{span}
\DeclareMathOperator{\im}{im}
\DeclareMathOperator{\rk}{rk}
\DeclareMathOperator{\GL}{GL}
\DeclareMathOperator{\Aut}{Aut}
\DeclareMathOperator{\ch}{char}
\DeclareMathOperator{\Fix}{Fix}


%----Analysis
\newcommand{\dd}[2][]{\frac{\partial^{#1}}{\partial {#2}^{#1}}}
\newcommand{\summ}{\sum\limits}
\newcommand{\norm}[1]{\left \vert \left \vert #1 \right \vert \right \vert}
\newcommand{\thicc}{\bigg}
\newcommand{\eps}{\varepsilon}
\newcommand*\cls[1]{\overline{#1}}


%--------Theorem environments--------
\newtheorem{definition}{Definition}[]
\newtheorem{lemma}{Lemma}[]
\newtheorem{corollary}{Corollary}[]
\newtheorem{theorem}{Theorem}[]
\theoremstyle{remark}
\newtheorem*{claim}{Claim}


\newenvironment{solution}
{\begin{proof}[Solution]}
{\end{proof}}


\makeatletter
\newcolumntype{"}{@{\hskip\tabcolsep\vrule width 1pt\hskip\tabcolsep}}
\makeatother

% --------Problem environment--------
\setlength\parindent{0pt}
\setcounter{secnumdepth}{0}
\newcounter{partCounter}
\newcounter{homeworkProblemCounter}
\setcounter{homeworkProblemCounter}{1}


\newenvironment{homeworkProblem}[1][-1]{
    \ifnum#1>0
        \setcounter{homeworkProblemCounter}{#1}
    \fi
    \section{Problem \arabic{homeworkProblemCounter}}
    \setcounter{partCounter}{1}
    \stepcounter{homeworkProblemCounter}
}


%--------Metadata--------
\title{MATH 7752 Homework 9}
\author{James Harbour}


\begin{document}
\maketitle


\begin{homeworkProblem}
  Let $K/L/F$ be a tower of algebraic extensions. Show that $K/F$ is separable if and only if $K/L$ and $L/F$ are separable.

  \begin{proof}\ \\
    \underline{$\implies$}: Suppose that $K/F$ is separable. Let $\alpha\in K$. As $\mu_{\alpha,L}\vert\mu_{\alpha,F}$, it follows that $\mu_{\alpha,L}$ is separable so $K/L$ is separable. As $L\sub K$ and every element of $K$ is separable over $F$, it follows that $L/F$ is also separable. \\

    \underline{$\impliedby$}: Assume that $K/L$ and $L/F$ are separable and
    suppose, for the sake of contradiction, that $\alpha\in K\setminus L$ is inseparable. As $\mu_{\alpha,L}\vert \mu_{\alpha,F}$, there exists some $p(x)\in L[x]$ and $k\geq 1$ such that $\mu_{\alpha,F}(x) = (\mu_{\alpha,L}(x))^k p(x)$ and $\gcd(\mu_{\alpha,L},p)=1$. As $\mu_{\alpha,F}$ is inseparable,
    \[
      0 = \mu_{\alpha,F}' = \mu_{\alpha,L}^{k-1}(k\mu_{\alpha,L}' p + \mu_{\alpha,L}p') \implies -k\mu_{\alpha,L}'p = \mu_{\alpha,L}'p,
    \]
    contradicting that $\gcd(\mu_{\alpha,L},p)=1$.
  \end{proof}

\end{homeworkProblem}


\begin{homeworkProblem}
  Let $K/F$ be a finite separable extension. Show that there is a finite number of fields $L$ such that $F\subseteq L\subseteq K$.

  \begin{proof}
    By the primitive element theorem, there exists an $\alpha\in K$ such that $K = F(\alpha)$. Let $n = \deg(\mu_{\alpha,F})$, so $[K:F]\leq n$. Consider a splitting field $E$ of $\mu_{\alpha,F}$. \\


    By induction over the number of roots of $\mu_{\alpha,F}$, it follows that $E/F$ is finite of degree at most $n!$.\\

    Moreover, $E/F$ is Galois and $|\Gal(E/F)| = [E:F]\leq n!$, and intermediate fields $L$ with $F\sub L\sub E$ correspond precisely to subgroups of $\Gal(E/F)$, of which there are finitely many, so there are \emph{a fortiori} finitely many intermediate fields $L$ with $F\sub L\sub F(\alpha)\sub E$.
  \end{proof}

\end{homeworkProblem}


\begin{homeworkProblem}
  Let $F$ be a field of $\ch(F)=p>0.$ Show that $F$ admits a finite inseparable extension $K/F$ if and only if $F$ is not perfect.

  \begin{proof}\ \\
    \underline{$\impliedby$}: Suppose that $F$ is not perfect. Take $\alpha\in F\setminus\phi(F)$. We claim first that the polynomial $f(x) = x^p-\alpha\in F[x]$ is irreducible. Fix an algebraic closure $\cls{F}$ of $F$ and identify $F$ with its copy inside $\cls{F}$. Let $\beta\in\cls{F}$ be a root of $f$. Then $\beta^p = \alpha$, so $x^p-\alpha = x^p-\beta^p = (x-\beta)^p\in\cls{F}[x]$, so if $f = gh$ for some $g,h\in F[x]$ then $g = (x-\beta)^n$ with $n<p$, whence $(x-\beta)^n = x^n -n\cdot\beta x^{n-1}+\cdots \not\in F[x]$ as $\beta\not\in F$.\\

    Now, consider the field $K = F[x]/(x^p-\alpha)$ and identify $F$ inside $K$. Then $[K:F] = p$ and $y^p-\alpha\in F[y]$ has a root $\cls{x}\in K$, whence $y^p-\alpha = y^p-\cls{x}^p = (y-\cls{x})^p$ and is thus not separable.\\

    \underline{$\implies$}: We proceed by contraposition. Suppose that $F$ is perfect. Let $K/F$ be a finite extension and suppose that $\alpha\in K$. Let $f(x) = \mu_{\alpha,F}\in F[x]$. By the argument in problem 5 part (a)'s proof, there exists a separable $g(x)\in F[x]$ and $n\geq 0$ such that $f(x) = g(x^{p^n})$. Consider the polynomial $g(x^{p^{n-1}})$. As $\phi$ is surjective, each of its coefficients have $p^{th}$ roots, so let $h(x^{p^{n-1}})\in F[x]$ be the polynomial obtained by replaceing every coefficient in $g(x^{p^{n-1}})$ with its $p^{th}$ root. Then $f(x) = g(x^{p^n}) = h(x^{p^{n-1}})^p$, whence irreduciblity of $f$ implies that $n=0$, and thus $f = g$ is separable, so $K/F$ is separable.
  \end{proof}

\end{homeworkProblem}


\begin{homeworkProblem}
  Let $F$ be a field of characteristic $p>0$ and let $K/F$ be an extension. \\

  \textbf{(a)}: Let $E=\{\alpha\in K:\alpha^{p^n}\in F,\text{ for some }n\geq 1\}$. Prove that $E$ is a subfield of $K$.

  \begin{proof}
    It is clear that $0,1\in E$. Suppose that $\alpha,\beta\in E$. Then there exist $n,m\in\N$ such that $\alpha^{p^n},\beta^{p^m}\in F$, whence
    \[
      (\alpha+\beta)^{p^{mn}} = (\alpha^{p^n})^m + (\beta^{p^m})^n \in F
    \]
    so $\alpha+\beta\in E$. Also $(-\alpha)^{p^n} = (-1)^{p^n}\alpha^{p^n}$, so $-\alpha\in E$, and $(\alpha\beta)^{p^{mn}}=(\alpha^{p^n})^m (\beta^{p^m})^n \in F$ so $\alpha\beta\in F$. Lastly, $(\frac{1}{\alpha})^{p^n} = \frac{1}{\alpha^{p^n}}\in F$, so $\frac{1}{\alpha}\in E$.
  \end{proof}

  \textbf{(b)}: Show that every $F$-automorphism of $K$ is automatically an $E$-automorphism.

  \begin{proof}
    Let $\sigma\in\Aut(K/F)$. Suppose that $\alpha\in E$, so there is some $n\in\N$ such that $\alpha^{p^n}\in F$. Then
    \[
      \sigma(\alpha)^{p^n} = \sigma(\alpha^{p^n}) = \alpha^{p^n}\implies (\sigma(\alpha)-\alpha)^{p^n} = 0
    \]
    so $\sigma(\alpha)=\alpha$.
  \end{proof}

\end{homeworkProblem}


\begin{homeworkProblem}
  Let $F$ be a field of characteristic $p>0$ and let $K/F$ be a finite extension. \\

  \textbf{(a)}: Let $\alpha\in K$. Show that either $\alpha^{p^n}\in F$ for some $n\geq 1$, or there exists some $m\geq 1$ such that $\alpha^{p^m}\not\in F$ and the element $\alpha^{p^m}$ is separable over $F$.

  \begin{proof}
    Suppose that $\alpha^{p^n}\not\in F$ for all $n\in \N$. Consider $f(x) = \mu_{\alpha,F}(x)$. If $f$ is separable, then we are done. Otherwise, $f'=0$, so there exists some $h\in F[x]$ with $\deg(h)<\deg(f)$ such that $f = h(x^p)$. Continuing in this way until decreasing degree forces us to stop, we find some $g\in F[x]$ and $n\in\N$ such that $f(x) = g(x^{p^n})$ and such that $g$ is separable, so $\alpha^{p^n}$ is separable.
  \end{proof}


  \textbf{(b)} Suppose that no element of $K\setminus F$ is separable over $F$. (Such extensions are called \textit{purely inseparable}). Deduce that for every $\alpha\in K$ there exists some $n\geq 1$ (depending on $\alpha$) such that $\alpha^{p^n}\in F$.

  \begin{proof}
    Let $\alpha\in K\setminus F$ and suppose for the sake of contradiction that $\alpha^{p^n}\not\in F$ for all $n\in \N$. Then by part (a), there is some $m\in \N$ such that $\alpha^{p^m}\not\in F$ and $\alpha^{p^m}$ is separable over $F$, contradicting the assumption that $K/F$ is purely inseparable.
  \end{proof}

\end{homeworkProblem}


\begin{homeworkProblem}
  The purpose of this problem is to show that the primitive element theorem is not true for inseparable extensions. Let $p$ be a prime number. Let $t$ be a transcendental element over $\F_p$ and let $F=\F_p(t)$. Let $s$ be a transcendental element over $F$ and let $K=F(s)$. Consider the polynomial $f(x)=(x^p-t)(x^p-s)\in K[x]$ and let $L$ be its splitting field.\\

  \textbf{(1)}: Prove that $[L:K]=p^2$.
  \begin{proof}
    Let $\alpha,\beta\in L$ such that $\alpha^p = t$ and $\beta^p = s$. Then $f(x) = (x-\alpha)^p(x-\beta)^p$, so $L = K(\alpha,\beta)$. By the same argument as in problem 3, these polynomials are irreducible, so $\mu_{\alpha,K} = x^p - t$ and $\mu_{\alpha,K} = x^p-s$. We claim that $\beta\not \in K(\alpha)$. Suppose, for the sake of contradiction, that there exist $p(x),q(x)\in K[x]$ such that $\beta = \frac{p(\alpha)}{q(\alpha)}$. Moreover, there exist $\tilde p,\tilde q \in K[x]$ such that $p(x)^p = \tilde p(x^p)$ and $q(x)^p = \tilde q(x^p)$. Then,
    \[
      \beta^p = \frac{p(\alpha)^p}{q(\alpha)^p} = \frac{\tilde p(t)}{\tilde q(t)} \implies s \cdot\tilde q(t) - \tilde p(t) = 0
    \]
    contradicting that $s$ is transcendental over $K$. Thus $[K(\alpha,\beta):K(\alpha)] = p$, so $[L:K] = p^2$.
  \end{proof}

  \textbf{(2)}: Show that for every $\gamma\in L$, it follows that $\gamma^p\in K$.

  \begin{proof}
    Let $\gamma\in L = K(\alpha,\beta)$. Then there exist $p(x,y),q(x,y)\in K[x,y]$ such that $\gamma = \frac{p(\alpha,\beta)}{q(\alpha,\beta)}$. Moreover, there are $u(x,y),v(x,y)\in K[x,y]$ such that $p(x,y)^p = u(x^p,y^p)$ and $q(x,y)^p = v(x^p,y^p)$. Then
    \[
      \gamma^p = \frac{p(\alpha,\beta)^p}{q(\alpha,\beta)^p} = \frac{u(t,s)}{v(t,s)}\in K.
    \]

  \end{proof}

  \textbf{(3)}: Show that the extension $L/K$ is not simple.

  \begin{proof}
    Suppose, for the sake of contradiction, that $L/K$ is simple. Then there exists some $\gamma\in L$ such that $L = K(\gamma)$. As $\gamma^p\in K$, it follows that $[L:K]\leq p$, contradicting that $[L:K] = p^2$.
  \end{proof}

\end{homeworkProblem}

\end{document}

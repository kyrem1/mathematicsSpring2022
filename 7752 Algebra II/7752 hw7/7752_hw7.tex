\documentclass[12pt,letterpaper]{article}

%--------Packages--------
\usepackage{amsmath, amsthm, amssymb}
\usepackage{xspace}
\usepackage{graphicx}
\usepackage{amssymb}
\usepackage{array}
\usepackage{braket}
\usepackage{multicol}
\usepackage{mathtools}
\usepackage{enumerate}
\usepackage{delarray}
\usepackage{mathtools}
\usepackage{fullpage}
\usepackage{faktor} % For quotients
\usepackage{mathrsfs}
% \usepackage{quiver}
% \usepackage{tikz}

% \usepackage{quiver}
\usepackage[linguistics]{forest}




%--------Page Setup--------

\pagestyle{empty}%

\setlength{\hoffset}{-1.54cm}
\setlength{\voffset}{-1.54cm}

\setlength{\topmargin}{0pt}
\setlength{\headsep}{0pt}
\setlength{\headheight}{0pt}

\setlength{\oddsidemargin}{0pt}

\setlength{\textwidth}{195mm}
\setlength{\textheight}{250mm}


%--------Macros--------

\newcommand{\ilm}[1]{\begin{psmallmatrix} #1 \end{psmallmatrix}}
\newcommand{\ilmb}[1]{\boxed{\begin{smallmatrix} #1 \end{smallmatrix}}}

\newcommand{\sub}{\subseteq}
\newcommand{\lcm}{\text{lcm}}
\newcommand{\ms}[1]{\mathscr{#1}}
\newcommand{\mc}[1]{\mathcal{#1}}
\newcommand{\mf}[1]{\mathfrak{#1}}
\newcommand{\sO}{\mathcal{O}}
\newcommand{\cyclic}[1]{\langle#1\rangle}
\newcommand{\units}[1]{#1 ^{\times}}
\newcommand{\la}{\langle}
\newcommand{\ra}{\rangle}
\newcommand{\lr}[1]{\left(#1\right)}
%----Switch phi and varphi
\let\temp\phi
\let\phi\varphi
\let\varphi\temp

\newcommand{\C}{\mathbb{C}}
\newcommand{\F}{\mathbb{F}}
\newcommand{\N}{\mathbb{N}\xspace}
\newcommand{\I}{\mathbb{I}\xspace}
\newcommand{\R}{\mathbb{R}\xspace}
\newcommand{\Z}{\mathbb{Z}\xspace}
\newcommand{\Q}{\mathbb{Q}\xspace}
\newcommand{\G}{\mathbb{G}\xspace}
\DeclareMathOperator{\Spec}{Spec}
\DeclareMathOperator{\res}{res}
\DeclareMathOperator{\Tr}{Tr}
\DeclareMathOperator{\ord}{ord}
\DeclareMathOperator{\Sym}{Sym}
\DeclareMathOperator{\dv}{div}
\DeclareMathOperator{\alb}{alb}
\let\Im\relax
\DeclareMathOperator{\Im}{Im}
\DeclareMathOperator{\et}{et}
\DeclareMathOperator{\ck}{coker}
\DeclareMathOperator{\Reg}{Reg}
\DeclareMathOperator{\Cor}{Cor}
\DeclareMathOperator{\Ac}{at}
\DeclareMathOperator{\supp}{supp}
\DeclareMathOperator{\Hom}{Hom}
\DeclareMathOperator{\Pic}{Pic}
\DeclareMathOperator{\Gal}{Gal}
\DeclareMathOperator{\fc}{frac}
\DeclareMathOperator{\Ann}{Ann}
\DeclareMathOperator{\Mod}{Mod}
\DeclareMathOperator{\Cone}{Cone}
\DeclareMathOperator{\FI}{FI}
\DeclareMathOperator{\End}{End}
\DeclareMathOperator{\Alb}{Alb}
\DeclareMathOperator{\Ext}{Ext}
\DeclareMathOperator{\ab}{ab}
\DeclareMathOperator{\Jac}{Jac}
\DeclareMathOperator{\coker}{coker}
\DeclareMathOperator{\fr}{frac}
\DeclareMathOperator{\spn}{span}
\DeclareMathOperator{\im}{im}
\DeclareMathOperator{\rk}{rk}
\DeclareMathOperator{\GL}{GL}


%----Analysis
\newcommand{\dd}[2][]{\frac{\partial^{#1}}{\partial {#2}^{#1}}}
\newcommand{\summ}{\sum\limits}
\newcommand{\norm}[1]{\left \vert \left \vert #1 \right \vert \right \vert}
\newcommand{\thicc}{\bigg}
\newcommand{\eps}{\varepsilon}
\newcommand*\cls[1]{\overline{#1}}


%--------Theorem environments--------
\newtheorem{definition}{Definition}[]
\newtheorem{lemma}{Lemma}[]
\newtheorem{corollary}{Corollary}[]
\newtheorem{theorem}{Theorem}[]
\theoremstyle{remark}
\newtheorem*{claim}{Claim}


\newenvironment{solution}
{\begin{proof}[Solution]}
{\end{proof}}


\makeatletter
\newcolumntype{"}{@{\hskip\tabcolsep\vrule width 1pt\hskip\tabcolsep}}
\makeatother

% --------Problem environment--------
\setlength\parindent{0pt}
\setcounter{secnumdepth}{0}
\newcounter{partCounter}
\newcounter{homeworkProblemCounter}
\setcounter{homeworkProblemCounter}{1}


\newenvironment{homeworkProblem}[1][-1]{
    \ifnum#1>0
        \setcounter{homeworkProblemCounter}{#1}
    \fi
    \section{Problem \arabic{homeworkProblemCounter}}
    \setcounter{partCounter}{1}
    \stepcounter{homeworkProblemCounter}
}


%--------Metadata--------
\title{MATH 7752 Homework 7}
\author{James Harbour}


\begin{document}
\maketitle

\begin{homeworkProblem}
  \textbf{(a)}: Consider the field $K=\Q(\sqrt{2},\sqrt{3})$. Prove that $[K:\Q]=4$.

  \begin{proof}
    As $[\Q(\sqrt{3}):\Q] = 2$, it follows that $[K:\Q(\sqrt{2})]\leq 2$. We claim that $\sqrt{3}\not\in Q(\sqrt{2})$. Suppose, for the sake of contradiction, that $\sqrt{3}\in \Q(\sqrt{2})$. Then there exist $a,b\in\Q$ such that $a+b\sqrt{2}= \sqrt{3}$. So $3=a^2+2ab\sqrt{2} + 2b^2$, whence $a$ or $b$ is $0$ as otherwise this would imply that $\sqrt{2}$ is rational which is absurd. If both are zero, the $3=0$ which is absurd, so at least one of them is nonzero. If $a=0, b\neq 0$, then $3=2b^2$, which is absurd as $3$ is odd. If $b=0, a\neq0$, then $3=a^2$ whence $a = \pm\sqrt{3}$ which is absurd as $a\in \Q$.\\

    Thus, $\sqrt{3}\not\in \Q(\sqrt{2})$, so $[K:\Q(\sqrt{2})]= 2$ whence \[ [K:\Q] = [K:\Q(\sqrt{2})][\Q(\sqrt{2}): \Q] = 4\]
  \end{proof}

  \textbf{(b)}: Let $L=\Q(\sqrt{2}+\sqrt{3})$. Show that $L=K$.

  \begin{proof}
    Clearly $\Q(\sqrt{2}+\sqrt{3})\sub \Q(\sqrt{2},\sqrt{3})$, so it suffices to prove the reverse inclusion. Let $\alpha = \sqrt{2}+\sqrt{3}$. By rationalizing, we find that $\frac{1}{\alpha} = \sqrt{3}-\sqrt{2}$ whence $\sqrt{2} = \alpha-\frac{1}{\sqrt{\alpha}}\in\Q(\alpha)$ and $\sqrt{3} = \alpha+\frac{1}{\sqrt{\alpha}}\in\Q(\alpha)$, so $\Q(\sqrt{2},\sqrt{3})\sub \Q(\alpha)$.
  \end{proof}
\end{homeworkProblem}

\begin{homeworkProblem}
  Let $S=\{n_1,\ldots, n_r\}$ be a finite set of positive integers with $n_i\geq 2$. For each $j\in\{1,\ldots,r\}$ let $\Q_j=\Q(\sqrt{n_1},\ldots,\sqrt{n_j})$. Moreover, set $\Q_0=\Q$. \\

  \textbf{(a)}: Prove that $[\Q_r:\Q]=2^m$ for some integer $0\leq m\leq r$. Moreover, show that the following set spans $\Q_r$ over $\Q$, \[P(S)=\{1\}\cup\{\sqrt{n}:n\text{ is a product of distinct elements from }S\}.\]

  \begin{proof}
    For all $i$, as $\sqrt{n_i}^2-n_i = 0$, it follows that $\mu_{n_i,Q_{i-1}}|x^2-n_i$ so $[Q_i:Q_{i-1}]\in \{1,2\}$. Thus, \[[Q_r:Q] = [Q_r:Q_{r-1}]\cdots[Q_1:Q_0] = 2^m\]for some $m\leq r$. As $\sqrt{S}:=\{\sqrt{n}:n\in S\}\sub P(S)$ and every element of $\sqrt{S}$ is algebraic over $Q$, $\Q_r = Q(P(S)) = Q[P(S)]$ as desired.
  \end{proof}

  \textbf{(b)}: Prove that $[\Q_r:\Q]<2^r$ if and only if $n_1$ is a complete square, or there exists $2\leq j\leq r$ such that $\sqrt{n_j}=\alpha+\beta\sqrt{n_{j-1}}$, for some $\alpha,\beta\in\Q_{j-2}$.

  \begin{proof}\ \\
    \underline{$\implies$}: Suppose that $[\Q_r:\Q] < 2^r$. If $n_1$ is not a complete square, then $[Q_1:Q_0]=2$ whence there is at least one $j\in\{2,\ldots, r\}$ such that $[\Q_j:\Q_{j-1}] = 1$. Then $\sqrt{n_j}\in\Q_{j-1} = \Q_{j-2}(\sqrt{n_{j-1}}) = \Q_{j-2}[\sqrt{n_{j-1}}]$, so there exist $a,b\in Q_{j-2}$ such that $\sqrt{n_j} = a + b\sqrt{n_{j-1}}$.\\

    \underline{$\impliedby$}: If $n_1$ is a complete square then $[\Q_1: \Q_0] = 1$ whence $[\Q_r:\Q]\leq 2^{r-1}<2^r$, so suppose that $n_1$ is not a complete square and that there exists $2\leq j\leq r$ such that $\sqrt{n_j}=\alpha+\beta\sqrt{n_{j-1}}$, for some $\alpha,\beta\in\Q_{j-2}$. Then $\sqrt{n_j}\in \Q_{j-2}[\sqrt{n_{j-1}}] = \Q_{j-2}(\sqrt{n_{j-1}}) = \Q_{j-1}$, whence $[\Q_j:\Q_{j-1}] = 1$ and thus $[\Q_r:\Q]\leq 2^{r-1}<2^r$.
  \end{proof}

  \textbf{(c)}: Suppose that the integers $n_1,\ldots, n_r$ are square-free and pairwise relatively prime. Prove that $[\Q_r:\Q]=2^r$. Conclude that the extension $L=\Q(T)$, where $T=\{\sqrt{n}:n\in\N, n \text{ square free}\}$ is an infinite algebraic extension of $\Q$. \\
\end{homeworkProblem}


\begin{homeworkProblem}
  Let $F$ be a field and $\alpha$ an algebraic element of odd degree over $F$ (i.e. the degree $[F(\alpha):F]$ is odd). Show that $F(\alpha^2)=F(\alpha)$. \\

  \begin{proof}
    Note that we have a tower of field extensions $F\sub F(\alpha^2)\sub F(\alpha)$. As $\alpha$ is a root of $x^2-\alpha^2\in F(\alpha^2)[x]$, it follows that $\mu_{\alpha,F(\alpha^2)}|x^2-\alpha^2$ and thus $[F(\alpha):F(\alpha^2)]\leq 2$. Suppose, for the sake of contradiction, that $F(\alpha^2)\neq F(\alpha)$. Then $[F(\alpha):F(\alpha^2)] = 2$, whence $[F(\alpha) : F] = [F(\alpha):F(\alpha^2)][F(\alpha^2):F] = 2[F(\alpha^2):F]$ is even, contradiction the assumption that $\alpha$ has odd degree over $F$.
  \end{proof}
\end{homeworkProblem}

\begin{homeworkProblem}
  Let $K/F$ be an algebraic extension. \\

  \textbf{(a)}: Let $F\subset R\subset K$ where $R$ is a subring of $K$. Prove that $R$ must be a subfield.

  \begin{proof}
    Let $\alpha\in R\setminus \{ 0\}$. Then as $K/F$ is algebraic and $\alpha\in K$, so $\alpha$ is algebraic over $F$. Hence, $F(\alpha) = F[\alpha]\sub R$, whence $\alpha^{-1}\in R$, so $R$ is a field.
  \end{proof}

  \textbf{(b)}: Show that (a) would be false if we dropped the assumption that $K/F$ is algebraic.

  \begin{proof}
    Suppose that $K/F$ is not algebraic. Take $\alpha \in K\setminus \{ 0\}$ transcendental over $F$. Then $F[\alpha]$ is a subring of $K$. We claim that $\frac{1}{\alpha}\not\in F[\alpha]$. Suppose, for the sake of contradiction, that $\frac{1}{\alpha}\in F[\alpha]$. Then there exist $b_0,\cdots,b_n\in F$ such that $f(x) = b_n x^n +\cdots + b_0\in F[x]$ has $f(\frac{1}{\alpha})=0$. Then
    \[ 0 = \alpha^n\cdot f\lr{\frac{1}{\alpha}} = \sum_{k=0}^{n}b_k\alpha^{n-k}\]
    whence $\alpha$ is algebraic over $F$, contradicting that $\alpha$ is transcendental over $F$.
  \end{proof}
\end{homeworkProblem}

\begin{homeworkProblem}
  Let $K/F$ be a finite field extension, $n=[K:F]$, and fix some basis $\Omega=\{\alpha_1,\ldots,\alpha_n\}$ of $K$ over $F$. For any $\alpha\in K$ define $T_\alpha:K\rightarrow K$ by $\beta\mapsto\alpha\beta$. Note that $T_\alpha\in\End_F(K)$.  Let $A_\alpha=[T_\alpha]_\Omega\in M_n(F)$ be the matrix of $T_\alpha$ with respect to $\Omega$.  \\

  \textbf{(a)}: Prove that the map $K\xrightarrow{\rho} M_n(F)$ given by $\alpha\mapsto A_\alpha$ is an injective ring homomorphism.

  \begin{proof}
    Note that, if $\alpha,\beta\in K$, then $(T_\alpha T_\beta) (\gamma) = T_\alpha (\beta\gamma) = \alpha\beta\gamma = T_{\alpha\beta}(\gamma)$ and $(T_\alpha+ T_\beta) (\gamma) = T_\alpha(\gamma) + T_\beta(\gamma) = (\alpha+\beta)\gamma = T_{\alpha+\beta}(\gamma)$ for all $\gamma\in K$, so $T_\alpha T_\beta = T_{\alpha\beta}$ and $T_\alpha + T_\beta = T_{\alpha+\beta}$. Thus
    \begin{align*}
      A_\alpha A_\beta &= [T_\alpha]_\Omega [T_\beta]_\Omega = [T_\alpha T_\beta]_\Omega = [T_{\alpha\beta}]_\Omega = A_{\alpha\beta}\\
      A_\alpha + A_\beta &= [T_\alpha]_\Omega +[T_\beta]_\Omega = [T_\alpha +T_\beta]_\Omega = [T_{\alpha+\beta}]_\Omega = A_{\alpha+\beta},
    \end{align*}
    so the map $\alpha\mapsto A_\alpha$ is a ring homomorhpism. As $\ker(\rho)\sub K$ is an ideal of the field $K$, it follows that $\ker(\rho)\in\{ 0,K\}$. Thus, it suffices to show that $\rho$ is nonzero, whence it would follow that $\ker(\rho) \neq K$ and thus $\ker(\rho) = 0$. To see this, note that $1\neq 0$ in $K$ and $\rho(1) = [T_1]_{\Omega} = [id_K]_{\Omega} \neq 0$ as $id_K (\alpha_i) = \alpha_i \neq 0$.
  \end{proof}

  \textbf{(b)}: Prove that the minimal polynomial of $\alpha$ over $F$ and the minimal polynomial of $A_\alpha$ coincide.

  \begin{proof}
    Let $\mu_\alpha = \sum_{k=0}^{s} c_k x^k\in F[x]$ be the minimal polynomial of $\alpha$ over $F$. Let $\{e_1,\cdots, e_n\}$ be the standard basis for $F^n$. On one hand, note that for $1\leq i\leq n$,
    \[
      \mu_{\alpha}(A_\alpha)(e_i) = \lr{\sum_{k=0}^{s}c_k A_{\alpha}^k} (e_i) = \sum_{k=0}^{s}c_k [T_\alpha ^k(\alpha_i)]  = \sum_{k=0}^{s}c_k \alpha^k\alpha_i = \mu_\alpha(\alpha)\alpha_i = 0
    \]
    whence $\mu_\alpha(A_\alpha) = 0$. Thus $\mu_\alpha\in \Ann(A_\alpha)$. \\

    On the other hand, suppose that $f(x)\in \Ann(A_\alpha)$. Observe that, for $\beta\in K$
    \[
      f(T_\alpha)(\beta) = \lr{\sum_{k=0}^{s}b_k T_\alpha ^k }(\beta) = \sum_{k=0}^{s}b_k \alpha^k\beta = T_{f\alpha}(\beta),
    \]
    so $f(T_\alpha) = T_{f(\alpha)}$. Then
    \[
      0 = f(A_\alpha) = \sum_{k=0}^{s}b_k [T_\alpha]_\Omega ^k = \left[\sum_{k=0}^{s} b_k T_\alpha ^k\right] = [f(T_\alpha)]_\Omega = [T_{f(\alpha)}]_\Omega = A_{f(\alpha)} = \rho(f(\alpha)),
    \]
    whence by injectivity of $\rho$, $f(\alpha) = 0$, i.e. $f(x)\in(\mu_\alpha)$.\\

    Thus $(\mu_\alpha) = \Ann(A_\alpha)$, so by uniqueness of the monic generators for each of these ideals, the minimal polynomial for $\alpha$ over $F$ and the minimal polynomial of $A_\alpha$ coincide.
  \end{proof}
\end{homeworkProblem}

\begin{homeworkProblem}
  Let $K/F$ be an extension of fields and let $F\subseteq K_1\subseteq K$ and $F\subseteq K_2\subseteq K$ be two subextensions of $K/F$. The \textit{compositum} of $K_1$ and $K_2$ is the smallest subfield of $K$ that contains both $K_1$ and $K_2$. \textbf{Notation:} We denote the compositum by $K_1 K_2$. \\

  \textbf{(a)}: Consider the $F$-algebra $K_1\otimes_F K_2$. Show that there exists a unique $F$-algebra homomorphism $\Phi:K_1\otimes_F K_2\rightarrow K_1 K_2$ such that $\Phi(a\otimes b)=ab$. Conclude that $[K_1 K_2:F]\leq [K_1:F][K_2:F]$.

  \begin{proof}
    Define a map $\phi:K_1\times K_2\to K_1 K_2$ by $\phi(a,b) = ab$. This map is clearly $F$-bilinear and $\phi(ac,bd) = acbd = abcd = \phi(a,b)\phi(c,d)$, so there exists a unique $F$-algebra homomorphism $\Phi:K_1\otimes_F K_2\rightarrow K_1 K_2$ such that $\Phi(a\otimes b)=ab$.

    % TODO Show Phi is surjective.
    If either $K_1$ or $K_2$ is infinite degree over $F$, then the inequality is trivially true, so assume $[K_1:F],[K_2:F]<+\infty$. Then, by rank nullity theorem,
    \[
      [K_1 K_2 : F] = \dim_F(K_1 K_2) \leq \dim_F(K_1\otimes K_2) = \dim_F(K_1)\dim_F(K_2) = [K_1:F][K_2:F].
    \]
  \end{proof}

  \textbf{(b)}: Assuming that $K_1,K_2$ are finite degree extensions over $F$, show that $K_1\otimes_F K_2$ is a field if and only if the above $\leq$ becomes an equality.

  \textbf{(c)}: Suppose that $K_1\cap K_2\neq F$. Prove that $K_1\otimes_F K_2$ is not a field. \\
\end{homeworkProblem}


\end{document}
